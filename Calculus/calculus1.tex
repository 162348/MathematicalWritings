\documentclass[uplatex, 12pt, a4paper]{jsarticle}
\title{微分積分学 石毛和弘\footnote{メールアドレスはishige@ms.u-tokyo.ac.jp} \\ 講義ノート}
\author{司馬博文 J4-190549 \\ hirofumi-shiba48@g.ecc.u-tokyo.ac.jp}
\date{2019年9月30日}
\pagestyle{headings}
\usepackage{amsmath}
\usepackage{amsfonts}
\usepackage{amsthm}
\newtheorem{theorem}{定理}
\newtheorem{definition}{定義}
\newtheorem{proposition}{命題}
\newtheorem{corollary}[proposition]{系}
\newtheorem{lemma}[proposition]{補題}
\usepackage[top=15truemm,bottom=15truemm,left=10truemm,right=10truemm]{geometry}
\usepackage{color}
\setcounter{secnumdepth}{4}
\usepackage{comment}
\begin{document}
\maketitle

\begin{abstract}微分積分学②では,まず(2変数の)偏微分とその応用を復習してから,1変数関数の積分,次に多変数関数の積分へと進む.\\Jacobianの計算などが,後期試験の花形になるであろう.最後に級数に触れて,微分積分学②の講義は締めくくられる.絶対収束という概念を導入して,より詳しく収束を議論する.\end{abstract}

\part{偏微分とその応用}
\section{復習}
\subsection{開集合と閉集合の概念}
\begin{definition}$D \subset \mathbb{R}$,$P \in D$\\
1, $P$が$D$の内点である $\Longleftrightarrow$ $\exists r>0 \, \mathrm{s.t.} \, B(P,r):=\{ Q \in \mathbb{R}^2 | d(P,Q)<R \} \subset D$ \\
\indent*ただし,$d$とは距離函数で,$d(P,Q)=\sqrt{(p_1-q_1)^2 + (p_2-q_2)^2}=|P-Q|$\\
2, $P$が$D$の外点である $\Longleftrightarrow$ $\exists r>0 \, \mathrm{s.t.} \, B(P,r) \cap D=\emptyset$ \\
$\Longleftrightarrow$ $P$が$D^c$の内点.\\
3, $P$が$D$の境界点である $\Longleftrightarrow$ $\exists r>0 \, \mathrm{s.t.} \, B(P,r) \cap D\neq \emptyset \and B(P,r) \cap D^c \neq \emptyset$ \\
\end{definition}
\noindent
*$D$の境界点が全て$D$に含まれる$\Longrightarrow$Dは閉集合.$D$の境界点が全て$D$に含まれない$\Longrightarrow$Dは開集合.($D$が開集合$\Longleftrightarrow D^c$が閉集合.\\
*微積分学の考察対象は大体全て「領域」で定義されている.

\begin{definition}[連結・領域]$D$の任意の点$P,Q$に対して,$P,Q$を結ぶ$D$内の曲線が存在する時,$D$を(弧状)連結という.特に,$D$が開集合でもある時に,これを「領域」という.
\end{definition}

\begin{definition}[点列]$\mathbb{R}^2$上の点列$\{Pn\}$が,$P\in \mathbb{R}^2$に収束する,とは,$$\lim_{n\to \infty} d(Pn,P)=0 \mathrm{(これを} \lim_{n\to \infty} Pn=P \mathrm{とも書く.} $$
\end{definition}
\noindent
*点列の極限は,存在すればただ一つである.\\
*収束する点列は有界である.\\
*収束列の全ての部分列は,同じ極限点に収束する.
\begin{theorem}[Bolzano-Weierstrassの定理]($\mathbb{R}^2$上の)有界な点列は,収束する部分列を持つ.\end{theorem}
\begin{theorem}[Cauchy列は収束する]点列$\{ Pn \}$が収束列であることと,点列$\{ Pn \}$がCauchy列であることは,$\mathbb{R}^n$上では同値である.ただし,Cauchy列であるとは,点列が$$\forall \epsilon >0, \exists N \in \mathbb{N} \forall n,m \in \mathbb{N} d(Pn,Pm)<\epsilon $$を満たすこと. \end{theorem}
\noindent
*(remark)$D$が開集合$\Longleftrightarrow D$から取り出した収束列の極限点は,必ず$D$に属する.

\begin{definition}[連続性]$D \subset \mathbb{R}^2$を集合,$f:D \Longrightarrow \mathbb{R}$を写像,$P_0 \in \mathbb{R}^2, l \in \mathbb{R}$に対して,$$ \lim_{P \to P_0} f(P)=l \\ \Longleftrightarrow \forall \epsilon >0, \exists \delta>0 \forall P\in D 0<d(P,P_0)<\delta \Longrightarrow |f(P)-l|<\epsilon$$特に,$P_0 \in D, l=f(P_0)$である時,$f$は点$P$で連続であるという.\end{definition}

\begin{theorem}[最大値・最小値定理]\label{maxmin}$D$を有界閉集合,$f$を$D$上連続な関数とすると,$f$は$D$上有界であり,その最大値と最小値を$D$の上に取る.\end{theorem}

\begin{definition}[偏微分]$f$を点$(x_0,y_0)$の近傍で定義された関数とする.$$\frac{\partial}{\partial x}f(x_0,y_0):=\lim_{h\to 0} \frac{f(x_0+h,y_0)-f(x_0,y_0)}{h}$$同様にして,$\frac{\partial}{\partial y}f$ $\frac{\partial^2}{\partial x^2}f$ $\frac{\partial^2}{\partial x \partial y}f$を定義する.\end{definition}

\begin{theorem}[Schwarzの定理]点$(a,b)$の周りで,$f_x, f_y, f_{xy}$が存在し,$f_{xy}$が$(a,b)$にて連続ならば,$f_{yx}(a,b)$は存在し,$f_{xy}(a,b)=f_{yx}(a,b)$を満たす.\end{theorem}
\begin{proof}$\varphi(x,y):=f(x,y)-f(x,b)$と置く.十分小さい$h,k>0$に対して,\begin{eqnarray*} \varphi(a+h,b+k) - \varphi(a,b+k) \\ &=& \varphi_x (a+\theta h,b+k)h (0<\exists \theta <1) \\ &=& \left[ f_x(a+\theta h,b+k) - f_x(a+\theta h,b) \right]  h\\  &=& f_{xy}(a+\theta h, b+\theta' k)hk (0<\exists \theta' <1) \end{eqnarray*}
一方,$\varphi$の定義より,$$\lim_{k\to 0} \frac{\varphi(x,b+k)}{k} = \lim_{k \to 0} \frac{f(x,b+k)-f(x,b)}{k} = f_y(x,b) $$よって,$$f_y(a+h,b) - f_y(a,b) \\ = \lim_{k \to 0} \frac{\varphi(a+h,b+k)-\varphi(a,b+k)}{k} \\ = h\lim_{h\to 0} f_{xy}(a+\theta h, b+\theta'h) $$故に,\begin{eqnarray*} f_{yx}(a,b)= \lim_{h\to0} \lim_{k\to0} f_{xy} (a+\theta h, b+\theta'h) \\ = f_{xy}(a,b) \hspace{10mm} f_{xy}\mathrm{の点}(a,b)\mathrm{での連続性より} \end{eqnarray*}
\end{proof}

\begin{definition}[全微分可能性]$f$を点$(a,b)$の近傍で定義された関数とする.$f$が点$(a,b)$で全微分可能であるとは,$\exists A,B \in \mathbb{R} \mathrm{s.t.} f(a+h,b+k)-f(a,b)=Ah+Bk+o(\sqrt{h^2+k^2})  \left( (h,k)\to(0,0) \right)$と定義する.\end{definition}
\noindent
*$f$が点$(a,b)$で全微分可能\\ $\Longrightarrow$ $f$は点$(a,b)$で偏微分可能\\ $\Longrightarrow$ $A=f_x(a,b), B=f_y(a,b)$

\begin{theorem}$f_x,f_y$が存在し,$f_x$または$f_y$が連続.この時,$f$は全微分可能である.
\end{theorem}
\noindent
*特に,$f$が$C^1$級の関数ならば,十分に全微分可能である.

\begin{theorem}[Chain Rule]$z=f(x,y), x=x(u,v), y=y(u,v)$の時,$$\frac{\partial z}{\partial u} = \frac{\partial z}{\partial x} \frac{\partial x}{\partial u} + \frac{\partial z}{\partial y} \frac{\partial y}{\partial u} \\  \frac{\partial z}{\partial v} = \frac{\partial z}{\partial x} \frac{\partial x}{\partial v} + \frac{\partial z}{\partial y} \frac{\partial y}{\partial v}$$が成り立つ.\end{theorem}

\begin{theorem}[Taylorの定理]$f$:領域$D$上の$C^n$級関数$(n=1,2,......)$\\線分$\{ (a+ht,b+kt) | 0<t<1 \}$が$D$に含まれているとする.ここで,$F(t):=f(a+th,b+tk)$と置くと,$$f(a+h,b+k)=F(1)\\ =F(0)+F'(0)+\frac{F''(0)}{2!}+...+\frac{F^{(n-1)}(0)}{(n-1)!}+\frac{F^{(n)}(\theta)}{n!}\\=f(a,b)+\left( h\frac{\partial}{\partial x}+ k\frac{\partial}{\partial y} \right)f(a,b)+...+\frac{1}{(n-1)!} \left( h\frac{\partial}{\partial x}+ k\frac{\partial}{\partial y} \right)^{(n-1)} f(a,b) + \frac{1}{n!} \left( h\frac{\partial}{\partial x}+ k\frac{\partial}{\partial y} \right)^n f(a+\theta h,b\theta k)   $$ \end{theorem}

\begin{theorem}[陰関数定理]$F=F(x,y)$:点$(a,b)$の近傍で定義された$C^1$級関数.点$(a,b)$において,$F(a,b)=0, F_y(a,b) \neq 0$とする.この時,$\exists I$:点$a$を含む閉区間$\exists \delta>0$s.t.$\forall x \in I$に対して$F(x,y)=0, |y-b|<\delta$を満たす.$y$は唯一つ,この$I$上の関数$y=f(x)$は$C^1$級関数であり,$$f'(x)=-\frac{F_x(x,f(x))}{F_y(x,f(x))}$$
\end{theorem}

\section{極限}
\noindent
問:周の長さが一定$(=2s)$の三角形のうちで,面積最大のものは正三角形であることを示せ.
\begin{proof}三角形の三辺の長さを$x,y,z$とする.対応する三角形の面積を$S$とすると,Heronの公式より,$$S^2=s(s-x)(s-y)(s-z) (2s=x+y+z) \\ =s(s-x)(s-y)(x+y-s)$$である.$f(x,y)=(s-x)(s-y)(x+y-s)$と置く.$x,y$が動く範囲は$$D:=\{ (x,y)|s-x>0, s-y>0 x+y-s>0 \}$$であるが,$f$に最大値が存在するには,$D$が閉集合でなくてはならない(すでに有界ではある).$D$の拡張$\overline{D}$,$$\overline{D} := \{ (x,y)|s-x \ge 0, s-y \ge 0, x+y-s \ge 0 \}$$上で$f$を考える.$\overline{D}$は有界閉集合で,$f$はその上で連続であるから,最大値最小値定理\ref{maxmin}より,$f$は$\overline{D}$上で最大値を取る.$f$は$D$上正,$D$の境界点上で零なので,最大値を取る点は$\overline{D}$の内点,つまり$D$上で取る.その点を$(x,y)\in D$とすると,$$f_x(x,y)=-(s-y)(x+y-s)+(s-x)(s-y)=0\\f_y(x,y)=-(s-x)(x+y-s)+(s-x)(s-y)=0\\ \Longrightarrow x=y=z$$よって,面積の最大値を与える三角形は正三角形である.
\end{proof}

\end{document}