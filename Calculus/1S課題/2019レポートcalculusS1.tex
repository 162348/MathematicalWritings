\documentclass[uplatex, 12pt, a4paper]{jsarticle}
\title{微分積分学S2ターム レポート}
\author{司馬博文 J4190549}
\date{2019年7月8日}
\usepackage{amsmath}
\usepackage{amsfonts}
\usepackage[top=15truemm,bottom=15truemm,left=10truemm,right=10truemm]{geometry}
\usepackage{color}
\begin{document}

\maketitle

\noindent
\textgt{問題1}:$a>0$とし,$$f(x)=x\log{(a+x)}$$を考える.このとき,$f$がマクローリン展開可能となる$x$の範囲を求め,それを説明せよ.\\
\textgt{答}:$$
\begin{cases}
|x| \le 1& a>1のとき \\
|x| < 1& a=1のとき\\
マクローリン展開可能なxは存在しない& 0<a<1のとき
\end{cases}$$
\textgt{説明}:求めるマクローリン展開可能となる$x$の条件は,関数$g(x)=\log{(a+x)}$のものと一致する.
Taylorの定理のCauchyの剰余項による表示により,この$g(x)$は,$x=0$の周りで,任意の$n \in \mathbb{N}$に対して,$\theta \in (0,1)$が存在して,$$g(x)= \sum_{k=0}^{n-1} \frac{f^{(k)}(0)}{k!} x^k + R_n(x) $$ $$ 但しR_n(x)=\frac{f^{(k)}(\theta x)}{(n-1)!} (x-\theta x)^{n-1} x$$と表せる.
したがって,$|R_n(x)| \to 0 (as \hspace{2mm} n \to \infty)$となる条件を求めれば良い.
$$\left|R_n(x)\right|=\left| \frac{1-\theta}{a+\theta x} \right| ^n \frac{|x|^n}{1-\theta}$$であるから,それは$\left| \frac{1-\theta}{a+\theta x} \right| \le 1$かつ$|x| \le 1$かつ「$| \frac{1-\theta}{a+\theta x} |$と$|x|$とは同時に1に恒常的に等しい訳ではない」という条件に等しい.この最後の条件は,$a=1$かつ$\theta =0$のとき,$|x|=1$ならば当てはまってしまうことに注意……($\sharp$)すれば,$|x| \le 1$の下では,第一の条件は$$|R_n(x)|=\left| \frac{1-\theta}{a+\theta x} \right| \le \frac{1-\theta}{a-\theta |x|} \le \frac{1-\theta}{a-\theta} \le 1$$即ち,十分条件は$$1-\theta \le a-\theta \Leftrightarrow 1 \le a$$つまり,$a \ge 1$の場合は,$(\sharp)$の場合を除いて,$|x| \le 1$の下にてマクローリン展開可能である.\par
次に$0<a<1$の場合を考える.このとき,$| \frac{1-\theta}{a+\theta x} |$の値は,$x$の値に関わらず,$\theta$が$0$から$1$の間で変化するにしたがって,$1$より大きかったり小さかったりして定まらない.したがって,この場合はマクローリン展開出来ない.

\vspace{10mm}

\noindent
\textgt{問題2}:$\sin{0.1}$を小数第5位まで求めよ.\\
\textgt{答}:$\sin{x}$を$x=0$を中心としてTaylor展開を考えることにより,$$\sin{x}=x-\frac{x^3}{6}+\frac{x^5}{120}+o(x^6)$$と表せる.これに$x=0.1$を代入することにより,$$\sin{0.1}=0.1-\frac{(0.1)^3}{6}+\frac{(0.1)^5}{120}+((0.1)^6以下の微小項)$$より,$$\sin{0.1}=0.1-0.0001666......+0.00000008+(小数点6桁以下の小数)=\underline{0.09983......}\hspace{5mm}・・・(答)$$
\vspace{10mm}

\noindent
\textgt{問題3}:次を満たす実数$A,\alpha$を求めよ.$$\frac{1}{x^3}-\frac{1}{\sin^3{x}}=Ax^{\alpha}+O(x^{\alpha+1}) \hspace{5mm} (|x| \to 0)$$\\
\textgt{答}:\begin{eqnarray*}
\frac{1}{x^3}-\frac{1}{\sin^3{x}} &=& \frac{1}{x^3}-\frac{1}{\left(x-\frac{x^3}{6}+O(x^5)\right)^3}\\
&=& \frac{1}{x^3}-\frac{1}{x^3-\frac{x^5}{2}+O(x^7)}\\
&=& \frac{1}{x^3} \left( 1-\frac{1}{1-\frac{x^2}{2}+O(x^4)} \right)\\
&=& \frac{1}{x^3} \left[ 1- \left( 1+ \left( \frac{x^2}{2} + O(x^4) \right) + \left(\frac{x^2}{2} + O(x^4)\right)^2 + O(x^4) \right) \right]\\
&=& \frac{1}{x^3} \left( -\frac{x^2}{2} + O(x^4) \right) \\
&=& -\frac{1}{2x} + O(x)\\
よって,\underline{A = -\frac{1}{2},\hspace{2mm} \alpha = -1} \hspace{10mm} ・・・(答)
\end{eqnarray*}
*注:或る関数が$x \to 0$にて$O(x)$のオーダーである時,$O(x^0)$即ち$O(1)$とも表せる.

\begin{eqnarray*}
    (1-\frac{1}{n^2})^n &=& \left{ (1-\frac{1}{n^2})^{n^2} \right}^{\frac{1}{n}} \\
    &\longrightarrow& e^0 = 1
\end{eqnarray*}


\end{document}