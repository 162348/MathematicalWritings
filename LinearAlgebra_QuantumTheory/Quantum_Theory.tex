\documentclass[uplatex, 12pt, dvipdfmx]{jsreport}
    
\title{線型代数から物質科学まで}
\author{司馬博文 J4-190549\\hirofumi-shiba48@g.ecc.u-tokyo.ac.jp}
\date{\today}
\pagestyle{headings} \setcounter{secnumdepth}{4}
\usepackage{amsmath, amsfonts, amsthm, amssymb, ascmac, color, comment, wrap fig}

\usepackage{mathtools}
\mathtoolsset{showonlyrefs=true} %labelを附した数式にのみ附番される.

\usepackage{tikz, tikz-cd}
\usepackage[all]{xy}
\def\objectstyle{\displaystyle} %デフォルトではxymatrix中の数式が文中数式モードになるので,それを直した.

%化学式をTikZで簡単に書くためのパッケージ.
\usepackage[version=4]{mhchem} %texdoc mhchem
%化学構造式をTikZで描くためのパッケージ.
\usepackage{chemfig}
%IS単位を書くためのパッケージ
\usepackage{siunitx}
%取り消し線を引くためのパッケージ
\usepackage{ulem}

%\rotateboxコマンドを,文字列の中心で回転させるオプション.
%他rotatebox, scalebox, reflectbox, resizeboxなどのコマンド.
\usepackage{graphicx}

%加藤晃史さんがフル活用していたtcolorboxを,途中改ページ可能で.
\usepackage[breakable]{tcolorbox}

%enumerate環境を凝らせる.
\usepackage{enumerate}

%日本語にルビをふる
\usepackage{pxrubrica}

%足助さんからもらったオプション
%\usepackage[shortlabels,inline]{enumitem}
%\usepackage[top=15truemm,bottom=15truemm,left=10truemm,right=10truemm]{geometry}

%以下,ソースコードを表示する環境の設定.
\usepackage{listings,jvlisting} %日本語のコメントアウトをする場合jlistingが必要
%ここからソースコードの表示に関する設定
\lstset{
  basicstyle={\ttfamily},
  identifierstyle={\small},
  commentstyle={\smallitshape},
  keywordstyle={\small\bfseries},
  ndkeywordstyle={\small},
  stringstyle={\small\ttfamily},
  frame={tb},
  breaklines=true,
  columns=[l]{fullflexible},
  numbers=left,
  xrightmargin=0zw,
  xleftmargin=3zw,
  numberstyle={\scriptsize},
  stepnumber=1,
  numbersep=1zw,
  lineskip=-0.5ex
}
%lstlisting環境で,[caption=hoge,label=fuga]などのoptionを付けられる.
\makeatletter
    \AtBeginDocument{
    \renewcommand*{\thelstlisting}{\arabic{chapter}.\arabic{section}.\arabic{lstlisting}}
    \@addtoreset{lstlisting}{section}
    }
\makeatother
%caption番号を「[chapter番号].[section番号].[subsection番号]-[そのsubsection内においてn番目]」に変更
\renewcommand{\lstlistingname}{program}
%caption名を"program"に変更

%%%
%%%フォント
%%%

%本文・数式の両方のフォントをTimesに変更するお手軽なパッケージだが,LaTeX標準数式記号の\jmath, \amalg, coprodはサポートされない.
% \usepackage{mathptmx}
%Palatinoの方が完成度が高いと美文書作成に書いてあった.
\usepackage[sc]{mathpazo} %オプションは,familyの指定.pplxにしている.
%2000年のYoung Ryuによる新しいTimes系.なおPalatinoもある.
% \usepackage{newtxtext, newtxmath}
%拡張数学記号.\textsectionでブルバキに!
\usepackage{textcomp, mathcomp}
\usepackage[T1]{fontenc} %8bitエンコーディングにする.comp系拡張数学文字の動作が安定する.
%AMS Euler.Computer Modernと相性が悪いとは…….
\usepackage{ccfonts, eulervm} %KnuthのConcrete Mathematicsの組み合わせ.
% \renewcommand{\rmdefault}{pplx} %makes LaTeX use Palatino in place of CM Roman.Do not use the Euler math fonts in conjunction with the default Computer Modern text fonts – this is ugly!

%%% newcommands
    %参考文献で⑦というのを出したかった.\circled{n}と打てば良い.LaTeX StackExchangeより.
\newcommand*\circled[1]{\tikz[baseline=(char.base)]{\node[shape=circle,draw,inner sep=0.8pt] (char) {#1};}}

%%%
%%% ショートカット 足助さんからのコピペ
%%%

\DeclareMathOperator{\grad}{\mathrm{grad}}
\DeclareMathOperator{\rot}{\mathrm{rot}}
\DeclareMathOperator{\divergence}{\mathrm{div}}
\newcommand\R{\mathbb{R}}
\newcommand\N{\mathbb{N}}
\newcommand\C{\mathbb{C}}
\newcommand\Z{\mathbb{Z}}
\newcommand\Q{\mathbb{Q}}
\newcommand\GL{\mathrm{GL}}
\newcommand\SL{\mathrm{SL}}
\newcommand\False{\mathrm{False}}
\newcommand\True{\mathrm{True}}
\newcommand\tr{\mathrm{tr}}
\newcommand\M{\mathcal{M}}
\newcommand\F{\mathbb{F}}
% \newcommand\H{\mathbb{H}} すでにある.
\newcommand\id{\mathrm{id}}
\newcommand\A{\mathcal{A}}
%\renewcommand\coprod{\rotatebox[origin=c]{180}{$\prod$}}
\newcommand\pr{\mathrm{pr}}
\newcommand\U{\mathfrak{U}}
\newcommand\Map{\mathrm{Map}}
\newcommand\dom{\mathrm{dom}}
\newcommand\cod{\mathrm{cod}}
\newcommand\supp{\mathrm{supp}}
%%% 複素解析学
\renewcommand\Re{\mathrm{Re}\;}
\renewcommand\Im{\mathrm{Im}\;}
\newcommand\Gal{\mathrm{Gal}}
\newcommand\PGL{\mathrm{PGL}}
\newcommand\PSL{\mathrm{PSL}}
%%% 解析力学
\newcommand\x{\mathbf{x}}
\newcommand\q{\mathbf{q}}
%%% 集合と位相
\newcommand\ORD{\mathrm{ORD}}

%%% 圏
\newcommand\Hom{\mathrm{Hom}}
\newcommand\Mor{\mathrm{Mor}}
\newcommand\Aut{\mathrm{Aut}}
\newcommand\End{\mathrm{End}}
\newcommand\op{\mathrm{op}}
\newcommand\ev{\mathrm{ev}}
\newcommand\Ob{\mathrm{Ob}}
\newcommand\Ar{\mathrm{Ar}}
\newcommand\Arr{\mathrm{Arr}}
\newcommand\Set{\mathrm{Set}}
\newcommand\Grp{\mathrm{Grp}}
\newcommand\Cat{\mathrm{Cat}}
\newcommand\Mon{\mathrm{Mon}}
\newcommand\CMon{\mathrm{CMon}}
\newcommand\Pos{\mathrm{Pos}}
\newcommand\Vect{\mathrm{Vect}}
\newcommand\FinVect{\mathrm{FinVect}}
\newcommand\Fun{\mathrm{Fun}}
\newcommand\Ord{\mathrm{Ord}}

%%%
%%% 定理環境 以下足助さんからのコピペ
%%%

\newtheoremstyle{StatementsWithStar}% ?name?
{3pt}% ?Space above? 1
{3pt}% ?Space below? 1
{}% ?Body font?
{}% ?Indent amount? 2
{\bfseries}% ?Theorem head font?
{\textbf{.}}% ?Punctuation after theorem head?
{.5em}% ?Space after theorem head? 3
{\textbf{\textup{#1~\thetheorem{}}}{}\,$^{\ast}$\thmnote{(#3)}}% ?Theorem head spec (can be left empty, meaning ‘normal’)?
%
\newtheoremstyle{StatementsWithStar2}% ?name?
{3pt}% ?Space above? 1
{3pt}% ?Space below? 1
{}% ?Body font?
{}% ?Indent amount? 2
{\bfseries}% ?Theorem head font?
{\textbf{.}}% ?Punctuation after theorem head?
{.5em}% ?Space after theorem head? 3
{\textbf{\textup{#1~\thetheorem{}}}{}\,$^{\ast\ast}$\thmnote{(#3)}}% ?Theorem head spec (can be left empty, meaning ‘normal’)?
%
\newtheoremstyle{StatementsWithStar3}% ?name?
{3pt}% ?Space above? 1
{3pt}% ?Space below? 1
{}% ?Body font?
{}% ?Indent amount? 2
{\bfseries}% ?Theorem head font?
{\textbf{.}}% ?Punctuation after theorem head?
{.5em}% ?Space after theorem head? 3
{\textbf{\textup{#1~\thetheorem{}}}{}\,$^{\ast\ast\ast}$\thmnote{(#3)}}% ?Theorem head spec (can be left empty, meaning ‘normal’)?
%
\newtheoremstyle{StatementsWithCCirc}% ?name?
{6pt}% ?Space above? 1
{6pt}% ?Space below? 1
{}% ?Body font?
{}% ?Indent amount? 2
{\bfseries}% ?Theorem head font?
{\textbf{.}}% ?Punctuation after theorem head?
{.5em}% ?Space after theorem head? 3
{\textbf{\textup{#1~\thetheorem{}}}{}\,$^{\circledcirc}$\thmnote{(#3)}}% ?Theorem head spec (can be left empty, meaning ‘normal’)?
%
\theoremstyle{definition}
 \newtheorem{theorem}{定理}[section]
 \newtheorem{axiom}[theorem]{公理}
 \newtheorem{corollary}[theorem]{系}
 \newtheorem{proposition}[theorem]{命題}
 \newtheorem*{proposition*}{命題}
 \newtheorem{lemma}[theorem]{補題}
 \newtheorem*{lemma*}{補題}
 \newtheorem*{theorem*}{定理}
 \newtheorem{definition}[theorem]{定義}
 \newtheorem{example}[theorem]{例}
 \newtheorem{notation}[theorem]{記法}
 \newtheorem*{notation*}{記法}
 \newtheorem{assumption}[theorem]{仮定}
 \newtheorem{question}[theorem]{問}
 \newtheorem{counterexample}[theorem]{反例}
 \newtheorem{reidai}[theorem]{例題}
 \newtheorem{problem}[theorem]{問題}
 \newtheorem*{solution*}{\bf{[解]}}
 \newtheorem{discussion}[theorem]{議論}
 \newtheorem{remark}[theorem]{注}
 \newtheorem{universality}[theorem]{普遍性} %非自明な例外がない.
 \newtheorem{universal tendency}[theorem]{普遍傾向} %例外が有意に少ない.
 \newtheorem{hypothesis}[theorem]{仮説} %実験で説明されていない理論.
 \newtheorem{theory}[theorem]{理論} %実験事実とその(さしあたり)整合的な説明.
 \newtheorem{fact}[theorem]{実験事実}
 \newtheorem{model}[theorem]{模型}
 \newtheorem{explanation}[theorem]{説明} %理論による実験事実の説明
 \newtheorem{anomaly}[theorem]{理論の限界}
 \newtheorem{application}[theorem]{応用例}
 \newtheorem{method}[theorem]{手法} %実験手法など,技術的問題.
 \newtheorem{history}[theorem]{歴史}
 \newtheorem{research}[theorem]{研究}
% \newtheorem*{remarknonum}{注}
 \newtheorem*{definition*}{定義}
 \newtheorem*{remark*}{注}
 \newtheorem*{question*}{問}
 \newtheorem*{axiom*}{公理}
 \newtheorem*{example*}{例}
%
\theoremstyle{StatementsWithStar}
 \newtheorem{definition_*}[theorem]{定義}
 \newtheorem{question_*}[theorem]{問}
 \newtheorem{example_*}[theorem]{例}
 \newtheorem{theorem_*}[theorem]{定理}
 \newtheorem{remark_*}[theorem]{注}
%
\theoremstyle{StatementsWithStar2}
 \newtheorem{definition_**}[theorem]{定義}
 \newtheorem{theorem_**}[theorem]{定理}
 \newtheorem{question_**}[theorem]{問}
 \newtheorem{remark_**}[theorem]{注}
%
\theoremstyle{StatementsWithStar3}
 \newtheorem{remark_***}[theorem]{注}
 \newtheorem{question_***}[theorem]{問}
%
\theoremstyle{StatementsWithCCirc}
 \newtheorem{definition_O}[theorem]{定義}
 \newtheorem{question_O}[theorem]{問}
 \newtheorem{example_O}[theorem]{例}
 \newtheorem{remark_O}[theorem]{注}
%
\theoremstyle{definition}
%
\raggedbottom
\allowdisplaybreaks

%証明環境のスタイル
\everymath{\displaystyle}
\renewcommand{\proofname}{\bf [証明]}
\renewcommand{\thefootnote}{\dag\arabic{footnote}}	%足助さんからもらった.どうなるんだ?

%mathptmxパッケージ下で,\jmath, \amalg, coprodの記号を出力するためのマクロ.TeX Wikiからのコピペ.
% \DeclareSymbolFont{cmletters}{OML}{cmm}{m}{it}
% \DeclareSymbolFont{cmsymbols}{OMS}{cmsy}{m}{n}
% \DeclareSymbolFont{cmlargesymbols}{OMX}{cmex}{m}{n}
% \DeclareMathSymbol{\myjmath}{\mathord}{cmletters}{"7C}
% \DeclareMathSymbol{\myamalg}{\mathbin}{cmsymbols}{"71}
% \DeclareMathSymbol{\mycoprod}{\mathop}{cmlargesymbols}{"60}
% \let\jmath\myjmath
% \let\amalg\myamalg
% \let\coprod\mycoprod

\begin{document}
\maketitle
\begin{abstract}
    現代の宇宙観に迫りたい.数学が支える点が違うから,そこから追って行くことに基礎を置いた.(Bourbaki)
    「世界は線型代数のことばで理解出来たね.」という1つの到達点を確認したい.
\end{abstract}
\tableofcontents

\part{線型代数の世界}
\chapter{行列と線型空間}

体$K$,その直積$K^n$やその元についての数ベクトルの概念はすでに与えられているとする.

\section{基底と次元}

\begin{shadebox}\begin{definition}[基底]\rm{}
    $V$を$K$-線型空間,$x_1,\cdots,x_n\in V$とする.このとき,以下の2条件は同値.これらの同値な条件を満たすことを,「$x_1,\cdots,x_n$は$V$の基底である」という.\\
    1. $\forall x\in V \, \exists ! a_1,\cdots, a_n\in K \, [x=a_1\cdot x_1 +\cdots +a_n\cdot x_n]$\\
    2. 以下のように定める写像$f$は可逆である.
    \begin{center}\begin{tikzcd}
        f:K^n \ar[r] \ar[d, phantom, "\rotatebox{90}{$\in$}"] & V \ar[d, phantom, "\rotatebox{90}{$\in$}"] \\
        (a_1,\cdots,a_n) \ar[r, mapsto] & x=a_1\cdot x_1 +\cdots +a_n\cdot x_n
    \end{tikzcd}\end{center}
    この写像$f$を「$V$の基底$x_1,\cdots,x_n$が定める写像」と呼ぶ.これは線型写像の公理を満たす.
\end{definition}\end{shadebox}

\begin{proof}
$a_1,\cdots,a_n$が条件1を満たすとは,任意の$x\in V$に対して,ただ1つの$K^n$の元が対応して,$x=a_1\cdot x_1 +\cdots +a_n\cdot x_n \in V$と表されること,即ち$f^{-1}(x)$は常に一元集合であることだから,$f$は全単射であることに等しい.このとき写像$f$は可逆になる.従って,2つの条件は同値.
\end{proof}

\section{行列表示}

\begin{shadebox}\begin{definition}[行列表示(matrix representation)]
    $K$を体とし,$V,W$を有限次元$K$-線型空間とする.$B=(x_1,x_2,\cdots,x_n), B'=(y_1,y_2,\cdots,y_m)$をそれぞれ$V,W$の基底とする.$g_B:K^n\longrightarrow V,\; g'_{B'}K^m\longrightarrow W$を,それぞれ基底$B,B'$が定める同型とする.
    この時,以下の図式を可換にする行列$A\in M_{mn}(K)$が存在する.
    \begin{center}\begin{tikzcd}
        V \ar[r,"f"] & W \ar[d, "g'^{-1}_{B'}"] \\
        K^n \ar[u, "g_B"] \ar[r, "f_A"'] & K^m
    \end{tikzcd}\end{center}
    この行列$A$を,\textbf{基底$B,B'$に関する$f$の行列表示}という.
\end{definition}\end{shadebox}

\begin{definition}[底の変換行列(transformation matrix)]
    $x_1,\cdots,x_n$と$y_1,\cdots,y_n$とを$V$の基底とする.$x_1,\cdots,x_n$を$y_1,\cdots,y_n$に写す$V$の自己同型の,基底$x_1,\cdots,x_n$に関する行列表示$A\in GL_n(K)$を,\textbf{$x_1,\cdots,x_n$から$y_1,\cdots,y_n$への底の変換行列}という.
\end{definition}

\chapter{自己準同型}
\section{最小多項式}
\subsection{特別な自己準同型}

\begin{definition}[Frobeniusのcompanion matrix] \\
    \rm{}monicな多項式$F(X) = X^n+a_1X^{n-1}+\cdots +a_{n-1}X+a_n \in K[X]$に対して,
        $$C(F)=\begin{bmatrix}
        0 & 0 & \dots & 0 & -a_n \\
        1 & 0 & \dots & 0 & -a_{n-1} \\
        0 & 1 & \dots & 0 & -a_{n-2} \\
        \vdots & \vdots & \ddots & \vdots & \vdots \\
        0 & 0 & \dots & 1 & -a_1
        \end{bmatrix}$$
    と定義される正方行列$C(F)\in M_n(K)$のことを\textbf{多項式$F$の同伴行列}と呼ぶ.
\end{definition}
\begin{definition}[Jordan matrix]\rm{}
        行列$$J(a,n)=\begin{bmatrix}
        a & 1 & 0 & \dots & 0 \\
        0 & a & 1 & \dots & 0 \\
        \vdots & \vdots & \ddots & \vdots & \vdots \\
        0 & 0 & \dots & a & 1 \\
        0 & 0 & \dots & 0 & a
        \end{bmatrix}$$
    を\textbf{Jordan行列}という.
\end{definition}

\section{固有値と対角化}
\begin{definition}[eigenspace, eigenvalue, eigenvector]
    $V$を$K$-線型空間とし,$f$を$V$の自己準同型とする.$V$の特別な部分空間$V_a$
    $$V_a := \{ x\in V \,|\, f(x)=ax \} = \mathrm{Ker} (f-a)$$
    のことを,\textbf{$a$に属する固有空間}という.$V_a\ne 0$である時,\textbf{$a$は$f$の固有値である}という.固有空間$V_a$の$0$でない元を,\textbf{固有値$a$の固有ベクトル}という.
\end{definition}

\begin{definition}[conjugate]
    $A,B\in M_n(K)$とする.$B=P^{-1}AP$を満たす$P\in GL_n(K)$が存在するとき,\textbf{$A$と$B$は共軛である}という.
\end{definition}

\begin{definition}[generalized eigenspace]
    $V$を$K$-線型空間とし,$f$を$V$の自己準同型とする.$a\in K$とし,$d$を$f$の最小多項式$\varphi$の根$a$の重複度とする.$V$の部分空間
    $$\widetilde{V}_a:=\mathrm{Ker}(f-a)^d$$
    を,\textbf{$a$に属する一般固有空間}という.
\end{definition}

\begin{definition}[generalized eigenspace decomposition]
    $V$を有限次元$K$-線型空間とし,$f$を$V$の三角化可能な自己準同型とする.$a_1,\cdots,a_r\in K$を$f$の固有値とし,$\widetilde{V}_{a_i}$を固有値$a_i$の一般固有空間とする.\\ この時,直和分解
    $$V=\widetilde{V}_{a_i}\oplus\cdots\oplus\widetilde{V}_{a_r}$$
    を,\textbf{$f$に関する$V$の一般固有空間分解}という.
\end{definition}

\clearpage

\chapter{双線型形式}

\begin{definition}[行列Aが定める双線型形式] \\
    $A\in M_{mn}(K)$とする.$x\in K^m,\,y\in K^n$に対して,$$b_A(x,y)={}^txAy\in K$$と置くと,$b_A:K^m\times K^n\longrightarrow K$は双線型形式である.
\end{definition}
\begin{definition}[標準双線型形式] \\
    写像$<\hspace{2mm},\hspace{2mm}>:V\times V^*\longrightarrow K$を$$<x,f>=f(x)\in K$$で定めると,$<\hspace{2mm},\hspace{2mm}>$は双線型形式である.
\end{definition}

\begin{proposition}
    $b:V\times W\longrightarrow K$を双線型形式とする.$V$の基底$B=(x_1,\cdots,x_m)$から$B'=(x_1',\cdots,x_m')$への底の変換行列を$P\in GL_m(K)$とし,$W$の基底$D=(y_1,\cdots,y_n)$から$D'=(y_1',\cdots,y_n')$への底の変換行列を$Q\in GL_n(K)$とする.
    $B,D$に関する$b$の行列表示を$A$とし,$B',D'$に関する$b$の行列表示を$A'$とすると,$$A'={}^tPAQ$$である.
\end{proposition}

\begin{proposition}[転置と随伴関手]
    $V,W$を有限次元$K$-線型空間とし,$b:V\times W\longrightarrow K$を非退化な双線型形式とする.$f:V\longrightarrow V$を線型写像とし,$f^*:W\longrightarrow W$をその右随伴写像とする.\\
    $B,B'$をそれぞれ$V,W$の基底とし,$B,B'$に関する$b$の行列表示を$P\in GL_n(K)$とする.$f$の$B$に関する行列表示が$A\in M_n(K)$ならば,$B'$に関する随伴写像$f^*$の行列表示は,
    $$P^{-1}{}^tAP\in M_n(K)$$
    である.
\end{proposition}

\begin{definition}$b$を$V$上の対称双線型形式とする.\rm{}\\
    1. 部分空間$W\subset V$に対し,$W^\perp = \{ x\in V\,|\, \forall y\in W \hspace{3mm}[b(x,y)=0] \}$を,\textbf{$W$の$b$に関する直交(orthogonal)}と言う.$V^\perp$を\textbf{$b$の核}と言う.\\
    2. $W\subset V$を部分空間とする.$b:V\times V\longrightarrow K$の制限$W\times W\longrightarrow K$が定める$W$の双線型形式を\textbf{$b$の$W$への制限}と呼び,$b_W$で表す.\\
    3. $V$の基底$x_1,\cdots,x_n$に関する$b$の行列表示が対角行列であるとき,$x_1,\cdots,x_n$は\textbf{$b$の直交基底}であると言う.$b$の行列表示が単位行列であるとき,$x_1,\cdots,x_n$は\textbf{正規直交基底(orthonormal basis)}であると言う.
\end{definition}

\begin{definition}[hermitian form]$V$を$\mathbb{C}$-線型空間とする.\rm{}\\
    1. 写像$h:V\times V\longrightarrow\mathbb{C}$が\textbf{エルミート形式}であるとは,次の条件を満たすことを言う.\\
    \hspace{3mm}(1) $\forall x,x',y\in V \hspace{3mm} [h(x+x',y)=h(x,y)+h(x',y)]$\\
    \hspace{3mm}(2) $\forall x,y,y'\in V \hspace{3mm} [h(x,y+y')=h(x,y)+h(x,y')]$\\
    \hspace{3mm}(3) $\forall a\in\mathbb{C}, x,y\in V \hspace{3mm} [h(ax,y)=h(x,\overline{a}y)=ah(x,y)]$\\
    \hspace{3mm}(4) $\forall x,y\in V \hspace{3mm} h(y,x)=\overline{h(x,y)}$\\
    2. $h:V\times V\longrightarrow\mathbb{C}$をエルミート形式とする.$\forall x\in V, x\ne 0 \hspace{3mm}[h(x,x)>0]$であるとき,$h$は\textbf{正定値}であると言う.\\
    3. $A\in M_n(\mathbb{C})$に対して,$A^*=\overline{{}^t\!A}$と置き,これを$A$の\textbf{随伴行列(adkoint matrix)}と呼ぶ.$A\in M_n(\mathbb{C})$が$A=A^*$を満たす時,$A$は\textbf{エルミート行列}であると言う.
\end{definition}
\begin{definition}$V$を$\mathbb{C}$-線型空間とし,$h:V\times V\longrightarrow\mathbb{C}$をエルミート形式とする.\rm{}\\
    1. $\mathbb{R}$-線型写像$r_h:V\longrightarrow V^*$が単射である時,$h$は\textbf{非退化}であるという.\\
    2. $h$が非退化であるとする.$V$の自己準同型$f$が随伴写像(共軛とも呼ぶ)$f^*$と等しい時,$f$は\textbf{$h$に関してエルミート変換(自己共軛変換)である}という.$f^*$が$f$の逆写像であるとき,$f$は\textbf{$h$に関してユニタリ変換である}という.
\end{definition}

\begin{center}\begin{tikzcd}
    V^* \ar[r, "i^*"] \ar[d, phantom, "\rotatebox{90}{$\in$}"] & W^* \ar[d, phantom, "\rotatebox{90}{$\in$}"] \\
    f:V\to K \ar[r, mapsto] & f|_W:W\to K
\end{tikzcd}\end{center}
$$(f+g)|_W = f|_W + g|_W \hspace{3mm}かつ\hspace{3mm} (af)|_W = a\cdot f|_W$$
よって$$W^\perp = \mathrm{Ker}(i^*:V^*\to W^*)$$
また$$W^{\rotatebox{180}{$\perp$}}=\bigcap_{f\in W}\mathrm{Ker}(f:V\to K)$$

\begin{center}\begin{tikzcd}
    S_V \ar[r, "\perp"] \ar[d, phantom, "\rotatebox{90}{$\in$}"] & S_{V^*} \ar[d, phantom, "\rotatebox{90}{$\in$}"] & S_V \ar[d, phantom, "\rotatebox{90}{$\in$}"] & S_{V^*} \ar[d, phantom, "\rotatebox{90}{$\in$}"] \ar[l, "\rotatebox{180}{$\perp$}"'] \\
    W \ar[r, mapsto] & W^\perp & W'^{\rotatebox{180}{$\perp$}} & W' \ar[l, mapsto]
\end{tikzcd}\end{center}
\begin{center}\begin{tikzcd}
    V \ar[r, "e_V"] \ar[d, phantom, "\rotatebox{90}{$\in$}"] & (V^*)^* \ar[d, phantom, "\rotatebox{90}{$\in$}"] & \\
    x \ar[r, mapsto] & ev_x : V^* \ar[r] \ar[d, phantom, "\rotatebox{90}{$\in$}"] & V^{**} \ar[d, phantom, "\rotatebox{90}{$\in$}"]   \\
    & f \ar[r, mapsto] & f(x)
\end{tikzcd}\end{center}

$C=$[有限次元$K$-線型空間]とし,$V,U\in C$を取る.
\begin{center}\begin{tikzcd}
    
\end{tikzcd}\end{center}

\begin{theorem}[Cauchy列による実数体の構成] \\
    $\mathbb{Q}^\mathbb{N}$を有理数列の空間,
    $$V=\{ x=(x_n)_{n\in\mathbb{N}} \in\mathbb{Q}^\mathbb{N} \,|\, \lim_{n\to\infty} x_nは収束する \}$$
    を収束列のなす部分空間,
    $$W=\{x=(x_n)_{n\in\mathbb{N}}\in\mathbb{Q}^\mathbb{N}\, |\, \lim_{x\to\infty} x_n=0\}$$
    を$0$に収束する列のなす部分空間とする.\\
    このとき,$x\in V$に対し,$\lim_{n\to\infty}x_n\in\mathbb{R}$を対応させる写像$f:V\longrightarrow \mathbb{R}$は,同型$\overline{f}:V/W\longrightarrow\mathbb{R}$をひきおこす.\\
    つまり,$W$を核とする全射線型写像$g:V\longrightarrow V/W\subset V$が存在して(標準全射),以下の図式は可換になる.
    \begin{center}
    \begin{tikzcd}
        V \ar[dr, "f"] \ar[d, "g"'] \\
        V/W \ar[r, "\overline{f}"'] & \mathbb{R}
    \end{tikzcd}
    \end{center}

\end{theorem}

\part{量子論の枠組み}

\part{量子化学:化学結合論から物性化学まで}

\chapter{元素の化学}
\section{宇宙内の元素}
George Gamow (Russia) 04-68

はBig Bang理論の始祖の一人とされているが,この理論を元素の理論と結びつけた.
\begin{hypothesis}[Gamowのinflation理論]\rm{}
    十分な高温下で,原子核反応が十分に起こったために,熱平衡に反応が至ると,宇宙の元素の主成分は原子核の結合エネルギーが最も大きいFe付近になるはずであるが,実際はHやHeである.

    従って,宇宙は超高温の状態からの冷却が,熱平衡に至らないくらいには急激に進んだはずである.
\end{hypothesis}
\begin{theory}[cosmic inflation theory]
    インフレーション理論では、宇宙は誕生直後の10-36秒後から10-34秒後までの間にエネルギーの高い真空(偽の真空)から低い真空(真の真空)に相転移し、この過程で負の圧力を持つ偽の真空のエネルギー密度によって引き起こされた指数関数的な膨張(インフレーション)の時期を経たとする。\footnote{https://ja.wikipedia.org/wiki/宇宙のインフレーション}
\end{theory}
\begin{fact}[1965]
    宇宙全体からの黒体放射(マイクロ波)が発見され,それによると現在の宇宙の温度が約$3\si{\kelvin}$と分かった.
\end{fact}
\begin{hypothesis}
    現在,宇宙に於ける全ての元素の合成過程は,星の中で起こる以下の6つの反応のいずれかとして理解される.\rm{}

    1(Hの燃焼).\, \ce{4^1H -> ^4He + 2e^+ + 2\nu}

    2(Heの燃焼).\, 水素が燃え尽きて重力収縮した$10^8\si{\kelvin}$環境で,\ce{^4He + ^4He -> ^8Be}などとして\ce{^12C, ^16O, ^20Ne}などの核種が生成する.
\end{hypothesis}

\chapter{結合の理論と構造式}

原子同士が結合して分子を作っている時(0でない原子価を持つとき)の電子の状態のことを指して「原子価状態」という.
この状態についての理論,ひいては電子の関与する結合についての理論を模索したい.

ルイスの構造式は量子力学前夜に開発された経験則である.そこから入り,Linus PaulingによるVB法に抜ける.

\section{原子価電子対反発則(Valence Shell Electron Pair Repulsion theory)}
「化学結合とは,電子対の共有である.」という,量子論前夜の第一近似.

\subsection{ルイス構造式(Lewis structure)}
Gilbert Lewis (America) 1875-46が論文"{\it The Atom and the Molecule}(16)"で提唱し,以降定着した経験的な記法.
しかし,(1)オクテット則の例外は原子番号が大きくなるほど多く,(2)実際の描像とのズレも大きいという欠点がある.

\begin{definition}[Lewis structure]
    元素記号の周りに,最外殻電子を黒点$\cdot$で,単結合/二重結合/三重結合を,対応する個数の黒点の対$:$か,価標$-$で表した構造式を,Lewisの構造式という.\rm{}
\end{definition}
\begin{example}[オクテット則の例外]\rm{}

    1.\, \ce{BF_3}において,Bの価電子は6つである.\ce{BeCl_2}において,Beの価電子は4つである.
    
    2.\, \ce{SF_6}において,Sの価電子は12個である.これは3d軌道が化学結合に関与しているためである.
\end{example}

\subsection*{形式電荷}

\begin{definition}[formal charge]
    Lewis構造式において,電子の振る舞いは,共有結合を形成しているか,孤立電子対であるかの2通りしかないと理想化しているが,この状況下において考えた個々の分子の電荷を形式電荷という.

    1. 非共有電子対は全てその原子に属する.

    2. 共有電子対は,2つの原子間で1つずつ保有する.
\end{definition}

\subsection{共鳴(28, Pauling)}

1つのLewis構造式では正確に状況が表せない場合に用いられる補正である.
数種類の構造の間を両方向の矢印で結んで,共鳴混成体として目的の構造を説明・表現する.

\begin{example}
    1. ベンゼン

    2. アミド:実はN原子から電子がOにまで流入する共役系を持つ為,アミド結合やペプチド結合は平面構造を取っている.
    \begin{center}
        \chemfig{R-[1]C(=[:90]O)-[-1]N(-[:-90]H)-[1]H} $\longleftrightarrow$ \chemfig{R-[1]C(-[:90]O^{-})=[-1]N^{+}(-[:-90]H)-[1]H}
    \end{center}

    *従って,Lewis構造式にて,共役系は単結合-と二重結合=との連続した部分に発生する事になる.

    3. 共役ブタジエンへの1,4-付加:これが起こることが,何よりも共鳴構造の存在の証明になる.
\end{example}

\subsection{VSEPR則}
分子の立体的な形を予測するのに,価電子の数の違いを用いてする模型.特にLewis構造式を定めた時に使える理論である.
また,分子の立体的な形ついての全く別の模型として,軌道の混成によっても説明される.

\begin{model}[VSEPR model]
    1. 原子の周りの立体構造は,電子対間の反発が最小になるように決定される.
\end{model}

\begin{fact}
    水の結合角は,正四面体配置の109.5度より小さな,104.5度になる.
    次の仮説を付け加えれば良い.
    
    2. 非共有電子対は,共有電子対よりも大きな空間を占める.
\end{fact}

\section{原子価結合理論(Valence Bond Theory)}
前述のLewis構造式で表される考え方の大前提には,価電子しか結合に関与しないというものがあった.これでは,炭素の結合の殆どを説明できない.

異なる原子の2つの(不対電子が所属する)最外殻軌道が,重ね合わさって,新しい分子軌道が作り出され,そこに電子対が入る,というパラダイムを
量子論の結果(Schrödinger方程式(26))を取り入れて,Paulingが整理した.

s-s, s-p, p-s軌道同士が重なると,軸について回転対称な軌道が出来る.これを$\sigma$結合という.
p-p, d-d軌道同士が重なると,軸について180°回転で反転する軌道が出来る.これを$\pi$結合という.
d-d軌道同士の重なりのうち,90°回転で符号反転する軌道が出来る,これを$\delta$結合という.

\subsection{昇位と混成軌道(30, Pauling)}

出来た軌道は,昇位(promotion)を伴って,すでに電子対が占めていた内部の軌道も参加して,混成軌道(hybrid orbital)という均された1つの軌道を作る.
全体として安定である.

\begin{example} 

    1. $sp^3$混成軌道:\ce{CH^4}の電子配置は$(1s)^2(sp^3)^4$
    \begin{eqnarray*}
        1&:& \frac{1}{\sqrt{2}}(2s)+\frac{1}{2}((2p_x)+(2p_y)+(2p_z)) \\
        2&:& \frac{1}{\sqrt{2}}(2s)+\frac{1}{2}((2p_x)-(2p_y)-(2p_z))\\
        3&:& \frac{1}{\sqrt{2}}(2s)+\frac{1}{2}(-(2p_x)+(2p_y)-(2p_z))\\
        4&:& \frac{1}{\sqrt{2}}(2s)+\frac{1}{2}(-(2p_x)-(2p_y)+(2p_z))
    \end{eqnarray*}

    2. $sp^2$混成軌道:電子配置は$(1s)^2(sp^2)^3(2p)^1$.この形が2つ集まって,2p軌道を重ねて$\pi$結合を作ったのがエチレン\ce{H_2C=CH_2}である.
    新しい軌道の状態ベクトルは\begin{eqnarray*}
        1&:& \frac{1}{\sqrt{3}}(2s)+\frac{\sqrt{2}}{\sqrt{3}}(2p_x) \\
        2&:& \frac{1}{\sqrt{3}}(2s)-\frac{1}{\sqrt{6}}(2p_x)+\frac{1}{\sqrt{2}}(2p_y)\\
        3&:& \frac{1}{\sqrt{3}}(2s)-\frac{1}{\sqrt{6}}(2p_x)-\frac{1}{\sqrt{2}}(2p_y)
    \end{eqnarray*}
    
    3. $sp$混成軌道:電子配置は$(1s)^2(sp)^2(2p)^2$
    \begin{eqnarray*}
        1&:& \frac{1}{\sqrt{2}}(2s)+\frac{1}{\sqrt{2}}(2p_x) \\
        2&:& \frac{1}{\sqrt{2}}(2s)-\frac{1}{\sqrt{2}}(2p_x)
    \end{eqnarray*}
\end{example}
これら残ったp軌道は$\pi$結合を形成するが,これをフロンティア軌道といい,フロンティア軌道論で議論される.

\subsection*{混成軌道のs性}
これが高いほど結合距離は短く,電子吸引性が増す.

\subsection*{問題点}
それぞれの電子対に,オーダーメイドの混成軌道1つずつを分配している.
つまり,軌道の数を電子対の数とみなせば,混成軌道の提供する空間的配置についての示唆はVSEPRのそれと一致する.
これだと,$\pi$結合を持つ二重結合が作る共役系に於ける特別な安定化構造を作ることと相性が悪い理論となってしまう.

自然界の指導原理は最大安定化であり,それの結果として対称的な軌道が出来ているというだけである.
結合相手との結合を最強にして,フロンティア軌道も含めて最安定構造を実現する過程にたまたま混成で説明できる均一化があったというだけである.

ただし,例えば,複雑な分子の形を予測するにあたって,非共有電子対を無視してルイスの構造式を書くと,概ね正しく予測できる.

\section{分子軌道法(Moleculer Orbital theory)}

\part{分子間力論とそれを基調とした宇宙観}

\begin{thebibliography}{99}
    \bibitem{斎藤毅}
        斎藤毅『線形代数の世界』
        大学数学の入門\circled{7}(東京大学出版会,2007)
    \bibitem{清水明}
        清水明『新版 量子論の基礎』
        新物理学ライブラリ=別巻2(サイエンス社,2004)
    \bibitem{ランダウ}
        L.D.ランダウ E.M.リフシッツ著,好村滋洋,井上健男訳『量子力学』
        ランダウ=リフシッツ物理学小教程(筑摩書房,2008)
    \bibitem{ポーリング}
        Linus Pauling, E. Bright Wilson, Jr. \textit{Introduction to Quantum Mechanics with Application to Chemistry}
        (Dover, 1985)
        
        内容は(McGraw-Hill, 1935)のものと同一である.
    \bibitem{化学と量子論}
        長倉三郎,中島威 編『化学と量子論』
        岩波講座現代化学1(岩波書店,1979)
    \bibitem{量子化学}
        真船文隆『量子化学―基礎からのアプローチ―』
        (化学同人,2008)
    \bibitem{化学の基礎}
        東京大学教育学部化学部会『化学の基礎77講』
        (東京大学出版会,2003)
    \bibitem{現代物性化学の基礎}
        小川桂一郎,小島憲道 編『新版 現代物性化学の基礎』
        (講談社,2010)
    \bibitem{平岡秀一}
        平岡秀一『溶液における分子認識と自己集合の原理―分子間相互作用』
        ライブラリ 大学基礎化学(サイエンス社,2017)
    \bibitem{木原太郎}
        木原太郎『原子・分子・遺伝子』
        (東京化学同人,1987)
\end{thebibliography}

\end{document}
