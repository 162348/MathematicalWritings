\documentclass[uplatex, dvipdfmx]{jsarticle}
\title{代数と幾何レポート 第3回}
\author{司馬博文 J4-190549}
\date{\today}
\pagestyle{empty} \setcounter{secnumdepth}{4}
\input{/Users/hirofumi.shiba48/Desktop/数理科学/preamble_CM.tex}
\begin{document}
\maketitle

\section*{問題28}

$1,i\in\C$が定める同型を
\[\xymatrix@R-2pc{
    \varphi:\R^2\ar[r]&\C\\
    \rotatebox[origin=c]{90}{$\in$}&\rotatebox[origin=c]{90}{$\in$}\\
    {\begin{pmatrix}a\\b\end{pmatrix}}\ar@{|->}[r]&a+bi
}\]
とすると,行列表示$A\in M_2(\R)$について,次の図式は可換である.
\[\xymatrix{
    \C\ar[r]^-{\alpha}&\C\\
    \R^2\ar[u]^-{\varphi}\ar[r]^-{\times A}&\R^2\ar[u]_-{\varphi}
}\]
従って,$e_1,e_2\in\R^2$について,
\begin{align*}
    \varphi^{-1}\circ\alpha\circ\varphi(e_1)&=\varphi^{-1}(\alpha(1))=\varphi^{-1}(a+bi)=\begin{pmatrix}a\\b\end{pmatrix}\\
    \varphi^{-1}\circ\alpha\circ\varphi(e_2)&=\varphi^{-1}(\alpha(i))=\varphi^{-1}(-b+ai)=\begin{pmatrix}-b\\a\end{pmatrix}
\end{align*}
であるから,行列表示は
\[ A=(Ae_1\;Ae_2)=\begin{pmatrix}a&-b\\b&a\end{pmatrix}. \]
\begin{flushright}$\blacksquare$\end{flushright}

\section*{問題37}

\begin{align*}
    \left(\int^x_0f(t)dt\right)'&=f(x),&
    \int^x_0f'(t)dt&=f(x),
\end{align*}
より,$F,G$は互いに逆射であるから,いずれの像も終域全体である:
\begin{align*}
    \Im F=F(C^\infty(\R))&=C^\infty(\R),&\Im G=G(C^\infty(\R))&=C^\infty(\R).
\end{align*}
また,核は
\begin{align*}
    \Ker F&=G(0)=\{f\in C^\infty(\R)\mid \forall f(x)= a\in\R\},&\Ker G&=F(0)=0.
\end{align*}
\begin{flushright}$\blacksquare$\end{flushright}

\end{document}