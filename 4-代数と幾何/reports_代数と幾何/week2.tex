\documentclass[uplatex, dvipdfmx]{jsarticle}
\title{代数と幾何レポート 第2回}
\author{司馬博文 J4-190549}
\date{\today}
\pagestyle{empty} \setcounter{secnumdepth}{4}
\input{/Users/hirofumi.shiba48/Desktop/数理科学/preamble_CM.tex}
\begin{document}
\maketitle
\begin{abstract}
    ボロクソに採点してください.よろしくお願いします.
\end{abstract}

\section*{問題8}
\subsection*{1}

$W$の任意の元は$a,b\in K$を用いて
\begin{equation}\label{eq-1}
    \begin{pmatrix}a\\b\\-a-b\end{pmatrix}=a(e_1-e_3)+b(e_2-e_3)
\end{equation}
と表せるから,$e_1-e_3,e_2-e_3$は$W$の生成系である.また(式(\ref{eq-1}))$=0$の時,
$ae_1+be_2-(a+b)e_3=0$であるが,$e_1,e_2,e_3$は$K^3$の基底だから$a=b=0$が従う.よって,
$e_1-e_3,e_2-e_3$は線型独立でもある.
\begin{flushright}$\blacksquare$\end{flushright}

\subsection*{2}

$a,b,c\in K$について,
\begin{align*}
    &a+b+c=0\land a=b=c\\
    \Leftrightarrow& \begin{cases}
        a=b=c,&Kの標数が3の時,\\
        a=b=c=0,&それ以外の時,
    \end{cases}
\end{align*}
であるから,
\[\begin{cases}
    W\cap W'=W',&Kの標数が3の時,\\
    W\cap W'=0,&それ以外の時.
\end{cases}\]
\begin{flushright}$\square$\end{flushright}

$K(e_1+e_2+e_3)=W'$より,$W+W'$の元は,$a,b,c\in K$を用いて,
\begin{align*}
    &a(e_1-e_3)+b(e_2-e_3)+c(e_1+e_2+e_3)\\
    =&(a+c)e_1+(b+c)e_2+(c-a-b)e_3
\end{align*}
と表せる.$e_1-e_3,e_2-e_3,e_1+e_2+e_3$を形式的に基底として3次元$K$-線型空間と見た$W\oplus W'$と,数ベクトル空間$K^3$の部分空間である$W+W'$の間に定まる線型写像
\[\xymatrix@R-2pc{
    f:W\oplus W'\ar[r]&W+W'\\
    \rotatebox[origin=c]{90}{$\in$}&\rotatebox[origin=c]{90}{$\in$}\\
    {\begin{pmatrix}a\\b\\c\end{pmatrix}}\ar@{|->}[r]&{\begin{pmatrix}a+c\\b+c\\c-a-b\end{pmatrix}}
}\]
が可逆であることと,$f^{-1}(0_{K^3})=0$であることは同値である.
(線型写像$h:V\to W$が全単射である時,$h^{-1}(0_W)=\{0_V\}$である.逆に$h^{-1}(0_W)=\{0_V\}$である時,任意に$w\in W$をとって$v_1,v_2\in f^{-1}(w)$とすると,$f(v_1-v_2)=f(v_1)-f(v_2)=w-w=0$より,$v_1=v_2$).
これが示せれば,$W+W'\simeq W\oplus W'$であり,また形式的な基底$(e_1-e_3,0),(e_2-e_3,0),(0,e_1+e_2+e_3)\in W\oplus W'$と$e_1,e_2,e_3\in K$が定める同型により$W\oplus W'\simeq K^3$であることから,($W+W'\subset K^3$より)$W+W'=K^3$とわかる.
\begin{align*}
    \begin{cases}
        a+c=0\\
        b+c=0\\
        c-a-b=0
    \end{cases}
    &\Leftrightarrow\begin{cases}
        a+(a+b)=0\\
        b+(a+b)=0\\
        c=a+b
    \end{cases}\\
    &\Leftrightarrow\begin{cases}
        a=b\\
        a+b+b=0\\
        c=a+b
    \end{cases}\\
    &\Leftrightarrow\begin{cases}
        a=b\\
        a+a+a=0\\
        c=a+a
    \end{cases}
\end{align*}
より,$K$の標数が3の時$f^{-1}(0)=\left\{\begin{pmatrix}0\\0\\0\end{pmatrix},\begin{pmatrix}1\\1\\2\end{pmatrix},\begin{pmatrix}2\\2\\1\end{pmatrix}\right\}$であり,
それ以外の時$f^{-1}(0)=0$.
よって,$K$の標数が3の時は$W'=K(e_1+e_2+e_3)\subset W$より,$W+W'=W$.それ以外の時は,上の$f$が同型を定めるが,$W+W'\subset K^3$だから,$W+W'=K^3$.
\begin{flushright}$\blacksquare$\end{flushright}

\begin{remark*}
    単に$e_1-e_3,e_2-e_3,e_1+e_2+e_3$が$K^3$の基底になるかどうかを$=0$の時に係数が全て$0$になるかどうかによって検証しているだけであり,写像$f$の可逆性
    とまで論じると大袈裟であることに気付きましたが,この際,言葉遣いと議論の運びが正しいかどうかを見ていただきたいです.
\end{remark*}

\section*{問題19}
\subsection*{1}

$f$の値域は$f(K^n)=\langle x_1,\cdots,x_n\rangle=:W\subset V$である.
$x_1,\cdots,x_n$が線型独立であることと,$x_1,\cdots,x_n$が$W$の基底であることは同値,
これは$f$の終域を狭めて得られる写像$g:K^n\to W$が同型である事に同値.
$i:W\to V$を包含写像とするとこれは単射だから,$f=i\circ g$が単射である事に同値.
\begin{flushright}$\blacksquare$\end{flushright}

\subsection*{2}

$x_1,\cdots,x_n$が$V$の生成系である時,
$f(K^n)=\langle x_1,\cdots,x_n\rangle =V$であるから
($f$は$x_1,\cdots,x_n$が定める線型写像だから,$f(K^n)=\{v\in K^n\mid \exists (a_1,\cdots,a_n)^T\in K^n,\; v=a_1x_1+\cdots+a_nx_n\}=\langle x_1,\cdots,x_n\rangle$),
$f$が全射である事に同値.
\begin{flushright}$\blacksquare$\end{flushright}

\end{document}