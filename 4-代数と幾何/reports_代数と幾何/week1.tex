\documentclass[uplatex, 12pt, dvipdfmx]{jsarticle}
\title{代数と幾何レポート}
\author{司馬博文 J4-190549}
\date{\today}
\pagestyle{empty} \setcounter{secnumdepth}{4}
\input{/Users/hirofumi.shiba48/Desktop/数理科学/preamble_CM.tex}
\begin{document}
\maketitle

\section*{問題1}

$2\in\Z$を取ると,$2b=1$を満たす数$b$は整数ではない.

\section*{問題2}

元$a\ne 0\in\F_p$を任意に取る.この$a$により定まる写像
\[\xymatrix@R-2pc{
    f_a:\F_p\ar[r]&\F_p\\
    \rotatebox{90}{$\in$}&\rotatebox{90}{$\in$}\\
    b\ar@{|->}[r]&ab
}\]
を考える.これが全単射であることを示すことにより,各$a\ne 0\in\F_p$
に対して逆像$f_a^{-1}(1)$の元はただ一つであることを導く.

今,$q,r\in\F_p$が$f_a(q)=f_a(r)$即ち$aq=ar$を満たしたとする.
すると,$a(q-r)=0\in\F$であるから,
$a,q,r$を$0<a,q,r<p$を満たす対応する整数と定め直したときに,
$a(q-r)\equiv 0\mod p$が成り立つ.
このとき,$0<a<p,-p<q-r<p$より,$q-r=0$即ち$q=r$が従う.

よって,$f_a$は単射である.今$\F_p$は有限集合だから,
$f_a$は全射でもある.

\begin{flushright}$\blacksquare$\end{flushright}

\section*{問題3}

\subsection*{(1)}

$x+x=x$ならば,両辺に逆元$(-x)$を加えて,
\begin{align*}
    (x+x) + (-x) &= x + (-x)\\
    x+(x+(-x)) &= 0 \\
    x+0&=0\\
    x&=0
\end{align*}
を得る.
\begin{flushright}$\blacksquare$\end{flushright}

\subsection*{(2)}

$0$は加法の中立元だから$0=0+0$より,$0x=(0+0)x$.
分配法則より,$(0+0)x=0x+0x$.この2つを合わせて$0x=0x+0x$を得るから,(1)の結果より$0x=0$.

\begin{flushright}$\blacksquare$\end{flushright}

\subsection*{(3)}

分配律による結果$0x=(1+(-1))x=x+(-1)x$を考える.左辺は(2)より$0x=0$だから,$x+(-1)x=0=x+(-x)$である.
両辺に$x$の逆元$-x$を加えることにより,$(-1)x=-x$を得る.

\begin{flushright}$\blacksquare$\end{flushright}

\section*{問題4}

問題の条件である
\begin{align}\label{q4-eq1}
    \begin{cases}
        x_i=e_i-e_{i+1}, &i=1,2,\cdots,n-1,\\
        x_n=c_1e_1+c_2e_2+\cdots+c_ne_n,
    \end{cases}
\end{align}
の関係から,$e_1,\cdots,e_{n-1}$を消去すると,
\begin{align}
    x_n&=c_1e_1+c_2e_2+\cdots+c_ne_n\\
    &=c_1(e_1-e_2) + (c_1+c_2)(e_2-e_3) + \cdots \\
    &\hphantom{{}={}}\cdots+ (c_1+\cdots+c_{n-1})(e_{n-1}-e_n)+(c_1+\cdots+c_n)e_n\\
    &=\sum^{n-1}_{j=1}\left(\sum^j_{i=1}c_i\right)x_j+Ce_n,\hspace{2cm}(C:=c_1+\cdots+c_nとした)\label{q4-eq2}
\end{align}
を得る.

\subsection*{(a)$\Rightarrow$(b)}

$C=0$と仮定すると,式\ref{q4-eq2}より,
\[ -\sum^{n-1}_{j=1}\left(\sum^j_{i=1}c_i\right)x_j+x_n=0 \]
を得る.$x_1,\cdots,x_n\in K^n$は$K$-線型空間$K^n$の基底としたから,
特に$x_n$の係数を比較して$1=0$が必要だが,これは$K$が体であることに矛盾.
従って,$C\ne 0$である.

\subsection*{(b)$\Rightarrow$(a)}

$x\in V$を任意に取る.$a_1,\cdots,a_n,b_1,\cdots,b_n\in K$として,
\begin{align}
    x&=a_1e_1+a_2e_2+\cdots+a_ne_n\label{q4-eq4}\\
    &=b_1x_1+b_2x_2+\cdots+b_nx_n\label{q4-eq3}
\end{align}
と表せたとする.
関係\ref{q4-eq1}の下で,式\ref{q4-eq3}は,
\begin{align*}
    x&=b_1(e_1-e_2) + b_2(e_2-e_3) + \cdots + b_{n-1}(e_{n-1}-e_n) + b_n(c_1e_1+c_2e_2+\cdots+c_ne_n)\\
    &=(b_1+b_nc_1)e_1 + (-b_1+b_2+b_nc_2)e_2 + \cdots + (-b_{n-2}+b_{n-1}+b_nc_{n-1})e_{n-1}  + (b_nc_n)e_n
\end{align*}
と同値であるが,$e_1,\cdots,e_n$は$K^n$の基底だから,式\ref{q4-eq4}と比較して,
\begin{align*}
    a_1 &= b_1+b_nc_1,\\
    a_2 &= -b_1+b_2+b_nc_2,\\
    \vdots\;\;&=\;\;\;\;\;\vdots\\
    a_{n-1} &= -b_{n-2}+b_{n-1}+b_nc_{n-1},\\
    a_n&=b_nc_n,
\end{align*}
が必要.これを逆に解いて,
\begin{align*}
    \begin{cases}
        b_j=\sum^j_{i=1}a_i-\frac{A}{C}\sum^j_{i=1}c_i,&j=1,2,\cdots,n-1,\\
        b_n=\frac{A}{C},
    \end{cases}
\end{align*}
を得る.
また,$e_1,\cdots,e_n$が$K^n$の基底であることより,
$a_1,\cdots,a_n$は一意的に定まっているから,
各係数$b_1,\cdots,b_n$はただ一通りに表せた事になる.
よって,$x_1,\cdots,x_n\in K^n$は基底である.
\begin{flushright}$\blacksquare$\end{flushright}






\section*{問題5}

$v,w\in V=C^\infty(\R)$とする.即ち,$v,w:\R\to\R$.
この時,関数$a(v+w)$と$av+aw$を考える.
任意の実数$x\in\R$に対して,実数の積を$\cdot$と表すと,
\begin{align*}
    a(v+w)(x) &= a\cdot (v+w)(x)\\
    &= a\cdot  (v(x)+w(x))\\
    &= a\cdot v(x)+a\cdot w(x)&\R 上の和に対する積の分配則\\
    (av+aw)(x) &= av(x)+aw(x)\\
    &= a\cdot v(x)+a\cdot w(x)
\end{align*}
より等しいから,関数$a(v+w),av+aw$は等しい.
よって,
\[ \forall a\in\R,\;v,w\in V,\; a(v+w)=av+aw \]

\begin{flushright}$\blacksquare$\end{flushright}

\end{document}