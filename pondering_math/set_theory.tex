\documentclass[uplatex, 12pt, a4paper, dvipdfmx]{jsarticle}
\title{Lebesgue積分への途}
\author{司馬博文 hirofumi-shiba48@g.ecc.u-tokyo.ac.jp}
\date{\today}
\pagestyle{headings} \setcounter{secnumdepth}{4}
\usepackage{amsmath, amsfonts, amsthm, mathptmx, amssymb, ascmac, color, comment, graphicx}
\usepackage{tikz, tikz-cd}
\usepackage[top=15truemm,bottom=15truemm,left=10truemm,right=10truemm]{geometry}
\newtheorem{theorem}{定理} \newtheorem{definition}{定義} \newtheorem{proposition}{命題} \newtheorem{corollary}[proposition]{系} \newtheorem{lemma}[proposition]{補題}
\renewcommand{\abstractname}{前書き}

\newenvironment{boxedlaw}[1]
{\begin{Sbox}\begin{minipage}{#1‎}‎\setcounter{mpfootnote}{\value{footnote}}}
{\end{minipage}\end{Sbox}\begin{center}\shadowbox{\TheSbox}
\setcounter{footnote}{\value{mpfootnote}}\end{center}}
\renewcommand*\thempfootnote{\arabic{mpfootnote}}

\begin{document}
\maketitle
\begin{abstract}
	数学世界は華厳哲学の説く事事無礙法界の風光を呈しています.数学世界のどんな片隅で説かれる理論も,それだけ独立し孤立しては在り得ない.だから,集合論も圏論も,「全て」やりましょう!将来Lebesgue積分を議論する際には,その他全ての数学世界を気にかけて居られるような,調和の精神\footnote{「これから数学をやりたいと思っておられる方に何よりもまず味わっていただきたいと思うのはアンリ・ポアンカレーの「数学の本体は調和の精神である」という言葉です.(中略).ここにいう調和とは真の中における調和であって,芸術のように美の中における調和ではありません.しかし同じく調和であることによって相通じる面があり,しかも美の中における調和のほうが感じとりやすいので,真の中における調和がどんなものかをうかがい知るにはすぐれた芸術に親しまれるのが最もよい方法だと思います.」(岡潔『数学を志す人に』平凡社 2016)}の体現者となって居りますように.
\end{abstract}
\tableofcontents\clearpage

\part{集合論}

文献\cite{須之内治男}では第0章が,文献\cite{吉田洋一}では第2章が,集合論の話題を扱っている.

\section{Introduction:集合論の精神と勉強の方針}

\subsection{集合論の精神}

数学世界の世界樹は2本,集合論と圏論である.この2つの理論/言語は,それぞれ,数学全体を形式化する能力を持つ.初めに出来たのが集合論で,例えば1939年からNicolas Bourbakiによって,集合論を基礎とした形式化が成されてから,暫く経つ.現在では,教養課程で教えられる微分積分学や線型代数学,そして集合論自身や今回の主眼である積分論も,驚くべき厳密性を持つ.\par

\begin{quote}
	ブルバキの『集合論』をほめる人に出会ったことがありませんが,私はとても気に入っています.この本がなければ,大学で数学を自分の専門として選ばなかったかもしれない,とさえ思っています.この本で,数学に対する信頼感が得られました.\footnote{栗原将人 1961- 専門は整数論 慶應義塾大学理工学部数理科学科教授}
\end{quote}

栗原先生の言うように,数学の厳密化,抽象化の流れを大きく前に推し進めたのは何と言ってもBourbakiの大作『数学原論(Éléments de mathématique)』であろう.

\begin{quote}
	元は微分積分学の現代的な教科書を書くのが彼らの目的だったが,作業が中途で肥大化し,その目的は捨て去られた.最終的には\underline{集合論の上に現代数学を厳密かつ公理的に打ち立てる}ことにその目標は向けられる.彼らはそこで,代数構造・順序構造・位相構造という三つの構造概念,フィルターなどいくつかの新しい概念や術語を導入し,現代数学に大きな影響を与えた.その完璧な厳密性と一般性を求める叙述はブルバキスタイルと呼ばれるようになる.ただしブルバキの狙いは,決して最大限の一般性ではなく,最大限の有効性を備えた一般性,最小限の一般化である.\footnote{https://ja.m.wikipedia.org/wiki/ニコラ・ブルバキ}
\end{quote}

\subsection{公理的方法}

現代では,このBourbakiが押し進めた現代数学の方法は,各分野に十分行き届いて来たと言えるだろう.それが,数学者に強力な後援を与えているパラダイムであると同時に,初学者の敷居を高くしていると言えるだろう.しかし,逆に言えば,この集合論特有の,数学的対象に対する厳密な取り扱い(特に「公理的方法」は現代数学のキーワードである)には,早めに慣れておけば,現代数学全般が非常に見渡しが良くなることは全く不思議ではない.数学で今後分からないことがあって,wikipediaで調べたとしても,以前よりずっと読み易く感じるための受け皿が,心の中に用意されることだろう.これが「公理的方法」の美点である.\par

Bourbakiが登場する1930年代以前までの数学は実用性志向で「アルゴリズム的」に教えられて来た.小学校で習う算術や,中高数学のチャート式のような教育を思い浮かべてほしい.従って,数学は「手が覚える技術」にどちらかとえいば近く,職人のような存在に近かっただろう.しかし公理的方法は,数理論理学という基礎論の発展が可能にした,知識の整理の営みであり,数学的真理の世界の描像である.この,数学の,アルゴリズム的存在から公理体系的存在への変化は,明らかな質的な変化である.今の僕はどう形容していいか分からないが,たしかに中高の頃の僕はこれ(公理的に整理された数学世界)を欲していて,大学に入って初めて目にした時は懐かしい気持ちさえした.従って,この歴史的展開も,自然だと僕には思われるのだ.\par

\begin{quote}
	古典的な理論からその基礎となっている命題を抽象的な形で取り出して,これらを公理とし,その公理系から出発して抽象的な理論を展開しようとする行き方は,現代数学の一般的傾向である.\footnote{\cite{吉田洋一}}
\end{quote}

\subsection{そこに圏論が加わった時}

最後に,この導入の節にて,圏論についても触れておきたい.現代数学の公理的方法は,すでに達成されて歩みを止めたと言う訳ではない.圏論は現在,この流れにさらに不連続的な質的変化を推し進める存在となりつつある.集合論と圏論という2本の世界樹が揃った数学世界は,真に血が通い命を宿し,動的に動き出す.Bourbakiが執筆活動を始めたのは1939年であるが,世界で初めて圏(category)という言葉が登場したのは1945年の論文においてである.

\begin{quote}
	ブルバキの影響は年と共に次第に低下していった.その理由の一つは,ブルバキの影響を受けた本が他にも出版されるようになり,ブルバキの本の独自色が失われたためである.またひとつには,重要と考えられるようになった別の抽象化,例えば圏論などをカバーしていないためでもある.ブルバキのメンバーの一人アイレンベルグは圏論の創始者であり,グロタンディークも圏論を積極的に論じた.だが圏論を導入するには,それまでに発表されてきたブルバキの著作に根本的な修正を与えなければならなかった.そのため圏論についてのブルバキの著作は準備されていたものの,結局は書かれなかった.\footnote{https://ja.m.wikipedia.org/wiki/ニコラ・ブルバキ}
\end{quote}

圏論は,基本的には,現代の集合論的数学の上に立てられる分野の1つであるが,本来的には,集合論と圏論は数学世界の2つの対等な軸であり,完全に対称に扱って良いものだと僕は思っている.それが本当に正しいかは,あと数年の数学者の活躍,または私の探求の成果を見て見ないと分からないが,圏論は現在急激に整備が進んでおり,数学世界はNicolas Bourbakiが数学界に公理的方法をもたらした歴史に匹敵するくらいの大規模な質的変化を迎えようとして居る機運を,東大で過ごし始めてから強く感じるようになった.\par
圏論への機運が,物理学,哲学,情報科学など,数理科学を中心にして広範囲で高まって居る良い例として,理論物理学者の谷村省吾先生の言葉を以下に紹介しよう.(谷村さんは最近Twitter\footnote{谷村省吾 名古屋大学情報学研究科 複雑系科学専攻 多自由度システム情報論講座 教授 @tani6s}上で,哲学者と物理学者との対談と題して哲学的問題(特に心の哲学)に対する物理学者として本気の考察と対話をなさって\footnote{http://www.phys.cs.is.nagoya-u.ac.jp/~tanimura/time/note.html},積極的に分野を横断した建設的な議論をしようという意志を感じます.尊敬してます.学問文化は人類の共有財産ですから,この動きが進めばいいなと思っています.)

\begin{quote}
圏論は,おそらく集合論よりも,物理学者のものの見方・考え方にフィットするのではないか,物理学者が暗黙のうちに使っている物理観・方法論みたいなものを圏論は明示的に言語化する能力を持っているのではないかと私は考えています.もっと言うと,何も物理学に話を限定する必要はなく,科学全般に共通するものの見方を,圏論は提供出来るのではないかと思います.\footnote{http://www.phys.cs.is.nagoya-u.ac.jp/~tanimura/lectures/tanimura-category.pdf}
\end{quote}

そして,この講演録の最後に付された謝辞にも興味深い言葉がある.

\begin{quote}
	謝辞:古結明男氏は,学部生時代からの同級生で,圏論だけでなく,私に広範な数学を教えてくれた方です.圏論に関する私の基本的な知識は古結氏から伝授されたものです.「21世紀の数学は圏論の矢印図式で書かれるようになると思う」というのが当時大学院生だった古結氏の予言です.
\end{quote}

今回は,Rebesgue積分論という題材を通じて,集合論と圏論への基礎的な理解が揃った時に得られる,数学の世界への動的で不羈自由な視点の獲得に挑戦したい.まずは集合論的な取り扱いという,現代数学が到達した最強の武器に慣れてから,米田先生のゼミの内容をよく復習してRieszの方法から積分論へと入堂し,オーソドックスなRebesgue積分論で返ってくることで,立体的な理解を得たいと思う.\par


牛腸先生がSセメでの授業で我々に託した,「現代数学において,定義や公理は実はそこまで大事じゃない.目くらましに合わないように.本当に大事なのは(特徴付ける)性質の方だ.」との言葉\footnote{谷村さんからの引用にある「暗黙のうちに使っている方法論みたいなものを,圏論は明示的に言語化する能力を持っているのではないか」との言葉の通り,牛腸さんのこの言葉に対する圏論の概念装置も存在します.それは「普遍性」だと思っています.}が,その時には完全に僕らのものとなっていますように.

\section{集合とは何か?(問題意識)}

集合論についての基礎事項は,殆どの入門的数学書の最初の話題である.数学を学ぶ大学初年度の学生はもはや見飽きたことだろう.その代表的な説明\footnote{今回の参考文献\cite{吉田洋一}\cite{須之内治男}も例に漏れず,「素朴な集合の定義」以上の言及はなかった.勿論,それで十分用に足りるからであるが,数学を志す者としてここは万全を期して学んでいこう.}は,例えば以下のようなものである.

\begin{itembox}[l]{素朴な集合の定義}
	集合(set)とは「ものの集まり(collection of things)」である.しかし,ものが集まっていれば何でも良いというわけではない.勝手な要素を取って来たときに,それが集合に入るか入らないか,明確にどちらかに定まる場合のみ,その「ものの集まり」を集合と言う.
\end{itembox}

この定義は抽象すれば,「集合とその元の間で,所属関係$\in$という二項関係が,排中的に定義出来ることが集合の条件だ」と言っている.この説明は本質を突いて居るからこそ,このような説明が広く採用されるのであるが,数学を志すのならこれでは足りない.一度公理的集合論の考察を経験してから,この文章に戻って来るのでは,この素朴な定義から得られる「これで良いんだ」という安心感が違う.

\begin{itembox}[l]{Russelのパラドックス}
	次の集合Rを考える.$$R=\{ x:x\notin x \}$$
	条件式$x\notin x$(これを$\varphi (x)$とする.日本語で表現すれば,「自分自身を要素として持たない集合」は真偽が明確に判定可能なので,$R$は「素朴な集合の定義」を満たす.ここで,$R$の$R$自身に対する所属関係$\in$を考えてみたい.$R$は集合だから,$R\in R$または$R\notin R$のいずれか一方のみが必ず成り立つ.前者が成り立つと仮定すると,$R$は$R$の元なので,$x=R$と代入すれば$\varphi (x)$は真になるはずであるが,主張$R\notin R$は仮定と矛盾.次に後者が成立するとすれば,$R\notin R$を満たすが,この条件は$\varphi (x)$に$x=R$を代入したものに他ならないから,$R$は$R$の元であるはずであるが,これは仮定に矛盾.従って,もっと根本の部分から矛盾しているようだ.\\
「このパラドックスをどう解釈すればいいか」「待てよ,そもそも集合とは何だ?」という議論が,公理的集合論を生んだ.
\end{itembox}


その他沢山のパラドックス

実は,先ほどの「素朴な集合の定義」を満たす「ものの集まり」は「集合」ではなく,「クラス(または類)」と呼ばれる.「集合」の要件を満たすものは「クラス」の一部分のみである.従って,クラスではあるが集合ではない「ものの集まり」が存在し,これを「真のクラス(proper class)」と言う.

RusselのParadoxで構成した集合$R$は真のクラスである.(つまり,ZFC公理系を満たさない.)ZFC公理系の内部から議論する際には「そのような集合$R$は作れない(想定できない)ため,そもそも二項関係$\in$を$R$について定義することが出来ない.」と言うことになる.

例えば圏の定義ではこんなこと全く気にしなくていい.集合だけがあまりに複雑であり,10の公理を必要とするし,「このような集合を考える」と宣言した時に「そんなこと考えられません」と言われる可能性を秘めているのは集合だけのように思われる.何故か?この問題意識を持ち続けるのが重要だと思う.集合論の第一線の研究者とも,共通する問題意識だと思われる.少なくとも,サマースクールに行っただけでは解決しなかったどころか,どうして集合について考える度に謎は深まるばかりで,議論は味を増していくばかりなのかそこはかとなく不思議であった.

ともかく,現在では,「集合とは何か?」に対して「ZFC公理系を充たすobjectである」と厳密に定義される.そしてZFC公理系は,一階述語論理という人工言語により表された公理系である.以下に日本語による解説も附しているが,それは人間が理解するための補助的な情報であり,従って数学的には省略可能だと言える.あってもなくても定義には関係しない.定義に直結する部分は全て一階述語論理により形式化されている.\par
現在の数学は全て集合の言葉によって定義されているから,簡単にまとめて仕舞えば,全ての数学はZFC公理系の上に成り立ち,ZFC公理系の10個の公理のいずれかを組み合わせて証明されることとなる.だから,「グラフ」「写像」「圏」などの概念も全て,究極には集合である.これらについて成り立つ定理は,どの本を見ても日本語で(或いは少なくとも何らかの自然言語で)書かれているであろうが,その表現は「何かしらの一階述語論理による論理式の略記だ.」と言えるのである.

\section{一階述語論理の補足}

前節の議論を定義という形で公理的な方法でまとめたい.その前に,一階述語論理の用語,特に形式言語についての用語の定義をここで確認しておく.その多くは日常用語と被っているから,術語(technical terms)だと気付かずにやり過ごすと,理解を著しく阻害する危険がある.\par
今でこそ数理論理学は数学の一分野と見なされるようになっているが,本来形式言語の用語は論理学者が温めて居たものであるから,普段の数学で触れる概念とこれまた異質な印象を受けるだろう.

\section{公理的集合論(Axiomatic Set Theory)}

\subsection{ZFC公理系}
公理的集合論において,集合を定める公理はどうすれば良いのかの議論があり,提出された案も多岐に渡るが,現在の数学で圧倒的主流となっている公理系の定め方が,ZFC公理系である.Ernst Zermelo (1871-1953, 独)が1908年に原型を提出し,Adolf Fraenkel (1891-1965)が1922年に分出公理(separation)を置換公理(replacement)で置き換えるという重要な修正を行ったために,その名が取られている.Cは選択公理(Axiom of Choice)を指す.

\begin{shadebox}
\begin{definition}[ZFC公理系(メタ言語による定義)]
	$L_\epsilon$を,\underline{非論理記号}として二項関係記号$\epsilon$のみを持った一階述語論理の\underline{言語}とする.以下の\underline{$L_\epsilon$-論理式}1から9までの\underline{公理系}をZFと呼び,1から10までの\underline{公理系}をZFCと呼ぶ.(今回はわかりやすさのために,丸括弧$()$も四角括弧$[]$もコロンもふんだんに使い,定数記号$\varnothing$や関数記号$\bigcup , \mathcal{P}, f$も適宜導入して言語を拡張しながら述べていく.矢印のスタイルは私の好みの問題である.)\\
	\rm{1, Extentionality}   $\forall x,y [\forall z(z\in x \Longleftrightarrow z\in y) \Longrightarrow x=y]$\\
	\hspace{5mm}*以降,$\forall x,y [\forall z(z\in x \Longrightarrow z\in y) \Longrightarrow x=y]$のみが成り立つ場合を,$x\subset y$と略記する\footnote{文献\cite{吉田洋一}では,この意味の二項関係に$\subseteq$を用いていて,記号$\subset$は$\subseteq \wedge \neq$の時の強調のための記号である,としているが,\cite{須之内治男}\cite{赤摂也}では採用されていない.}ことにする(逆も然り).すると,Extentionalityは$(x\subset y \wedge y\subset x)\Longleftrightarrow x=y$と書き直せる.\\
	2, \rm{Empty}   $\exists ! x \forall y (y\notin x)$\\
	\hspace{5mm}*以降,この一つしかない$x$を$\varnothing$と書くこととする.\\
	3, \rm{Pairing}   $\forall x,y \exists z [x\in z \wedge y\in z ]$\\
	\hspace{5mm}*ここで,組(tuple)と呼ばれる対象を$<x,y>:=\{ \{ x\} ,\{ x,y\} \}$と定義する.\\
	4, \rm{Union}  \; $\forall x \exists u [u=\bigcup x := \{ z:\exists y \in x (z\in y)\} ]$ \footnote{\cite{Kunen}などでは,$\forall \mathcal{F} \exists A \forall Y \forall x [x\in Y \wedge Y\in \mathcal{F}\longrightarrow x\in A]$が採用されているが,いずれも同じ意味である.こちらの場合は,任意の集合$\mathcal{F}$について,集合$A=\bigcup \mathcal{F}$が構成可能であることを主張する.} \\
	5, \rm{Infinity}   $\exists z [\varnothing \in z \wedge \forall x \in z (x\bigcup \{ x \} \in z)]$\\
	6, \rm{Power}   $\forall x \exists z [z=:\mathcal{P} (x) = \{ y:y \subset x \} ]$\\
	7, \rm{Replacement(schema\footnote{公理図式とは,$\varphi$を言語$L_\epsilon$内の任意の論理式として成り立つ公理のことである.})}   $\forall x,y,z [\underline{\varphi(x,y) \wedge \varphi (z,y)} \Longrightarrow y=z] \Longrightarrow \forall a \exists b [b=\{ y:\exists x\in a \varphi (x,y)\} ]$ \footnote{\cite{Kunen}では,$x,y,A,w_1,w_2,\dots,w_n$を$\varphi$の自由変項とすれば,$\forall A \forall w_1 \forall w_2 \dots \forall w_n [\forall x (x\in A \longrightarrow \exists ! y \varphi) \longrightarrow \exists B \forall x (x\in A \longrightarrow \exists y (y\in B \wedge \varphi))]$と表現されているが,等価である.} \\
	\hspace{5mm}*以降,下線部を満たす二項関係$R$を写像と呼び,$f$などと書く.$b$を$f$による集合$a$の像と呼び,$b=f(a):=\{ f(x) : x\in a \}$などと書く.\\
	8, \rm{Regularity\, /\, Foundation}   $\forall a [a\neq \varnothing \Longrightarrow \exists x\in a (x\cap a \neq \varnothing )] $\\
	9, \rm{Choice}   $\forall a \exists f [fは関数 \wedge \forall x\in a (x\neq \varnothing \Longrightarrow f(x)\in x)]$\\
	\hspace{5mm}*但し,自然言語「$f$は関数」とは,$fは関係 \wedge \forall x,y,z[(<x,y>\in f \wedge <x,z>\in f ) \Longrightarrow y=z$の略記であり,自然言語「fは関\\ \hspace{5mm}係」とは,$\forall u \in f \exists x,y \in \bigcup \bigcup f[u=<x,y>]$の略記である.\\
	10, \rm{Specification(schema)} \; \rm{For each formula $\varphi$ with free variables among} $x,z,w_1,w_2,\dots,w_n$, $\forall z \forall w_1 \forall w_2 \dots \forall w_n \exists y \forall x [x\in y \Longleftrightarrow x\in z \wedge \varphi]$ \\
	\hspace{5mm}*以降,これを$y=\{ x\in z: \varphi (x,w_1,w_2,\dots ,w_n)\}$と書く.
\end{definition}	
\end{shadebox}

下線部分は全て前節の意味で用いている.数学は自然言語を用いて議論するが,それは,本来的には一階述語論理という形式言語を用いて全ての定理は$L_\epsilon$-論理式を用いて表現されるべきであるが,そんなものはもはや人間には扱えないので,\underline{自然言語を用いて,$L_\epsilon$-論理式を略記する}.つまり,「曖昧な定義のまま用いる単語は存在せず,数学の全てのstatementの裏には何らかの論理式が隠れている」という意識は,特に集合論を議論する際に殆ど全ての数学者が無意識に持っている共通了解だと言って良いだろう.\footnote{なお,現実には全ての数学概念に裏付けがある訳では勿論ない.この形式体系による裏付けを進めていく分野が数学基礎論であり,これを完成させるのが数学基礎論の夢である.そんな世界はつまらないと思うだろうか?どうであろう,私にも分からないが,急激に基礎を彫り込んだその先には,究極の応用範囲の広さが待っていると信じて疑えないのだ.}

空集合公理(2, empty)は独立ではなく,置換公理(7, Replacement)から,導くことが出来る.論理式$\varphi (x,y)$を変項の取り方に依らず,常に偽となるようなもの(例えば$x\neq x \wedge y\neq y$など)とすれば良い.すると,下線部が偽であるから,置換公理(7, Replacement)の前提条件を自明な形で満たすから,ある勝手に取った集合$a$に対して,$b={y\in a : aの元xに対して x\neq x かつ y\neq yである}$となる.これを満たす元は存在しないから空集合である.なお,空集合の一意性(空集合公理(2, Empty)の存在記号$\exists$の隣の!の部分)は,外延性公理(1, Extentionality)から従う.\par
また,分出公理(10, specification)も

\textbf{以下,特に注釈がない限りZFC公理系の下で議論しているものとし,記述には専ら日本語を用いる.}

\subsection{集合の定義}

\begin{shadebox}\begin{definition}[集合(メタ理論)]
	ZFC公理系満たす数学的対象$x,y,z\dots$のことを集合と呼ぶ.
\end{definition}\end{shadebox}
この定義がwell-definedであるためには,そもそもZFC公理系の議論領域は空でないことを確認しなきゃいけないが,これは2,Emptyや6,Powerにより保証される.従って,少なくとも1つは集合が存在する.\footnote{なお,集合の存在性を,$\exists x(x=x)$などと表現して,公理系のうちに含める流儀もある.\cite{Kunen}など.}

このような集合の定義は,Hilbertの幾何学の公理系\footnote{Hilbertが1899に発表した書籍『幾何学基礎論』で展開された,Euclid幾何学を公理的方法でまとめ直した幾何学理論体系.これが,数学に於ける公理的方法の先駆けだとされる.公理的方法の美点として,人々の思考様式が変わった.公理を取り替えることで,全く別の世界線が想定し得るのである.そのような幾何学を非Euclid幾何学という.}に於ける「点」と「直線」と「平面」の定義に似ている.どちらも「無定義用語(primitive notion)」と呼ばれる.「点」も「直線」も「平面」も「集合」も,一切定義されず,ただ他の対象との関係の中で定義される.\footnote{圏論の萌芽?} つまり,議論のスタート地点には「公理」しかない.「じゃあ集合って何なの?」という点については,人間ならば誰もが初源的なイメージを持っているので,それに任せて直接は定義せず,「こういう性質を満たすもの全部だ」と宣言する訳である.そして,実際に人間がどういう解釈を想定するかを,「構造」または「モデル」という\footnote{数理論理学に於ける「モデル」とは,形式言語によって表記された公理系に対して,意味を与える「構造」のことである.日本語を学んだことのない外国人が,日本語で書かれた文章を読めないのは,ひらがなや漢字が文字としてしか認識出来ず,自分の中に持つ観念と一切結びつかないからである.モデル理論はまさにそれと同じである.形式言語は一切の意味を持たない,記号列の生成規則であり,何らかのモデルを立てて初めて議論が始まる.言語学で言えば,数学のモデル理論は,文字列に解釈を与える意味論に当たる.言語学の統語論に当たるのは数学の証明論である.英語の文Colorless green ideas sleep furiously.は統語論的には完璧な文であるが,全く意味論的には崩壊しており,これを満たすモデルは普通の人は持ち合わせないだろう.従って,この文章は統語論的には生成可能ではあるが,主張としては人類にとって何の価値も持たない.こうして,数学基礎論では,公理系が無矛盾であることを,「それを満たすモデルが存在すること」と考える.(一般化されたGödelの完全性定理)}.しかし,公理的方法において肝心な点は,そのことが厳密性を崩さないという点である.もし,「点」「直線」「平面」を,「みかん」「鉛筆」「テーブル」だと思ってモデルを作っても良い.その後の公理で定義される「平行」「合同」などの関係を満たすように,自分でルールを決められるならば.そしてその公理を満たすモデルについて,普段は成り立つだろうと思われていた定理が成り立たないことを示せたら,それは大発見となる.集合についてもそうである.ある日宇宙人が地球にやって来て数学談話を我々としたとしよう.彼らは集合について,我々と全く違う認知の仕方をしているかもしれないが,所属関係$\in$をどう捉えていようと,公理について納得いただけるようであれば,全く同等に集合についての定理や,その後の数学について対話が可能であろう.

\begin{shadebox}\begin{definition}[クラス(ZF)]
	$\varphi (u,u1,u2,\dots)$を$L_\epsilon$-論理式とし,$z1,\dots,zn$を集合とする.$$\{w:\varphi (w,z1,\dots,zn)\}$$と表される対象を\textbf{クラス}と呼ぶ.集合$x$が存在して,$$x=\{w:\varphi (w,z1,\dots,zn)\}$$とは表せないとき,これを\textbf{真のクラス}と呼ぶ.
\end{definition}\end{shadebox}

論理式を用いて構成出来る「ものの集まり」は,あまりにバラエティに富み過ぎていて,とんちを聞かせれば病的なものを沢山想定出来ることが発覚した.従って,その「ものの集まり」の中でも,はっきりと範囲が定まり,論理式によって表現出来るものをクラスと呼び,次に数学的に比較的簡単に取り扱い可能な素直なものを集合と呼べるようにZFC公理系として整備し,クラスについては「集合の集まり」として間接的に言及する.\par
例えばRusselの逆理で構成した$R$は真のクラスであり,ZFC公理系を満たさない.

\subsection{Russelの逆理の解決}

\subsection{集合の内包的記法の書き方}
分出公理(Axiom schema of specification)は,英語でAxiom schema of restrected comprehensionとも言う.内包(comprehension)による記法として習った記法,例えば$\varphi$を$x$を変項とする論理式として$\{ x | \varphi (x) \}$などと言うようにして集合を表そうとする記法は,強力すぎて,とても直感に反する例まで作り出すことが可能であることが判明したわけだから,この記法に制限をつけることで(この記法で書ける「ものの集まり」をクラスと呼び,「集合」として良いものはその一部のみとすることで)解決しようというのである.それが,公理的集合論の取った途であり,現行の数学はその上に咲いている.\par
例えば,筆者が大学に入って初めて受けた数学の演習の授業で,こんな問題が出た.

\begin{itembox}[l]{問題}
	正の平方数全体を表す集合を,集合の記法を使って何通りかで表せ.
\end{itembox}
読者はどう答えるだろうか?当時の僕は調子に乗って6つほど回答して提出したが,$S=\{ x | \sqrt{x} \in \mathbb{N} \}$という回答だけ不正解となった.
\begin{itembox}[l]{集合の内包的記法についての注意}
	$X$を所与の集合とする.集合を表す記法には2通りある.\\
	置換公理(7, replacement)を用いて,関数$f$を用いて$\{ f(x) | x\in X \}$と,「$x$が集合$X$の元を走るとき,$f(x)$の値として取り得る値全体の成す集合」として表すもの.または,分出公理(10, specification)を用いて$\{ x\in X | (xについての条件式) \}$と,「$X$の元$x$であって,"$x$の条件式"を満たすような元のなす$X$の部分集合」として表すもの.
\end{itembox}
基本的にはこの2つのどちらかであり,この形をしているなら,例えば$\{ \sum^n_{k=1} (2k-1) | n\in\mathbb{N} \}$と書いても,定義上集合であることは保証される(自然数全体の集合$\mathbb{N}$を集合だと認めるなら).

では,公理的集合論は,「回避」という方法でしかRusselの逆理に立ち向かえなかったのであろうか?

\section{主要概念の集合論からの世界観}

\subsection{直積}
この概念は極めて重要であり,絶対に集合論の一番最初の段階で定義されるが,写像の概念がこの直積概念の上に定義されるというのはよく考えると不思議な構造である.\footnote{こうして圏論を待ち望む心が,問題意識という形で醸成されていく}


\begin{thebibliography}{99}
	\bibitem{吉田洋一} 『ルベグ積分入門』 吉田洋一 (ちくま学芸文庫 2015)
	\bibitem{須之内治男} 『ルベーグ積分入門(応用解析の基礎5)』第4版 須之内治男 (内田老鶴圃 1987)
	\bibitem{伊藤清三} 『ルベーグ積分入門(数学選書(4))』 伊藤清三 (裳華房 1963)
	\bibitem{赤摂也} 『現代数学概論』 赤摂也 (筑摩書房 2019)
	\bibitem{Kunen} Kenneth Kunen. Set Theory (Studies in Logic: Mathematical Logic and Foundations). College Publications. England and Wales. (2011).
\end{thebibliography}
\end{document}