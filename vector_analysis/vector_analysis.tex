\documentclass[uplatex, 12pt, a4paper]{jsarticle}
\title{電磁気学で使う数学 清野和彦\\講義ノート}
\author{司馬博文 J4-190549\\hirofumi-shiba48@g.ecc.u-tokyo.ac.jp}
\date{\today}
\pagestyle{headings}
\usepackage{amsmath}
\usepackage{amsfonts}
\usepackage{amsthm}
\newtheorem{theorem}{定理}
\newtheorem{definition}{定義}
\newtheorem{proposition}{命題}
\newtheorem{corollary}[proposition]{系}
\newtheorem{lemma}[proposition]{補題}
\usepackage[top=15truemm,bottom=15truemm,left=10truemm,right=10truemm]{geometry}
\usepackage{mathptmx}
\usepackage{amssymb}
\usepackage{ascmac}
\usepackage{color}
\usepackage{comment}
\setcounter{secnumdepth}{4}
\begin{document}
\maketitle

\part{「場」を記述するということ}

\section{Introduction}

ベクトル解析という分野は,元々電磁気学の基本法則であるMaxwell方程式を,最も効果的に叙述する言語を考える試みの中で考案された.ベクトル解析の基礎は,Willard Gibbs(1839-1903 米)とOliver Heaviside(1850-1925)によって定式化された.Edwin Wilson(1879-1964 米)が,GibbsがYale大学で行った講義を元に出版した"Vector Analysis"(1901)が,史上初のベクトル解析に就いての教科書であり,これによって現在の殆どの基本術語と記法が確立され,物理学において常識的な概念となっていった.

ベクトル解析は非常に言語性の高い記法体系を持つ.線積分の記号は,特に数理的な基礎付けについて考えなくとも,直感的に書き下すことが出来る.これがベクトル解析が特有に持つ性質だと言えるだろう.\par
その証拠に,電磁気学をはじめ,現在では多くの物理現象はベクトル場やテンソル場として記述されるため,ベクトル解析は物理学の基本言語として欠かせない.古典力学を高校で学ぶときでさえ,現代ではベクトルを基本言語として用いる.しかし,勿論であるが,ベクトルという概念が誕生したのは,古典力学の登場(17世紀)よりもずっと後の話であり,当時は直交座標系を用いた解析幾何学的表現や,Hamiltonの四元数を用いた記法が一般的であったが,いずれも現在はその立ち位置をベクトルに取って代わられて居る.ベクトルの概念が線形代数学の完成によって数学的に厳密になったのも,同時期の話である.(ベクトルを用いて量子力学を定式化したWerner Heisenberg(1901-1976 独)も,線形代数を習って居なかった.)そして,数学としても,一般にn次元多様体上で,微分形式という概念によって同様の理論を展開できる.この理論の,3次元Euclid空間に話を限ったバージョンがベクトル解析だ,と言うことになる.\par
数学と物理学はこのようにして,相互に刺激し合いながら共に進んでいく学問である.そこで,線形代数学の整備と教育システムへの普及が終わったのがつい最近であるように,次の課題は微分形式である.幸い現代社会にはTechnologyも行き渡るようになって来ていて,数学を伝える手段は活字による数学書か,黒板による授業の2つに1つだけに頼らなくても済むようになった.
\begin{quote}
    数学者の立場から見ると,本当は微分形式は「電磁気学という学問の数学」に入っているべきものだと思うのですが,物理学の「常識」としては,残念ながら微分形式は電磁気学よりも進んだ物理学を学ぶまでは触れないのが普通のようです.
\end{quote}
\noindent
次の課題は明らかである.

\section{ベクトル場とスカラー場の定義}

そもそもMaxwell方程式とは何についての方程式であるか?\\
Maxwell方程式とは,ベクトル場$\mathbf{E,B}$を未知の対象とする微分方程式である.\\
ではベクトル場とは何か?\\
時空内の元$(x,y,z,t)$を引数により,空間ベクトル$(u,v,w)$を返すベクトル値関数である.\\
だが,Maxwell方程式は決して「4変数関数3つについての微分方程式」ではない.「関数の組」と「ベクトル場」とは別物である\textcolor{blue}{(??)}\\
こうして,我々は「場の微分(2種類)」「場の積分(2種類)」を,「関数の微積分」から拡張させる形で,別に考えていかなければならない.その過程で「関数の組」と「ベクトル場」とは別物であることが了解されるであろう.

そして引き続き,Maxwell方程式についての数学的な考察を深めていくことによって,「scaler potentialとvector potentialについての微分方程式」へと書き換えられ,「量子電磁力学QED」へと話を進めることが出来る.

そう,Maxwell方程式とは微分系が正式である.電磁気学は局所的な理論であり,使うときに積分して使う.しかし,直感的な理解のし易さも,歴史的な電磁気学の発展も,積分系から入ったため,此処でも積分系から話を進める.次に,ベクトル場の微分も独立に定義して,「ベクトル場の上での微積分学の基本定理」を導く.

\subsection{ベクトルとスカラーの定義}

以降,単にベクトルと言った場合には,3次元Euclid空間$\mathbb{R}^3$を存在の場とする幾何ベクトルであるとする.有向線分は,以下の見地に立ったときには幾何ベクトルとも言う.$\mathbb{R}^3$内の全ての有向線分からなる集合について,「始点同士と終点同士を結んだ際に平行四辺形を成す」という条件によって同値関係を導入した際の同値類のことを指す.つまり,平行移動して重なる有向線分は,互いに同じものとみなすと約束する.するとこの商集合は和(矢印の足し合わせ)と実数倍について線型空間$V$をなすためにベクトルの名を持つ.\\
ベクトルは,空間$\mathbb{R}^3$の基底を定める毎に成分表示による数ベクトル表現を持つ.そして,基底変換に対してその表現は必ず規則的に変化する,基底の変換行列が表す線型写像によって移される,という意味で.

一方でスカラーとは,空間$\mathbb{R}^3$の基底の取り方に依らず,値が変化しない数を言う.

以上の議論を定義にまとめる.\\
\begin{shadebox}
\begin{definition}[vector,scaler] \label{vector,scaler}
    (ベクトルもスカラーも,電磁気学では,空間$U$のみを存在の場と考えれば十分である.なお,三次元空間について,線型空間という見方もなく,定まった座標系もない,単に空間内の点$P,Q,R,\ldots$の集合という意味を強調するため,以降$\mathbb{R}^3$ではなく$U$と書く.$U$に1つ座標系を導入する度に同一視出来る.)ベクトルとは,座標系と同じ変換を受ける,$U$内の幾何ベクトル(またはそれで表される物理量)のことである.スカラーとは,座標系の取り方に依らずに値の変わらない数(またはそれによって表される物理量)のことである.
\end{definition}\end{shadebox}
「向きを持つか,大きさしか持たないか」の議論は,この定義を噛み砕いた結果である.そもそも自然言語における「向き」という言葉は,情報の中でも特に,見方によって変わる情報のことを指すからである.\\

\noindent
*3次元Euclid空間$\mathbb{R}^3$を,我々の存在する物理空間の数理モデルとして使用することの意味について(\textcolor{blue}{未編集})\\
*ベクトルを「幾何ベクトルだ」と呼んでしまっているように,矢印は線型空間を成すことを前提としている.そしてそれは数ベクトルの成す線型空間と同型だから,「座標系と同じ変換」と言った時はその場その場で
成分表示を設定して,行列によって基底変換を施す姿を想像してもいいが,ともかく物理学で言うベクトルとは,そう言った座標変換に規則正しく「一緒に変化」するような矢印のことに他ならない.\textcolor{blue}{言葉を浪費した説明}\\
\textcolor{blue}{*清野先生は,この意味でのベクトルを$\vec{v}$というように書き,数ベクトルを$\mathbf{v}$と書き,表示する際には線形代数学の手法に沿って縦ベクトルで書く.どういうことを意味するのだろう.「空間ベクトル」「幾何ベクトル」「数ベクトル」「縦ベクトル」とはなんだろう?}\\
\textcolor{green}{今のところの回答}縦ベクトル:縦に数を並べて表示すると約束したn-組は,線型空間を成す.横ベクトルも同様であるが,行列積の主流な定義を採用した際に噛み合わせが良いのは,縦ベクトルの方である.以上2つの,「数の組」という実体からなる線型空間の元をベクトルと呼ぶ.幾何ベクトルとは矢印のことに他ならなくて,空間の次元を意識したときに「空間ベクトル」「平面ベクトル」「直線ベクトル」などのように呼び分ける,という理解?

\subsection{場の定義}

\begin{shadebox} \begin{definition}[fields]
    各時間毎の空間$U$の各点について,スカラーを1つ対応させる写像を「スカラー場」といい,ベクトルを1つ対応させる写像を「ベクトル場」という.
\end{definition} \end{shadebox} 


このようにして,時空全体を定義域に取った写像のことを場と呼んで,これ全体を考察対象にする.従って,それを分解して「4変数関数n個」とは言いたくない.\textcolor{blue}{もっとほりこめる}

スカラー場とは実数値関数に他ならないが,「3変数」となる前である.$U$という,空間内の点の集合という意味づけしかしていない素朴に物理的な対象を考えたいのだ.例えば座標系を導入して,$\mathbb{R}^3$との一対一対応をしたときに,スカラー場は確かに3変数実数値関数$f:\mathbb{R}^3 \longrightarrow \mathbb{R}$と同一視できる(点$P$が$(x,y,z)$として表される$\Longrightarrow f(x,y,z)=\varphi(P) \in \mathbb{R}$).が,あくまで概念としては別物として区別して定義する.以降,スカラー場を$\varphi : U' \longrightarrow \mathbb{R}$と書く.

同様にして,$V$を幾何ベクトルのなす$\mathbb{R}$-線型空間とすると,ベクトル場とは,写像$\vec{F}:U' \Longrightarrow V$のことである.これはまさに素朴に,空間に矢印が生えている姿を想像していただければそれをそのままモデリングしたものに他ならない.ここに,$U$に適当な座標を導入し,$U$上の点と$V$の元に成分表示を与えると,3変数ベクトル値関数$\mathbf{F}(x,y,z) := \left( \begin{array}{c} F_1(x,y,z) \\ F_2(x,y,z) \\ F_3(x,y,z) \end{array} \right)$と見做せる.こちらを「ベクトル場の成分表示」と言う.

こうして,空間に対して座標系に依らないアプローチをかけていく準備が出来た.現代の物理教育で座標の導入に慣れ過ぎたために最初は違和感があるかもしれないが,座標系という人間の恣意的な切り解き方を乗り越えて,より高い理論を見ようとする試みが,ベクトルという概念の最終到達点である.

\noindent
*他に,テンソルを1つ対応させるテンソル場,スピノル(spinor)を対応させるスピノル場などを考えることもあるが,電磁気学では当分ここまでで十分.\\
*以降簡単のために時間$t$は省略して書くが,「時間を止めて,ある瞬間について考えている」とも,「時間によって変化しない系を考えている」とも解釈してもよい.空間についての考察が全て終わった後に,時間を引数とした一変数関数として考え,その微分を考察すれば,Maxwell方程式については十分である.時間とはそういう存在である.「静電場」という物理用語は,このものの見方を意味する.\textcolor{blue}{???}

\subsection{場の上の積分領域:位相空間論の言葉}

\section{場の積分:「関数の積分」を,場にどう拡張するか?}


\end{document}