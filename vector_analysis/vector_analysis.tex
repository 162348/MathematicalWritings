\documentclass[uplatex, 12pt, a4paper]{jsarticle}
\title{電磁気学で使う数学 清野和彦\\講義ノート}
\author{司馬博文 J4-190549\\hirofumi-shiba48@g.ecc.u-tokyo.ac.jp}
\date{\today}
\pagestyle{headings}
\usepackage{amsmath}
\usepackage{amsfonts}
\usepackage{amsthm}
\newtheorem{theorem}{定理}
\newtheorem{definition}{定義}
\newtheorem{proposition}{命題}
\newtheorem{corollary}[proposition]{系}
\newtheorem{lemma}[proposition]{補題}
\usepackage[top=15truemm,bottom=15truemm,left=10truemm,right=10truemm]{geometry}
\usepackage{mathptmx}
\usepackage{amssymb}
\usepackage{ascmac}
\usepackage{color}
\usepackage{comment}
\setcounter{secnumdepth}{4}
\begin{document}
\maketitle

\part{「場」を記述するということ}

\section{Introduction}

ベクトル解析という分野は,元々電磁気学の基本法則であるMaxwell方程式を,最も効果的に叙述する言語を考える試みの中で考案された.ベクトル解析の基礎は,Willard Gibbs(1839-1903 米)とOliver Heaviside(1850-1925)によって定式化された.Edwin Wilson(1879-1964 米)が,GibbsがYale大学で行った講義を元に出版した"Vector Analysis"(1901)が,史上初のベクトル解析に就いての教科書であり,これによって現在の殆どの基本術語と記法が確立され,物理学において常識的な概念となっていった.

ベクトル解析は非常に言語性の高い記法体系を持つ.線積分の記号は,特に数理的な基礎付けについて考えなくとも,直感的に書き下すことが出来る.これがベクトル解析が特有に持つ性質だと言えるだろう.\par
その証拠に,電磁気学をはじめ,現在では多くの物理現象はベクトル場やテンソル場として記述されるため,ベクトル解析は物理学の基本言語として欠かせない.古典力学を高校で学ぶときでさえ,現代ではベクトルを基本言語として用いる.しかし,勿論であるが,ベクトルという概念が誕生したのは,古典力学の登場(17世紀)よりもずっと後の話であり,当時は直交座標系を用いた解析幾何学的表現や,Hamiltonの四元数を用いた記法が一般的であったが,いずれも現在はその立ち位置をベクトルに取って代わられて居る.ベクトルの概念が線形代数学の完成によって数学的に厳密になったのも,同時期の話である.(ベクトルを用いて量子力学を定式化したWerner Heisenberg(1901-1976 独)も,線形代数を習って居なかった.)そして,数学としても,一般にn次元多様体上で,微分形式という概念によって同様の理論を展開できる.この理論の,3次元Euclid空間に話を限ったバージョンがベクトル解析だ,と言うことになる.\par
数学と物理学はこのようにして,相互に刺激し合いながら共に進んでいく学問である.そこで,線形代数学の整備と教育システムへの普及が終わったのがつい最近であるように,次の課題は微分形式である.幸い現代社会にはTechnologyも行き渡るようになって来ていて,数学を伝える手段は活字による数学書か,黒板による授業の2つに1つだけに頼らなくても済むようになった.
\begin{quote}
    数学者の立場から見ると,本当は微分形式は「電磁気学という学問の数学」に入っているべきものだと思うのですが,物理学の「常識」としては,残念ながら微分形式は電磁気学よりも進んだ物理学を学ぶまでは触れないのが普通のようです.
\end{quote}
\noindent
次の課題は明らかである.

\section{ベクトル場とスカラー場の定義}

そもそもMaxwell方程式とは何についての方程式であるか?\\
Maxwell方程式とは,ベクトル場$\mathbf{E,B}$を未知の対象とする微分方程式である.

ではベクトル場とは何か?\\
時空内の元$(x,y,z,t)$を引数により,空間ベクトル$(u,v,w)$を返すベクトル値関数である.\\
だが,Maxwell方程式は決して「4変数関数3つについての微分方程式」ではない.「関数の組」と「ベクトル場」とは別物である.「場」とは,数学的な必然性があって作られた概念であるという訳ではなく,物理現象を捉えるための数理モデルだと思った方が良い.\textcolor{blue}{(??)物理学の「場」の概念は,数学的な必然性があると言うよりも,多分物理的要請から作られた数理モデルに過ぎない.}\\
こうして,我々は「場の微分(2種類)」「場の積分(2種類)」を,「関数の微積分」から拡張させる形で,別に考えていかなければならない.その過程で「関数の組」と「ベクトル場」とは別物であることが了解されるであろう.

そして引き続き,Maxwell方程式についての数学的な考察を深めていくことによって,「scaler potentialとvector potentialについての微分方程式」へと書き換えられ,「量子電磁力学QED」へと話を進めることが出来る.

そう,Maxwell方程式とは微分系が正式である.電磁気学に限らず「場」とは局所的な理論であり,宇宙の果てのことは考えなくて済む.使うときに積分して使う.しかし,直感的な理解のし易さも,歴史的な電磁気学の発展も,積分系から入ったため,此処でも積分系から話を進める.次に,ベクトル場の微分も独立に定義して,「ベクトル場の上での微積分学の基本定理」を導く.

\subsection{今回の議論領域}

我々は電磁気的現象を包括的に説明する数理的枠組みを構築しようとしている.それに当たって,「この中で議論します」という空間を定義しないと,始まらない.その空間とは何かと言えば,我々が今生きている物理空間のことであるが,これをまずはどうモデル化して用意して,議論を始めれば良いか.

\begin{shadebox}\begin{definition}[空間$U$]
空間$U$を「物理空間内に存在する点$P,Q,R\dots$全体の集合,と定義する.
\end{definition}\end{shadebox}
以降,まずはベクトルとスカラーとは何かを定義し,次に場を定義してから,微積分の定義へと進んでいくが,その全ての議論の舞台はこの$U$であるとする.\\
この定義の意味を明らかにしよう.我々の済む三次元空間には,初めから線型空間という見方もなく,定まった座標系もない,単に空間内の点$P,Q,R,\ldots$の集合としてまずはモデル化する.この意味を強調するため,$\mathbb{R}^3$ではなく$U$とした.$U$に1組の座標系を導入する度に,実数の3-組の集合$\mathbb{R}^3$と同一視出来る.そして実数の組とは周知の通り,自然に$\mathbb{R}$-線型空間と見做せる.\\
さて,では$U$の濃度が連続体濃度である(座標系を入れれば$\mathbb{R}^3$と同一視出来る)という仮定を暗にしたことは特筆しておく.空間とはどんなに拡大しても密に点が詰まっていて,「隙間」がない連続的なものである,という仮定を置いたことになる.現時点ではこれで何の問題もないどころか,適切な仮定をしたと思える.実際,このモデルのまま,古典電磁気学は最後まで展開される.\\
なお,時間$t$については,$U$とは別に引数とし,$U$の定義とは分離しておく.簡単のためにまずは時間$t$は止めてから考察し,最後に変数$t$を動かす,という順番で考察したいからだ.

\subsection{ベクトルとスカラーの定義}

\underline{以降,単にベクトルと言った場合には,3次元Euclid空間$\mathbb{R}^3$を存在の場とする幾何ベクトルであるとする.}有向線分は,以下の見地に立ったときには幾何ベクトルとも言う.$\mathbb{R}^3$内の全ての有向線分からなる集合について,「始点同士と終点同士を結んだ際に平行四辺形を成す」という条件によって同値関係を導入した際の同値類のことを指す.つまり,平行移動して重なる有向線分は,互いに同じものとみなすと約束する.するとこの商集合は和(矢印の足し合わせ)と実数倍について線型空間$V$をなすためにベクトルの名を持つ.この幾何ベクトルを表記するのに,有向線分の記法$\overrightarrow{AB}$を受け継いで,$\vec{v}$などと書くこととする.

\textcolor{blue}{(改善の余地)ベクトルは,空間$\mathbb{R}^3$の基底を定める毎に,「成分表示」による数ベクトル表現を持つ.そして,基底変換に対してその表現は必ず規則的に変化する,基底の変換行列が表す線型写像によって正しく移される,という意味で.}\\
一方でスカラーとは,空間$\mathbb{R}^3$の基底の取り方に依らず,値が変化しない数を言う.

以上の議論を定義にまとめる.\\
\begin{shadebox}
\begin{definition}[vector, scaler] \label{vector, scaler}
    ベクトルとは,見方を変えた際に,座標系と同じ変換を受ける,$U$内の幾何ベクトル(またはそれで表される物理量)のことである.スカラーとは,座標系の取り方に依らずに値の変わらない数(またはそれによって表される物理量)のことである.
\end{definition}\end{shadebox}
一般書でよくされるような「ベクトルとスカラーの違いというのは,向きの情報も持つか,大きさしか持たないかの違いだ」というような表現は,この定義を噛み砕いた結果である.そもそも自然言語における「向き」という言葉は,情報の中でも特に,見方によって変わる情報の指すという定義を暗に含んでいる.\\

\noindent
*ベクトルを「幾何ベクトルだ」と呼んでしまっているように,矢印は線型空間を成すことを前提としている.そしてそれは数ベクトルの成す線型空間と同型だから,「座標系と同じ変換」と言った時はその場その場で
成分表示を設定して,行列によって基底変換を施す姿を想像してもいいが,ともかく物理学で言うベクトルとは,そう言った座標変換に規則正しく「一緒に変化」するような矢印のことに他ならない.\textcolor{blue}{言葉を浪費した説明}\\

ベクトルについての用語を整理しておく.\\
幾何ベクトル$\vec{v}$のうち,3次元空間を存在の場とするものを「空間ベクトル」,2次元,1次元空間を存在の場とするものを夫々「平面ベクトル」「直線ベクトル」と呼ぶ.\\
有向線分の同値類ではなく,数の組という数学的対象からなる線型空間の元を「数ベクトル」といい,太字を用いて$\mathbf{v}$などと表す.数の組であれば,横に並べられていても,縦に並べられていても,行列のように横にも縦にも両方向に並べられていても,数ベクトルとしては表現の違いでしかなくて本質的に同一である.しかし,縦と横を使い分ける慣習は線型代数学の理論によって既に存在する上,混同して良いことはないので,「横ベクトル」「縦ベクトル」「行列」と呼び分け,数ベクトル空間の元は縦ベクトルを用いて表現する.

\subsection{場の定義}

\begin{shadebox} \begin{definition}[fields]
    各時間毎の空間$U$の各点について,スカラーを1つずつ対応させる写像$\varphi$を「スカラー場」といい,(幾何)ベクトルを1つずつ対応させる写像$\overrightarrow{F}$を「ベクトル場」という.
\end{definition} \end{shadebox} 


このようにして,「時空全体を定義域に取った実数値函数」がスカラー場,「時空全体を定義域に取った幾何ベクトル値写像」のことをベクトル場と呼んで,これ全体を考察対象にする.時空には座標系を入れるたびに$\mathbb{R}^4$と同一視出来,こうしてスカラー場は4変数函数$f:\mathbb{R}^4\longrightarrow \mathbb{R}$と同一視され(点$P$が$(x,y,z)$として表される$\Longrightarrow f(x,y,z)=\varphi(P) \in \mathbb{R}$),より簡単に解析可能になる.ベクトル場の値は数ではなく直線だから,それを数ベクトル値函数のように分解して「4変数関数n個だ」などとはすぐには考えない.まさに,「\textbf{時空上の各点に,思い思いの方向に草が生えている}」様子がベクトル場である.時空に座標系を定めれば,4変数ベクトル値関数$\mathbf{F}(x,y,z) := \left( \begin{array}{c} F_1(x,y,z) \\ F_2(x,y,z) \\ F_3(x,y,z) \end{array} \right)$と見做せる.こちらを「ベクトル場の成分表示」と言う.

こうして,空間に対して座標系に依らないアプローチをかけていく準備が出来た.現代の物理教育で座標の導入に慣れ過ぎたために最初は違和感があるかもしれないが,座標系という人間の恣意的な切り解き方を乗り越えて,より高い理論を見ようとする試みが,ベクトルという概念の最終到達点である(はず).(微分形式とか)

\noindent
*他に,テンソルを1つずつ対応させるテンソル場,スピノル(spinor)を1つずつ対応させるスピノル場などを考える理論もあるが,電磁気学では当分ここまでで十分.\\
*以降簡単のために時間$t$は省略して,スカラー場もベクトル場もあたかも3変数函数であるかのように書くが,「時間を止めて,ある瞬間について考えている」とも,「時間によって変化しない系を考えている」とも解釈してもよい.空間についての考察が全て終わった後に,時間を引数とした一変数関数として考え,その微分を考察すれば,Maxwell方程式については十分である.時間とはそういう存在である.「静電場」という物理用語は,まさにこの簡略化のことを意味する.

\subsection{場の上の積分領域:位相空間論の言葉}

\section{場の積分:「関数の積分」を,場にどう拡張するか?}


\end{document}