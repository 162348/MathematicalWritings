\documentclass[uplatex, 12pt, a4paper]{jsarticle}
\title{Lebesgue積分とSobolev空間1 米田剛 \\ 講義ノート}
\author{司馬博文 J4-190549 \\ hirofumi-shiba48@g.ecc.u-tokyo.ac.jp}
\date{2019年9月30日}
\pagestyle{headings}
\usepackage{amsmath}
\usepackage{amsfonts}
\usepackage{amsthm}
\newtheorem{theorem}{定理}
\newtheorem{definition}{定義}
\newtheorem{proposition}{命題}
\newtheorem{corollary}[proposition]{系}
\newtheorem{lemma}[proposition]{補題}
\usepackage[top=15truemm,bottom=15truemm,left=10truemm,right=10truemm]{geometry}
\usepackage{color}
\usepackage{mathptmx}
\usepackage{amssymb}
\usepackage{ascmac}
\setcounter{secnumdepth}{4}
\usepackage{comment}
\begin{document}
\maketitle

\part{非可算無限・非可測集合}
\section{実数の非可算性}
\noindent
\begin{shadebox}
\begin{proposition}[実数の非可算性]実数の部分集合である$(0,1):=\{x \in \mathbb{R} : 0<x<1\}$中に存在する実数は非可算無限である.\end{proposition} \end{shadebox}
\begin{proof}[証明(Cantorの対角線論法)]$(0,1)$内の実数が可算であると仮定する.すると,実数の全ての元を,以下のように並べることができる.\\
\begin{eqnarray*} \left. \begin{array}{c} 0.1325......\\0.552.......\\0.9833......\\ \vdots \end{array} \right\}  \dots \star^1 \hspace{30mm}\\
\hspace{30mm} \Longrightarrow \left( \begin{array}{lc} 0.a_{11}a_{12}a_{13} & \hspace{7mm} \\ 0.a_{21}a_{22}a_{23} & \hspace{7mm} \\ 0.a_{31}a_{32}a_{33} & \hspace{7mm} \\ \hspace{7mm} \vdots  & \text{以下同様に附番していく.} \end{array}  \right.
\end{eqnarray*}
ここで,$\overline{a}_{ij}$を$a_{ij}$ではない勝手な一桁の自然数とする.これを用いて,新たな数$0.\overline{a}_{11} \overline{a}_{22} \overline{a}_{33}......$を作ると,これは$(0,1)$内の実数であるのにも関わらず,$\star^1$の中には存在しない数であるから,矛盾.
よって実数は非可算である.\end{proof}

\noindent
これを踏まえて,非可測集合を構成しよう.
\section{実数の非可測集合 Vitali Monsters (Herrlich)}
\vspace{5mm}

\noindent
長さが定まらない$\mathbb{R}$の部分集合$\Lambda$を構成することを考える.\\
まず長さとはなんだろう?ここで考えたい「集合の長さ」とは,例えば区間$[0,1]$だと$1$,一般に区間$[a,b]$だと$b-a$だと考えられるようなもののことだ.では,集合$[0,1]\cup[3,4]$は?当然$2$と定義すべきだろう.この感覚を,厳密な定義に落とし込む1つのやり方がLebesgue測度である.ここでは「集合の長さ」はこの直感的な定義で十分であるから,深入りしない.以下,集合の濃度ではなく,長さのことを$|A|$などと書く.\\さて,実数の部分集合で,この「長さ」の概念が考えられないような集合(数学的に言えば「Lebesgue不可測集合」)は作れるだろうか?そんな場合があるのだろうか?20世紀に入るまで,その存在性は誰にも分からなかった.\\[3mm]
\begin{shadebox}\begin{definition}[Vitali集合]$\mathbb{R}$の部分集合$\Lambda$を「任意の実数$x$に対して,一意に$r \in \Lambda$と$q\in \mathbb{Q}$が存在し,$x=r+q$と表せるようなもの」と定義する.\end{definition}\end{shadebox}
\noindent
*Giuseppe Vitaliは1875-1932にかけてのイタリアの数学者である.実数の部分集合の中で,不可測なものが存在することを示した(Vitaliの定理,1905)(というより実例を初めて作った)のが彼である.\\
*すると,$x$が有理数の時,$r$も有理数である.$x$が無理数の時,$r$も無理数である.$\Lambda$は一対一対応はするから実数と同じ濃度であろう.(正しいこと言っているけどあまり自明な事実ではない.この文章は要削除)\\
*つまりこれは,$\mathbb{R}$を$\mathbb{Q}$-線型空間と見なした時の基底の冪集合に等しい.(多分等濃って意味で.)\\
*なお,このVitali集合の存在は,選択公理を仮定して初めて示される.一気に選び出すことを含意しているからだ.つまり,選択公理を仮定しない宇宙では,実数の部分集合に不可測集合が存在するかどうかは未解決である(と思う).\\
*このような集合は,不可算に無限個存在する.

\noindent
ここで,便宜上,$\Lambda \subset (-1,1)$となるように,代表元の選出を調整する.(選出公理はここまで強いのだろうか?いや,選び方は支持できるのだから,ここは選択公理の守備範囲ではないのだろうな.)\\
*例:$x=\pi$の時は,$|r|=|\pi-q|<1$となるように有理数$q$をとる.この場合は$q=3$ととれば,$r=0.1415926535......$となり,$r\in(-1,1)$である.有理数は稠密だから,このようになる$q$は常に取れる.\\[5mm]

\begin{shadebox}\begin{proposition}[集合$\Lambda$の長さ]集合$\Lambda$の長さは定まらない.\end{proposition}\end{shadebox}
\begin{proof}[証明]
集合$\Lambda$の定め方から,$(-1,1)$区間内の任意の実数$x$は,$\Lambda$の元$r$と$(-2,2)$内の有理数$q$を使って,$x=r+q$と表される.ここで,集合$\Lambda$の平行移動$V_n$を,$q_n$を$(-2,2)$内の$n$番目の有理数として,$$V_n = \{ \lambda+q_n | \lambda \in \Lambda \}$$
と与えると,$$(-1,1) \subset \bigcup^{\infty}_{n=1}V_n \subset (-3,3) \dots \star^2$$となることがわかる.そして,$\{V_n\}^{\infty}_{n=1}$は互いに素である.なぜなら,もし自然数$j,k$が存在して,$V_j \cap V_k \neq \varnothing$だったとすると,$x \in V_j \cap V_k$が取れて,$x=r_j+q_j=r_k+q_k$という二通りの表現が得られて,$\Lambda$の定義に矛盾.よって,$V_j \cap V_k = \varnothing (\rm{for} \hspace{1mm} \forall k,j \in \mathbb{N})$である.よって,$$\left| \bigcup^{\infty}_{k=1}V_k \right| = \sum^{\infty}_{k=1}|V_k|$$よって,$\star^2$より,$$|(-1,1)| \leq \left| \bigcup^{\infty}_{k=1}V_k \right| = \sum^{\infty}_{k=1}|V_k| = \sum^{\infty}_{k=1}|V+q_k| = \sum^{\infty}_{k=1}|V| \leq |(-3,3)| $$であり,結局$$2 \leq |\Lambda| \sum^{\infty}_{k=1} 1 \leq 6$$である.一つの定数の無限和は0であるか無限大に発散するかのいずれかであるから,そのいずれの場合にしろ,この式を満たす$|\Lambda|$は存在しない.
\end{proof}



\end{document}