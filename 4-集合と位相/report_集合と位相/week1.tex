\documentclass[uplatex, 12pt, dvipdfmx]{jsarticle}
\title{集合と位相 第1回レポート}
\author{司馬博文 J4-190549}
\date{\today}
\pagestyle{empty} \setcounter{secnumdepth}{4}
\input{/Users/hirofumi.shiba48/Desktop/数理科学/preamble_CM.tex}
\begin{document}
\maketitle

\section*{1}

\subsection*{(1)}

\[\forall a\in\R^m,\;\forall\epsilon>0,\exists\delta>0,\;\|x-a\|<\delta\Rightarrow\|f(x)-f(a)\|<\epsilon \]
否定は,
\[ \exists a\in\R^m,\;\exists\epsilon>0,\forall\delta>0,\;\|x-a\|<\delta\land\|f(x)-f(a)\|\ge\epsilon \]

\subsection*{(2)}

\[ \forall\epsilon>0,\;\exists N>0, n,m>N\Rightarrow \|x_n-x_m\|<\epsilon \]
否定は,
\[ \exists\epsilon>0,\;\forall N>0, n,m>N\land \|x_n-x_m\|\ge\epsilon \]

\section*{2}

\subsection*{(1)}

$\forall x\in B\cup(\cap_\gamma A_\gamma)$について,
\begin{align*}
    &x\in B\cup(\cap_\gamma A_\gamma)\\
    \Leftrightarrow&x\in B\lor x\in\cap_\lambda A_\lambda\\
    \Leftrightarrow&x\in B\lor(\forall\lambda\in\Lambda,\; x\in A_\lambda)\\
    \Leftrightarrow&\forall\lambda\in\Lambda,\; (x\in B\land x\in A_\lambda)\\
    \Leftrightarrow&\forall\lambda\in\Lambda,\; (x\in B\cap A_\lambda)\\
    \Leftrightarrow&x\in \cap_\lambda (B\cap A_\lambda),
\end{align*}
が成り立つから,$B\cup(\cap_\gamma A_\gamma)=\cap_\lambda (B\cap A_\lambda)$.

\subsection*{(2)}



\end{document}