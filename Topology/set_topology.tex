\documentclass[uplatex, 12pt, dvipdfmx]{jsreport}
\title{集合と位相ノート}
\author{司馬博文}
\date{\today}
\pagestyle{headings} \setcounter{secnumdepth}{4}
\usepackage{amsmath, amsfonts, amsthm, mathptmx, amssymb, ascmac, color, comment, graphicx, wrap fig}
\usepackage[shortlabels,inline]{enumitem}	%足助さんからもらったオプション
\usepackage{tikz, tikz-cd}
\usepackage[top=15truemm,bottom=15truemm,left=10truemm,right=10truemm]{geometry}
%以下足助さんからのコピペ
\newtheoremstyle{StatementsWithStar}% ?name?
{3pt}% ?Space above? 1
{3pt}% ?Space below? 1
{}% ?Body font?
{}% ?Indent amount? 2
{\bfseries}% ?Theorem head font?
{\textbf{.}}% ?Punctuation after theorem head?
{.5em}% ?Space after theorem head? 3
{\textbf{\textup{#1~\thetheorem{}}}{}\,$^{\ast}$\thmnote{(#3)}}% ?Theorem head spec (can be left empty, meaning ‘normal’)?
%
\newtheoremstyle{StatementsWithStar2}% ?name?
{3pt}% ?Space above? 1
{3pt}% ?Space below? 1
{}% ?Body font?
{}% ?Indent amount? 2
{\bfseries}% ?Theorem head font?
{\textbf{.}}% ?Punctuation after theorem head?
{.5em}% ?Space after theorem head? 3
{\textbf{\textup{#1~\thetheorem{}}}{}\,$^{\ast\ast}$\thmnote{(#3)}}% ?Theorem head spec (can be left empty, meaning ‘normal’)?
%
\newtheoremstyle{StatementsWithStar3}% ?name?
{3pt}% ?Space above? 1
{3pt}% ?Space below? 1
{}% ?Body font?
{}% ?Indent amount? 2
{\bfseries}% ?Theorem head font?
{\textbf{.}}% ?Punctuation after theorem head?
{.5em}% ?Space after theorem head? 3
{\textbf{\textup{#1~\thetheorem{}}}{}\,$^{\ast\ast\ast}$\thmnote{(#3)}}% ?Theorem head spec (can be left empty, meaning ‘normal’)?
%
\newtheoremstyle{StatementsWithCCirc}% ?name?
{6pt}% ?Space above? 1
{6pt}% ?Space below? 1
{}% ?Body font?
{}% ?Indent amount? 2
{\bfseries}% ?Theorem head font?
{\textbf{.}}% ?Punctuation after theorem head?
{.5em}% ?Space after theorem head? 3
{\textbf{\textup{#1~\thetheorem{}}}{}\,$^{\circledcirc}$\thmnote{(#3)}}% ?Theorem head spec (can be left empty, meaning ‘normal’)?
%
\theoremstyle{definition}
 \newtheorem{theorem}{定理}[section]
 \newtheorem{corollary}[theorem]{系}
 \newtheorem{proposition}[theorem]{命題}
 \newtheorem*{proposition*}{命題}
 \newtheorem{lemma}[theorem]{補題}
 \newtheorem*{lemma*}{補題}
 \newtheorem*{theorem*}{定理}
 \newtheorem{definition}[theorem]{定義}
 \newtheorem{axiom}[theorem]{公理}
 \newtheorem{example}[theorem]{例}
 \newtheorem{notation}[theorem]{記号}
 \newtheorem*{notation*}{記号}
 \newtheorem{assumption}[theorem]{仮定}
 \newtheorem{question}[theorem]{問}
 \newtheorem{reidai}[theorem]{例題}
 \newtheorem{remark}[theorem]{注}
% \newtheorem*{remarknonum}{注}
 \newtheorem*{definition*}{定義}
 \newtheorem*{remark*}{注}
 \newtheorem*{question*}{問}
%
\theoremstyle{StatementsWithStar}
 \newtheorem{definition_*}[theorem]{定義}
 \newtheorem{question_*}[theorem]{問}
 \newtheorem{example_*}[theorem]{例}
 \newtheorem{theorem_*}[theorem]{定理}
 \newtheorem{remark_*}[theorem]{注}
%
\theoremstyle{StatementsWithStar2}
 \newtheorem{definition_**}[theorem]{定義}
 \newtheorem{theorem_**}[theorem]{定理}
 \newtheorem{question_**}[theorem]{問}
 \newtheorem{remark_**}[theorem]{注}
%
\theoremstyle{StatementsWithStar3}
 \newtheorem{remark_***}[theorem]{注}
 \newtheorem{question_***}[theorem]{問}
%
\theoremstyle{StatementsWithCCirc}
 \newtheorem{definition_O}[theorem]{定義}
 \newtheorem{question_O}[theorem]{問}
 \newtheorem{example_O}[theorem]{例}
 \newtheorem{remark_O}[theorem]{注}
%
\theoremstyle{definition}
%
\raggedbottom
\allowdisplaybreaks
%
\everymath{\displaystyle}
\renewcommand{\proofname}{\bf [証明]}
\renewcommand{\thefootnote}{\dag\arabic{footnote}}	%足助さんからもらった.どうなるんだ?
%以下coprodを出力するためのマクロ.
\DeclareSymbolFont{cmletters}{OML}{cmm}{m}{it}
\DeclareSymbolFont{cmsymbols}{OMS}{cmsy}{m}{n}
\DeclareSymbolFont{cmlargesymbols}{OMX}{cmex}{m}{n}
\DeclareMathSymbol{\myjmath}{\mathord}{cmletters}{"7C}
\DeclareMathSymbol{\myamalg}{\mathbin}{cmsymbols}{"71}
\DeclareMathSymbol{\mycoprod}{\mathop}{cmlargesymbols}{"60}
\let\jmath\myjmath
\let\amalg\myamalg
\let\coprod\mycoprod
\begin{document}
\tableofcontents
\vspace{1cm}
これは斎藤毅『集合と位相』(東京大学出版会.2016),松坂和夫『集合・位相入門』(岩波書店.2015)を読んでまとめたノートである.
\chapter{集合論}

\section{集合の扱い}

\begin{proposition}$A,B,C\subset X$とする.次の2つの条件は同値である.\rm{}\\
    (1)\, $A\cup B\subset C$ \\
    (2)\, $A\subset C\wedge B\subset C$
\end{proposition}

\begin{proposition}[部分集合Aによって生成される同値関係]
    $X$を集合とする.$A\subset X\times X$とし,${}^t\!A=\{ (x,y)\in X\times X|(y,x)\in A\}$と置く.$X$の元の対$(x,y)$についての次の条件$(R_A)$は$X$上の同値関係である.\\
    $(R_A)$:自然数$n\ge 0$と,$X$の元$x=x_0,x_1,\cdots,x_n=y$で,全ての$i\in \{ m\in\mathbb{N} | 1\le m\le n \}$に対して$(x_{i-1},x_i)\in A\cup {}^t\!A$を満たすものが存在する.
\end{proposition}
AP1.4.4から.$A$は$X$上の有向グラフを,$A\cup {}^t\!A$は$X$上の無向グラフを表し,条件$(R_A)$は$X$の2つの元が道で結ばれていることを表し,集合$X$をいくつかの道に直和分解する,と捉えられる.

\section{写像の扱い}

\begin{itembox}[l]{写像の集合論的構成}
	\begin{definition}[mapping]\rm{} \\
		1. 部分集合$\Gamma\subset X\times Y$が「$X$から$Y$への写像のグラフである」とは,$$\forall x\in X\, \exists y\in Y \, ((x,y)\in\Gamma\wedge \forall z\in Y((x,z)\in\Gamma\Rightarrow y=z))$$が成立することを言う.\\
        2. $\Gamma\subset X\times Y$を,$X$から$Y$への写像のグラフとするとき,3つ組$f=(\Gamma,X,Y)$を写像$X\to Y$と呼び,$f:X\to Y$と表す.\\
        また,$$\mathrm{Map}(X,Y)=\{ f\in P(X\times Y)\times \{ X\}\times\{ Y\} | f=(\Gamma,X,Y), \Gamma はXからYへの写像のグラフ \} $$
	\end{definition}
\end{itembox}

\begin{itembox}[l]{様々な写像}
	\begin{definition}[restriction mapping]
		$A$を集合$X$の部分集合とする.以下のように定める「写像を,その$A$での制限に写す」写像を\textbf{制限写像}という.
		\begin{center}\begin{tikzcd}
			\mathrm{Map}(X,Y) \ar[r] \ar[d, phantom, "\rotatebox{90}{$\in$}"] & \textrm{Map}(A,Y) \ar[d, phantom, "\rotatebox{90}{$\in$}"] \\
			f:X\to Y \ar[r, mapsto] & f|_A:A\to Y
		\end{tikzcd}\end{center}
	\end{definition}
	\begin{definition}[evaluation mapping]
		$x$を集合$X$の元とする.以下のように定める「写像を,その$x$での値に写す」写像$\mathrm{ev}_x$を\textbf{$x$での値写像}という.
		\begin{center}\begin{tikzcd}
			\mathrm{ev}_x:\mathrm{Map}(X,Y) \ar[r] \ar[d, phantom, "\rotatebox{90}{$\in$}"] & Y \ar[d, phantom, "\rotatebox{90}{$\in$}"] \\
			f:X\to Y \ar[r, mapsto] & f(x)
		\end{tikzcd}\end{center}
		また,次の写像$e$も定義されるが,定まった名前はない.
		\begin{center}\begin{tikzcd}
			\mathrm{e}:\mathrm{Map}(X,Y)\times X \ar[r] \ar[d, phantom, "\rotatebox{90}{$\in$}"] & Y \ar[d, phantom, "\rotatebox{90}{$\in$}"] \\
			(f,x) \ar[r, mapsto] & f(x)
		\end{tikzcd}\end{center}
		この写像の第二変数$x\in X$を定める度に,値写像を得る.
	\end{definition}
	\begin{definition}[composition]
		写像の合成に関連して,以下の写像が定まる.なお,各写像を$f:X\to Y, g:Y\to Z$とする.
		\begin{center}\begin{tikzcd}
			\circ :\mathrm{Map}(Y,Z)\times\mathrm{Map}(X,Y) \ar[r] \ar[d, phantom, "\rotatebox{90}{$\in$}"] & \mathrm{Map}(X,Z) \ar[d, phantom, "\rotatebox{90}{$\in$}"] \\
			(g,f) \ar[r, mapsto] & g\circ f \\
		\end{tikzcd}\end{center}
		また,各写像を$f:X\to Y, g:W\to X, h:Y\to Z$として,
		\begin{center}\begin{tikzcd}
			f_*:\mathrm{Map}(W,X) \ar[r] \ar[d, phantom, "\rotatebox{90}{$\in$}"] & \mathrm{Map}(W,Y) \ar[d, phantom, "\rotatebox{90}{$\in$}"]\\
			g:W\to X \ar[r, mapsto] & f\circ g:W\to Y \\
			f^*:\mathrm{Map}(Y,Z) \ar[r] \ar[d, phantom, "\rotatebox{90}{$\in$}"] & \mathrm{Map}(X,Z) \ar[d, phantom, "\rotatebox{90}{$\in$}"]\\
			h:Y\to Z \ar[r, mapsto] & h\circ f:X\to Z
		\end{tikzcd}\end{center}
	\end{definition}
\end{itembox}

\begin{proposition}[部分集合と特性関数]
	$X$を集合とする.$X$の部分集合$A$に対して,特性関数$\chi_A:X\to 2$を対応させる写像$\chi :P(X)\to \mathrm{Map}(X,2)$は可逆である.
\end{proposition}

\begin{proposition}[普遍性による特徴付け]$f:X\to Y$を写像とする.次の2つの条件は同値である.\rm{}\\
    1.\, $f$は可逆である.\\
    2.\, 任意の集合$Z$に対して,写像$f^*:\mathrm{Map}(Y,Z)\to\mathrm{Map}(X,Z)$は可逆である.
\end{proposition}

\section{集合族}

\subsection{集合族の定義}
集合族自体は,添字集合と呼ばれる集合から,集合の集合への写像として定義される.「集合族の間の演算」として,集合間の演算を,無限項の場合にまで含めて統一的に定義できることになる(写像も集合なので,写像の積も含めて).自然数の集合論における定義と響き合って,記法$X^n$もうまく配置集合の記法として説明される.特に,自然数の計算規則(指数法則など)は全て集合論的な構成として説明がつくようだ.
いつだってそうだ,写像によって無限項の場合が議論されるのは微分積分学の列という対象も含めて共通で,その基盤が集合族という考え方となる.組やそれに定義される射影なども,無限集合の場合にまで一般化される.

一方で,こういった集合論的定義とは裏腹に,直和や積なども遍性から定義できるが,対象としては段々と複雑で不自然になって来ており,集合論の方からは選択公理の議論が生じる.
なお,線型空間の基底は集合族$n\to V$のうち,その値域が,一次独立な$V$-生成系となるものである,とすると,「基底を取る」という語が理解しやすい.このように基底を集合族として定式化するなら,その基底の存在性は選択公理とZF上同値になる.
つまり,具体的な構成を,一般の場合にcanonicalに与えることは出来ない.

        \begin{definition}[集合族]
            以下の写像を,\textbf{集合$I$で添字づけられた集合の族}という.
            \begin{center}\begin{tikzcd}
                (X_i)_{i\in I}:I \ar[r] \ar[d, phantom, "\rotatebox{90}{$\in$}"] & \mathfrak{X} \ar[d, phantom, "\rotatebox{90}{$\in$}"] \\
                i \ar[r, mapsto] & X_i
            \end{tikzcd}\end{center}
            $\mathfrak{X}\subset P(X)$である時,族$(X_i)_{i\in I}$を\textbf{集合$X$の部分集合族}という.\\
            $I=\varnothing$である時,族$(X_i)_{i\in I}$は包含写像$i:\varnothing\to\mathfrak{X}$であり,特に\textbf{空な族}という.\\
            有限族の場合は,例えば$I=n\in\mathbb{N}$である時,$X_1,X_2,\cdots,X_n$というような列挙が可能である.\\
            集合$\mathcal{A}\subset P(X)$を集合族と言ってしまう場合は,この$\mathcal{A}$を添字集合とした包含写像$(A)_{A\in\mathcal{A}}:\mathcal{A}\to P(X)$のことを指す,と形式的に考えられる.この場合,$\bigcup_{A\in\mathcal{A}}A$を指して,$\bigcup\mathcal{A}$などと書く.
            $\bigcup\mathcal{A}=X$となる時,集合族$\mathcal{A}$を$X$の被覆という.($\mathcal{A}\subset P(X)$である必要はないこともある.)
        \end{definition}
        \begin{remark}
            線型空間$0$の基底は空な族$\varnothing\to\{ 0\}$である($0$は加法の中立元かつ逆元).その次元は$|0|=|\varnothing |=0$である.
        \end{remark}
        \begin{definition}[集合の演算の一番一般的な形/無限項への拡張]
            $$\bigcup_{i\in I}X_i:=\{ x\in X\, |\, \exists i\in I x\in X_i \},\hspace{3mm}\bigcap_{i\in I}X_i:=\{ x\in X\, |\, \forall i\in I x\in X_i \}$$
            今までの$X\cup Y$などは,$I=2$など$I\in\mathbb{N}$となる有限族の場合と捉えられる.添字集合$I$の概念をはっきりさせることで,この2つの集合演算(構成)と一階述語論理との対応が対称的に理解される.\\
            $I=\varnothing$の場合,$\bigcup_{i\in\varnothing} X_i=\varnothing ,\bigcap_{i\in\varnothing} X_i=X$
        \end{definition}
        \clearpage
        \begin{definition}[無縁和と直和]
            合併$\cup_{i\in I}X_i$が,$\forall i,j\in I \, i\ne j\Rightarrow X_i\cap X_j=\varnothing$を満たす時,特に\textbf{無縁和}であるという.これを強調して$\coprod_{i\in I}X_i$で表す.\\
            族$(X_i)_{i\in I}$に対して,次のように構成された集合を直和と呼ぶ.
            $$\coprod_{i\in I}X_i := \{ (x,i)\in X\times I\, |\, x\in X_i \}$$
            この族$(X_i)_{i\in I}$の合併$\cup_{i\in I}X_i$が無縁和になるときは,標準写像$j_k$
            \begin{center}\begin{tikzcd}
                j_k:X_k \ar[r] \ar[d, phantom, "\rotatebox{90}{$\in$}"] & \coprod_{i\in I}X_i \ar[d, phantom, "\rotatebox{90}{$\in$}"] \\
                x \ar[r, mapsto] & (x,k)
            \end{tikzcd}\end{center}
            によって,各$X_k$と$j_k(X_k)=X_k\times \{k\}$は同一視できる.$\forall i,j\in I\, i\ne j\Rightarrow X_i\cap X_j=\varnothing$のとき,自然に誘導された標準写像によって$\cup_{i\in I}X_i\simeq \coprod_{i\in I}X_i$となるから,無縁和にも直和の記号を混用して同じもののように扱う.
        \end{definition}
        \begin{definition}[積]集合$\cup_{i\in I}X_i$の部分集合族$(X_i)_{i\in I}$について,次を満たす$\mathrm{Map}(I,X)$の部分集合を\textbf{積}と呼ぶ.
            $$\prod_{i\in I}X_i := \{ (x_i)_{i\in I}\in \mathrm{Map}(I,X)\, |\, \forall i\in I \, x_i\in X_i \}$$
            この元$(x_i)_{i\in I}$は値の取り方について特別な条件(各$i\in I$に対して,先に決まっている写像$(X_i)_{i\in I}$に対して,$(x_i)_{i\in I}(i)\in X_i$である必要がある.)を満たした写像$I\to \cup_{i\in I}X_i$で,これを\textbf{$(X_i)_{i\in I}$の元の族}という.\\
            $I=n$であるとき,次の標準写像による同型が存在するから,一般の$I$に対しても,$(x_i)_{i\in I}(j) (j\in I)$を$j$成分と呼ぶ.
            \begin{center}\begin{tikzcd}
                \prod_{i\in n}X_i \ar[r] \ar[d, phantom, "\rotatebox{90}{$\in$}"] & X_0\times \cdots\times X_{n-1} \ar[d, phantom, "\rotatebox{90}{$\in$}"] \\
                (x_i)_{i\in n} \ar[r, mapsto] & (x_0,\cdots,x_{n-1})
            \end{tikzcd}\end{center}
            次の写像を$j$成分への射影と呼ぶ.
            \begin{center}\begin{tikzcd}
                \mathrm{pr}_j:\prod_{i\in I}X_i \ar[r] \ar[d, phantom, "\rotatebox{90}{$\in$}"] & X_j \ar[d, phantom, "\rotatebox{90}{$\in$}"] \\
                (x_i)_{i\in I} \ar[r, mapsto] & x_j
            \end{tikzcd}\end{center}
            全ての$i\in I$に対して,$X_i=\cup_{i\in I}X_i=X$であるとき,積$\prod_{i\in I}X$は単に写像の集合$\mathrm{Map}(I,X)$と一致し,これを$X^I$と書く.\underline{$X^n$という表記の一般化と見れる}.\\
            従って,特に$X^0=\{\varnothing\to X\}$は,$X=0$の時も,包含写像のみを元とする,一元集合である.$id_0:0\to 0$を$0$と書くこととすると,$0^0=1$が成り立つ(集合としての相等).(随分怪しい話だ.圏としての相等と言った方が良いかも知れない).
        \end{definition}
        \begin{axiom}[選択公理:直積の性質への言及として]
            上のように集合族の積を定義するとき,$\exists i\in I\, X_i=\varnothing$である場合は,$\prod_{i\in I}X_i=\varnothing$となる.そうではない場合,$\prod_{i\in I}X_i\ne\varnothing$となる.
        \end{axiom}
        \begin{remark}
            積集合は正しい構成をしていて,必ず集合となる.また,全ての$i$について$X_i\ne\varnothing$である場合は特に,$\cup_{i\in I}X_i$は空でない.従って,各$i\in I$に対して,ある全体集合$\cup_{i\in I}X_i$の,$I$と同じ個数存在する空でない部分集合$X_i$から,それぞれ一つずつ元$x_i\in X_i$を選び取ることが出来る(論理式$\forall i\in I\, x_i\in X_i$を満たす集合$(x_i)_{i\in I}$が存在する)ことを主張している.
            $I$が有限集合である場合は,$I=n$である場合と同一視でき,帰納的に証明できる.一方で,元$(x_i)_{i\in I}$がしっかり書き下せる場合も多い.
            しかし$I$を一般の集合とすると,これは集合論の公理である.
        \end{remark}
以上,集合に対する演算子$\cup,\cap,\coprod,\prod$を定義した.
\begin{definition}[写像の積]
    2つの添字集合を共有した積集合$(X_i)_{i\in I},(Y_i)_{i\in I}$について,各写像$f_i:X_i\to Y_i$から以下のように構成される写像を,写像の族$(f_i)_{i\in I}$の積と呼ぶ.
    \begin{center}\begin{tikzcd}
        \prod_{i\in I}f_i:\prod_{i\in I}X_i \ar[r] \ar[d, phantom, "\rotatebox{90}{$\in$}"] & \prod_{i\in I}Y_i \ar[d, phantom, "\rotatebox{90}{$\in$}"] \\
        (x_i)_{i\in I} \ar[r, mapsto] & (f_i(x_i))_{i\in I}
    \end{tikzcd}\end{center}
    特に,$X_i=X, Y_i=Y, f_i=f:X\to Y$である時,$\prod_{i\in X}X_i=\mathrm{Map}(I,X), \prod_{i\in Y}Y_i=\mathrm{Map}(I,Y)$となり,ある一定の$f:X\to Y$に対して,$(f_i)_{i\in I}=f_*$である.
\end{definition}


\begin{proposition}[集合族とその演算についての分配則とde Morgan則]
    $$\left(\bigcup_{i\in I}X_i\right)\cap Y = \bigcup_{i\in I}(X_i\cap Y), \hspace{3mm}\left(\bigcap_{i\in I}X_i\right)\cup Y = \bigcap_{i\in I}(X_i\cup Y)$$
    $$X\setminus\left(\bigcup_{i\in I}X_i\right)=\bigcap_{i\in I}(X\setminus X_i), \hspace{3mm}X\setminus\left(\bigcap_{i\in I}X_i\right)=\bigcup_{i\in I}(X\setminus X_i)$$
\end{proposition}
$I=2$の場合は,ベン図を使って証明した.

\begin{shadebox}\begin{definition}[積の普遍性]
    $A,B,C$はある圏の対象とする.次の条件を満たすとき,$C$は$A\times B$と書かれる.

    2つの射$\pi_1:C\to A,\pi_2:C\to B$が存在し,各$f_1:X\to A, f_2:X\to B$について,唯一つ$f:X\to C$が存在して次の図式が可換になる.

    \begin{center}\begin{tikzcd}
        & X \ar[dl] \ar[d] \ar[dr] & \\
        A & C \ar[l, "\pi_1"] \ar[r, "\pi_2"'] & B
    \end{tikzcd}\end{center}
\end{definition}\end{shadebox}
特に,これが集合の圏であった場合,$X=1$とした時の条件から,$\pi_1\circ f = (f_1:X\to A) \simeq a$より,各$(f:1\to C) \simeq (f_1,f_2) \simeq (a,b)$であり,$f:1\to C$のそれぞれとは,$C$の元1つ1つと一対一対応するから,$C=A\times B$の各元は$(a,b)$と(少なくとも)同一視できることは,普遍性の主張に含まれていることがわかる.

\begin{remark}\rm{}
    上の状況下で,
    \begin{center}\begin{tikzcd}
        \varphi : \hom_C(C,A)\times\hom_C(C,B) \ar[r] \ar[d, phantom, "\rotatebox{90}{$\in$}"] & \hom_{[C^{op},Set]}(h^C,\hom_C(-,A)\times\hom_C(-,B)) \ar[d, phantom, "\rotatebox{90}{$\in$}"] \\
        (f_1,f_2) \ar[r, mapsto] & \varphi_{(f_1,f_2)}
    \end{tikzcd}\end{center}
    というbijectionが存在する.特に,$(\pi_1,\pi_2)$の像は次の自然変換である.
    \begin{center}\begin{tikzcd}
        \varphi_{\pi_1,\pi_2}:\hom_C(-,C) \ar[r] \ar[d, phantom, "\rotatebox{90}{$\in$}"] & \hom_C(-,A)\times\hom_C(-,B) \ar[d, phantom, "\rotatebox{90}{$\in$}"] \\
        f \ar[r, mapsto] & (\pi_1\circ f, \pi_2\circ f)
    \end{tikzcd}\end{center}
    これは対象$C$が$A,B$と何の関係が無くても成り立つ.この時,この自然変換$\varphi_{\pi_1,\pi_2}$が可逆でもある時,$C=A\times B$と書き,この唯一の射の組$(\pi_1,\pi_2)$を射影という.

    また従って以上より,積の普遍性は,米田の補題の特別な場合に付けた名前である.
\end{remark}

\begin{proposition}[集合の和と積との関係]\rm{}
    次が成り立つ.

    $$1.\, \bigcup_{i\in I}A_i = \mathrm{pr}_1(\coprod_{i\in I}A_i)$$
    $$2.\, \bigcap_{i\in I}A_i = \delta^{-1}(\prod_{i\in I}A_i)$$
\end{proposition}

\begin{shadebox}\begin{proposition}[積の普遍性]
    $(T)_{i\in I}, (X_i)_{i\in I}$を集合の族,$(f_i)_{i\in I}$を写像$f_i:T\to X_i$の族とする.$(X_i)_{i\in I}$の直積を$X=\prod_{i\in I}X_i$とする.$(f_i)_{i\in I}$の直積は
    \begin{center}\begin{tikzcd}
        \prod_{i\in I}f_i:\prod_{i\in I}T=\mathrm{Map}(I,T) \ar[r] \ar[d, phantom, "\rotatebox{90}{$\in$}"] & \prod_{i\in I}X_i=X \ar[d, phantom, "\rotatebox{90}{$\in$}"]\\
        (t_i)_{i\in I} \ar[r, mapsto] & (f_i(t_i))_{i\in I}
    \end{tikzcd}\end{center}
    であるが,今回特に,$(t_i)_{i\in I}$が定値写像$(t)_{i\in I}$となる場合に注目し,次の写像を$(f_i)\subset \prod_{i\in I}f_i$とする.
    \begin{center}\begin{tikzcd}
        (f_i):\mathrm{Map}(I,T) \ar[r] \ar[d, phantom, "\rotatebox{90}{$\in$}"] & X \ar[d, phantom, "\rotatebox{90}{$\in$}"]\\
        (t)_{i\in I} \ar[r, mapsto] & (f_i(t))_{i\in I}
    \end{tikzcd}\end{center}
    これは結局次の写像$f$と同一視できる.
    \begin{center}\begin{tikzcd}
        f:T \ar[r] \ar[d, phantom, "\rotatebox{90}{$\in$}"] & X \ar[d, phantom, "\rotatebox{90}{$\in$}"]\\
        t \ar[r, mapsto] & (f_i(t))_{i\in I}
    \end{tikzcd}\end{center}
    この時,こうして定義した$f$は,条件$$\forall i\in I \hspace{2mm} f_i=\mathrm{pr}_i\circ f$$
    によって特徴付けられる.
\end{proposition}\end{shadebox}
\begin{proof}
    $j\in I, t\in T$を任意に取る.$f(t)=(f_i(t))_{i\in I}$より,$\mathrm{pr}_j(f(t))=f_j(t)$である.

    $g:T\to X$が任意の$i\in I$に対して$f_i=\mathrm{pr}_i\circ g$を満たすとする.$t\in T$を任意に取り,$g(t)=(x_i)_{i\in I}$とする.すると,全ての$j\in I$に対して,$\mathrm{pr}_j(g(t))=x_j=f_j(t)$が仮定から成り立つが,これは$g$が各$t\in T$に対して,写像$I\ni i\mapsto f_i(t)\in X_i$を対応づけていることを表す.この対応は$f$に他ならず,$f=g$である.
\end{proof}
この普遍性は,任意の$i\in I$に対して全ての写像$\mathrm{pr}_i$について同時に成り立つと言及しているから,写像1つを定めるほどの強さを持つ主張である.このことを,ある図式を用いて,「次の図式を可換にする$f$がただ一つ存在する」とよく表現される.
写像$f:T\to \prod_{i\in I}X_i$も積といい,$(f_i)$と表すらしいが,写像$f_i:T\to X_i$の族$(f_i)_{i\in I}$の積$\prod_{i\in I}f_i:\mathrm{Map}(I,T)\to\prod_{i\in I}X_i$とは別物である.
定値写像$t_j:T\ni t\to t_j\in T$を用いて,各$j\in I$について$f\circ\mathrm{pr}_j = \prod_{i\in I}f_i\circ t_{j*}$が成り立つ.
\begin{proposition}
    定値写像$t_j:T\ni t\to t_j\in T$を用いて,各$j\in I$について$f\circ\mathrm{pr}_j = \prod_{i\in I}f_i\circ t_{j*}$が成り立つ.
\end{proposition}
\begin{remark}
    特に$X_i=T$でもあり,写像$f_i:T\to T$の族$(id_T)_{i\in I}$の積$T\to\mathrm{Map}(I,T)$を対角写像$\delta$と呼ぶ.
    \begin{center}\begin{tikzcd}
        \delta :T \ar[r] \ar[d, phantom, "\rotatebox{90}{$\in$}"] & \mathrm{Map}(I,T) \ar[d, phantom, "\rotatebox{90}{$\in$}"] \\
        t \ar[r, mapsto] & (t)_{i\in I}=\mathrm{Map}(I,\{ t\} )
    \end{tikzcd}\end{center}
    対角写像の値域を対角集合といい,その$n=2$の場合の特性関数をクロネッカーのデルタと言う.
\end{remark}
圏論的には,組$(X, (\mathrm{pr}_i)_{i\in I})$を直積と呼ぶ.対象の族$(t(pr_i))_{i\in I}$の直積が,複数存在するなら,それらは同型である(可逆な射が存在する)ことが,普遍性から証明できる.

\section{商集合と写像の標準分解}

\begin{proposition}[canonical decomposition]
    $f:X\to Y$を写像とする.\\
    1.\, 次の図式を可換にする写像$\overline{f}$が唯一つ存在する.この分解$f=i\circ\overline{f}\circ q$を\textbf{$f$の標準分解}という.
    \begin{center}\begin{tikzcd}
        X \ar[r, "f"] \ar[d, "q"'] & Y \\
        X/R_f \ar[r, dotted, "\overline{f}"] & f(X) \ar[u, "i"']
    \end{tikzcd}\end{center}
    2.\, 写像$\overline{f}$は可逆である.これを\textbf{$f$によって引き起こされる可逆写像}と呼ぶ.\\
    $f$が定める同値関係$R_f$についての商集合$X/R_f$を,\textbf{$f$の余像}と呼ぶ.
\end{proposition}

\section{写像の全射・単射による分解と「引き起こされた写像」}

\begin{proposition}[単射と一般の写像]\rm{}
    $i:X\to Y$を単射,$T$を勝手な集合,$f:T\to Y$を写像とする.次の2つの条件は同値である.
    
    1.\, $f(T)\subset i(X)$である.

    2.\, 下の図式を可換にする写像$g:T\to X$が存在する.
    \begin{center}\begin{tikzcd}
        X \ar[r, rightarrowtail, "i"] & Y \\
        T \ar[u, dashed, "g"] \ar[ur, "f"'] &
    \end{tikzcd}\end{center}
\end{proposition}
$f(T)\supsetneq i(X)$の時,$g$をどう取っても$f(T)\setminus i(X)\ne\varnothing$となってしまうため,写像として一致し得ない.

\begin{definition}[同値関係の関係]\label{def-relationship-between-equivalence-relation}
    同値関係$R_p$が$R_f$より細かいとは,論理式
    \[ \forall x,x'\in X \, p(x)=p(x')\Rightarrow f(x)=f(x') \]
    が成り立つと言うことである.この時,それぞれのグラフを$C_p, C_f$とすれば,
    \[ C_p \subset C_f \]
    と同値である.
\end{definition}

\begin{proposition}[全射と一般の写像]\label{prop-induced-mapping}\rm{}
    $X,Y,Z$を集合,$p:X\to Y$を全射,$f:X\to Z$を写像とする.

    1.\, 次の条件(1)と(2)は同値である,

    (1)\, $f=g\circ p$を満たす写像$g:Y\to Z$が存在する.
    \begin{center}\begin{tikzcd}
        X \ar[r, twoheadrightarrow, "p"] \ar[dr, "f"'] & Y \ar[d, dashed, "g"]\\
        & Z
    \end{tikzcd}\end{center}

    (2)\, 全射$p$が定める同値関係$R_p$は,写像$f$が定める同値関係$R_f$よりも細かい.

    2.\, いま,$R_p$が$R_f$よりも細かいとする.この時,次の2つの条件は同値である.

    (1)\, $f=g\circ p$を満たすこの$g:\to Z$は単射である.

    (2)\, $R_p$と$R_f$は同値である.
\end{proposition}
\begin{remark}
    写像$p$の時点で重要な何かが潰れていなければいい.このための条件は,「写像が定める同値関係」として,共通する始域$X$上の関係,またはそのグラフ(部分集合)の包含関係などで議論できる.
    $R_p$の方が$R_f$よりも細かければ,より豊富な情報を含んでいて還元出来ない部分はないから,$g$を上手く潰すように設定すれば,$f=g\circ p$と出来る.

    なお,2つの同値関係の間の関係として,「よりも細かい」とは,$\forall x,x'\in X \, p(x)=p(x')\Rightarrow f(x)=f(x')$が成り立つと言うことである.この逆も成り立つ時,2つの同値関係は同値であると言う.
\end{remark}
「気持ち」と定式化された「理論」の違いをご覧に入れたい.おそらくこれは定義\ref{def-relationship-between-equivalence-relation}の同値関係同士の「細かい」と言う関係の定式化が上手だからである.
でもそれにしても$(2)\Rightarrow (1)$の証明は,今までの集合論の議論が要点を得ていることを実感する,大海の上を,非常に頑健でかつ絶妙に配置された足場を飛びながら自由に旅をしているの感がある.

まず$(1)\Rightarrow (2)$は,$R_f=R_{g\circ p}$であるが,$R_{g\circ p}$は,$R_p$よりも$g$の分だけ同値類が統合されて粗くなっている($g$が全単射でない限り).従って,$R_f$は$R_p$よりも粗い.

次に,$(2)\Rightarrow (1)$は,$p$が引き起こす可逆写像$\tilde{p}$により$Z\simeq X/R_p$だから,下の図式を可換にするような$g':X/R_p\to Z$を構成すれば良い.
\begin{center}\begin{tikzcd}
    X/R_p \ar[r, dashed, "g'"] \ar[dr, twoheadrightarrow, tail, "\tilde{p}" near end] & Z \\
    X \ar[u, "q_p"] \ar[ur, "f" near start] \ar[r, twoheadrightarrow, "p"] & Y \ar[u, dashed, "g"']
\end{tikzcd}\end{center}
これは,$R_p$の同値類を巧妙に潰して$R_f$にするような$g'$,即ち$f(x)=f(x'),\hspace{0.5em} x,x'\in X \Rightarrow g(p(x))=g(p(x'))$の仕事をしてくれる$g$を選べば良い.

これ以上踏み込めない感覚がするのは,$(1)\Rightarrow (2)$も$(2)\Rightarrow (1)$も,上記の議論では集合論的見地から,具体的な要素について論理を用いて論証していないからであろう.それを実行するには正しい道具の整備を訓練が居る,さもないとこの「所感」のように,表面だけさらって正しいような気がしてしまう.それにしてもここが突破出来るとはとても思えなかった,集合論の威力はここにある.
\begin{proof}
    $(1)\Rightarrow (2)$は\[ \forall x,x'\in X \, p(x)=p(x')\Rightarrow f(x)=f(x') \]を示せば良い.いま,実際$p(x)=p(x')$を満たす$x,x'\in X$について,$q(p(x))=q(p(x'))$であるから,$f(x)=f(x')$が従う.

    次に$(2)\Rightarrow (1)$を考える.写像$g$を構成するために,写像
    \begin{center}\begin{tikzcd}
        (p,f):X \ar[r] \ar[d, phantom, "\rotatebox{90}{$\in$}"] & Y\times Z \ar[d, phantom, "\rotatebox{90}{$\in$}"] \\
        x \ar[r, mapsto] & (p(x),f(x))
    \end{tikzcd}\end{center}
    を考える.この値域$(p,f)(X)=\{ (p(x),f(x))\mid x\in X \}=:\Gamma_g$は(A)写像のグラフとなっており,そして(B)このグラフが定める写像$(Y,Z,\Gamma_g)=:g$が求める唯一つの写像であることを示す.

    (B)については,全ての$x\in X$について,$g$の定め方より$g(p(x))=f(x)$が成り立つから,確かにこれは$f=g\circ p$を満たす写像である.

    (A)\,$\Gamma_g$が写像のグラフとなっていることの証明を,$\mathrm{pr}_1:Y\times Z\to Y$を第一射影として,$\mathrm{pr}_1|_{\Gamma_g}$が全単射であることを示すことによって行う.
    $\mathrm{pr}_1|_{\Gamma_g}\circ (p,f)=id_Y\circ p=p$より,$p$は全射であるから$\mathrm{pr}_1|_{\Gamma_g}$も全射である.また,$(y,z),(y',z')\in\Gamma_g$について$\mathrm{pr}_1(y,z)=\mathrm{pr}_1(y',z')$即ち$y=y'$即ち
    $\exists x,x'\in X \,\mathrm{s.t.}\, p(x)=p(x')$ならば,$R_p\subset R_f$より,$f(x)=f(x')$即ち$z=z'$より,$\mathrm{pr}_1|_{\Gamma_g}$は単射でもある.
\end{proof}
\begin{remark}
    この証明の始め方自体がキーとなっている.集合論という方法論を完全に乗りこなしているかのような,先を見据えた定式化によって,いとも簡単に論理の意図を手繰り寄せる証明で,びっくりした.
\end{remark}
\begin{proof}
    $(2)\Rightarrow(1)$. $R_p=R_f$の時,$X/R_p=X/R_f$であるから,$p,f$の標準分解は,可逆写像$\tilde{p}:X/R_p\to Y$と単射$\overline{f}:X/R_p\to Z$を定める.
    \begin{center}\begin{tikzcd}
        X \ar[r, "p"] \ar[d, "q"'] \ar[dr, "f"' near end, "\circlearrowright"' near start] & Y \ar[d, "g"] \\
        X/R_p \ar[ur, "\tilde{p}" near end, "\circlearrowright"' near start] \ar[r, "\overline{f}"'] \ar[d, dashed, "\tilde{f}"] & Z \\
        f(X) \ar[ur, dashed, "i"']
    \end{tikzcd}\end{center}
    この図式は結局全体として可換であり($f=g\circ p$かつ$f=\overline{f}\circ q$より,$\overline{f}\circ q=g\circ p$を得る.これと$p=\tilde{p}\circ q$より),$\overline{f}\circ\tilde{p}^{-1}=g$となる.従って$g$は全射である.

    $(1)\Rightarrow(2)$.$g$が単射ならば,$g(y)=g(y')\Rightarrow y=y'$より,
    \begin{eqnarray*}
        p(x)=p(x') &\Leftrightarrow& g(p(x))=g(p(x')) \\
        &\Leftrightarrow& f(x)=f(x')
    \end{eqnarray*}
    より,$R_f=R_p$である.
\end{proof}

\begin{definition}[induced mapping]\rm{}
    $p:X\to Y$を全射とし,$f:X\to Z$を写像とする.
    \begin{center}\begin{tikzcd}
        X \ar[r, rightarrowtail, "i"] & Y \\
        T \ar[u, dashed, "g"] \ar[ur, "f"'] &
    \end{tikzcd}\end{center}
    $p$が定める同値関係$R_p$が,$f$が定める同値関係$R_f$より細かい時,$y\in Y$に対して,\textbf{$f(x)=g\circ p(x)\in Z$は$x\in X$の取り方に依らない}といい,\textbf{写像$g$はwell-definedである}という.なお,
    この写像$g$を\textbf{$f$によって引き起こされた写像}という.
\end{definition}
\begin{remark}
    あまり世界観の具体例の想像がつかない.
\end{remark}

\begin{corollary}[商集合の普遍性]
    $R$を集合$X$上の同値関係とし,$q:X\to X/R$を商写像とする.\rm{}
    
    1.\, 写像$f:X\to Y$について,次の2条件は同値である.

    (1)\, 次の図式を可換にする写像$g:X/R\to Y$が存在する.これは$f$によって引き起こされた写像である.
    \begin{center}\begin{tikzcd}
        X \ar[r, "q"] \ar[dr, "f"'] & X/R \ar[d, "g"] \\
        & Y
    \end{tikzcd}\end{center}

    (2)\, $R$は,$f$が定める同値関係$R_f$より細かい.

    2.\, $R'$を$Y$の同値関係とし,$q':Y\to Y/R'$を商写像とする.写像$f:X\to Y$に対して,次の2条件は同値である.

    (1)\, 写像$g:X/R\to Y/R'$で,次の図式を可換にするものが存在する.
    \begin{center}\begin{tikzcd}
        X \ar[r, "f"] \ar[d, "q'"'] & Y \ar[d, "q'"] \\
        X/R \ar[r, "g"'] & Y/R'
    \end{tikzcd}\end{center}

    (2)\, $C\subset X\times X$を$R$のグラフとし,$C'$を$R'$のグラフとすると,$C\subset (f\times f)^{-1}(C')$である.
\end{corollary}
\begin{proof}
    $1.$全射$p$について命題\ref{prop-induced-mapping}を適用して得る主張である.なお,$q$が定める同値関係$R_q$とは$R$に他ならない.
    
    $2.$全射$q'\circ f$について命題\ref{prop-induced-mapping}を適用して得る主張である.
\end{proof}

\begin{definition}[Universal propoerty of quotient set]
    写像$q:X\to X'$について,任意の集合$Y$と写像$f:X\to Y$に対して,次の図式を可換にする$g$が存在するとき,この$X'$を,$q$が定める同値関係$R_q$による商集合といい,$q$をその商写像と呼ぶ.
    \begin{center}\begin{tikzcd}
        X \ar[r, "q"] \ar[dr, "f"'] & X' \ar[d, "g"] \\
        & Y
    \end{tikzcd}\end{center}
\end{definition}
\begin{remark}
    これは$q$が全射であるための条件となっている.でもこのままでは明らかに,全ての$f$に対応できるわけではない,$f$が全単射であった場合,$q$は自明な同値関係による商写像を与える全単射である.
\end{remark}

\section{数の構成}

\begin{definition}[Natural numbers]\rm{}
    次を満たす集合$\mathbb{N}$を自然数と呼ぶ.

    1.\, $\varnothing\in\mathbb{N}$

    2.\, $\forall n(n\in\mathbb{N}\Rightarrow n\cup\{ n\}\in\mathbb{N})$

    3.\, $\forall A((A\subset\mathbb{N}\wedge\varnothing\in A\wedge\forall n(n\in A\Rightarrow n\cup\{n\}\in A))\Rightarrow A=\mathbb{N})$
\end{definition}
\begin{remark}\rm{}
    1.2.を満たす最小の閉包が自然数であるから,条件$P(n)$を1.2.の場合について示せれば,自然数$\mathbb{N}$全体で成り立つことを得る.この自然数の定義上の約束を数学的帰納法と呼ぶ.

    また,$a_0\in X_0$と,$a_0\in X_0,\cdots,a_n\in X_n$がすでに定まっている際に$a_{n+1}\in X_{n+1}$を与えるルールを定めると,列$a=(a_n)\in\prod_{n\in\mathbb{N}}X_n$を定めたことになる.これを帰納的定義(recursive definition)という.
\end{remark}

\begin{proposition}[well-definedness of recursive definition]\rm{}
    $(X_n)_{n\in\mathbb{N}}$を集合列とし,$c\in X_0$とする.

    1.\, $(f_n)_{n\in\mathbb{N}}$を写像$f_n:X_0\times\cdots\times X_n\to X_{n+1}$の族とする.この時,列$(a_n)\in\prod_{n\in\mathbb{N}}X_n$であって,$a_0=c, a_{n+1}=f_n(a_0,\cdots,a_n)(n\in\mathbb{N})$を満たすものは,唯一つ存在する.

    2(AC).\, $(F_n)_{n\in\mathbb{N}}$を写像$F_n:X_0\times\cdots\times X_n\to P(X_{n+1})\setminus\varnothing$の族とする.この時,列$(a_n)\in\prod_{n\in\mathbb{N}}X_n$であって,$a_0=c, a_{n+1}\in F_n(a_0,\cdots,a_n)(n\in\mathbb{N})$を満たすものは,唯一つ存在する.
\end{proposition}
\begin{remark}\rm{}
    構成数学と非構成数学とで,使える道具の差ACを目の当たりにしている.
\end{remark}

\begin{definition}[algebraic/order structure of the natural numbers]
    $m\in\mathbb{N}$への加算と乗算を,それぞれ次のようにして,帰納的に定義する.
    \[ m+0=m, m+(n+1):=(m+n)+1, \]\[ m\cdot 0=0, m\cdot (n+1):=(m\cdot n)+m \]
    順序関係を$m\le n:\Leftrightarrow m\subset n$と定める.
\end{definition}
\begin{remark}
    なお,$m\in n$は$m<n$を定める.これが自然数の特徴かもしれない.
\end{remark}

整数は,自然数の差演算についての閉包として構成できる.
\begin{definition}[Integers]
    $\mathbb{N}^2$上の同値関係$\sim$を,$(n,m)\sim (n',m'):\Leftrightarrow n+m'=n'+m$として定義する(2数の差が同じ).この時,$\mathbb{Z}:=\mathbb{N}^2/\sim$を整数全体の集合と呼ぶ.同値類$\overline{(n,m)}$を$n-m$で表すこととする.

    $Z:=\{(n,m)\in\mathbb{N}^n\mid n=0\lor m=0\}$は$\mathbb{Z}$の完全代表系である.単射$\mathbb{N}\to\mathbb{Z}:n\mapsto n-0$により,$\mathbb{N}$を$\mathbb{Z}$の部分集合と同一視し,$0-n=:-n$と表すこととする.
\end{definition}

\begin{definition}[algebraic/order structure of integers]
    \[ (n-m)+(n'-m'):= (n+n')-(m+m')=\overline{(n+n',m+m')} \]
    \[ (n-m)\cdot (n'-m'):=(nn'+mm')-(mn'+nm')=\overline{(nn'+mm',mn'+nm')} \]
    として$\mathbb{N}^2/\sim$上の加法と乗法を定義し,順序関係は$n-m\le_\mathbb{Z} n'-m' :\Leftrightarrow n+m'\le_\mathbb{N} n'+m$で定める.
    また,$n\in\mathbb{N}$の時,$n,-n\in\mathbb{Z}$の絶対値を$|n|,|-n|=n$と定める.
\end{definition}

有理数は,整数の除算についての閉包として構成できる.
\begin{definition}[Rational numbers]
    $\{ (n,m)\in\mathbb{Z}^2\mid m>0 \}$上に同値関係$\sim$を,$(n,m)\sim (n',m'):\Leftrightarrow nm'=n'm$として定める(2数の比が同じ).$\mathbb{Q}:=\{ (n,m)\in\mathbb{Z}^2\mid m>0 \}/\sim$を有理数全体の集合という.
    同値類を$\overline{(n,m)}=:\frac{n}{m}$と表す.$Q:=\{(n,m)\in\mathbb{Z}^2\mid m>0\land \gcd(n,m)=1\}$はこの完全代表系である.
    標準全射$p:\{ (n,m)\in\mathbb{Z}^2\mid m>0 \}\to\mathbb{Q}$の$Q$への制限の逆写像$p|_Q^{-1}:\mathbb{Q}\to Q$は,有理数に対して,その既約分数表現の分子と分母の組を対応させる写像である.単射$\mathbb{Z}\to\mathbb{Q}:n\mapsto\frac{n}{1}$により,$\mathbb{Z}$を,$\mathbb{Q}$の部分集合と同一視する.
\end{definition}

\begin{definition}[algebraic/order structure of the rational numbers]
    \[ \frac{n}{m} +_\mathbb{Q} \frac{n'}{m'} := \frac{nm'+_\mathbb{Z}mn'}{mm'} \]
    \[ \frac{n}{m}\cdot_\mathbb{Q}\frac{n'}{m'} := \frac{nn'}{mm'} \]
    順序関係を$\frac{n}{m}\le\frac{n'}{m'}:\Leftrightarrow nm'\le n'm$で定め,絶対値は$\left| \frac{n}{m} \right|:\Leftrightarrow \frac{|n|}{m}$で定める.
\end{definition}
四則演算が自由に出来る体まで構成できた.最終到達点が実数の構成である.
これは$\mathbb{Q}$の部分集合から構成される$P(\mathbb{Q})$内部の存在で,何ら集合としての華やかな発展はない.ただし,これは華やかな位相情報を持ち,それが有理数との主な違いである.

\chapter{位相空間論}

\section{実数と位相}

\subsection{実数の構成}

\begin{definition}[Dedekind's cut]\rm{} 

    1.\, $\mathbb{Q}$の部分集合$L$が次の3条件を満たすとき,$L$はデデキントの切断であるという.

    (1)\, $\varnothing\subsetneq L\subsetneq\mathbb{Q}$

    (2)\, $x\in L \,\wedge\, y\le x \,\Rightarrow\, y\in L$

    (3)\, $x\in L \,\Rightarrow \exists y\in L \,\,\, x<y$

    2.\, デデキントの切断$L$を実数と呼び,実数全体の集合を$\mathbb{R}:=\{ L\in P(\mathbb{Q})\mid Lは切断 \}$と書く.


    3.\, 実数$L,M$について,順序関係を$L\le M :\Leftrightarrow L\subset M,\hspace{1em} L<M :\Leftrightarrow \subsetneq$と定める.
\end{definition}

\begin{proposition}\rm{}\label{prop-order-of-the-real-numbers} 

    1.\, $r\in\mathbb{Q}\Rightarrow L(r):=\{ x\in\mathbb{Q}\mid x<r \}$is a Dedekind's cut. が成り立つ.

    2.\, 実数$L, L(r)\,(r\in\mathbb{Q})$について,次の3条件は同値である.


    (1)\, $r\in L$

    (2)\, $L(r)< L$

    (3)\, $L(r)\ngeq L$
\end{proposition}
\begin{proof}\rm{}
    $1.$少なくとも$r-1\in L$であり,また$r\notin L$より,条件(1)を満たす.有理数体上の順序関係の推移性より,(2)も成り立つ.$x\in L$を勝手に取った時,$x<\frac{x+r}{2}<r$となる$\frac{x+r}{2}\in L(r)$が作れるから,(3)も成り立つ.

    $2.\,(1)\Rightarrow(2)$.$r\notin L(r)$より$L(r)\neq L$であるが,勝手に取った$x\in L(r)$について,$x<r$だから$r\in L$と併せて$x\in L$が従う.従って,$L(r)\subsetneq L$である.

    $(2)\Rightarrow(3)$.$L(r)\subsetneq L$とは$L\setminus L(r)\neq\varnothing$ということであるから,$L(r)\nsupseteq L$が従う.

    $(3)\Rightarrow(1)$.$L(r)\ngeq L$の時,$L\setminus L(r)\neq\varnothing$より,$x\in L\setminus L(r)$が取れる.$x\notin L \Leftrightarrow r<x$であるが,$x\in L$より,条件(2)から$r\in L$を得る.
\end{proof}
\begin{remark}\rm{}
    $L$がDedekind's cutならば,$\exists r\in\mathbb{Q}\,\mathrm{s.t.}\, L=L(r)$とはならない点が,実数が有理数の拡張になっている点である.
    つまり,写像$\mathbb{Q}\ni r\mapsto L(r)\in\mathbb{R}$は単射である.系\ref{corollary-order-of-the-real-numbers}より,この写像は順序構造を保つから,この埋め込みによって$\mathbb{Q}$を$\mathbb{R}$の部分集合として同一視する.
    これを,順序の言葉で指定した抽象的な集合の集まりとして実数を定義することによって成し遂げているのが魔法使いみたいだ.

    また,この命題は(1)をよく抽出したと思う.$x\in M$となった途端に,$x$より大きいが$M$には含まれる有理数は無限個存在するから,$L(x)<M$なのである.これはそのまま有理数の実数上での稠密性の翻訳になっているのだな.
\end{remark}

\begin{corollary}[実数体の順序]\rm{}\label{corollary-order-of-the-real-numbers} 

    1.\, $r,s\in\mathbb{Q}$について,$r<s$と$L(r)<L(s)$とは同値である.
    
    2.\, $L,M\in\mathbb{R}$について,$L\leq M, M\leq L$のいずれかが成り立つ.また,次が成り立つ.
    \[ \forall L,M\in\mathbb{R} \, L<M \Longrightarrow \exists s\in\mathbb{Q} (L<L(s)<M) \]
\end{corollary}
\begin{proof}\rm{}
    $1.$命題\ref{prop-order-of-the-real-numbers}より,各$L(r)<L(s)\Leftrightarrow r\in L(s)\Leftrightarrow r<s$.

    $2.$命題\ref{prop-order-of-the-real-numbers}の$(2)\Leftrightarrow(3)$より,$L<M$または$L\ngeq M$である.従って,$L\le M$または$L\ge M$である.
    今,$L<M:\Leftrightarrow L\subsetneq M$とすると,勝手に取った$x\in M\setminus L$に対して,$x\in M$より$L(x)<M$が,$x\notin L$より$L(x)\nless L$即ち$L(x)\ge L$が,命題\ref{prop-order-of-the-real-numbers}より従い,$L\le L(x)<M$が成り立つ.
    今,$M$について条件(3)を用いて,$x<s$を満たす$s\in M$を取り直すことにより,再び命題\ref{prop-order-of-the-real-numbers}から,$L\le L(x)<L(s)<M$が成り立つ.
\end{proof}

有理数体の順序を引き継ぐ埋め込みによって実数を定義したが,その演算の構造は,次のように定義することで,集合として構成された有理数$\mathbb{Q}$から自然に,というより無意識的に?引き継ぐ.
\begin{definition}[実数の演算]\label{def-algebraic-structure-of-real-numbers}
    
    $L,M\in\mathbb{R}$とする.和を次のように定義する.
    \[ L+M = \{ x+y\in\mathbb{Q}\mid x\in L, y\in M \} \]
    $L$の加法逆元を,$L':=\{x\in\mathbb{Q}\mid \forall y\in L (x+y<0)\}$を用いて,次のように定義する.
    \[ -L :=\{x\in\mathbb{Q}\mid \exists y\in L'(x<y)\} \]
    これは確かに切断になっており,$L+(-L)=(-L)+L=L(0)$を満たす.また,$L\ge L(0)\Leftrightarrow -L\le L(0)$となる.

    積を次のように定義する.
    \[ L\cdot M=(-L)\cdot (-M) := \{ x\in\mathbb{Q}\mid \exists y\in L, z\in M \,\mathrm{s.t.}\, y>0, z>0, x<yz \} \]
    $(-L)\cdot M=L\cdot (-M)=:-(L\cdot M)$と定める.$L=L(0)$または$M=L(0)$である場合は,$LM=L(0)$と約束する.
\end{definition}
\begin{remark}\rm{}
    $L,M$が有理数と同一視出来る場合について議論すると様子が掴みやすい.$s\in\mathbb{Q}$として$L=L(s)$である場合h,$L'=L(-s)\cup\{-s\}$である.これに対して,最大元を省いた集合を$-L$と定義している.

    この定義がうまくいくのは全てDedekind's cutの定義が絶妙なのである.$L+M=L(r)+L(s)$と表される場合は退化していて分かりにくいが,$L,M=L(s)$となる$s\in\mathbb{Q}$が見つからない場合でもこの定義は整合的にできている.
    つまりは,$L=L(x)(x\in\mathbb{R})$を,実数を一切登場させることなく,順序関係$<$のことばだけで指定可能であるということを言っている.これに成功している時点で,数や距離以外の情報/ことばの体系が実数には含まれていることが予感される.
    これらのことばの自然言語への翻訳の一部が,「上界」「上限」として用意されている.
\end{remark}
\begin{proposition}[実数体]
    定義\ref{def-relationship-between-equivalence-relation}による実数の演算について,$\mathbb{Q}$に引き続き体となっている.
\end{proposition}

\begin{proposition}[上限の特徴付け]
    $\varnothing\ne A\subset\mathbb{R}$とする.実数$S\in\mathbb{R}$に対して,次の3つの条件は同値である.\rm{}

    1.\, $S$は$A$の上限である.$S=\min\{ x\in\mathbb{R}\mid \forall a\in A\,a\le x \}$

    2.\, $\forall L\in A \;(L\le S)$かつ$\forall T<S \; \exists L\in A\; (T<L)$
\end{proposition}
\begin{proof}
    上界$S$が上限であるとは,$S$が上界のうち最小のものであるということである.即ち,$T<S$を満たす全ての$T\in\mathbb{R}$は上界ではない,つまり,$\exists L\in A\; (T<L)$.
\end{proof}

\begin{theorem}[実数の連続性]
    $A\subset\mathbb{R}$とする.$A\ne\varnothing$かつ上に有界ならば,$A$の上限が存在する.
\end{theorem}
\begin{proof}
    $S:=\bigcup_{L\in A}L\subset\mathbb{Q}$と構成すれば,これは確かに切断となっており,$A$の上限に他ならないことを示す.

    (1)仮定$A\ne\varnothing$より,切断$L\in A$が存在するから,$\varnothing\subsetneq L\subset S$.また$A$は上に有界だから,切断$M\in\mathbb{R}$が存在して$A\subset M\subsetneq\mathbb{Q}$.
    
    (2),(3)$x\in S$を任意に取ると,或る切断$L\in A$が存在して$x\in L$である.従って,$\forall y<x\; y\in L$かつ$\exists z>x z\in L$である.よって,$\forall y<x\; y\in S$かつ$\exists z>x z\in S$であり,確かに$S$も切断.

    $S$が求める上限であることを示す.$\forall a\in A\; a\le S$は,$S$の定義上任意の$a\in A$について$a\subset S$であることから従う.
    また,既に示した$A\subset M\subsetneq\mathbb{Q}$より,$S$は上界のうち最小のものであることが分かる.
\end{proof}
\begin{remark}
    なんだよ,上限の特徴付けの方を使うわけではないのか,と思ったが,その試みの中で,切断において$L(x)\subsetneq \cup A$と,$L(x)\in A$は同値だと気付いた.これは自然数の定義と,切片の議論と,似ている.
\end{remark}

以降$[-\infty,\infty]:=\mathbb{R}\coprod\{-\infty,\infty\}\;(-\infty\ne\infty)$という記法を用いると,この範囲で実数の部分集合は必ず上限を持つ.
ただし,$A\subset [-\infty,\infty]$について,$\sup A=-\infty \; (A\subset \{-\infty\}の時)$とする.
写像$f:X\to\mathbb{R}\coprod\{-\infty,\infty\}\;(-\infty\ne\infty)$についても同様に定める.
$\sup f(X)=:\sup_{x\in X}f(x)$とも書く,あるいは同値な条件を下に添えて書く.

\begin{definition}[実数列の収束]
    $(x_n)\in{}^{<\omega}\mathbb{R},a\in\mathbb{R}$とする.\textbf{数列$(x_n)$が$a$に収束する}とは,次が成り立つことである.
    \[ \inf_{m\ge 0}\left(\sup_{n\ge m}|x_n-a|\right)=0 \]
    この関係を$\lim_{n\to\infty}x_n=a$と書き,$a$を極限という.

    数列$(x_n)$が有界であるという時には,その値域が上に有界であることを言う.
\end{definition}
\begin{remark}
    $\sup_{n\ge m}|x_n-a|$とは,$m$番目以降の項の,$a$からの距離の振れ幅の範囲が,この中に収まることを意味する.数列が収束するとは,$m$を十分大きく取ることで,その範囲をいくらでも小さくする/$0$に近づけることが出来ることを意味する.

    この定義なら,絶対値の構造を備える距離空間一般について拡張できそうである.
\end{remark}

\begin{proposition}[実数上の$\epsilon-\delta$論法]\rm{}$(x_n)$を十数列とする.
    
    1.\, 次の2条件は同値である.

    (1)\, $\lim_{n\to\infty}x_n=a$

    (2)\, $\forall r\in\mathbb{R}_{>0}\; \exists m\in\mathbb{N} \; \forall n\in\mathbb{N}_{n\ge m} \; |x_n-a|<r$

    2.\, 収束する数列$(x_n)$は有界である.

    3.\, $(x_n)$は有界かつ単調増加であるとする.$s=\sup_{x\ge n}x_n$とすれば,$\lim_{n\to\infty}x_n=s$である.
\end{proposition}
\begin{remark}\rm{}
    2.の逆はそのままでは成り立たないが,Bolzano-Weierstrassの定理が成り立つ.

    3.は実数の連続性の特徴付けとなる.
\end{remark}

\section{Euclid空間上の開集合}
以降$n\in\mathbb{N}$として,$n$次元Euclid空間$\mathbb{R}^n$の位相を考えるための言葉を整備する.
距離の概念から位相の言葉を定義し,そのうち位相の概念の一般化の足掛かりとなるような性質をみる.

\begin{definition}[内積と長さ]
    $x=(x_1,\cdots,x_n),y=(y_1,\cdots,y_n)\in\mathbb{R}^n$に対し,$\langle x,y\rangle =x_1y_1+\cdots +x_ny_n\in\mathbb{R}$と定めた内積を,\textbf{標準内積}といい,$||x||:=\sqrt{\langle x,x\rangle}$を\textbf{$x$の長さ}という.
\end{definition}
\begin{definition}[距離]\rm{}
    $x=(x_1,\cdots,x_n),y=(y_1,\cdots,y_n)\in\mathbb{R}^n$に対し,
    \[ d(x,y)=\sqrt{(x_1-y_1)^2+\cdots +(x_n-y_n)^2} \]
    を,2点$x,y$間の距離という.この,2点の距離をベクトル$x-y$の長さ$d(x,y)=||x-y||$によって定めた距離を備えた系$(\mathbb{R}^n,d)$をEuclid空間という.
\end{definition}

\begin{proposition}[距離の性質]\rm{}$x,y,z\in\mathbb{R}^n$に対して,次の3つが成り立つ.

    1(non-negativity,identity of indicernibles).\, $d(x,y)\ge 0$で,等号成立条件は$x=y$である.

    2(symmetricity).\, $d(x,y)=d(y,x)$

    3(subadditivity).\, $d(x,z)\le d(x,y)+d(y,z)$
\end{proposition}

開区間の定義を$n$次元に拡張すると,球という概念が表面化する.「端点」と呼べる部分が一気に無限個になる.
\begin{definition}[開集合]\rm{}

    1.\, $p\in\mathbb{R}^n, \delta\in\mathbb{R}_{>0}$とする.
    $$B_\delta (p)=\{ q\in\mathbb{R}^n \mid d(q,p)<\delta \} $$
    を開球という.
    
    2.\, $U\subset\mathbb{R}^n$が$\mathbb{R}^n$-開集合であるとは,次の論理式を満たすことである.
    $$\forall p\in U,\, \exists\delta >0,\, B_\delta (p)\subset U$$

    3.\, $A\subset\mathbb{R}^n$の補集合$\mathbb{R}^n\setminus A$が$\mathbb{R}^n$-開集合である時,$A$は$\mathbb{R}^n$-閉集合であるという.集合$\{d(x,y)\in\mathbb{R}\mid x,y\in A\}$が有界である時,$A$は有界であるという.
\end{definition}

\begin{example}\rm{}

    1.\, $\varnothing,\mathbb{R}^n$はいずれも,開集合かつ閉集合である.前者は自明な形で,後者は普通に開集合の定義を満たし,2つは互いに補集合であるからである.

    2.\, $a\in\mathbb{R}^n$とする.$\mathbb{R}^n\setminus \{a\}(n=0,1,2,\cdots)$は開集合なので,$\{a\}$は閉集合である.また,$n>0$の時,$a$を中心とした$\{a\}$に含まれる開球は存在しないので,$\{a\}$は開集合でない.$n=0$の時は,$\mathbb{R}^0=\{id_0\}\simeq 1$となり,全ての部分集合が開集合でもあり,閉集合でもある.

    3.\, $\ge m<n$とし,$\mathbb{R}^m$を$\mathbb{R}^n$の部分集合$\{(x_1,\cdots,x_m,0,\cdots,0)\mid (x_1,\cdots,x_m)\in\mathbb{R}^m\}$と同一視すると,$\mathbb{R}^n\setminus\mathbb{R}^m$はあいも変わらず開集合より$\mathbb{R}^m$は$\mathbb{R}^n$-閉集合であるが,$\mathbb{R}^m$は,全く行けない次元が$n-m$次元あるので,$\mathbb{R}^n$上の開球を中に含めることは出来ず,開集合ではない.
\end{example}

\begin{proposition}
    $\varnothing,\mathbb{R}^n$以外に,$\mathbb{R}^n$の部分集合であって,開集合でも閉集合でもあるものは存在しない.
\end{proposition}
\begin{proof}
    $\mathbb{R}^n$が連結であるかららしいが,何故かは分からない.
\end{proof}

\begin{proposition}[開集合の特徴付け1]\rm{}
    $U\subset\mathbb{R}^n$とする.次の2条件は同値である.

    1.\, $U$は開集合である.

    2.\, $U$は開球の族の和集合である.
    
\end{proposition}
\begin{proof}
    1$\to$2を示す.$U$は開集合だから,全ての点$p$について,対応する開球$B_{\delta_p}(p)$が存在し,$B_{\delta_p}(p)\subset U$を満たす.
    従って,$U':=\bigcup_{p\in U}B_{\delta_p}(p)$とすれば,即座に$U'\subset U$である.また,$p\in U$に対して$p\in B_{\delta_p}(p)\subset U$だったのだから,$p\in U'$であるため,$U'\supset U$でもある.従って,$U=U'=\bigcup_{p\in U}B_{\delta_p}(p)$を得る.

    2$\to$1を示す.開集合の族$U:=\bigcup_{\lambda\in\Lambda}B_\lambda$を考える.勝手に取った点$p\in U$に対して,対応する$\lambda\in\Lambda$と開球$B_\lambda$が存在して,$p\in B_\lambda$を満たす.仮に$B_\lambda=B_{\delta}(q)$だったとすると,$r=\delta -||p-q||$として,$B_{r}(p)$は,$B_{r}(p)\subset B_\lambda\subset U$を満たす.
    こうして各点$p$に対して,それを中心として$U$に含まれる開球が存在するから,この$U:=\bigcup_{\lambda\in\Lambda}B_\lambda$は開集合である.
\end{proof}

主にこちらが,一般の集合についても位相の言葉を考えるにあたって,基点となる.
\begin{proposition}[開集合の特徴付け2]\rm{}

    1.\, $(U_i)_{i\in I}$が$\mathbb{R}^n$-開集合の族であるならば,合併$\bigcup_{i\in I}U_i$も$\mathbb{R}^n$-開集合である.

    2.\, $(U_i)_{i\in I}$が$\mathbb{R}^n$-開集合の有限族であるならば,共通部分$\bigcap_{i\in I}U_i$も$\mathbb{R}^n$-開集合である.
\end{proposition}
\begin{remark}
    $A$を一般に$\mathbb{R}^n$の部分集合とすると,無限集合族を用いて$\bigcap_{x\in\mathbb{R}^n\setminus A}\mathbb{R}^n\setminus\{x\}$として$A$が表現出来てしまう.すごい怖い.
\end{remark}

\begin{definition}[数列の収束の点列への拡張]
    $(x_m)\in{}^{<\omega}\mathbb{R}^n,a\in\mathbb{R}^n$とする.\textbf{点列$(x_m)$が$a$に収束する}とは,次が成り立つことである.
    \[ \lim_{m\to\infty}d(x_m,a)=0 \]
    この関係を$\lim_{m\to\infty}x_m=a$と書き,$a$を極限という.

    点列$(x_m)$が有界であるという時には,その値域が$\mathbb{R}^n$の有界な部分集合であることを言う.

\end{definition}
\begin{remark}
    2点の間の距離という実数値関数を利用して,実数列の収束から点列の収束を定めた.
\end{remark}

\begin{proposition}[点列の収束の位相的特徴付け]\rm{}
    $(x_m)\in{}^{<\omega}\mathbb{R}^n, a\in\mathbb{R}^n$とする.次の3条件は同値である.

    1.\, $\lim_{m\to\infty}x_m=a$

    2.\, $\forall r\in\mathbb{R}_{>0}\;\exists l\in\mathbb{N}\;\forall m\in\mathbb{N}_{\ge n}\; d(x_m,a)<r$

    3.\, $a$を元として含む任意の開集合$U\subset\mathbb{R}^n$について,$\{m\in\mathbb{N}\mid x_m\notin U\}$は有限集合である.

    条件3.を「十分大きな$n$について$x_m\in U (m\ge n)$である,ということがある.
\end{proposition}
\begin{remark}
    点列が収束することを開集合のことば
    によって純粋に表現することに成功したわけであるが,閉集合のことばだとどうなるのであろうか?
\end{remark}

\section{連続写像}



\section{}

\end{document}