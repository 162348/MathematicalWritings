\documentclass[uplatex,dvipdfmx]{jsarticle}
\title{現代物理学レポート(担当:立川裕二先生)\\超分子の化学と相分離の生物学}
\author{司馬博文 J4-190549}
\pagestyle{plain} \setcounter{secnumdepth}{4}
\input{/Users/hirofumi.shiba48/Desktop/数理科学/preamble_CM.tex}
\begin{document}
\maketitle
\begin{abstract}
    日本の昔の物理学者木原太郎(1917-2001)は「分子間力」を基調とした自然観を軸に研究活動・教育活動を行い,
    核と電子の相互作用による現象である原子から,原子同士の相互作用である分子,果てには分子間相互作用で組み立てられている
    タンパク質やDNAなどの生体機構から遺伝子までを「分子間力」の視点から統一的に扱った
    著作『原子・分子・遺伝子』\cite{木原太郎-遺伝子}を残しているが,この作品には些か時代が追い付いていなかったように思える.
    
    現在,種々の分子間力を理論的に調べる学問分野は超分子化学と呼ばれており,
    様々な強度や性質を持った分子間の相互作用の織り成す
    複雑なネットワークが少しずつ解明されつつあり,またそれを応用して新たな人工超分子の作成なども進んでいる.
    また,分子間力の理論的応用としては,分子生物学と細胞生物学の橋渡しに当たる新分野として相分離生物学という分野が,
    これまでの科学研究の蓄積の上に,今現在,確立しつつある.
    この分野は作品\cite{木原太郎-遺伝子}の先を行き,生命現象の理解へ向けた大きなparadigm shiftを
    含むと筆者は理解している.
    
    このレポートでは,著作\cite{木原太郎-遺伝子}の分子間力を基調とした世界観の現代的な継承と言える超分子化学のトピックを
    いくつかの例を軸として紹介し,その見方がどのように(相分離)生物学に活かされているか,
    そして相分離生物学のものの見方の何処が革新的であるかを解説する.
\end{abstract}
\tableofcontents

\section{Introduction}

今回の「現代物理学」の講義(の前半)では,量子論の枠組みとqubit系が取り上げられた.
理科類に所属する駒場生としては,必修科目「構造化学」「物性化学」で量子化学の基本的
議論(分子軌道法による化学結合の説明など)を学んだが,その際には触れられなかったより基本的な枠組みについての
数理が学べたので,理解が地続きになったのが非常に気持ちが良かった.
地続きな学問分野で言えば,駒場生には幅広い学びの機会があるもので,いわば量子論を基礎付ける立ち位置にある
と言える「線形代数学」も必修として,
また構造化学・物性化学の理論の応用に当たる分野として「超分子化学」の授業が総合科目として学べる.

文献\cite{木原太郎-遺伝子}は,日本の物理学者が,
「分子間の力」をモチーフとして,原子から分子を通じてタンパク質と遺伝子までの
「ミクロな自然の科学」を統一的に扱った書籍である.
筆者はその原子に関する量子力学的議論にも,ウイルスの自己集合に関する超分子化学的な議論にも
妥協やごまかしのない,一つのテーマからの通分野的な理解の試みに心酔したのだった.
この試みを模倣すべく,対応する内容を粗方学びきったのがこの2Sセメスターということで,
その内容と,それに加えてこのスペクトラムの先に位置付けられる,
超分子化学と分子生物学の橋渡しに当たる新興分野「相分離生物学」の内容のうちいくつかを,
主に文献\cite{平岡秀一}や\cite{白木賢太郎}を学習した経験を基にまとめる.

\section{超分子化学とは如何なる分野か}

原子間に形成される化学結合の代表といえば,共有結合である.
共有結合は,第一義的には「電子対の共有」が作る結合と理解され,
また量子化学の見地からは,原子軌道同士の重なり合いによる安定化の機構
として理解出来た.共有結合は非常に強い結合であるため,
これらで結ばれた原子はひとまとまりとして考えて良い.
これを\textbf{分子}という.

こうして共有結合を抽象化してしまえば,より弱いが多様な力の働きが見えてくる.
これらを総称して「分子間力」という.分子間力の正体は基本的には電磁気的な相互作用であるが,
その発生機序は分子に依って異なり,水素結合やvan der Waals力など様々な名前がついて分類・研究されている.
分子間力により集合した安定な分子の集合体を\textbf{超分子}という.
生体物質からの例としては,DNAの二重螺旋構造や複雑な折り畳み構造を持ったポリペプチド鎖が集まって出来ているタンパク質,
酵素と基質の相互作用や生体膜などが,超分子として研究されている.
分子間力の強さの測定や力の発生のメカニズムの理論的解明,
また新たな超分子のデザインと実用を研究する化学分野を\textbf{超分子化学}
という.

また,共有結合を抽象化して分子間力に注目すると,溶媒と溶質の対称的な関係が見えてくる.
例えば,溶質同士の分子間相互作用が強ければ,これはもちろん溶媒内で自己集合するであろう.
一方で,溶媒同士の分子間相互作用が十分に強い場合も,溶質はこれらに排斥される形で,
溶媒内で自己集合する.これを\textbf{疎溶媒効果}という.
水は特にこの効果の強い溶媒であって,特にこの場合を\textbf{疎水効果}という.
これらの現象は溶液の熱力学の枠組みで,エントロピーやエンタルピー
または自由エネルギーなどの熱力学的パラメータを用いて定量的に議論される.
このような議論が超分子化学である.

\section{自己集合現象とは何か}

自己集合は,2分子からなる自己集合現象(\textbf{分子認識}という)と
その間の\textbf{協同性(cooperativity)}という2つの要素から理解される.
協同性には大きく分けて2種類存在する.分子が事前に結合している場合は,
集合するにあたってのエントロピー低下が事前に抑えられるために集合し易くなる
という\textbf{chelate協同性}と,複数の結合が形成される多段階反応において,1つ目の結合が
conformationの変化などを引き起こすために間接的に
2つ目の結合の形成を促進するという\textbf{allosteric協同性}である.

分子認識の代表例に,cyclodextrinがある.
まずamyloseとは,glucoseの1位と4位の炭素が
glycoside結合によって連結した直鎖状高分子である.
これはamylopectinと違って熱水に溶解する.
これを酵素が分解すると,グルコースが環状に繋がった
環状オリゴ糖であるcyclodextrinを得る.
cyclodextrinは外側に並んだ-OH基のために外側は親水的で,内側は疎水的なので,
水に疎水分子と共に溶解させると,疎水効果により
その疎水分子はcyclodextrinの内部に包摂される.

これが,各種の医薬品にて薬用成分の安定化剤としてクリームの成分に,
あるいは苦い薬用成分をシクロデキストリンに包摂する服薬補助成分として,
またはわさびの辛味成分を閉じ込めて保存性を高めた商品「練りわさび」\footnote{わさびの辛みは揮発成分であるため,チューブ形で保存を想定されている商品ではシクロデキストリンに包んでおく.},
消臭・除菌ブームの火付け役である「ファブリーズ」,
口臭予防タブレットなどに利用されている\cite{有賀克彦}.
\begin{quote}
    「成分表に,シクロデキストリンあるいはサイクロデキストリンと記載されている市販品以外でも,
    環状オリゴ糖やとうもろこし成分などと書いてあればシクロデキストリンを用いている可能性が高い.\cite{有賀克彦}
\end{quote}

また,東京大学大学院総合文化研究科広域科学専攻の寺尾潤先生は,シクロデキストリンを用いた
応用的な研究を行っている.シクロデキストリンをnmスケールの半導体ワイヤやグラフェンナノリボンなどの
極細電荷チャネル界面に組み込み,配位能を有するポルフィリンなどの遷移金属錯体と組み合わせることで,
高次の分子認識を発現する超分子・錯体センサを導入することが研究例の1つとして挙げられている\footnote{寺尾研究室HP http://park.itc.u-tokyo.ac.jp/terao/}.
なお,ポルフィリンとはピロールという窒素原子を含む五員環化合物が4つ対称に集まった形の環状化合物であり,
分子全体に$\pi$-共役系が広がっているため,そこで発生する誘起双極子による分子間相互作用(London分散力という)
が超分子化合物を作る原動力となり易く,超分子化学の顔の1つに挙げられる化合物である.
実際,ヘモグロビンが酸素を捕まえる際に用いるFe(II)イオンは,
ヘモグロビンのポルフィリン環の$\pi$-共役系に存在する.しかし,ポルフィリン環の分子骨格には収まらずに,
嵌った状態でいる(この状態をヘムのdoming状態という).ここに酸素分子が結合すると,Fe(II)イオンはFe(III)イオンに
酸化され,イオン半径は小さくなるために,ポルフィリン環の平面に収まることが出来,(擬)正八面体構造をなす\footnote{六配位はN原子5つとO原子1つからなるため,正確に正八面体構造にはならない}.
この構造変化が,ヘモグロビン内に存在する結合部位4つの間に,allosteric協同性を起こす.
即ち,ヘモグロビンでは「酸素が1つでも結合すると,ヘモグロビンのサブユニット間の構造変化
によって次の酸素が結合し易くなる」という協同性が存在するが,
これにはヘム間相互作用という名前がついている.
これが,ヘモグロビンの酸素運搬効率の高さに大きく寄与している.
この協同性の存在により,基質濃度に対する飽和曲線はsigmoidalになるので,酸素濃度が足りている居るか居ないかに
鋭敏に反応するスイッチにような役割を果たすのだ.
なお,酸素の貯蔵が主な役割であるミオグロビンには,結合部位は1つしか存在しない.

%\section{人工の超分子}

\section{タンパク質の自己集合について}

超分子化学の主要な題材の一つに,タンパク質の折り畳み構造(これを\textbf{folding}という)の研究がある.
タンパク質はDNA上の遺伝情報から翻訳されたポリペプチド鎖が,折り畳まれて出来たものであるが,
これがfoldingを形成する際の機構は分子間力に因る.
先ほどのヘモグロビン内のallosteric協同効果と同様のもの\footnote{タンパク質内で水素結合が生じると,amido結合の繰り返し構造からなるタンパク質は結合部位のconformationが変化するので,次の水素結合はできやすくなる}がタンパク質のfoldingでも起こり,
また多くの様々な要因やシャペロン(chaperone)と呼ばれる「他のタンパク質のfoldingを手助けするタンパク質」により,
ドミノ倒しの様に一気にタンパク質は想定された三次構造(これをタンパク質の\textbf{天然状態(natural state)}という)を完成させる.
一方で,何らかの理由でタンパク質がこの本来の天然状態から変性し,
線維状($\beta$-sheet構造)になると,自己集合する.
これを\textbf{アミロイド(amyloid)}といい,これらタンパク質のなす四次構造をcross-$\beta$構造と呼ぶ.
まず$\beta$-sheet構造とは,$\beta$-strandという3〜10ほどの長さのポリペプチド鎖が,
3本ほど互いに水素結合を形成して出来るねじれ\footnote{非常に緩やかな右回りの螺旋になる}
やひだのあるシート構造であり,数々のタンパク質構造に典型的なものである.
$\beta$-sheet構造を持った線維状のタンパク質同士は,飛び出たひだの部分などに残るイオン性の残基同士の
分子間相互作用により,残酷なほど安定な超分子を形成する(これがcross-$\beta$構造である).
この超分子的な構造をタンパク質の四次構造と呼ぶ\footnote{タンパク質の一次構造とは共有結合,二次構造はアミノ酸主鎖の折り畳み方,三次構造がfoldingを形成する要因であるアミノ鎖同士の相互作用である.}.
なお,cross-$\beta$構造の名前に含まれている語crossとは,X線回折の際に特徴的に見られる直行する回折縞から名付けられたものである.

この変性したタンパク質の自己集合現象は,直感の通り,病気の原因となり得る.
器官内へのアミロイドの過剰な蓄積(amyloidosisと呼ばれる)が,
アルツハイマー症や筋萎縮性側索硬化症など様々な神経変性疾患の原因であるとされている.

しかし,実はこの変性タンパク質は,タンパク質の天然状態内にも存在し,
この「天然変性タンパク質」の自己集合は,生体内の正常な機構としても起こる.

その名は\textbf{プリオン(prion)}である.
プリオンとは,何かしらの理由により,正常なタンパク質が異常化することを媒介する機能を持った,即ち
「増殖性」「感染性」のあるタンパク質のことであり,結果として本来のタンパク質の機能を不全に陥れ,
アミロイドの蓄積を誘導するものもある.1990年代に流行した狂牛病などが,プリオン症の例として有名である.
これは病死した牛の肉骨粉を別の牛の飼料に混ぜて居たために,極微量含まれていた異常型プリオン
を食べて感染が爆発した事例である.
このように,プリオンは人間だけでなく牛や,さらには酵母にも存在し,
全生物に普遍的に見られるものである.
従って,プリオンの存在には進化的な理由,生物の存続に欠かせない決定的な役割があるはずである.
これに対して一つの明晰な説明を提出するに至ったのが,相分離生物学である.

\section{相分離生物学のものの見方}

超分子化学は,共有結合よりも弱い分子間力の描き出す複雑な化学反応の様相を理解する学問であった.
そこでは,溶質同士の相互作用を考えるだけでなく,溶媒との相互作用と溶媒同士の相互作用も無視できない
ということが,基本的な結果なのであった.

生体内には,「膜のない細胞小器官(オルガネラ)」が存在する.
1830年頃には,光学顕微鏡を用いてアフリカツメガエルの卵細胞の核小体には膜が無いことは知られていた.
が,当時はこの事実を適切に評価するには十分な準備が整っていなかったようだ.
2009年にこの事実が次のような形で再発見された.研究\cite{Clifford Brangwynne Anthony Hyman 09 Science}では,
線虫の卵細胞にはP顆粒(生殖顆粒)と呼ばれる独自の構造物があり,
卵割が進む前に片側に集まることが着目された.
このメカニズムを,天然変性タンパク質の蛍光化によって可視化したところ,
初め細胞内に分散していたが次第に集合・融合して成長するという「相分離」が見られたという.

細胞内でのこのような液-液相分離(LLPS: Liquid-liquid phase separation)は普遍的で,
核小体やストレス顆粒,さらには一時的に出来る小さなものも含めて,\textbf{膜のないオルガネラ(membrane-less organelles)}と呼び,
また細胞核の中に存在する分子密度の高い領域を,\textbf{液滴(droplet),condensate, coacervate}などという.
まだ定まった呼び名は存在しないようだ.
また,dropletの形成にタンパク質が関わる場合,これを\textbf{タンパク質の五次構造}として数える見方も提案されている.
タンパク質の四次構造とはタンパク質間の分子間相互作用であったが,
この液滴の形成はそれよりも弱い力によって生じ,環境の変化に応答して集合し,その信号が消えると解離するものである.
しかしながら,このdropletの形成の仕方とそれを誘発するタンパク質間の弱い相互作用の種類は,
分子進化の過程で保存されて来た定まった機構であることが示唆されるためである\cite{Molecular Evolution}.

このdropletは,天然変性タンパク質やRNAのようなポリイオンが,
静電相互作用やカチオン-$\pi$相互作用,$\pi$-$\pi$相互作用,短いクロス$\beta$構造などで安定化したもので,
1つの広義の超分子とみなせる.
生体機能を,タンパク質などの生体分子一つ一つが演じる機構であるというよりもむしろ,
この(生物学的)相分離によるdropletの動的な形成を機能の単位として注目する観点から捉え直す試みが\textbf{相分離生物学}
である.顕微鏡で見える実体として人類に最初に発見されたものが一つ一つの生体分子であったが,それらはいわば
dropletが動的に形成されたり消滅したりすることで提供される「場」という台本に支配される演者でしかなく,
生体機能の担い手としてはdropletからの方が圧倒的に見通し良く理解できるというパラダイムシフトがこの分野
の本質である.

\section{dropletから理解する,プリオン(様)タンパク質の機能}

さて,プリオンもdroplet形成の観点からその本来の機能を理解できる.
プリオンの研究では,特に哺乳類への感染など,取り扱うに当たって危険がない出芽酵母が
モデル生物として用いられる.出芽酵母が持つプリオンSup35は685個のアミノ酸からなる翻訳を終結させる働きを持ったタンパク質で,3つのドメインを持つ.
C末端側には,実際に翻訳を止める機能ドメインがある.N末端側のNドメイン,
その中央のMドメインは構造を持たない天然変性領域を持つ.
この2つのドメインがアミロイドを形成して,その機構が他の細胞へ伝播し\footnote{異常構造を持つタンパク質が細胞を超えて伝わり,別の細胞でも異常構造を生み出す現象を\textbf{伝播}という.},感染性因子プリオンとして振る舞う.
また,Sup35の機能ドメインのみを残しても酵母は生き続けるので,N,Mドメインは直接は生存には関与しない.

研究\cite{Yeast Prion}では,Sup35に緑色蛍光タンパク質(GFP)を結合させ,
出芽酵母の中でSup35がどの様に存在しているかを観察したところ,
通常の条件で生育させると細胞の中に広く遍く存在していた.
一方でグルコースを与えずに飢餓条件で生育させるとSup35がサブミクロン程度の大きさのdropletを形成した.
また,グルコースを枯渇させると細胞内のpHが弱酸性へ傾くことが知られていたので,
これが相分離の引き金になっていることを確認した.
即ち,弱酸性ではSup35は丸いゲル状の構造を形成し,pHを中性に戻すとゲルは再び溶解した.
なお,イオン強度を数百mMにするとSup35はゲル化しなかったため,
電気的な相互作用がゲル化の駆動力となっていると示唆される.
ここでもちろん,Sup35の機能ドメインのみのタンパク質を作成して調べたところゲル化せず,
これは弱酸性条件で不可逆に変性して凝集し,中性に戻しても機能は回復されなかった.
また,数億年前に分種したと考えられる出芽酵母Saccharomyces cerevisiaeと
分裂酵母Schizosaccharomyces pombeのいずれでも確認されたため,
Sup35のN,Mドメインは元々進化的にゲル化することで機能ドメインを守る働きを担っていたと考えられる.

\begin{quote}
    酵母のSup35は翻訳を終結させる働きを持ったプリオンだが,天然変性ドメインを持っている.
    このドメインによって,環境からのストレスに応答してdropletを形成でき,
    タンパク質の不可逆な失活を防いでいたのである.この働きがプリオンの本来の働きであり,
    疾患を引き起こす性質はいわば副作用だったのである.このようなメカニズムを考えると,
    これから潜在的なプリオンとしてのタンパク質がたくさん発見されていくだろう.\cite{白木賢太郎}
\end{quote}

このSup35のように,一般に固有の立体構造を持たない長い領域を持つタンパク質を\textbf{天然変性タンパク質}または
\textbf{プリオン様タンパク質}という.
天然変性タンパク質の固有の立体構造を持たない領域には,
類似した配列が繰り返し現れる\textbf{低複雑性ドメイン}または\textbf{プリオン様ドメイン}と呼ばれる領域が3箇所ほどある.
感染性を持つプリオンに限らず,このプリオン様タンパク質一般の機能についての研究の例として,次がある.

DNAは非常に長いので,クロマチンと呼ばれるタンパク質に約1.65回巻きつけられた構造
(ヌクレオソームという)が数珠繋ぎのようになった構造(これをクロマトソームという)
で保管されており,そのうち転写が活発でない部分は,葡萄の房の様な形で密集した形でコンパクトに保存されている.
この状態であるクロマチンを,ヘテロクロマチン(heterochromatin)という.反対に転写が活発で,DNAが
すぐに利用可能なようにクロマチン同士が疎になっている領域はユークロマチン(euchromatin)という.
この2つの形態は温度に依存して相互に移行するが,
そのメカニズムはクロマチンタンパク質という「溶質」だけに注目していても見えてこない.

ショウジョウバエのheterochromatin protein 1 (HP1)は3種類ある.
そのうちの1つHP1$\alpha$は206個のアミノ酸からなる天然変性タンパク質である.
これは高濃度かつ22$^\circ\mathrm{C}$環境下でdropletを形成する\cite{HP1a}.
このdropletはイオン濃度が高いと形成されないため,静電相互作用(塩橋)や水素結合によるものであろうと考えられる.
さらにHP1aをGFP(Green Fluorescent Protein)と結合させると細胞内にぶつぶつが発現したので,
確かにこのプリオン様タンパク質は液滴を形成することが分かる.
このHP1aは次の様にしてheterochromatinの不活化機構を担っていると考えられる.
HP1aは,ヒストンH3の9番目のリシンにメチル基が2,3個修飾された構造であるH3K9me2/3がDNAと結合している領域を認識して相互作用し,
この相互作用が広がる様にしてdropletを形成し,結果的に他の部位もdropletに取り込まれる形でDNA構造全体がコンパクトになり,
この領域の遺伝子が不活化し,ヘテロクロマチンとなる.
こうして,このdropletが場として主導権を握る存在であり,
むしろchromatinはそれに流されるだけの実体であると考えることで,
heterochromatinの高い温度感受性に説明がつく.
またこの機構は真核生物に普遍的なもので,人間のHP1タンパク質についても同様の機構が存在すると予想される\cite{HP1a-human}.

\section{相分離生物学が見せる新たな生命像}

オランダ飢饉(1944-1945)を胎児として経験して生まれた子供は,
肥満・高脂血症・糖尿病・統合失調症を発症しやすくなったという報告がある\cite{epigenetics}.
これは,すでに形成されつつあった子供の遺伝状態が,メチル化などの化学修飾を受けることで後天的に変化して,
それが大人になるにつれて発現したものと思われる.
このメカニズムも相分離の観点から説明できると思われる.

メチル化やリン酸化などの化学修飾は,反応としては局所的で小さなものだが,
いずれも水溶液中での溶解性を大きく変化させるものが多い.
即ち,dropletの形成のし易さを変化させていると考えられる.

こうして,従来のDNAを中心としたセントラルドグマ以外の遺伝的活動や表現型の変化を誘起するメカニズムが,
dropletの動的な形成という生物学的相分離の観点から明晰に理解できるのではないかという希望がある.
新たな分野の興隆を,共時代的に目撃している.

\begin{thebibliography}{99}
    \bibitem{木原太郎-遺伝子}
        木原太郎『原子・分子・遺伝子』
        (東京化学同人,1987)
    \bibitem{平岡秀一}
        平岡秀一『溶液における分子認識と自己集合の原理―分子間相互作用』
        ライブラリ 大学基礎化学(サイエンス社,2017)
    \bibitem{有賀克彦}
        有賀克彦『賢くはたらく超分子』
        岩波科学ライブラリー103(岩波書店,2005)
    \bibitem{白木賢太郎}
        白木賢太郎『相分離生物学』
        (東京化学同人,2019)
    \bibitem{Clifford Brangwynne Anthony Hyman 09 Science}
        Clifford P. Brangwynne, et al. "Germline P Granules Are Liquid Droplets That Localize by Controlled Dissolution/Condensation" Science 324, 1729 (2009).
    \bibitem{HP1a}
        Strom, A., Emelyanov, A., Mir, M. et al. "Phase separation drives heterochromatin domain formation" Nature 547, 241–245 (2017). https://doi.org/10.1038/nature22989
    \bibitem{HP1a-human}
        Larson, Adam G et al. “Liquid droplet formation by HP1α suggests a role for phase separation in heterochromatin” Nature vol. 547,7662 (2017): 236-240. doi:10.1038/nature22822
    \bibitem{Yeast Prion}
        T. M. Franzmann et al., "Phase separation of a yeast prion protein promotes cellular fitness", Science, 359(6371), pii: eaao5654 (2018).
    \bibitem{epigenetics}
        Amanda Vlahos, Toby Mansell, Richard Saffery and Boris Novakovic, "Human placental methylome in the interplay of adverse placental health, environmental exposure, and pregnancy outcome" PLoS Genet 15(8) (2019): e1008236. https://doi.org/10.1371/journal.pgen.1008236
    \bibitem{Molecular Evolution}
        E. H. McConkey, "Molecular evolution, intracellular organization, and the quinary structure of proteins", Proc. Natl. Acad. Sci. USA, 79(10), 3236-3240 (1982).
\end{thebibliography}

\end{document}