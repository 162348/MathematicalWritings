\documentclass[uplatex, 12pt, dvipdfmx]{jsarticle}
\title{複素解析学Ⅰレポート}
\author{司馬博文 J4-190549}
\date{\today}
\pagestyle{empty} \setcounter{secnumdepth}{4}
\usepackage{amsmath, amsfonts, amsthm, amssymb, ascmac, color, comment, wrap fig}

\setcounter{tocdepth}{2}
%2はsubsectionまで
\usepackage{mathtools}
%\mathtoolsset{showonlyrefs=true} %labelを附した数式にのみ附番される.

%%% 生成されるPDFファイルにおいて、\tableofcontents によって書き出された目次をクリックすると該当する見出しへジャンプしたり、 さらには、\label{ラベル名} を番号で参照する \ref{ラベル名} や thebibliography環境において \bibitem{ラベル名} を文献番号で参照する \cite{ラベル名} においても番号をクリックすると該当箇所にジャンプする
\usepackage[dvipdfmx]{hyperref}
\usepackage{pxjahyper}

\usepackage{tikz, tikz-cd}
\usepackage[all]{xy}
\def\objectstyle{\displaystyle} %デフォルトではxymatrix中の数式が文中数式モードになるので,それを直した.

%化学式をTikZで簡単に書くためのパッケージ.
\usepackage[version=4]{mhchem} %texdoc mhchem
%化学構造式をTikZで描くためのパッケージ.
\usepackage{chemfig}
%IS単位を書くためのパッケージ
\usepackage{siunitx}

%取り消し線を引くためのパッケージ
\usepackage{ulem}

%\rotateboxコマンドを,文字列の中心で回転させるオプション.
%他rotatebox, scalebox, reflectbox, resizeboxなどのコマンド.
\usepackage{graphicx}

%加藤晃史さんがフル活用していたtcolorboxを,途中改ページ可能で.
\usepackage[breakable]{tcolorbox}

%足助さんからもらったオプション
% \usepackage[shortlabels,inline]{enumitem}
% \usepackage[top=15truemm,bottom=15truemm,left=10truemm,right=10truemm]{geometry}

%enumerate環境を凝らせる.
\usepackage{enumerate}

%日本語にルビをふる
\usepackage{pxrubrica}

%以下,ソースコードを表示する環境の設定.
\usepackage{listings,jvlisting} %日本語のコメントアウトをする場合jlistingが必要
%ここからソースコードの表示に関する設定
\lstset{
  basicstyle={\ttfamily},
  identifierstyle={\small},
  commentstyle={\smallitshape},
  keywordstyle={\small\bfseries},
  ndkeywordstyle={\small},
  stringstyle={\small\ttfamily},
  frame={tb},
  breaklines=true,
  columns=[l]{fullflexible},
  numbers=left,
  xrightmargin=0zw,
  xleftmargin=3zw,
  numberstyle={\scriptsize},
  stepnumber=1,
  numbersep=1zw,
  lineskip=-0.5ex
}
%lstlisting環境で,[caption=hoge,label=fuga]などのoptionを付けられる.

%%%
%%%フォント
%%%

%本文・数式の両方のフォントをTimesに変更するお手軽なパッケージだが,LaTeX標準数式記号の\jmath, \amalg, coprodはサポートされない.
\usepackage{mathptmx}
%Palatinoの方が完成度が高いと美文書作成に書いてあった.
% \usepackage[sc]{mathpazo} %オプションは,familyの指定.pplxにしている.
%2000年のYoung Ryuによる新しいTimes系.なおPalatinoもある.
% \usepackage{newtxtext, newtxmath}
%拡張数学記号.\textsectionでブルバキに!
% \usepackage{textcomp, mathcomp}
% \usepackage[T1]{fontenc} %8bitエンコーディングにする.comp系拡張数学文字の動作が安定する.
%AMS Euler.Computer Modernと相性が悪いとは…….
% \usepackage{ccfonts, eulervm} %KnuthのConcrete Mathematicsの組み合わせ.
% \renewcommand{\rmdefault}{pplx} %makes LaTeX use Palatino in place of CM Roman.Do not use the Euler math fonts in conjunction with the default Computer Modern text fonts – this is ugly!

%%% newcommands
    %参考文献で⑦というのを出したかった.\circled{n}と打てば良い.LaTeX StackExchangeより.
\newcommand*\circled[1]{\tikz[baseline=(char.base)]{\node[shape=circle,draw,inner sep=0.8pt] (char) {#1};}}

%%%
%%% ショートカット 足助さんからのコピペ
%%%

\DeclareMathOperator{\grad}{\mathrm{grad}}
\DeclareMathOperator{\rot}{\mathrm{rot}}
\DeclareMathOperator{\divergence}{\mathrm{div}}
\newcommand\R{\mathbb{R}}
\newcommand\N{\mathbb{N}}
\newcommand\C{\mathbb{C}}
\newcommand\Z{\mathbb{Z}}
\newcommand\Q{\mathbb{Q}}
\newcommand\GL{\mathrm{GL}}
\newcommand\SL{\mathrm{SL}}
\newcommand\False{\mathrm{False}}
\newcommand\True{\mathrm{True}}
\newcommand\tr{\mathrm{tr}}
\newcommand\M{\mathcal{M}}
\newcommand\F{\mathbb{F}}
\renewcommand\H{\mathbb{H}}
\newcommand\id{\mathrm{id}}
\newcommand\A{\mathcal{A}}
\renewcommand\coprod{\rotatebox[origin=c]{180}{$\prod$}}
\newcommand\pr{\mathrm{pr}}
\newcommand\U{\mathfrak{U}}
\newcommand\Map{\mathrm{Map}}
\newcommand\dom{\mathrm{dom}}
\newcommand\cod{\mathrm{cod}}
\newcommand\supp{\mathrm{supp}\;}
\newcommand\Ker{\mathrm{Ker}\;}
%%% 複素解析学
\renewcommand\Re{\mathrm{Re}\;}
\renewcommand\Im{\mathrm{Im}\;}
\newcommand\Gal{\mathrm{Gal}}
\newcommand\PGL{\mathrm{PGL}}
\newcommand\PSL{\mathrm{PSL}}
%%% 解析力学
\newcommand\x{\mathbf{x}}
\newcommand\q{\mathbf{q}}
%%% 集合と位相
\newcommand\ORD{\mathrm{ORD}}
%%% 形式言語理論
\newcommand\REGEX{\mathrm{REGEX}}

%%% 圏
\newcommand\Hom{\mathrm{Hom}}
\newcommand\Mor{\mathrm{Mor}}
\newcommand\Aut{\mathrm{Aut}}
\newcommand\End{\mathrm{End}}
\newcommand\op{\mathrm{op}}
\newcommand\ev{\mathrm{ev}}
\newcommand\Ob{\mathrm{Ob}}
\newcommand\Ar{\mathrm{Ar}}
\newcommand\Arr{\mathrm{Arr}}
\newcommand\Set{\mathrm{Set}}
\newcommand\Grp{\mathrm{Grp}}
\newcommand\Cat{\mathrm{Cat}}
\newcommand\Mon{\mathrm{Mon}}
\newcommand\CMon{\mathrm{CMon}}
\newcommand\Pos{\mathrm{Pos}}
\newcommand\Vect{\mathrm{Vect}}
\newcommand\FinVect{\mathrm{FinVect}}
\newcommand\Fun{\mathrm{Fun}}
\newcommand\Ord{\mathrm{Ord}}
\newcommand\eq{\mathrm{eq}}
\newcommand\coeq{\mathrm{coeq}}

%%%
%%% 定理環境 以下足助さんからのコピペ
%%%

\newtheoremstyle{StatementsWithStar}% ?name?
{3pt}% ?Space above? 1
{3pt}% ?Space below? 1
{}% ?Body font?
{}% ?Indent amount? 2
{\bfseries}% ?Theorem head font?
{\textbf{.}}% ?Punctuation after theorem head?
{.5em}% ?Space after theorem head? 3
{\textbf{\textup{#1~\thetheorem{}}}{}\,$^{\ast}$\thmnote{(#3)}}% ?Theorem head spec (can be left empty, meaning ‘normal’)?
%
\newtheoremstyle{StatementsWithStar2}% ?name?
{3pt}% ?Space above? 1
{3pt}% ?Space below? 1
{}% ?Body font?
{}% ?Indent amount? 2
{\bfseries}% ?Theorem head font?
{\textbf{.}}% ?Punctuation after theorem head?
{.5em}% ?Space after theorem head? 3
{\textbf{\textup{#1~\thetheorem{}}}{}\,$^{\ast\ast}$\thmnote{(#3)}}% ?Theorem head spec (can be left empty, meaning ‘normal’)?
%
\newtheoremstyle{StatementsWithStar3}% ?name?
{3pt}% ?Space above? 1
{3pt}% ?Space below? 1
{}% ?Body font?
{}% ?Indent amount? 2
{\bfseries}% ?Theorem head font?
{\textbf{.}}% ?Punctuation after theorem head?
{.5em}% ?Space after theorem head? 3
{\textbf{\textup{#1~\thetheorem{}}}{}\,$^{\ast\ast\ast}$\thmnote{(#3)}}% ?Theorem head spec (can be left empty, meaning ‘normal’)?
%
\newtheoremstyle{StatementsWithCCirc}% ?name?
{6pt}% ?Space above? 1
{6pt}% ?Space below? 1
{}% ?Body font?
{}% ?Indent amount? 2
{\bfseries}% ?Theorem head font?
{\textbf{.}}% ?Punctuation after theorem head?
{.5em}% ?Space after theorem head? 3
{\textbf{\textup{#1~\thetheorem{}}}{}\,$^{\circledcirc}$\thmnote{(#3)}}% ?Theorem head spec (can be left empty, meaning ‘normal’)?
%
\theoremstyle{definition}
 \newtheorem{theorem}{定理}[section]
 \newtheorem{axiom}[theorem]{公理}
 \newtheorem{corollary}[theorem]{系}
 \newtheorem{proposition}[theorem]{命題}
 \newtheorem*{proposition*}{命題}
 \newtheorem{lemma}[theorem]{補題}
 \newtheorem*{lemma*}{補題}
 \newtheorem*{theorem*}{定理}
 \newtheorem{definition}[theorem]{定義}
 \newtheorem{example}[theorem]{例}
 \newtheorem{notation}[theorem]{記法}
 \newtheorem*{notation*}{記法}
 \newtheorem{assumption}[theorem]{仮定}
 \newtheorem{question}[theorem]{問}
 \newtheorem{counterexample}[theorem]{反例}
 \newtheorem{reidai}[theorem]{例題}
 \newtheorem{problem}[theorem]{問題}
 \newtheorem*{solution*}{\bf{[解]}}
 \newtheorem{discussion}[theorem]{議論}
 \newtheorem{remark}[theorem]{注}
 \newtheorem{universality}[theorem]{普遍性} %非自明な例外がない.
 \newtheorem{universal tendency}[theorem]{普遍傾向} %例外が有意に少ない.
 \newtheorem{hypothesis}[theorem]{仮説} %実験で説明されていない理論.
 \newtheorem{theory}[theorem]{理論} %実験事実とその(さしあたり)整合的な説明.
 \newtheorem{fact}[theorem]{実験事実}
 \newtheorem{model}[theorem]{模型}
 \newtheorem{explanation}[theorem]{説明} %理論による実験事実の説明
 \newtheorem{anomaly}[theorem]{理論の限界}
 \newtheorem{application}[theorem]{応用例}
 \newtheorem{method}[theorem]{手法} %実験手法など,技術的問題.
 \newtheorem{history}[theorem]{歴史}
 \newtheorem{research}[theorem]{研究}
% \newtheorem*{remarknonum}{注}
 \newtheorem*{definition*}{定義}
 \newtheorem*{remark*}{注}
 \newtheorem*{question*}{問}
 \newtheorem*{axiom*}{公理}
 \newtheorem*{example*}{例}
%
\theoremstyle{StatementsWithStar}
 \newtheorem{definition_*}[theorem]{定義}
 \newtheorem{question_*}[theorem]{問}
 \newtheorem{example_*}[theorem]{例}
 \newtheorem{theorem_*}[theorem]{定理}
 \newtheorem{remark_*}[theorem]{注}
%
\theoremstyle{StatementsWithStar2}
 \newtheorem{definition_**}[theorem]{定義}
 \newtheorem{theorem_**}[theorem]{定理}
 \newtheorem{question_**}[theorem]{問}
 \newtheorem{remark_**}[theorem]{注}
%
\theoremstyle{StatementsWithStar3}
 \newtheorem{remark_***}[theorem]{注}
 \newtheorem{question_***}[theorem]{問}
%
\theoremstyle{StatementsWithCCirc}
 \newtheorem{definition_O}[theorem]{定義}
 \newtheorem{question_O}[theorem]{問}
 \newtheorem{example_O}[theorem]{例}
 \newtheorem{remark_O}[theorem]{注}
%
\theoremstyle{definition}
%
\raggedbottom
\allowdisplaybreaks

%証明環境のスタイル
\everymath{\displaystyle}
\renewcommand{\proofname}{\bf [証明]}
\renewcommand{\thefootnote}{\dag\arabic{footnote}}	%足助さんからもらった.どうなるんだ?

%mathptmxパッケージ下で,\jmath, \amalg, coprodの記号を出力するためのマクロ.TeX Wikiからのコピペ.
% \DeclareSymbolFont{cmletters}{OML}{cmm}{m}{it}
% \DeclareSymbolFont{cmsymbols}{OMS}{cmsy}{m}{n}
% \DeclareSymbolFont{cmlargesymbols}{OMX}{cmex}{m}{n}
% \DeclareMathSymbol{\myjmath}{\mathord}{cmletters}{"7C}
% \DeclareMathSymbol{\myamalg}{\mathbin}{cmsymbols}{"71}
% \DeclareMathSymbol{\mycoprod}{\mathop}{cmlargesymbols}{"60}
% \let\jmath\myjmath
% \let\amalg\myamalg
% \let\coprod\mycoprod
\begin{document}
\maketitle

\section*{[R3]}

$n\in\N$について定まる線型写像$F$:
\[\xymatrix@R-2pc{
    \C^{n+1}\ar[r]^-F&{\C[x,y]}\\
    \rotatebox[origin=c]{90}{$\in$}&\rotatebox[origin=c]{90}{$\in$}\\
    {\begin{pmatrix}\alpha_1\\\vdots\\\alpha_n\end{pmatrix}}\ar@{|->}[r]&\sum^n_{k=0}\alpha_kx^{n-k}y^k=P(x,y)=f(z)
}\]
について,$\C[x,y]$の部分空間
\[ V:=\left\{f(z)=P(x,y)\in\C[x,y]\;\middle|\; \frac{\partial f}{\partial\overline{z}}=0\right\} \]
の逆像の次元を求める問題である.

まず,次の補題を証明する.

\begin{lemma*}
    $n=1,2,\cdots$とする.族$(x^{n-k}y^k)_{k=1,\cdots,n}$は,実2変数の複素係数多項式からなる線型空間$\C[x,y]$上,線型独立である.
\end{lemma*}
\begin{proof}
    $n$についての帰納法で示す.
    $n=1$の時,$x,y$は独立変数であるから,$\alpha_1,\alpha_2\in\C$として$\alpha_1x+\alpha_2y=0$ならば,
    特に$x=0,y=0$の場合をそれぞれ考えると$\alpha_1=\alpha_2=0$(例えば$x=0$とした時,$\alpha_2y=0$.任意の$y\in\R$についてこれを満たすには$\alpha_2=0$が必要).
    よって,$\C[x,y]$上線型独立.

    今,$n=1,\cdots,m-1$について族$(x^{n-k}y^k)_{k=1,\cdots,n}$が線型独立であると仮定し,$n=m$の場合を示す.
    $x^m,x^{m-1}y,\cdots,xy^{m-1},y^m$に対して,$\alpha_0,\cdots,\alpha_{m}\in\C$とし,
    $\alpha_0x^m+\alpha_1x^{m-1}y+\cdots+\alpha_{m-1}xy^{m-1}+\alpha_my^m=0$とする.まず,$x=0,y=0$の場合をそれぞれ考えると,$\alpha_0=\alpha_m=0$である.
    すると,$\alpha_1x^{m-1}y+\cdots+\alpha_{m-1}xy^{m-1}=xy(\alpha_1x^{m-2}+\cdots+\alpha_{m-1}y^{m-2})=0$である.特に$xy\ne 0$の場合を考えると,$\alpha_1x^{m-2}+\cdots+\alpha_{m-1}y^{m-2}=0$.
    すると帰納法の仮定より,$x^{m-2},x^{m-3}y,\cdots,xy^{m-3},y^{m-2}$は線型独立だから,$\alpha_1=\cdots=\alpha_{m-1}=0$.以上より,$n=m$の場合も成り立つ.
\end{proof}

$n=0$の時,$P(x,y)=\alpha_0$と定数関数であり,これは常に正則.従って,$F^{-1}(V)=\C$で,一次元.

$n=1,2,\cdots$の時,
\[P(x,y)=\alpha_0x^n+\alpha_1x^{n-1}y+\alpha_2x^{n-2}y^2+\cdots+\alpha_{n-2}x^2y^{n-2}+\alpha_{n-1}xy^{n-1}+\alpha_ny^n\]
であるから,補題より,
\begin{align*}
    \frac{\partial f}{\partial\overline{z}}=0
    &\Leftrightarrow \frac{\partial f}{\partial x}+i\frac{\partial f}{\partial y}=0\\
    &\Leftrightarrow (n\alpha_0x^{n-1}+(n-1)\alpha_1x^{n-2}y+\cdots+2\alpha_{n-2}xy^{n-2}+\alpha_{n-1}y^{n-1})\\
    &\hphantom{\Leftrightarrow}\;+i(\alpha_1x^{n-1}+2\alpha_2x^{n-2}y+\cdots+(n-1)\alpha_{n-1}xy^{n-2}+n\alpha_ny^{n-1})=0\\
    &\Leftrightarrow (n\alpha_0+i\alpha_1)x^{n-1}+((n-1)\alpha_1+2\alpha_2i)x^{n-2}y+\cdots\\
    &\hphantom{\Leftrightarrow}\;\cdots+(2\alpha_{n-2}+i(n-1)\alpha_{n-1})xy^{n-2}+(\alpha_{n-1}+in\alpha_n)y^{n-1}=0\\
    &\Leftrightarrow \begin{cases}
        n\alpha_0+i\alpha_1=0\\
        (n-1)\alpha_1+2\alpha_2i=0\\
        \hphantom{(n-1)\alpha_1}\vdots\\
        2\alpha_{n-2}+i(n-1)\alpha_{n-1}=0\\
        \alpha_{n-1}+in\alpha_n=0
    \end{cases}\;\;\;(\because 補題)
\end{align*}
この$n$本の連立方程式により,部分空間$F^{-1}(V)$は1次元に定まる.なぜなら,$\alpha_0\in\C$を任意に定めると,1本目の式により$\alpha_1$が定まり,それと2本目により$\alpha_2$が定まり,
以降$\alpha_n$まで一意に定まるからである.

よって以上より,$\dim(F^{-1}(V))=1$.
\begin{flushright}$\blacksquare$\end{flushright}

\section*{[R4]}

適宜$\C\simeq\R^2$による同一視をすることで,変数$z\in U, x:=\Re z,y:=\Im z\in\R$と関数$u,v:\R^2\to\R$を,次の図のように置く.
\[\xymatrix{
    &\C&\\
    U\simeq\R^2\ar@{-->}[ur]^-{h}\ar[rr]^-{f}\ar@{}[d]|{\rotatebox[origin=c]{90}{$\in$}}&&V\simeq\R^2\ar[ul]_-{g}\ar@{}[d]|{\rotatebox[origin=c]{90}{$\in$}}\\
    z=x+yi\ar@{|->}[rr]&&{f(z)=u\begin{pmatrix}x\\y\end{pmatrix}+iv\begin{pmatrix}x\\y\end{pmatrix}}
}\]

多変数の実ベクトル値関数$\R^2\to\R^2\to\R^2(\simeq\C)$についての連鎖律より,$\frac{\partial h}{\partial z}$は次のように計算できる.ただし,式中の$\cdot$は終域$\C$上の積とした.
\begin{align*}
    2\frac{\partial h}{\partial z}&=\frac{\partial h}{\partial x}-i\frac{\partial h}{\partial y}=\frac{\partial (g\circ f)}{\partial x}-i\frac{\partial (g\circ f)}{\partial x}\\
    &=(g_x\circ f\cdot u_x+g_y\circ f\cdot v_x)-i(g_x\circ f\cdot u_y+g_y\circ f\cdot v_y)\\
    &=g_x\circ f(u_x-iu_y)+g_y\circ f(v_x-iv_y)\;\;\; (体\R の分配法則により,式を整理した).
\end{align*}

一方で,
\[\frac{\partial g}{\partial z}\circ f\cdot \frac{\partial f}{\partial z}+\frac{\partial g}{\partial\overline{z}}\circ f\cdot\frac{\partial\overline{f}}{\partial z}\]
も,
\[\frac{\partial f}{\partial x}=\frac{\partial(u+iv)}{\partial x}=\frac{\partial u}{\partial x}+i\frac{\partial v}{\partial x}\]
より,
\begin{align*}
    &\frac{\partial g}{\partial z}\circ f\cdot \frac{\partial f}{\partial z}+\frac{\partial g}{\partial\overline{z}}\circ f\cdot\frac{\partial\overline{f}}{\partial z}\\
    =&\left(\frac{\partial g}{\partial x}-i\frac{\partial g}{\partial y}\right)\circ f\cdot\left(\frac{\partial f}{\partial x}-i\frac{\partial f}{\partial y}\right) + \left(\frac{\partial g}{\partial x}+i\frac{\partial g}{\partial y}\right)\circ f\cdot\left(\frac{\partial\overline{f}}{\partial x}+i\frac{\partial\overline{f}}{\partial y}\right)\\
    =&(g_x-ig_y)\circ f(u_x+iv_x-i(u_y+iv_y)) + (g_x+ig_y)\circ f(u_x-iv_x-i(u_y-iv_y))\\
    =&(g_x-ig_y)\circ f(u_x+v_y+i(v_x-u_y)) + (g_x+ig_y)\circ f(u_x-v_y-i(v_x+u_y))\\
    =&2g_x\circ f(u_x-iu_y)+2g_y\circ f(v_x-iv_y)
\end{align*}
であるから,
\[\frac{\partial h}{\partial z}=\frac{\partial g}{\partial z}\circ f\cdot \frac{\partial f}{\partial z}+\frac{\partial g}{\partial\overline{z}}\circ f\cdot\frac{\partial\overline{f}}{\partial z}\]
を得る.
\[\frac{\partial h}{\partial\overline{z}}=\frac{\partial g}{\partial z}\circ f\cdot \frac{\partial f}{\partial\overline{z}}+\frac{\partial g}{\partial\overline{z}}\circ f\cdot\frac{\partial\overline{f}}{\partial\overline{z}}\]
も同様.
\begin{flushright}$\blacksquare$\end{flushright}

\end{document}