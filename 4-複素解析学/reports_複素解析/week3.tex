\documentclass[uplatex, 12pt, dvipdfmx, twocolumn]{jsarticle}
\title{複素解析学Ⅰ 第三回レポート}
\author{司馬博文 J4-190549}
\date{\today}
\pagestyle{empty} \setcounter{secnumdepth}{4}
\input{/Users/hirofumi.shiba48/Desktop/数理科学/preamble_CM.tex}
\begin{document}
\maketitle

\section*{[R5]}

部分和の列$\left(S_N:=\sum^N_{n=0}|a_n|\right)_{n\in\N}$は収束するから,この極限を$S\in\R$と置く.

まず,新たな列$\left(T_n\right)_{n\in\N}$を
\[T_n=S_{\max_{1\le i\le n}f(i)}\]
で定めると,この列は収束し,$\lim_{n\to\infty}T_n=S$であることを示す.
作り方から,値域について$\{T_n\}\subset\{S_n\}$であるから,列$(T_n)$は上に有界.
また$(T_n)$は単調増加列であることより,確かに実数列$(T_n)$は収束する:$\lim_{n\to\infty}T_n\le S$.
これから,$\lim_{n\to\infty}T_n\ge S$でもあることをみる.任意の$\epsilon>0$に対して,十分大きな$N>0$が存在して
\[ S-\epsilon<S_n(<S)\;\;\;(n\ge N) \]
が成り立つから,$N':=\{f^{-1}(N),N\}$と取れば,$n\le \max_{1\le i\le n}f(i)$より$T_n=S_{\max_{1\le i\le n}f(i)}\ge S_n$(等号成立は,$f|_{n+1}:\{0,1,\cdots,n\}\to\{0,1,\cdots,n\}$が全単射である時)より,
\[S-\epsilon<S_n\le T_n(<S)\;\;\;(n>N')\]
が成り立つ.よって,$\lim_{n\to\infty}T_n=S$.

最後に,この$(T_n)$を用いて,$\left(S'_N:=\sum^N_{n=0}|a_{f(n)}|\right)_{n\in\N}$が$S$に収束することを示す.この列$(S'_N)$も単調増加列で,$S'_N\le T_N\;(N\in\N)$であるから極限値をもち,これを$S'$とすれば$S'\le S$である.
$S\le S'$を示す.いま$(T_n)$は$S$に収束するから,任意の$\epsilon>0$に対して,十分大きな$N$を取れば,
\[S-\epsilon<T_n(<S)\;\;\;(n\ge N)\]
を満たす.
このとき,$N$に対して$M:=\max_{1\le i\le N}f(i)$と置けば$T_N=a_0+a_1+\cdots+a_M$であるが,さらに$N'=\max_{1\le i\le M}f^{-1}(i)$と置けば,$N'\ge N$で,
\[S-\epsilon<T_N\le S'_n(<S)\;\;\;(n>N')\]
が成り立つ.
よって,$S=S'$.
\begin{flushright}$\blacksquare$\end{flushright}

\section*{[R6]}

\end{document}