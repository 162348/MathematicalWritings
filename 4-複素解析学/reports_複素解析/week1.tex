\documentclass[uplatex, 12pt, dvipdfmx]{jsarticle}
\title{複素解析学Ⅰレポート}
\author{司馬博文 J4-190549}
\date{\today}
\pagestyle{empty} \setcounter{secnumdepth}{4}
\input{/Users/hirofumi.shiba48/Desktop/数理科学/preamble_CM.tex}
\begin{document}
\maketitle
\section*{[R1]}
\subsection*{(a)}
$z\in\C$とする.$\Im z=\frac{z-\overline{z}}{2i}\Leftrightarrow (\overline{z}-z)i=2\Im z$より,
\begin{align*}
    |\psi(z)|^2 &= \psi(z)\overline{\psi(z)}\\
    &= \frac{z-i}{z+i}\overline{\frac{z-i}{z+i}}\\
    &= \frac{z-i}{z+i}\frac{\overline{z}+i}{\overline{z}-i}\\
    &= \frac{|z|^2+(-\overline{z}+z)i+1}{|z|^2+(\overline{z}-z)i+1}\\
    &= \frac{|z|^2-2\Im z+1}{|z|^2+2\Im z+1} &\cdots\cdots(*)
\end{align*}
であるが,$z\in\R$だから$\Im z=0$より,
\begin{align*}
    |\psi(z)|^2= \frac{z^2+1}{z^2+1}=1
\end{align*}
\begin{flushright}
    $\blacksquare$
\end{flushright}

\subsection*{(b)}

$(*)$より,$\Im z>0$の時,$\frac{|z|^2-2\Im z+1}{|z|^2+2\Im z+1}<1$より,$|\psi(z)|<1$.
従って,$(0<)|\psi(z)|<1$.
\begin{flushright}$\blacksquare$\end{flushright}

\section*{[R2]}
\subsection*{(a)$\Rightarrow$(b)}

(a)より,次が成り立つ.
\begin{equation}\label{1}
    \forall\varepsilon>0,\;\exists\delta>0,\;|a-z|<\delta\Rightarrow|f(a)-f(z)|<\varepsilon.
\end{equation}
$a$に収束する数列$\{z_n\}$を任意に取ると,次が成り立つ.
\begin{equation}\label{2}
    \forall\delta>0,\;\exists N>0,\;n>N\Rightarrow|a-z_n|<\delta.
\end{equation}
任意の$\epsilon>0$を取ると,\ref{1}より,$|a-z|<\delta\Rightarrow|f(a)-f(z)|<\varepsilon$を満たす$\delta>0$が存在し,この$\delta$に対して
\ref{2}より,$n>N\Rightarrow|a-z_n|<\delta$を満たす$N>0$が存在する.
以上より,次の論理式,即ち$\lim_{n\to\infty}f(z_n)=f(a)$が示せた.
\[\forall\epsilon>0,\;\exists\delta>0,\;\exists N>0,\;n>N\Rightarrow|a-z_n|<\delta\Rightarrow|f(a)-f(z_n)|<\varepsilon.\]
数列$\{z_n\}$は任意にとったから,(b)が示せた.

\begin{flushright}$\blacksquare$\end{flushright}

\subsection*{(b)$\Rightarrow$(a)}

Gauss平面上の点$w\in\C$に対して,$z^w_n=a+\frac{w-a}{n}\;(n=1,2,3,\cdots)$と定めることにより,点$w$を通り$a$に収束する($|z_n-a|$の値が単調減少するという意味で)単調な数列$\{z_n^w\}_{n=1,2,\cdots}$が取れる.
(こうして,数列の族$(\{z^w_n\}_{n=1,2,\cdots})_{w\in\C}$を定めた).
任意に$\varepsilon>0$を取る.これに対して,各$w\in\C$に対して$\{z^w_n\}$は$a$に収束するから,(b)から非負整数$N^w\ge 0$が存在し,$n>N^w\Rightarrow|f(z^w_n)-f(a)|<\varepsilon$を満たす(こうして,族$\{N^w\}_{w\in\C}$が定まる).
このとき,$\delta=\min_{w\in\C}|z^w_{N+1}-a|$とすれば,$|z-a|<\delta\Rightarrow|f(z)-f(a)|<\varepsilon$が成り立つことを示せば良い.

$|w-a|<\delta$を満たす$w\in\C$を任意に取る.
これを通る数列$\{z_n^w\}$について,$n>N\Rightarrow |f(z_n^w)-f(a)|<\varepsilon$が成り立つ.
いま,$\{z_n^w\}$は単調だったから,$\delta$は$\delta=\min_{w\in\C}|z^w_{N+1}-a|$と定めたことから,
$|w-a|<\delta$を満たす$w$に対して,$w=z^w_m$を満たす整数$m>N$が存在する.
従って,$|f(z_m^w)-f(a)|=|f(w)-f(a)|<\varepsilon$が導かれる.

以上より,次の主張,即ち(a)が示せた.
\[ \forall\varepsilon>0,\;\exists\delta>0,\;|z-a|<\delta\Rightarrow|f(z)-f(a)|<\varepsilon \]
\begin{flushright}$\blacksquare$\end{flushright}

\subsection*{(b)$\Rightarrow$(a)(より簡潔な回答)}
対偶を示す.(a)が成り立たないとすると,$\exists\varepsilon>0,\;\forall\delta>0,\;|z-a|<\delta\land|f(z)-f(a)|\ge\varepsilon$.
この時の$\varepsilon$を一つ取り,$\epsilon_0$とする.
ここで,勝手に$a$に収束する数列$\{z_n\}$を取る.
すると$\forall\varepsilon>0,\;\exists N>0,n>N\Rightarrow|z_n-a|<\varepsilon$.
従って,$\epsilon_0$に対して,$\forall\delta>0,\;\exists N>0,\;n>N\Rightarrow|f(z_n)-f(a)|\ge\epsilon_0$が成り立つから,
$\lim_{n\to\infty}f(z_n)\ne f(a)$.よって(b)の否定が導けた.

\begin{flushright}$\blacksquare$\end{flushright}
\end{document}