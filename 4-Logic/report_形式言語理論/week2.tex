\documentclass[uplatex, dvipdfmx]{jsarticle}
\title{形式言語理論 第2回レポート}
\author{司馬博文 J4-190549}
\date{\today}
\pagestyle{empty} \setcounter{secnumdepth}{4}
\input{/Users/hirofumi.shiba48/Desktop/数理科学/preamble_CM.tex}
\begin{document}
\maketitle

\section*{1,2}

緑色部分が,示した言語を受理する決定性オートマトンで,
黄色部分も含めると,非決定性オートマトンである.

\begin{center}\begin{figure}[h]\centering
    \includegraphics[width=10cm]{形式言語理論.jpg}
\end{figure}\end{center}

\section*{3}

その言語$L$を受理する有限な決定性または非決定性のオートマトンを構成すれば良い.

\end{document}