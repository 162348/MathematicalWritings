\documentclass[uplatex, dvipdfmx]{jsarticle}
\title{計算の理論 レポート問題2.1}
\author{司馬博文 J4-190549}
\date{\today}
\pagestyle{empty} \setcounter{secnumdepth}{4}
\usepackage{amsmath, amsfonts, amsthm, amssymb, ascmac, color, comment, wrap fig}

\usepackage{mathtools}
\mathtoolsset{showonlyrefs=true} %labelを附した数式にのみ附番される.

\usepackage{tikz, tikz-cd}
\usepackage[all]{xy}
\def\objectstyle{\displaystyle} %デフォルトではxymatrix中の数式が文中数式モードになるので,それを直した.

%化学式をTikZで簡単に書くためのパッケージ.
\usepackage[version=4]{mhchem} %texdoc mhchem
%化学構造式をTikZで描くためのパッケージ.
\usepackage{chemfig}
%IS単位を書くためのパッケージ
\usepackage{siunitx}
%取り消し線を引くためのパッケージ
\usepackage{ulem}

%\rotateboxコマンドを,文字列の中心で回転させるオプション.
%他rotatebox, scalebox, reflectbox, resizeboxなどのコマンド.
\usepackage{graphicx}

%加藤晃史さんがフル活用していたtcolorboxを,途中改ページ可能で.
\usepackage[breakable]{tcolorbox}

%enumerate環境を凝らせる.
\usepackage{enumerate}

%日本語にルビをふる
\usepackage{pxrubrica}

%足助さんからもらったオプション
%\usepackage[shortlabels,inline]{enumitem}
%\usepackage[top=15truemm,bottom=15truemm,left=10truemm,right=10truemm]{geometry}

%以下,ソースコードを表示する環境の設定.
\usepackage{listings,jvlisting} %日本語のコメントアウトをする場合jlistingが必要
%ここからソースコードの表示に関する設定
\lstset{
  basicstyle={\ttfamily},
  identifierstyle={\small},
  commentstyle={\smallitshape},
  keywordstyle={\small\bfseries},
  ndkeywordstyle={\small},
  stringstyle={\small\ttfamily},
  frame={tb},
  breaklines=true,
  columns=[l]{fullflexible},
  numbers=left,
  xrightmargin=0zw,
  xleftmargin=3zw,
  numberstyle={\scriptsize},
  stepnumber=1,
  numbersep=1zw,
  lineskip=-0.5ex
}
%lstlisting環境で,[caption=hoge,label=fuga]などのoptionを付けられる.
\makeatletter
    \AtBeginDocument{
    \renewcommand*{\thelstlisting}{\arabic{chapter}.\arabic{section}.\arabic{lstlisting}}
    \@addtoreset{lstlisting}{section}
    }
\makeatother
%caption番号を「[chapter番号].[section番号].[subsection番号]-[そのsubsection内においてn番目]」に変更
\renewcommand{\lstlistingname}{program}
%caption名を"program"に変更

%%%
%%%フォント
%%%

%本文・数式の両方のフォントをTimesに変更するお手軽なパッケージだが,LaTeX標準数式記号の\jmath, \amalg, coprodはサポートされない.
% \usepackage{mathptmx}
%Palatinoの方が完成度が高いと美文書作成に書いてあった.
\usepackage[sc]{mathpazo} %オプションは,familyの指定.pplxにしている.
%2000年のYoung Ryuによる新しいTimes系.なおPalatinoもある.
% \usepackage{newtxtext, newtxmath}
%拡張数学記号.\textsectionでブルバキに!
\usepackage{textcomp, mathcomp}
\usepackage[T1]{fontenc} %8bitエンコーディングにする.comp系拡張数学文字の動作が安定する.
%AMS Euler.Computer Modernと相性が悪いとは…….
\usepackage{ccfonts, eulervm} %KnuthのConcrete Mathematicsの組み合わせ.
% \renewcommand{\rmdefault}{pplx} %makes LaTeX use Palatino in place of CM Roman.Do not use the Euler math fonts in conjunction with the default Computer Modern text fonts – this is ugly!

%%% newcommands
    %参考文献で⑦というのを出したかった.\circled{n}と打てば良い.LaTeX StackExchangeより.
\newcommand*\circled[1]{\tikz[baseline=(char.base)]{\node[shape=circle,draw,inner sep=0.8pt] (char) {#1};}}

%%%
%%% ショートカット 足助さんからのコピペ
%%%

\DeclareMathOperator{\grad}{\mathrm{grad}}
\DeclareMathOperator{\rot}{\mathrm{rot}}
\DeclareMathOperator{\divergence}{\mathrm{div}}
\newcommand\R{\mathbb{R}}
\newcommand\N{\mathbb{N}}
\newcommand\C{\mathbb{C}}
\newcommand\Z{\mathbb{Z}}
\newcommand\Q{\mathbb{Q}}
\newcommand\GL{\mathrm{GL}}
\newcommand\SL{\mathrm{SL}}
\newcommand\False{\mathrm{False}}
\newcommand\True{\mathrm{True}}
\newcommand\tr{\mathrm{tr}}
\newcommand\M{\mathcal{M}}
\newcommand\F{\mathbb{F}}
% \newcommand\H{\mathbb{H}} すでにある.
\newcommand\id{\mathrm{id}}
\newcommand\A{\mathcal{A}}
%\renewcommand\coprod{\rotatebox[origin=c]{180}{$\prod$}}
\newcommand\pr{\mathrm{pr}}
\newcommand\U{\mathfrak{U}}
\newcommand\Map{\mathrm{Map}}
\newcommand\dom{\mathrm{dom}}
\newcommand\cod{\mathrm{cod}}
\newcommand\supp{\mathrm{supp}}
%%% 複素解析学
\renewcommand\Re{\mathrm{Re}\;}
\renewcommand\Im{\mathrm{Im}\;}
\newcommand\Gal{\mathrm{Gal}}
\newcommand\PGL{\mathrm{PGL}}
\newcommand\PSL{\mathrm{PSL}}
%%% 解析力学
\newcommand\x{\mathbf{x}}
\newcommand\q{\mathbf{q}}
%%% 集合と位相
\newcommand\ORD{\mathrm{ORD}}

%%% 圏
\newcommand\Hom{\mathrm{Hom}}
\newcommand\Mor{\mathrm{Mor}}
\newcommand\Aut{\mathrm{Aut}}
\newcommand\End{\mathrm{End}}
\newcommand\op{\mathrm{op}}
\newcommand\ev{\mathrm{ev}}
\newcommand\Ob{\mathrm{Ob}}
\newcommand\Ar{\mathrm{Ar}}
\newcommand\Arr{\mathrm{Arr}}
\newcommand\Set{\mathrm{Set}}
\newcommand\Grp{\mathrm{Grp}}
\newcommand\Cat{\mathrm{Cat}}
\newcommand\Mon{\mathrm{Mon}}
\newcommand\CMon{\mathrm{CMon}}
\newcommand\Pos{\mathrm{Pos}}
\newcommand\Vect{\mathrm{Vect}}
\newcommand\FinVect{\mathrm{FinVect}}
\newcommand\Fun{\mathrm{Fun}}
\newcommand\Ord{\mathrm{Ord}}

%%%
%%% 定理環境 以下足助さんからのコピペ
%%%

\newtheoremstyle{StatementsWithStar}% ?name?
{3pt}% ?Space above? 1
{3pt}% ?Space below? 1
{}% ?Body font?
{}% ?Indent amount? 2
{\bfseries}% ?Theorem head font?
{\textbf{.}}% ?Punctuation after theorem head?
{.5em}% ?Space after theorem head? 3
{\textbf{\textup{#1~\thetheorem{}}}{}\,$^{\ast}$\thmnote{(#3)}}% ?Theorem head spec (can be left empty, meaning ‘normal’)?
%
\newtheoremstyle{StatementsWithStar2}% ?name?
{3pt}% ?Space above? 1
{3pt}% ?Space below? 1
{}% ?Body font?
{}% ?Indent amount? 2
{\bfseries}% ?Theorem head font?
{\textbf{.}}% ?Punctuation after theorem head?
{.5em}% ?Space after theorem head? 3
{\textbf{\textup{#1~\thetheorem{}}}{}\,$^{\ast\ast}$\thmnote{(#3)}}% ?Theorem head spec (can be left empty, meaning ‘normal’)?
%
\newtheoremstyle{StatementsWithStar3}% ?name?
{3pt}% ?Space above? 1
{3pt}% ?Space below? 1
{}% ?Body font?
{}% ?Indent amount? 2
{\bfseries}% ?Theorem head font?
{\textbf{.}}% ?Punctuation after theorem head?
{.5em}% ?Space after theorem head? 3
{\textbf{\textup{#1~\thetheorem{}}}{}\,$^{\ast\ast\ast}$\thmnote{(#3)}}% ?Theorem head spec (can be left empty, meaning ‘normal’)?
%
\newtheoremstyle{StatementsWithCCirc}% ?name?
{6pt}% ?Space above? 1
{6pt}% ?Space below? 1
{}% ?Body font?
{}% ?Indent amount? 2
{\bfseries}% ?Theorem head font?
{\textbf{.}}% ?Punctuation after theorem head?
{.5em}% ?Space after theorem head? 3
{\textbf{\textup{#1~\thetheorem{}}}{}\,$^{\circledcirc}$\thmnote{(#3)}}% ?Theorem head spec (can be left empty, meaning ‘normal’)?
%
\theoremstyle{definition}
 \newtheorem{theorem}{定理}[section]
 \newtheorem{axiom}[theorem]{公理}
 \newtheorem{corollary}[theorem]{系}
 \newtheorem{proposition}[theorem]{命題}
 \newtheorem*{proposition*}{命題}
 \newtheorem{lemma}[theorem]{補題}
 \newtheorem*{lemma*}{補題}
 \newtheorem*{theorem*}{定理}
 \newtheorem{definition}[theorem]{定義}
 \newtheorem{example}[theorem]{例}
 \newtheorem{notation}[theorem]{記法}
 \newtheorem*{notation*}{記法}
 \newtheorem{assumption}[theorem]{仮定}
 \newtheorem{question}[theorem]{問}
 \newtheorem{counterexample}[theorem]{反例}
 \newtheorem{reidai}[theorem]{例題}
 \newtheorem{problem}[theorem]{問題}
 \newtheorem*{solution*}{\bf{[解]}}
 \newtheorem{discussion}[theorem]{議論}
 \newtheorem{remark}[theorem]{注}
 \newtheorem{universality}[theorem]{普遍性} %非自明な例外がない.
 \newtheorem{universal tendency}[theorem]{普遍傾向} %例外が有意に少ない.
 \newtheorem{hypothesis}[theorem]{仮説} %実験で説明されていない理論.
 \newtheorem{theory}[theorem]{理論} %実験事実とその(さしあたり)整合的な説明.
 \newtheorem{fact}[theorem]{実験事実}
 \newtheorem{model}[theorem]{模型}
 \newtheorem{explanation}[theorem]{説明} %理論による実験事実の説明
 \newtheorem{anomaly}[theorem]{理論の限界}
 \newtheorem{application}[theorem]{応用例}
 \newtheorem{method}[theorem]{手法} %実験手法など,技術的問題.
 \newtheorem{history}[theorem]{歴史}
 \newtheorem{research}[theorem]{研究}
% \newtheorem*{remarknonum}{注}
 \newtheorem*{definition*}{定義}
 \newtheorem*{remark*}{注}
 \newtheorem*{question*}{問}
 \newtheorem*{axiom*}{公理}
 \newtheorem*{example*}{例}
%
\theoremstyle{StatementsWithStar}
 \newtheorem{definition_*}[theorem]{定義}
 \newtheorem{question_*}[theorem]{問}
 \newtheorem{example_*}[theorem]{例}
 \newtheorem{theorem_*}[theorem]{定理}
 \newtheorem{remark_*}[theorem]{注}
%
\theoremstyle{StatementsWithStar2}
 \newtheorem{definition_**}[theorem]{定義}
 \newtheorem{theorem_**}[theorem]{定理}
 \newtheorem{question_**}[theorem]{問}
 \newtheorem{remark_**}[theorem]{注}
%
\theoremstyle{StatementsWithStar3}
 \newtheorem{remark_***}[theorem]{注}
 \newtheorem{question_***}[theorem]{問}
%
\theoremstyle{StatementsWithCCirc}
 \newtheorem{definition_O}[theorem]{定義}
 \newtheorem{question_O}[theorem]{問}
 \newtheorem{example_O}[theorem]{例}
 \newtheorem{remark_O}[theorem]{注}
%
\theoremstyle{definition}
%
\raggedbottom
\allowdisplaybreaks

%証明環境のスタイル
\everymath{\displaystyle}
\renewcommand{\proofname}{\bf [証明]}
\renewcommand{\thefootnote}{\dag\arabic{footnote}}	%足助さんからもらった.どうなるんだ?

%mathptmxパッケージ下で,\jmath, \amalg, coprodの記号を出力するためのマクロ.TeX Wikiからのコピペ.
% \DeclareSymbolFont{cmletters}{OML}{cmm}{m}{it}
% \DeclareSymbolFont{cmsymbols}{OMS}{cmsy}{m}{n}
% \DeclareSymbolFont{cmlargesymbols}{OMX}{cmex}{m}{n}
% \DeclareMathSymbol{\myjmath}{\mathord}{cmletters}{"7C}
% \DeclareMathSymbol{\myamalg}{\mathbin}{cmsymbols}{"71}
% \DeclareMathSymbol{\mycoprod}{\mathop}{cmlargesymbols}{"60}
% \let\jmath\myjmath
% \let\amalg\myamalg
% \let\coprod\mycoprod
\begin{document}
\maketitle
\begin{abstract}
    河村先生の授業にて,「計算で解けない問題」の例として,ポストの文字列揃え問題(以下PCPとする)を学んだ.
    この問題が計算不可能であることを,
    Turing機械による「計算」「計算で解ける」ことの定義を用いて,「PCPを解くTuring機械は存在しない」ことを
    (自分の言葉で)証明を構成する形で確認した.
    
    その過程を,次の3つの段階に分解してまとめた.

    1. Turing機械が停止するかどうかを判定する問題(以降HALTとする)は決定不可能である(定理\ref{thm-1}).

    2. PCPは別の問題(PCPw1)に書き換えても等価である(命題\ref{prop-1},\ref{prop-2}).

    3. HALTからPCPw1に帰着する算法が存在するから,これが解けるならば1に矛盾する(定理\ref{thm-2}).
\end{abstract}

\section{停止性判定問題HALTは計算不可能である}

次の定理が成り立つため,停止性判定問題は計算不可能であると言える.

\begin{theorem}[停止性判定機械の非存在]\label{thm-1}
    次の問題を解く機械は存在しない.
    \begin{quotation}停止性判定問題

        1(入力). 機械$M$と入力$x$を表す二進数表記による自然数からなる組$(M,x)$の全て.

        2(出力). $M$が停止する場合は,出力$M(x)$.停止しない場合は記号$\times$.
    \end{quotation}
\end{theorem}
\begin{proof}
    停止性判定問題を解く機械$M_0$の存在を認めて,矛盾を導く.
    このとき,任意の機械$M$と入力$x$に対して,その停止性についての情報を得られるのだから,それに挙動を追加しただけの次のような2つの機械$M_1,M_2$を構成できる.

    \begin{quotation}
        入力$(M,x)$に対して,$M(x)$が停止するならば停止せず($\times$を出力し),$M(x)$が停止しないならば停止する($\bigcirc$を出力する)機械$M_1$.

        入力$x$に対して,$x$をコードされた機械と見たときの機械$x$について,$x(x)$が停止するならば停止せず($\times$を出力し),$x(x)$が停止しないならば停止する($\bigcirc$を出力する)機械$M_2$.
    \end{quotation}

    このとき,$M_1(x,x)\simeq M_2(x)$が成り立つ.即ち,$M_2$は関数$M_1:\mathbb{N}^2\to\{\bigcirc,\times\}$の定義域を$\Delta\subset\mathbb{N}^2$に制限し,1変数にしたものに他ならない.
    しかしこのとき,機械$M_2(x)$は自然数の中にコードされて居らず,そのようなものを作り出してしまったことになる.よって矛盾.
\end{proof}

\section{PCPとPCPw1は,Turing等価である}

\begin{screen}
    PCP (Post's Correspondence Problem) : 有限列が上下1組書かれた札(を表す文字列)が有限種類・各種類可算無限個与えられる.これらを有限枚並べて,上下の文字列を一致させることが出来るかを表す文字列$\{\bigcirc,\times\}$を返せ.

    PCPw1 (PCP with the 1st card designated) : 有限列が上下1組書かれた札\underline{と区別されたそのうちの1枚}(を表す文字列)が有限種類・各種類可算無限個与えられる.これらを,\underline{区別された1枚を先頭として}有限枚並べて,上下の文字列を一致させることが出来るかを表す文字列$\{\bigcirc,\times\}$を返せ.
\end{screen}

\begin{proposition}\label{prop-1}
    PCPはPCPw1に帰着する.
\end{proposition}
\begin{proof}
    PCP問題はカードの枚数$n$について,そのそれぞれを最初に使うカードとして指定した場合の$n$回のPCPw1問題に等価である.
\end{proof}

\begin{proposition}\label{prop-2}
    PCPw1はPCPに帰着する.
\end{proposition}
\begin{proof}
    $n$枚の札と$1$枚の指定札からなるPCPw1問題を考える.計$n+1$枚の札$\frac{\xi}{\eta}=(\xi,\eta)$(但し$\xi,\eta$は記号の有限列)について,$\xi$内の任意の記号$x$の出現を全て$!x$で置換し,$\eta$内の任意の記号$y$の出現を全て$y!$で置換した$n+1$枚のカードを作成する.
    次に,指定札について,同じような置換を施した上で$\eta$の先頭に$!$を追加したカードを1枚作成する.これは,唯一上下について両方とも$!$が先頭にくる札であり,一致させるにはこの札を先頭に用いるしかない.最後に,上のカードに記号$!$が1つだけ足りない点を除いて記号列が全て上下一致した場合に,$\bigcirc$判定が出来るように,札$\frac{!?}{?}$を作成する.
    こうして作成した計$n+3$枚のカードについてのPCPは,所与のPCPw1に等価になる.
\end{proof}

\section{HALTはPCPw1に帰着する}

\begin{screen}
    HALT : 機械$M$と入力$x$を表す2進数表記による自然数の組$(M,x)$の全てを入力として取り,$M$が停止する場合は出力$M(x)$を,停止しない場合は記号$\times$を出力する.

    HALT' : 機械$M$と入力$x$を表す2進数表記による自然数の組$(M,x)$の全てを入力として取り,$M$が停止する場合は出力$M(x)$を出力して\underline{テープ上の文字を全て消去し,ヘッドをテープ左端に揃えてから終了状態に至り},停止しない場合は記号$\times$を出力する.
\end{screen}
すると,2つの問題は互いに帰着し,等価である.

\begin{theorem}\label{thm-2}
    停止前に追加の行動を要求する停止性判定問題HALT'は,最初のカードの指定つきポストの文字列合わせ問題PCPw1に帰着する.
\end{theorem}
\begin{proof}
まず,Turing機械の挙動を文字列にコードする方法を準備する.
簡単のため,勝手なTuring機械$M=(\Sigma,I,b;Q,q_0,q_h;\delta)$について,そのアルファベットは$\Sigma=\{0,1\}$とし,$Q=\{ q_0,q_1,\cdots,q_k,q_h \}$とする.
すると各状態は,tape(のうち十分大きくとった有限領域,即ち入力$x$と同じ長さ)上に存在するcell内の記号の列(例えば$010110\cdots 010$)の中に,headが指し示すcellに書き込まれている記号の直前に,headの状態$q_i$を挿入した文字列(例えば初期状態から2回右に遷移した場合$01q_i0110\cdots 010$など)を用いて表せる.
すると,この各状態を表す記号列を,separate symbol \% などを用いて区切って繋げることで,或るTuring機械の挙動全体をコードすることが出来る.

入力$(M,x)$に応じて,次のような手順でカードを作る.
まず,上部分は空欄のカード$\frac{ }{ q_0x\% }$を作成し,これを最初に使うべきカードとして指定する.
次に,separate カード$\frac{\%}{\%}$を追加する.
続いて,機械$M$のアルファベットに応じて,その元(tape symbol)1つを上下同じ文字書いたカード,従ってここでは$\Sigma_M=\{0,1\}$であるからカード$\frac{0}{0}, \frac{1}{1}$を用意する.
次に,遷移規則$\delta$の算譜に応じて,例えば$\delta(q_i,a,R)=(q_j,b)$の場合は右遷移カード$\frac{q_ia}{bq_j}$,$\delta(q_i,a,L)=(q_j,b)$の場合は左遷移カード$\frac{aq_i}{q_jb}$を作成する.
最後に,受理状態$q_h$についてカード$\frac{aq_h}{q_h}$を全てのアルファベット$a\in \Sigma$について作成する.

すると,これらの有限枚のカードについてのPCPw1問題は,最初のカードの下の内容を上が追随するために次のカードから始まり,次のseparateカードを迎えるまでは,カードの上の文字列はTuring機械$M$の初期状態を表す文字列を模倣する.その時カードの下の文字列としてあり得るパターンは,Turing機械$M$の算譜$\delta$に記載されていたもののみである.
これがseparate カード$\frac{\%}{\%}$を挟んで続く.最後のカード$\frac{aq_h}{q_h}$(を含むようなseparateカードで区切られたブロック)を迎えることが出来る場合は,「全ての文字を消去した結果,headは左端に到達した結果停止する」ことに対応するから,$\bigcirc$を出力,もしそのような並べ方がない場合はTuring機械が停止しないことがわかるので$\times$を出力すれば良い.
\end{proof}

\begin{thebibliography}{99}
    \bibitem{wikipedia}
    en.wikipedia.orgによる"Post Correspondence Problem"のページを参考にした.
    
    %なお,この記事の参照した節 Proof sketch of undecidability は次の書籍を参考にしている.

    %\bibitem{Sipser}
    %Michael Sipser (2005). "A Simple Undecidable Problem". Introduction to the Theory of Computation (2nd ed.). Thomson Course Technology. pp. 199–205. ISBN 0-534-95097-3.
\end{thebibliography}

\end{document}