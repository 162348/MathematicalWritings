\documentclass[uplatex, 12pt, dvipdfmx]{jsarticle}
\title{形式言語理論 第1回レポート}
\author{司馬博文 J4-190549}
\date{\today}
\pagestyle{empty} \setcounter{secnumdepth}{4}
\input{/Users/hirofumi.shiba48/Desktop/数理科学/preamble_CM.tex}
\begin{document}
\maketitle

\section{ }

$f:\N\to\N$を,定値写像$f=1$と定める:$\forall x\in \N,\;f(x)=1$.
すると,例えば$W=\N\setminus\{1\}$とすれば,$f(f^{-1}(W))=f(\varnothing)=\varnothing\ne W$.
$f=g, V=\{1\}$とすれば,$g^{-1}(g(V))=g^{-1}(1)=\N\ne V$.

\section{ }

二項関係$R\subset X\times Y$が次の2条件を満たせば良い.
\begin{enumerate}
    \item [右一意性]$\forall (x,y),(x',y')\in R,\; x=x'\Rightarrow y=y'$.
    \item [左全域性]$\forall a\in X,\; \exists (x,y)\in R,\; a=x$.
\end{enumerate}

\section{ }

それぞれの概念の定義を次の通りとする.
\begin{definition*}[prefix, sufflix, subword, subsequence]
    アルファベット$\Sigma$による2つの記号列$x,y\in\Sigma^*$について,
    \begin{enumerate}
        \item $\exists u\in\Sigma^*, x=yu$が成り立つ時,$y$を$x$の\textbf{接頭語}と言う.
        \item $\exists u\in\Sigma^*,x=uy$が成り立つ時,$y$を$x$の\textbf{接尾語}と言う.
        \item $\exists u,w\in\Sigma^*,x=uyw$が成り立つ時,$y$を$x$の\textbf{部分語}と言う.
        \item $x=x_1\cdots x_n$と部分列$1\le i_1<\cdots<i_m\le n$が存在して,$y=x_{i_1}\cdots x_{i_m}$を満たす$y$を,\textbf{部分系列}と言う.
    \end{enumerate}
\end{definition*}
$x=x_1\cdots x_n$とすると,
\begin{description}
    \item[prefix] 先頭から$\epsilon,x_1,x[1:2],\cdots,x[1:n-1],x$の$n+1$個.
    \item[suffix] 同様に$n+1$個.
    \item[subword] 列$x_1\cdots x_n$に中に2本の区切り棒$|$を定める定め方は,${}_{n+1}C_2=\frac{n(n+1)}{2}$個.
\end{description}

\section{ }

\begin{center}\begin{figure}[h]\centering
    \includegraphics[width=10cm]{tmp-40.jpg}
    \includegraphics[width=10cm]{tmp-41.jpg}
\end{figure}\end{center}

\end{document}