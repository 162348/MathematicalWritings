\documentclass[uplatex, 12pt, dvipdfmx]{jsarticle}
\title{}
\author{司馬博文 J4-190549\\hirofumi-shiba48@g.ecc.u-tokyo.ac.jp}
\date{\today}
\pagestyle{empty} \setcounter{secnumdepth}{4}
\input{/Users/hirofumi.shiba48/Desktop/数理科学/preamble_CM.tex}
\begin{document}

\section{Categories}

\subsection{Introduction}
\subsection{Functions of sets}
\subsection{Definition of a category}
\begin{definition}[Category] 

    1. 対象$A,B,C,\cdots$というものがある.

    2. 射$f,g,h,\cdots$というものがある.

    3. 各射には$\mathrm{dom}(f)=A, \mathrm{cod}(f)=B$という対象が紐づけられていて,その関係を$f:A\to B$と書く.

    4. $\mathrm{cod}(f)=\mathrm{dom}(g)$を満たす射$f,g$に対し,$g\circ f:\mathrm{dom}(f)\to \mathrm{cod}(g)$という射が定義される.

    5. 各対象$A$には$1_A:A\to A$という特別な射が定義される(単位射).

    6. 射は結合律を満たす.$h\circ (g\circ f)=(h\circ g)\circ f$

    7. 単位射は合成について単位的である.$f:A\to B$として,$f\circ 1_A=f=1_B\circ f$
\end{definition}

\subsection{Examples of categories}

1. 集合の圏$\mathbf{Sets}$と,有限集合の圏$\mathbf{Sets}_\mathrm{fin}$
\begin{example*}[集合の圏から,対象の集合と射の集合に特定の制限を付け加えることで,自由に部分圏が作れる他の例.] 

    1. 対象:有限集合,射:単射 は合成について閉じる.

    2. 対象:集合,射:ファイバーが高々2元集合である写像 は合成について閉じないので圏ではない.

    3. 対象:集合,射:ファイバーが高々有限集合である写像 は合成について閉じる.「高々可算である」でも大丈夫そう.

    4. 対象:集合,射:ファイバーは無限集合である多価写像 は恒等写像がこれを満たさないので,単位射の特徴付けを満たす射が存在しなくなる.
\end{example*}

2. Category of structured sets
\begin{definition*}[具体圏]
    圏$C$が,忘却関手$U:C\to\mathbf{Set}$を持つとき,これを具体圏と呼ぶ.
\end{definition*}

3. 順序集合と単調写像の圏$\mathbf{Pos}$

4. 二項関係の圏$\mathbf{Rel}$:写像は特別な二項関係と見れるから,$\mathbf{Sets}$はこの部分圏である.

射$f:A\to B$は$A\times B$の部分集合で,単位射$1_A$は恒等写像$id_A$のグラフと同じグラフを持つ関係,即ち自明な同値関係$=_A$になる.
合成は,2つの関係$R\subset A\times B, S\subset B\times C$から作れる「相対関係$(a,c)\in S\circ R:\Leftrightarrow \exists (a,b)\in R, (b,c)\in S$」として作れば確かに閉じている.

5. 有限圏としての自然数:射は順序関係である.

6. 圏の圏$\mathbf{Cat}$
\begin{definition}[Functor]
    関手$F:\mathbf{C}\to\mathbf{D}$とは,次を満たす対象写像と射写像の組である.

    1. $F(f:A\to B)=F(f):F(A)\to F(B)$

    2. $F(1_A)=1_F(A)$

    3. $F(g\circ f)=F(g)\circ F(g)$
\end{definition}

7. 圏としてのpreorder:任意の2つの間に射が1つしか存在しない圏(細い圏).従って,Hesse図の上に作った自由圏そのものである.

\begin{definition*}[thin category]
    圏$C$が次の条件を満たす時,細い圏であるという.
    
    任意の2つの対象$x,y\in C$について,
    \begin{center}\begin{tikzcd}
        x \ar[r, "f"] \ar[r, "g"'] & g
    \end{tikzcd}\end{center}
    となっている時,必ず$f=g$である.
\end{definition*}
\begin{remark*}
    細い圏に於いて,2つの対象間で双方向に射が存在する場合,これは互いに逆射になる.
\end{remark*}
\begin{proposition*}
    細い圏は,prosetと同型で,posetと同値である.
\end{proposition*}
\begin{remark}
    細い圏は全てposetと同型である,としなかったのがむしろ圏論特有の自由度の高さ,表現力の豊かさとなっている.
\end{remark}
\begin{proof}
    圏$C$の対象の集合を集合$P$とし,その間の関係$x\le y$を
    \[ x\le y:\Leftrightarrow \exists f:x\to y\in C \]
    と定めると,この関係は反射性と推移性を満たし,前順序集合(preordered set)となる.
    今,関手$F:C\to P$を対象集合は$1_P$,射集合は$f:x\to y\mapsto x\le y$とすると,これはいずれも可逆で,確かに圏の同型である.

    この時,集合$P$について,次のように約束する.
    \[ x\le y\land y\le x\Rightarrow x=y \]
    すると集合$P/=$は順序集合(partially ordered set)である.
    関手$F':C\to P/=$は厳密な意味では可逆ではない.
\end{proof}

8. 圏としてのposet:poset categoriesと呼ぶ.perorder categoryを同型な対象について畳み込んだもの.

9. 位相空間からの例
\begin{proposition*}
    $T_0$ spaces $X$ are posets under the specialization ordering:
    \[ x\le y \Leftrightarrow \forall U\in O(X)\; (x\in U\Rightarrow y\in U) \]
\end{proposition*}

10. 数理論理学からの例:演繹体系に付随する圏 category of proofs
対象を式とし,その間に証明がある$\varphi\vdash\psi$時,射$\varphi\to\psi$を定義する.

11. 計算機科学からの例:関数型プログラミング言語Lに付随する圏$C(L)$
対象は$L$のデータ型,射は関数とする.単位射はdo nothing programで,合成は関数の連続適用$g\circ f=f:g$である.

12. 集合$X$に付随する離散圏$\mathbf{Dis}(X)$

13. 単一対象圏としてのmonoid

射が対象の間に持つ構造「2つの対象と順番付きで紐づけられている」と「単位射の存在」と「合成についての閉性(=推移性)」とを,そっくりそのまま,順序関係に翻訳すれば前順序である.
射自体の持つ構造「結合性」と「単位射の存在」を,代数構造に翻訳すればモノイドである.いずれも最低限の圏である.
それぞれに付加構造として対称性を加えれば,半順序と群を得る.半順序とモノイドが,この本の主要な例になる.

8., 13.の観点から,posetの射とは関手だし,monoidの射も関手と見做せる.

\subsection{Isomorphisms:とCayley表現関手}

\begin{definition}[同型]
    圏$C$に於いて,次を満たす射$f:A\to B$を同型という.
    \[ \exists g:B\to A\in C\; g\circ f=1_A \land f\circ g=1_B \]
\end{definition}
\begin{remark*}[例えばPosでは,全単射な射は,同型だとは限らない.]
    射を何らかの写像だとすると,この同型であるための条件は全単射であることと同値.従ってこの定義は,具体圏に於ける台写像の「全単射」性を一般の圏に写し取ったものに思える.
    だから,全単射でないのに同型になることはないはずだ.
    だが,全単射な射は可逆だとは限らない.例えばPosやTopなどである.

    これは台集合への情報の与え方に依るのだろうか.グラフによる与え方(つまり「より細かい」という語が定義できるような構造)だと,このようなことが起こる?
    より「細かい」構造が定義されている集合へ向けた射は,全単射であろうと可逆ではない.
\end{remark*}

\begin{definition}[群]
    群とは,可逆なモノイドのことである.従って,全ての射が同型であるような単一対象圏のことである.
\end{definition}

\begin{theorem*}[Cayley]
    群$G=(G,\cdot,e,{}^{-1})$は,$\mathrm{Aut}(G)$の或る部分群と同型になる.
\end{theorem*}
\begin{proof}
    Cayley representation $\overline{G}\subset\mathrm{Aut}(G)$を構成する.各$g\in G$に対して,$\overline{g}\in\overline{G}\subset\mathrm{Aut}(G)$を次のような射として定める.
    \begin{center}\begin{tikzcd}
        \overline{g}=g^*:G\ar[r] \ar[d, phantom, "\rotatebox{90}{$\in$}"] & G \ar[d, phantom, "\rotatebox{90}{$\in$}"] \\
        h \ar[r, mapsto] & g\cdot h
    \end{tikzcd}\end{center}
    この時,$\overline{G}$は群になっていることを,写像$F:G\to \overline{G}$が群の射であることを示すことによって確認する.
    $F(f\cdot g)=F(f)\circ F(g)$は$G$の演算$\cdot$の結合性より,また$F(e)=1_G$も成り立つ.
    なお,各射の可逆性については,$F(f\cdot f^{-1})=F(f)\circ F(f^{-1})=1_G=F(e)$より成り立つ.

    群の射$F:G\to \overline{G}$の逆射$H$を構成する.
    \begin{center}\begin{tikzcd}
        H:\overline{G}\ar[r] \ar[d, phantom, "\rotatebox{90}{$\in$}"] & G \ar[d, phantom, "\rotatebox{90}{$\in$}"] \\
        \overline{g} \ar[r, mapsto] & g=\overline{g}(e)
    \end{tikzcd}\end{center}
    これについて,確かに$F\circ H=1_{\overline{G}}, H\circ F=1_G$が成り立つ.従って,$G\simeq \overline{G}$
\end{proof}

\begin{remark}[Two different levels of isomorphisms]
    構成した群$\overline{G}\subset\mathrm{Aut}(G)$の元である,$g$を集合$G$に左から作用させる写像$\overline{g}$は,群$G$の置換であり,集合の同型である.
    一方,構成した関手$F,H$は群の同型である.
\end{remark}

\begin{theorem*}
    任意の圏$C$は,或る具体圏と同型である.
\end{theorem*}
\begin{proof}
    圏$C$から,同型な圏$\overline{C}$を構成する.関手$\overline{ }:C\to\overline{C}$の対象写像を次のように定める.
    \begin{center}\begin{tikzcd}
        C \ar[r] \ar[d, phantom, "\rotatebox{90}{$\in$}"] & \overline{C} \ar[d, phantom, "\rotatebox{90}{$\in$}"] \\
        c \ar[r, mapsto] & \overline{c}=\{ f\in\mathrm{arr}(C)\mid \mathrm{cod}(f)=c \}
    \end{tikzcd}\end{center}
    射関手を次のように定める.
    \begin{center}\begin{tikzcd}
        C \ar[r] \ar[d, phantom, "\rotatebox{90}{$\in$}"] & \overline{C} \ar[d, phantom, "\rotatebox{90}{$\in$}"]\\
        g:c\to d \ar[r, mapsto] & \overline{g}=g^*:\hom_C(-,c)\to\hom_C(-,d)
    \end{tikzcd}\end{center}
    ただし,この写像$g^*$は,任意の対象$x\in C$に対して,
    \begin{center}\begin{tikzcd}
        \hom_C(x,c) \ar[r] \ar[d, phantom, "\rotatebox{90}{$\in$}"] & \hom_C(x,d)\ar[d, phantom, "\rotatebox{90}{$\in$}"]\\
        f:x\to c \ar[r, mapsto] & g\circ f:x\to d
    \end{tikzcd}\end{center}
    と対応づける写像(関手の射/自然変換)である.
    この関手は可逆であり,逆関手の$\overline{x}\in\overline{C}$成分は射写像は次の通りである.
    \begin{center}\begin{tikzcd}
        \overline{C} \ar[r] \ar[d, phantom, "\rotatebox{90}{$\in$}"] & C \ar[d, phantom, "\rotatebox{90}{$\in$}"]\\
        \overline{g}:\hom_C(-,c)\to\hom_C(-,d) \ar[r, mapsto] & \overline{g}(1_c)
    \end{tikzcd}\end{center}
\end{proof}
\begin{remark*}
    こうして,対象$c$を「$c$に入射する射」に写し,射を「$c$に入射するHom集合同士の自然変換」に写す構成関手$\overline{ }:C\to\overline{C}$を「Cayley表現関手」と呼ぶことにしようか.
    群論でのCayley表現のアナロジーとして,ここでも「表現」という語が,「対象$c$を取り巻く射の動きを定式化することで,$c$の内部構造が漏れ出しているのを捉える」という精神を感じる.

    これが「表現」という術語の出処であろう.この時点ではまだ素朴の意味で「$C$の表現$\overline{C}$」という感覚である.
    また,これが「ホム関手」「ホム集合」という概念の出処でもある.集合での表現を持つから,我々の「具体」性という得意分野に引きずりこめるのだ.
    また,何度も本文内で注意されるが,集合に頼り過ぎないで,純粋に圏論的なまま理論を豊かにしていくのも大事である.(群論だってそうなのだろう).
    例えば,一般の圏を白紙から考えるとき,対象の間の射全体の集まりは「集合」であるとは限らないのだ.

    また,2-圏の起こりにも見える.また戻って来たい.
\end{remark*}

\subsection{Constructions on categories}

前章の終わりに出て来たCayley representationの考え方が,圏論の階層性の萌芽の全てなんじゃないか.
これを分解したような圏の構成法についての言葉を整理する.


\subsubsection{Product}

\begin{itembox}[l]{$C\times D$}
    対象:$(c,d)\in \mathrm{obj}(C)\times \mathrm{obj}(D)$

    射:$(f,g)\in \mathrm{mor}(C)\times\mathrm{mor}(D)$

    合成,単位射:徹底的に「要素毎」の考え方(直積の普遍性)
\end{itembox}

\subsubsection{Opposite}

\begin{itembox}[l]{$C^{op}$}
    対象:同じ

    射:$f:C\to D\in C$に対して,$f^*:D^*\to C^*\in C^{op}$

    合成:順序を逆にしたもの

    $C^{op}=(C,M,t,s,c\circ w,e)$
\end{itembox}.

この時の構成関手${}^{op}$は良く関手を分解するときに用いる.

dualityとは,ある圏が,別の圏の反対(の部分圏)になるという対応が成り立つこと(を主張する命題のこと)である.

\subsubsection{arrow category:lifting}

\begin{itembox}[l]{arrow category $\overrightarrow{C}$}
    対象:射

    射$g:(f:A\to B)\to (f':A'\to B')$:足を結びつける$C$の射と頭を結びつける$C$の射の組$g:=(g_1,g_2)$,すなわち,次の$f'\circ g_1=g_2\circ f$を主張する可換図式
    \begin{center}\begin{tikzcd}
        A \ar[r, "g_1"] \ar[d, "f"'] & A' \ar[d, "f'"]\\
        B \ar[r, "g_2"] & B'
    \end{tikzcd}\end{center}
    合成:成分毎$(h_1,h_2)\circ (g_1,g_2)=(h_1\circ g_1, h_2\circ g_2)$,または,可換図式を繋げて外回りを取ること
    
    単位射:$1_f=(1_A,1_B)$
\end{itembox}

対象は射$f:A\to B$だが,要は$(A,B)$,これはどう考えても$C\times C$あるいは$[2,C]$と同型になる.
\begin{proposition*}[arrow category と product category の関係]
    次の関手が存在する.
\begin{center}\begin{tikzcd}
    C & \overrightarrow{C} \ar[l, "\mathrm{dom}"'] \ar[r, "\mathrm{cod}"] & C
\end{tikzcd}\end{center}
    即ち,対象$f:A\to B\in\overrightarrow{C}$について,その定義域に写す関手と,その終域に写す関手とが,射影に相当する.
\end{proposition*}

\begin{remark*}[nLab "lift"より]
    arrow categoryの射としての可換図式$(g_1,g_2)$を,「$f_1,f_2$間のlifting problem (between $f_1$ and $f_2$)」とも言う.
    \begin{center}\begin{tikzcd}
        A \ar[r, "g_1"] \ar[d, "f"'] & A' \ar[d, "f'"]\\
        B \ar[r, "g_2"] \ar[ur, dashed, "\exists\gamma"] & B'
    \end{tikzcd}\end{center}
    上図を可換にする$\gamma:B\to A'$が存在する時,これをliftと呼び,このlifting problem $(g_1,g_2)$はsolution $\gamma$を持つ,と言う.
    またこのliftが一意的である時,$f$と$f'$は直交する($f\perp f'$)と言う.
    定義\ref{def-projective-object}参照.
\end{remark*}

\subsubsection{slice / over category:頭を共通の対象に突っ込んだ射と,足元の移動}

圏$C$と対象$c\in C$について,
\begin{itembox}[l]{$C/c$}
    対象:$\{ f\in\mathrm{arr}(C)\mid \mathrm{cod}(f)=c \}$
    
    射:2対象$f:x\to c, f':x'\to c$に対して両足を結ぶ$C$の射.つまり,次の$C$の図式を可換にする射$a:x'\to x\in C$($f=f'\circ a$).
    \begin{center}\begin{tikzcd}
    x\ar[rr, "a"] \ar[dr, "f"'] & & x' \ar[dl, "f'"] \\
    &c&
    \end{tikzcd}\end{center}
\end{itembox}
\begin{proposition*}[slice category と arrow category の関係]
    これは,対象を,終域を$c$とする射のみに限ったため,arrow categoryの充満部分圏である.
\end{proposition*}

対象について,そのcodomain $c$を忘れ,射$a:(x,c)\to (x',c)$についても$c$を忘れれば,先ほどの$\mathrm{cod}$に当たるものが,忘却関手$C/c\to C$となる.
(これは一種の具体圏だったのか?)
\begin{proposition*}[slice category と product category の関係]
    次の関手が存在する.
\begin{center}\begin{tikzcd}
    C & \overrightarrow{C} \ar[l, "\mathrm{dom}"'] \ar[r, "\mathrm{cod}"] & C
\end{tikzcd}\end{center}
    即ち,対象$f:x\to c\in\overrightarrow{C}$について,その定義域に写す関手と,その終域に写す関手(=)とが,射影に相当する.
\end{proposition*}

\begin{itembox}[l]{数学原論での導入$C_A$}
    圏$C=(C,M,s,t,c,e)$とその対象$A\in C$に対して,$A$上の圏$C_A=(C_A, M_A, s_A, t_A, c_A, e_A)$を,以下のように定める.

    1. $C_A=\{ f\in M\mid t(f)=A \}$.この,圏$C$における$A$を的とする射$f:B\to A$を「$A$上の対象」という.これを$B$と書いてしまうことも多い.

    2. $M_A=\{(g,k)\in M_s\underset{C}{\times}{}_tM\mid g\in C_A\}$.即ち,$B$を$D$に写すover categoryの射$(g,k)$とは,$f\in C_A$に対して,的である$g\in C_A$と,それと足下で合成可能な射$k$との組で,これは$f=g\circ k\in C_A$を$g\in C_A$を満たすもの,即ち次を可換にするものである.
    この時第二成分である$k$のみをさして「$A$上の射」とも呼ぶ.
    \begin{center}
        \begin{tikzcd}
            B \ar[rr, "k"] \ar[dr, "f"'] & & D\ar[dl, "g"] \\
            & A &
        \end{tikzcd}
    \end{center}

    3. $s_A:M_A\to C_A$は合成$c$の$M_A$への制限.

    4. $t_A:M_A\to C_A$は第一射影.

    5. $c_A$は$c_A((h,l),(g,k))=(h, l\circ k)$
    \begin{center}
        \begin{tikzcd}
            D \ar[r, "k"] & C \ar[r, "l"] \ar[rr, "h\circ l=s{(l,h)}=g", bend left=40] & B \ar[r, "h"] & A
        \end{tikzcd}
    \end{center}

    6. $e_A(f)=(f,1_{s(f)})$で定める.
\end{itembox}
\begin{remark*}
    これまた3.4.は「道」みたいな定め方だな.形式的には結構技巧的なものになっている.
    \begin{quote}
        一般的に幾何的な対象のなす圏を考えているときは$A$上の圏といえば$C_A$であり,代数的な対象を考えているときは$A$上の圏というと$C^A$になる.
    \end{quote}
\end{remark*}

\subsection*{slice category 構成の定める関手}

$C$の射$g:c\to d$に対して,自然変換$g^*:C/c\to C/d$が定まる.
\begin{center}\begin{tikzcd}
    C/c \ar[r] \ar[d, phantom, "\rotatebox{90}{$\in$}"] & C/d\ar[d, phantom, "\rotatebox{90}{$\in$}"]\\
    f:x\to c \ar[r, mapsto] & g\circ f:x\to d \\
    a:(f:x\to c)\to (f':x'\to c) \ar[r] & a:(g\circ f:x\to d)\to (g\circ f':x'\to d)
\end{tikzcd}\end{center}

従って,slice categoryの構成は,関手("composition functor")$C/(-):C\to \mathbf{Cat}$を定める.

これは勝手な圏$C$に対して,関手圏としての表現を与えるCayley表現関手のアナロジー,と思うことが出来る.
Cayley表現関手は,抽象的な群や圏を,「射の集合」として具体化する.slice categoryの構成は,圏を「圏の圏」とする.
というより,これに忘却関手$U:\mathrm{Cat}\to\mathrm{Sets}$を合成した関手であった.
\[ \overline{ } = U\circ C/(-) \]

\begin{remark*}
    composition functorは,Hom関手の射写像である.
    これをこの本ではCayley representationから導入し,圏論としてはまずslice categoryを定義したわけだ.どっちが主軸だろう?

    いや,Hom関手が要はover categoryの構成写像なのか!?
\end{remark*}

\begin{proposition*}[coslice category]
    \[ (-)/C = C/(-)\circ {}^{op} \]
    である.
\end{proposition*}

\begin{example}
    \[\mathrm{Sets}_*\simeq 1/\mathrm{Sets}\]
    となる.何故なら,$\mathrm{Sets}_*$の射$f:(A,a)\to (B,b)$は,圏$1/\mathrm{Sets}$の射$f$と対応し,対象は図式の中の通り,$1$から出る射$a,b$と対応させれば良い.
    \begin{center}
        \begin{tikzcd}
            1 \ar[r, "a"] \ar[dr, "b"'] & A \ar[d, "f"]\\
            & B
        \end{tikzcd}
    \end{center}
    こうして同型が構成できる.
\end{example}

\begin{proposition*}
    $1\in C$を圏$C$に於ける終対象とする.この時,
    \[ C\simeq C/1 \]
\end{proposition*}
\begin{proof}
    対象写像を$a\in C$を,唯一の射$a\to 1$に写す写像とする.
    この後どのように射写像を定めれば,可逆な関手$C\to C/1$を定められるのかが分からない.
\end{proof}

\subsection*{Posに於けるslice category}

\begin{definition*}[principal ideal]
    順序集合$(P,\le)$の部分集合$I$がイデアルであるとは,次の3条件が成り立つことをいう.

    1. $I\ne\varnothing$

    2(lower set). $\forall x\in I\; \forall y\in P\; (y\le x\Rightarrow y\in I)$

    3(directed set). $\forall x,y\in I\; \exists z\in I\; (x\le z\land y\le z)$

    イデアル$I\subset P$が主イデアルであるとは,単元生成されたイデアル($p$を含む最小のイデアル)のことをいう.
    即ち,$\downarrow(p)=\{q\in P\mid p\le p\}$である.
\end{definition*}
\begin{remark*}
    2.の形の条件は群論の時点から見たことがある.idealは元々抽象代数からの借入語である.

    directed setは有向集合と呼ばれる.前順序集合(preorder)のうち,どの2元も\underline{上界}を持つものをいう.即ち,半束ならば有向集合だが,有向集合だからと言って半束であるとは限らない(上界は複数あっても上限が存在するとは限らない).
\end{remark*}

\begin{proposition*}[principal ideal]
    poset category $P$について,
    \[ P/p \simeq \downarrow (p) \]
\end{proposition*}

\subsection{Free categories}

\subsubsection{Free monoid}

\begin{definition*}[free functor from Mon]
    自由関手$M:\mathrm{Sets}\to\mathrm{Mon}$とは,集合$A$から,$A^*=W(A):={}^{<\omega}\!A$を台集合として,concatenation演算子$*$を積とし,空列$-$を単位元としたモノイド$(A^*,*,-)$に対応させる関手である.
\end{definition*}
\begin{remark*}
    $A=\varnothing$の時$M(A)$は自明なモノイドであり,$A=1$の時,一進法表記した$M(A)=\mathbb{N}$である.

    $A=\mathbb{N}$とした場合,集合上では$\mathbb{N}\simeq U(M(\mathbb{M}))$であるが,モノイドとしては全く違う.
\end{remark*}

\begin{definition*}[UMP of "freeness"]
    $M(A)$が集合$A$から生成される自由モノイドである(即ち,関手$M:\mathrm{Sets}\to\mathrm{Mon}$が自由関手である)とは,次の条件を満たすことである.

    任意のモノイド$N\in\mathrm{Mon}$と,それとの任意の写像$f:A\to U(N)$に対応して,モノイドの射$\overline{f}:M(A)\to N$が唯一つ存在して,次の図式を可換にする$i:A\to U(M(A))$が存在する.
    \begin{center}
        \begin{tikzcd}
            U(M(A)) \ar[r,dashed, "U(\overline{f})"] & U(N)\; (\mathrm{on}\hspace{0.5em}\mathrm{Sets}) & M(A) \ar[r,dashed, "\overline{f}"] & N \; (\mathrm{on}\hspace{0.5em}\mathrm{Mon}) \\
            A \ar[u,dotted, "i"] \ar[ur, "f"']
        \end{tikzcd}
    \end{center}
    論理構造を明確にするために,この条件を,集合とモノイドの組$(A,M(A))$に対する条件として形式化すると,
    \[ \exists i:A\to U(M(A))\; \forall f:A\to U(N)\; \exists! \overline{f}:M(A)\to N \; (f=U(\overline{f})\circ i) \]
    \rightline{$\blacksquare$}
\end{definition*}

\begin{proposition}
    勝手な集合$A$について,$A$上の自由モノイド$M(A)=(A^*,*,-)$は,上の普遍性を満たす.
\end{proposition}
\begin{proof}
    $A$を勝手な集合,$N=(N,\cdot,u)$を勝手なモノイド,$f:A\to U(N)$を勝手な写像とする.
    一意的な$\overline{f}:M(A)\to N$と,$i:A\to M(A)$とを順に構成し,$f=U(\overline{f})\circ i$を満たすようにできることを示せばいい.
    $f$を用いて,写像$\overline{f}$を
    $$\begin{cases}
        \overline{f}(-) &= u \\
        \overline{f}(a_1,\cdots,a_i) &= f(a_1)\cdot\cdots\cdot f(a_i)\;\;\; (a_1,\cdots,a_i\in A)
    \end{cases}$$
    で定めると,確かにこれはモノイドの射である.
    これに続いて,$i:A\to U(M(A))$を包含写像とすると,
    \begin{eqnarray*}
        f(a)&=& U(\overline{f})\circ i(a)\;\;\; (\forall a\in A) \\
        &=& U(\overline{f})(a)
    \end{eqnarray*}
    が成り立つ.ここで,$\overline{f}$は一意的に定めることができているので,確かに$M(A)$はUMPを満たす.
\end{proof}
\begin{remark*}[自由モノイドの普遍性として妥当な直観的理由]
    $M(A)$の生成系$A$のそれぞれについて行き先を定める写像$f:A\to U(N)$から,モノイドの射$\overline{f}:M(A)\to N$を一般的に(=関手的に),そして一意的に構成可能である.
    次元というか,軌道のようなものは潰れるかもしれないが.
    この時,足元は生成系$A$から,その上の自由モノイド$M(A)$にまで持ち上がる.
    それに対応するのが包含写像$i:A\to M(A)$である.
    このような性質を持つのが,生成系$A$を指定した時に付随して確定するモノイド$M(A)$のことで,このようなものはとても標準的で自然なものだから,とりあえず「自由」と呼ぶ.が,このように,背後には関手がある.

    この時,モノイドの射$\overline{f}$の存在性が"no noise"に対応し(純粋なモノイド性のみを持ち,それ以上の,公理に含まれない非自明な関係は一切持たない.だから,
    任意のモノイド$N$に対してこのような射が存在する.),一意性が"no junk"に対応する(生成元に関しての閉包であり,生成元とモノイドの公理には行き先が指定されておらず,自由度が残っているような元は全く含まれていない).
\end{remark*}

\begin{proposition}[UMPが対象を同型を除いて一意に定める]
    $M,N\in\mathrm{Mon}, i:A\to U(M), j:A\to U(N)$を写像とし,$(A,M),(A,N)$はいずれも自由モノイドの普遍性を満たすとする.この時,$M\sim N$である.
\end{proposition}
\begin{proof}
    $(A,M)$の普遍性に対して$f=j$を適用し,$(A,N)$の普遍性に対して$f=i$を適用すれば,互いに逆射となるモノイドの射$\overline{i}, \overline{j}$を得る.
\end{proof}
\begin{example*}
    $M(1)\simeq\mathbb{N}$として,UMPにより同型を除いて一意に定まる.
\end{example*}

\subsubsection{Free Category}
今回はグラフの道として定式化するが,analogousな定式化は他にも存在する.それらをまとめてpath categoryと呼ぶ.

\begin{definition*}[有向グラフ]
    集合$V(G),E(G)$とその間の写像$s:E(G)\to V(G), t:E(G)\to V(G)$の4つ組$G=(V(G), E(G), s,t)$を有向グラフと言う.
    \begin{center}
        \begin{tikzcd}
            E \ar[r, "s"] \ar[r, "t"] & V 
        \end{tikzcd}
    \end{center}

    頂点の有限列$(e_1,\cdots,e_n)\in {}^{<\omega}\! E(G)$であって,$t(e_i)=s(e_{i+1})\; (i=1,\cdots,n-1)$を満たすものを道という.
\end{definition*}

\begin{definition*}[有向グラフ上の自由圏]
    有向グラフ$G$に対して,$G$から生成される自由圏$\mathbf{C}(G)$を,$C(G)=(V, \mathrm{path}(E),\mathrm{dom}, \mathrm{cod}, \circ ,e)$定める.

    \begin{center}
        \begin{tikzcd}
            M_s\underset{C}{\times}{}_tM \ar[r, "\circ"] & M \ar[r, "\mathrm{dom}"] \ar[r, "\mathrm{cod}"' near start] & C \ar[l, "e" near start]
        \end{tikzcd}
    \end{center}

    1. 対象は頂点とする.

    2. 射は,$G$の道とする.$(e_1, \cdots, e_n)$を道としたとき,射を$e_n\cdots e_1$と書くこととする.

    3. 合成は道の結合とする.あるいは,文字列$e_n\cdots e_1$の結合と考えても良い.

    4. 各頂点$v$に対して,単位射$1_v$を考える.
\end{definition*}
\begin{example*}
    $V(G)=1$であった場合,$C(G)$は一点対象圏であり,$E(F)$上の自由モノイド(に付随する圏)と同型になる.
\end{example*}

\begin{definition*}[有向グラフ準同型]
    グラフの射$h:G\to H$とは,次を可換にする写像の組$(h_0, h_1)$である.
    \begin{center}
        \begin{tikzcd}
            G_1 \ar[r, "s"] \ar[d, "h_1"'] & G_0 \ar[l, "t"] \ar[d, "h_0"] \\
            H_1 \ar[r, "t"'] \ar[r, "s"] & H_0
        \end{tikzcd}
    \end{center}
\end{definition*}

\begin{definition*}[圏の忘却関手]
    忘却関手$U:\mathrm{Cat}\to \mathrm{Graphs}$を
    \begin{center}\begin{tikzcd}
        M_s\underset{C}{\times}{}_tM \ar[r, "\circ"] & M \ar[r, "\mathrm{dom}"] \ar[r, "\mathrm{cod}"' near start] & C \ar[l, "e" near start]
    \end{tikzcd}\end{center}
    を,台グラフ
    \begin{center}\begin{tikzcd}
        M \ar[r, "\mathrm{dom}"] \ar[r, "\mathrm{cod}"'] & C
    \end{tikzcd}\end{center}
    に写す,6組$(C,M,s,t,c,e)$を4組$(C,M,s,t)$に情報を落とす行為だとみなせる.
\end{definition*}
\begin{remark*}
    一般に忘却関手は「集合の付加構造を落とす」関手であるが,圏自体の忘却関手も,圏論的な定式化を使えば,全く同じような議論「グラフの付加構造を落とす」ものとして理解できる.

    一般に,射をそのままedgeと見做してしまっているので,自然な有向グラフとは程遠く(推移性が無駄に残っている),$C$と$U$は互いに逆関手ではない.
    $U$の左随伴が自由圏構成関手$C$である.
\end{remark*}

\begin{definition*}[自由圏の普遍性]
    \[ \exists i:G\to U(C(G))\; \forall h:G\to U(D) \; \exists !\overline{h}:C(G)\to D\; (U(\overline{h})\circ i=h) \]
\end{definition*}
\begin{example*} 

    1. グラフ$(1, E, \Psi)$上の自由圏は単一対象圏となる.

    2. グラフ$(2, \{0\to 1\})$上の自由圏は,finite category $2$と同型.

    3. グラフ\begin{center}
        \begin{tikzcd}
            A \ar[r, "e"] & B \ar[l, "f"]
        \end{tikzcd}
    \end{center}上の自由圏は,無限個の道が存在するために,無限個の射が存在する
\end{example*}

\subsection{Foundations: large, small, and locally small}

\begin{definition}[finite, small, locally small, large]
    圏$(C,M,s,t,c,e)$が有限であるとは,$C,M$が有限集合であることをいう.
    小さいとは,$C,M$が集合であることをいう.
    $C,M$のいずれかでも集合ではない場合,大きいという.
\end{definition}

\begin{definition}[hom-setとlocally small]
    $\hom_C(X,Y)=\{ f\in C_1\mid f:X\to Y \}$が集合であるとき,局所的に小さいという.圏が小さい場合は局所的にも小さい.
\end{definition}
\begin{remark*}
    圏を考える際,「$\bigcirc\bigcirc$全体のなす圏」など,大抵は言及が雑すぎるので,その圏は大抵小さくはならない.しかし,対象の集合$C$は(従って$M$も)集合にはならずとも,対象を集合とした圏Sets, Pos, Top, GroupにおいてHom集合は大抵集合になる.

    その主な理由は,条件を無条件に化しただけでは(例えば群の公理など)そのメンバーが集合になるとは限らないから,クラスについての知見が必要になるからである.
    例えば[有限集合]FinSetも小さくない.全ての集合$X$について$\{X\}$とすれば有限集合を作れるからである.同様の理由で[小さな圏]Catも自身は圏として小さくない.

    しかし,[遺伝的有限集合]SetsFinとすれば小さくできる.$C=V_\omega$はZFCの下では集合である.
    即ち,FinSetは小さな圏と同値であるため,「本質的に小さい」と言える.
\end{remark*}


\begin{remark}
    Posetとして見た$\mathbb{R}$は小さい圏だが,具体圏ではない.(structured setではないから?)

    Posは具体圏だが小さくはない.
\end{remark}

\begin{remark*}[Grothendieck宇宙と到達不可能基数は同じ!? nLab "finite category"]
    (Locally) finite categories may also be called (locally) $\omega$-small; this generalises from $\omega$ (the set of natural numbers) to (other) inaccessible cardinals (or, equivalently, Grothendieck universes).
\end{remark*}

\subsection{Exercises}

\section{Abstract structures}
圏論的な言葉だけで,圏の対象や射を特徴付ける性質を述べる.
このような性質のことをひとまずabstruct characterizationと呼ぶこととする.UMPはその良い例である.

\subsection{Epis and monos:代数的に定めた,圏論的単射・全射の条件}

\subsubsection{まずは射の代数の話をしよう.}

単射は左に付いても単位的な働きしかせず,全射は右に付いても単位的な働きしかしない,まるで筒抜け,という
写像の合成における振る舞いのみを抽出して,次の圏論的概念を定める.
\begin{definition}[圏論的単射・全射]射$f:A\to B\in C$について

    1. 次(左簡約可能条件)を満たすとき,monomorphismという.
    \[ \forall C\in C\;\;\; \forall g,h:C\to A\;\;\; (fg=fh\Rightarrow g=h) \]
    \begin{center}
        \begin{tikzcd}
            C \ar[r, "g"] \ar[r, "h"'] & A \ar[r, "f"] & B
        \end{tikzcd}
    \end{center}

    2. 次(右簡約可能条件)を満たすとき,epimorphismという.
    \[ \forall D\in C\;\;\; \forall i,j:B\to D\;\;\; (if=jf\Rightarrow i=j) \]
    \begin{center}
        \begin{tikzcd}
            A \ar[r, "f"] & B \ar[r, "i"] \ar[r, "j"'] & D
        \end{tikzcd}
    \end{center}
\end{definition}

\begin{proposition}[Setsでは全射も全写も同値である]
    $f\in \mathrm{Sets}$に於ては,次は同値.

    1. $f$はmonicである.

    2. $f$は単射である.
\end{proposition}
\begin{proof}
    1.$\Rightarrow$2.について.
    $f:X\to Y$をmonicとすると,特に$g,h:1\to X$について,$g\ne h\Rightarrow fg\ne fh$である.
    ここで,$1$は一点集合であるから,$g\ne h\Leftrightarrow g(0)\ne h(0)$,$fg\ne fh\Leftrightarrow f(g(0))\ne f(h(0))$が成り立つことに注意すると,これは$f$が単射である条件である.

    2.$\Rightarrow$1.について.
    $f:X\to Y$を単射として,勝手な$Z$について$g,h:Z\to X$を取る.ここで$g\ne h$を仮定する.すると,$g(c)\ne h(c)$を満たす$c\in C$が存在するから,その$c$について$f\circ g(c)\ne f\circ h(c)$より,$fg\ne fh$を得る.
\end{proof}

\begin{example}[具体圏に於けるmonic射は台写像もSets上でmonicである.]
    具体圏において,monosは,「単射な準同型」に一致する.これは,具体圏中の自由対象の普遍性から示せる性質である.
    上記の証明で,$A$の元と射$1\to A$とを同一視した行為が,UMP of $M(1)$により,具体圏の中にも同じ状況が写される,パラレルな議論が展開できる.
    従って,具体圏のmonosは,Setsでもmonicである.

    これは,$1$という存在が特殊なんだと見方が変えられる.
\end{example}

\begin{proposition*}次の2条件は同値である.

    1. Monにおいて,射$f:M\to N$がmonicである.
    
    2. 台写像$U(f):U(M)\to U(N)$はモノイドの射$f:M\to N$を定め,またSete上でmonicである.
\end{proposition*}
\begin{proof}
    1.$\Rightarrow$2.は成り立つ.

    2.$\Rightarrow$1.を考える.
    まず写像$x,y:1\to U(M)$を$x\ne y$となるように取る.取れない場合は自明に$f$は左簡約可能になる.
    すると,自由モノイド$M(1)$の普遍性により,モノイドの射$\overline{x},\overline{y}:M(1)\to M$が$\overline{x}\ne\overline{y}$を満たしながら一意的に存在する.
    この時,$U(f)$はmonicであるとしたから,$x\ne y$より,$U(f)x\ne U(f)y$である.これと再び自由モノイド$M(1)$の普遍性より,$f\overline{x}\ne f\overline{y}$を満たす射$M(1)\to N$が存在する.
    よって,射$f$はmonicである.
\end{proof}

\begin{example}[Posに於ける状況は退化してしまっている.Posでは射の存在性の$0,1$コードしか大事じゃなく,合成は全く使わない次元だからであろう.]
    Posにおいて,射$p\le q$はmonicかつepicである.何故なら,Hom-setの元は高々1つであるから.$g=h$とならないような$g,h:r\to p$は取れない.
\end{example}

\begin{example}[具体圏でepicな射は,Sets上でもepicとは限らない.]\label{example-epis-notnecessarily-surjective}
    次の命題は,Sets上でepic(=surjective)ではないのに,Mon上ではepicである例である.

    そうすると,Mon上でmonicかつepicでも,台写像は全単射ではない可能性がある.ここらへんから,次の命題\ref{prop-iso_is_monoepi}の逆は確かに成り立たないことがわかる.
    \begin{proposition*}
        包含写像(monicな集合の射)$i:(\mathbb{N},+,0)\to(\mathbb{Z},+,0)$はepicなモノイドの射であるが,特に全射(集合の射としてepic)ではない.
    \end{proposition*}
    \begin{proof}
        $i$がMon上でepicであるとは,$i$の右簡約則,即ち,$f,g:(\mathbb{Z},+,0)\to (M,*,u)$について,$f|_\mathbb{N}=g|_\mathbb{N}\Rightarrow f=g$を示せば良い.

        いま,$f|_\mathbb{N}=g|_\mathbb{N}$とすると,実は$f(-1)=g(-1)$を示せば,各$f(-i)=g(-i)\; (i=2,3,4,\cdots)$も得る.
        \begin{eqnarray*}
            f(-1) &=& f(-1)*g(0) \\
            &=& f(-1)*g(1-1) \\
            &=& f(-1)*g(1)*g(-1) \\
            &=& f(-1)*f(1)*g(-1) \\
            &=& f(0)*g(-1) \\
            &=& g(-1)
        \end{eqnarray*}
        $i$は包含写像であり,$\mathbb{Z}\setminus\mathbb{N}\ne\varnothing$だから,明らかに集合の全射ではない.
    \end{proof}
\end{example}

\subsubsection*{同型射との関係}

可逆性によって同型射を,簡約可能性によってmonic, epic射を定義した.両側簡約可能だからと言って可逆だとは限らないが,逆は勿論成り立つ.
群論の初歩でも似たような議論があるように,非常に代数的に定義している,まるで射の代数である.
\begin{proposition}[可逆な射は簡約可能である]\label{prop-iso_is_monoepi}
    同型射は,monicかつepicである.
\end{proposition}
\begin{proof}
    $m:B\to C$は逆射を$e:C\to B$とする同型であるとする.すると,$mx=my$と仮定すれば,$x=(em)x=e(mx)=e(my)=y$を導けるから,$m$はmonicである.
\end{proof}

\clearpage

\subsubsection{Sections and retractions:右逆元と左逆元の対}
可逆性は簡約可能性より強い概念になる.従って,全てのsectionはmonicだし,全てのretractionはepicである.
\vspace{1cm}
\begin{definition}[左右可逆性をsplitで表す,右逆元をsection,左逆元をretractionと呼ぶ.] 

    1. 左(右)逆射を持つ射を\textbf{split} mono(epi)と呼ぶ.

    2. $es=1_A$を満たす射$s:A\to X, e:X\to A$について,$e$に対する右逆元$s$をsectionまたはsplittingとよび,$s$に対する左逆元$e$をretractionと呼ぶ.
    この時に引き戻される所の対象$A$を$X$のretractと呼ぶ.
    \begin{center}
        \begin{tikzcd}
            A \ar[r, "s"] \ar[dr, "1_A"'] & X\ar[d, "e"] \\
            & A
        \end{tikzcd}
    \end{center}
    \rightline{$\blacksquare$}
\end{definition}
\vspace{1cm}
\begin{remark*}
    可逆性については,関手は単位射や合成を保存するので,mono/epiのsplitablilityも保存する.
    これは例\ref{example-epis-notnecessarily-surjective}で,忘却関手$U:\mathrm{Mon}\to\mathrm{Sets}$が,splitじゃないepiである包含写像$i:\mathbb{N}\to\mathbb{Z}$を保存しなかった(Sets上ではepicではなくなった)例と対照的である.
\end{remark*}

\vspace{3cm}

\begin{example}[Setsでの様子と,選択公理]
    Sets上では,
    
    1. $\varnothing\to A$という種のものを除いて,monoである(左簡約可能)こと,split monoである(左可逆)こと,単射であることは全て同値.
    
    2. epicである(右簡約可能)こと,split epicである(右可逆)こと,全射であることは全て同値.

    \vspace{3cm}
    なお,Setsにおいて「全射ならば右逆元が存在する」という方向の条件は,選択公理と同値である.
    
    実際,$e:E\to X$を全射とする.即ち,各fiber $E_x=e^{-1}(\{x\})$は空でない.このとき,$es=1_X$を満たす切断$s$が存在するとは,$E$上の集合の族$(E_x)_{x\in X}$から,各
    $x$について,$s$の値とすべき$s(x)\in E_x$を選び出すことに等しい.

    \vspace{3cm}
    逆に,空でない集合の族$(E_x)_{x\in X}$が与えられたとき,$E=\{(x,y)\in X\times \bigcup_{x\in X}E_x\mid x\in X, y\in E_x\}$と定めて,この$E$からの全射$e:E\to X$を第一射影(添え字集合への射影)として定めれば,再び,これの切断$s:X\to E$は選択関数である.
    $e\circ s=1_X$を満たす切断$s$が存在するということは選べているということになる.

    これらは全て,集合の族$(E_x)_{x\in X}$という対象が,射$e:E\to X$として圏論的にも翻訳されていることによる.(This has much wider applications...)
\end{example}
\vspace{3cm}

\begin{definition*}[射影的対象]\label{def-projective-object}対象$P$がprojectiveであるとは,

    次の条件
    \[ \forall f:P\to X, e:E\twoheadrightarrow X \;\;\;\;\; \exists \overline{f}:P\to E\;\;\;\;\; (e\circ\overline{f}=f) \]
    \begin{center}
        \begin{tikzcd}
            & E \ar[d,twoheadrightarrow, "e"] \\
            P \ar[ur,dotted,  "\overline{f}"] \ar[r, "f"'] & X
        \end{tikzcd}
    \end{center}
    が成り立つことをいう.
    即ち,射影的対象から出る任意の射$f:P\to X$は,全射$e:E\to X$を通じて分解する.
    この条件(left lifting propoerty against epimorphisms)を"$f$ lifts across $e$"とも表現する.また,$\overline{f}$をliftとも言う.
\end{definition*}
\vspace{1cm}
\begin{remark*} 
    
    1. $f$として,$f=id_P:P\to P$を取り,epiとして$e:E\to P$を取ると,存在が保証される所の$\overline{f}:P\to E$とは切断に他ならないから,射影的対象に入射する全てのepiはsplitする.
    \begin{center}
        \begin{tikzcd}
            & E \ar[d, twoheadrightarrow, "e"] \\
            P \ar[ur, dashed, "\overline{f}"] \ar[r, "id_P"'] & P
        \end{tikzcd}
    \end{center}

    \vspace{1cm}

    2. "Projective objects may be thought of as having a more "free" structure, thus permitting "more arrows.""
    
    \vspace{1cm}

    3. 次の命題が成り立つ.
    \begin{proposition*}
        選択公理を認めるならば,全てのSetsの対象は射影的である.
    \end{proposition*}
    \begin{proof}
        $P$を勝手な集合とする.各$x\in X$について,$e^{-1}(\{x\})=E_X(\ne\emptyset)$と置く.選択公理より,各$s(x)\in E_x$を選び出す切断$s:X\to E$が存在する.

        $P=\bigcup_{x\in X}f^{-1}(\{x\})$と分割できるが,$f^{-1}(\{x\})\subset P$が空でないなら,その各元に$s(x)\in E_X$を対応させることで,次の図式を可換にする$\overline{f}:P\to E$が構成できる.
        \begin{center}
            \begin{tikzcd}
                & E \ar[d,twoheadrightarrow, "e"] \\
                P \ar[ur,dotted,  "\overline{f}"] \ar[r, "f"'] & X
            \end{tikzcd}
        \end{center}
        よって,$P$は射影的対象である.
    \end{proof}

    \vspace{3cm}

    4. 次の命題が成り立つ.
    \begin{proposition*}
        全ての圏$C$について,射影対象$P$のretract $A$は射影対象である.
    \end{proposition*}
    \begin{proof}
        いま,$A$は$P$のretractだから,$e\circ s=1_A$とし,$X,Y$を勝手な対象,$f:A\to X$を勝手な射,$g:Y\twoheadrightarrow X$を勝手なepi射とする.
        \begin{center}
            \begin{tikzcd}
                Y \ar[r, "g"] & X \\
                A \ar[u, dotted, "\overline{f\circ e}\circ s"] \ar[ur, "f"] \ar[r, "s"] & P \ar[ul, dashed, "\overline{f\circ e}"'] \ar[u, "f\circ e"'] \ar[l, "e"]
            \end{tikzcd}
        \end{center}
        $P$は射影的対象だから,$f\circ e=\overline{f\circ e}\circ g$を満たす射$\overline{f\circ e}:P\to Y$が存在する.これの両辺に右から$s$との合成をとると,$f\circ e\circ s=f=\overline{f\circ e}\circ g\circ s$である.
        
        従って,条件を満たす射$\overline{f\circ g}\circ s$が存在し,$A$も射影的対象である.
    \end{proof}

    \vspace{3cm}
\end{remark*}

\subsection{Initial and terminal objects}

\begin{definition}
    圏$C$において,

    1. 任意の対象$C$について,$\hom(0,C)$が一元集合である時,$0\in C$を始対象と言う.

    2. 任意の対象$C$について,$\hom(C,1)$が一元集合である時,$1\in C$を終対象と言う.
\end{definition}

\begin{proposition}
    始対象は同型を除いて一意である.
\end{proposition}
\begin{proof}
    $1,1'\in C$がいずれも始対象であるとする.
    この時,$\hom(1,1'),\hom(1'1)$はいずれも一元集合で,$\hom(1,1),\hom(1',1')$もいずれも一元集合だから,その元は互いに逆射である.
    従って,$1\simeq 1'$.
\end{proof}
\vspace{3cm}

\begin{example} 

    1. Setsにて,空集合は始対象であり,一点集合が終対象である.
    この時,空集合全体の集合は1つであるが,一点集合全体の集合は唯一つではない.
    従って,$\mathrm{Set}\simeq\mathrm{Set}^{op}$は成り立たず,一般に関手${}^{op}$は同型にならない.

    \vspace{1cm}

    2. 同様にして,Catでは,空圏$0$が始対象であり,$1$が終対象である.

    \vspace{1cm}

    3. Grp, Mon, Vectでは,自明な群が始対象でも終対象でもある.

    \vspace{1cm}

    4. Ringsでは$\mathbb{Z}$が始対象であり,零環が(環とするなら)終対象である.

    \vspace{1cm}

    5. Poset categoryでは,始対象は存在すれば最小元であり,終対象は存在すれば最大元である.
    これらを持つようなPosetを有界という.特に,Boolean algebraは双方を持つ.

    \vspace{1cm}

    6. ブール代数の圏BAでの始対象は$2$,終対象は$1$である.

    \vspace{1cm}

    7. slice category $C/X$において,対象$1_X:X\to X$は終対象である.
    slice category の対象とは$X$に差し込む$C$の射$f$のことだったから,この場合$\hom(f,1_X)=\{f\}$となる.
    \begin{center}
        \begin{tikzcd}
            Y \ar[rr, "f"] \ar[dr, "f"'] & & X \ar[dl, "id_X"] \\
            & X &
        \end{tikzcd}
    \end{center}
\end{example}
\vspace{1cm}

\begin{definition*}
    ブール代数とは,次を満たす6組$B=(B,\land,\lor,0,1,\lnot)$である.

    \[ 0\le a \]
    \[ a\le 1 \]
    \[ a\le c\;\land\; b\le c\Leftrightarrow a\lor b\le c \]
    \[ c\le a \;\land\; c\le b \Leftrightarrow c\le a\land b \]
    \begin{equation}\label{formula-BooleanAlgebra-introduction-of-not}
        a\le\lnot b\Leftrightarrow a\land b=0
    \end{equation}
    \begin{equation}\label{formula-BooleanAlgebra-excluded-middle}
        \lnot\lnot a=a
    \end{equation}
\end{definition*}
\vspace{1cm}
\begin{example*}
    冪集合$P(X)$は,$\land=\cap, \lor=\cup, 0=\varnothing, 1=X, \lnot=X\setminus -$として,束どころか,Boolean algebraをなす.
    特に$X$が一点集合のときは真理値$2=\{0,1\}$である.$X$が空集合のときは自明なブール代数$\{0=1\}$となる.
\end{example*}
\vspace{1cm}

\subsection{Generalized elements}

\subsubsection*{始対象への射}

\begin{example*}
    1. Sets, Poset category では,対象$A$から始対象$0$への射$A\to 0$が存在する時,$A= 0$である.

    \vspace{1cm}

    2. Mon, Groupsでは,始対象と終対象は一致するので,全ての対象$A$について射$A\to 0$も一意に取れる.
\end{example*}

\vspace{1cm}

\begin{definition*}[ultrafilter]
    部分集合$F\subset B$がブール代数$B$のfilterであるとは,

    1(closed upward). \begin{equation}\label{formula-BAfilter-closed-upward}
        a\in F\;\land\; a\le b\Rightarrow b\in F
    \end{equation}

    2(closed under meets). \begin{equation}\label{formula-BAfilter-closed-under-meets}
        a\in F\;\land\; b\in F\Rightarrow a\land b\in F
    \end{equation}

    を満たすことである.$F$を真に包含するfilter $F'$が$F'=B$しか存在しない時,filter $F$を極大(maximal)であるという.またこの$F$をultrafilterともいう.
\end{definition*}

\vspace{1cm}

\begin{proposition*}[ブール代数上のultrafilterの同値な条件]
    ブール代数$B$と真のフィルター$F\subset B$について,次の2条件は同値である.

    1. $F$はultrafilterである.

    2. 任意の$b\in B$について,$b\in F$か$\lnot b\in F$かのいずれか一方である.
\end{proposition*}
\begin{proof}
    1.$\Rightarrow$2.を考える.$b\le b$と式\ref{formula-BooleanAlgebra-introduction-of-not}($a\le c\;\land\; b\le c\Leftrightarrow a\lor b\le c$)を併せると,$b\land\lnot b=0$を得る.
    従って,$b\in F\;\land\; \lnot b\in F$とすると,$0\in F$となって,超filter $F$が真のフィルターであることに矛盾する.
    一方で,$b,\lnot b\notin F$と仮定すると,$b/B=\uparrow (b)=\{a\in B\mid b\le a\}$は真のfilter($0\notin b/B$)であり,$F$よりも真に大きい.従って,$F$が超filterであることに矛盾.
    よって,$b\in F$と$\lnot b\in F$のいずれか一方である.
    
    2.$\Rightarrow$1.を考える.$F$を真のフィルターとして,任意の$b\in B$について,$b\in F$か$\lnot b\in F$かのいずれか一方であるとする.
    $b\in B\setminus F$を任意にとり,これを含む$F$より大きい真のフィルター$F'$を考えると,$b\notin F$より,$\lnot b\in F$であるから,$b,\lnot b\in F'$となる.すると,フィルターの定義より$b\land \lnot b=0\in F'$となるから,これは$F'=B$である.
    従って,フィルター$F$は極大である.
\end{proof}
\vspace{3cm}

\begin{proposition*}
    ブール代数の射$p:B\to 2$は,$B$上の超フィルターと一対一に対応する.
    \[ \mathrm{Hom}_{\mathrm{BA}}(B,2)\simeq \mathrm{St}(B) \]
\end{proposition*}
\begin{proof}
    写像$U:\hom_{\mathrm{BA}}(B,2)\to \mathrm{St}(B)$を,各$p:B\to 2$に対して$U_p:=p^{-1}(1)$と置く.
    こうして定めた$U_p$は超filterになって居る.何故ならば,次の3条件が満たされるからである.
    
    1. $a\in U_p,a\le b$とすると,$p$は関手だから,$1=p(a)\le p(b)$となる.$1$は最大元だから$p(b)=1$より,式\ref{formula-BAfilter-closed-upward}を満たす.
    
    2. $a,b\in U_p$とすると,$p(a)=p(b)=1$で,$p(a)\land p(b) = 1\in U_p$であるから,式\ref{formula-BAfilter-closed-under-meets}を満たす.

    3. 性質「任意の$b\in B$について,$b\in U_p$か$\lnot b\in U_p$かのいずれか一方である.」も,$b\in U_p$と仮定すれば,$p$はブール代数の射だから$p(\lnot b)=\lnot p(b)=0$より$\lnot b\notin U_p$で,逆も成り立つので成り立つ.

    逆に,写像$p:\mathrm{St}(B)\to\hom_{\mathrm{BA}}(B,2)$を,
    超フィルター$U\subset B$に対して,射$p_U:B\to 2$($p_U(b)=1:\Leftrightarrow b\in U$)を対応させる写像として定める.超フィルター$U$に$b$か$\lnot b$のいずれかが入って居ることにより,$p_U$は構造$\lnot$を保ち,確かにブール代数の射になる.

    こうして定めた写像$U,p$は互いに逆写像になって居る.
\end{proof}
\vspace{3cm}
\begin{remark*}
    1. ブール代数の射$B\to 2$は,真理値表の行1つに対応する.例えば$B$を$P(2)$から生成したものとすると,
    \begin{center}
        \begin{tikzcd}
            & 1 & \\
            a\ar[ur] & & \lnot a \ar[ul] \\
            & 0 \ar[ul] \ar[ur] &
        \end{tikzcd}
    \end{center}
    
    2つの超フィルターが,$b:=\lnot a$とかけば,それぞれ次の行に対応する.
    \[\begin{array}{|l|c||r|}\hline
            a & b & a\lor b  \\ \hline
            0 & 1 & 1  \\\hline
            1 & 0 & 1  \\ \hline
    \end{array}\]

    \vspace{1cm}
    2. 以上の議論と類比的なことがRingsでの,始対象への射$A\to\mathbb{Z}$で起こる.これが対応するのは超フィルターの代わりに,prime idealと呼ばれる.
\end{remark*}
\vspace{1cm}

\subsubsection*{終対象からの射}

\begin{example} 

    1. Setsにて,$X\simeq\mathrm{Hom}_{Sets}(1,X)$
\vspace{1cm}

    2. Pos Topにて,$\mathrm{Hom}_{Pos}(1,P)$は,$P$の台集合$U(P)$に対応する.
    \vspace{1cm}

    3. 一般の圏$C$において,$\mathrm{Hom}_C(1,A)$の元を,$A$のglobal elements, points, constantsなどという.
\vspace{1cm}

    4. Sets, Poset category, Topにて,全ての点$x:1\to P$において$fx=gx$が成り立つことと,$f=g$であることは同値である.
    \vspace{1cm}

    5. Monにて,$\mathrm{Hom}_{Mon}(1,M)$は「$0\in 1$を$M$の単位元$u_M$に対応させる射$1\to M$」の一点集合である.従って,任意の射$h,j:M\to N$に対して,全ての(1つしかないが)$x:1\to M$について,$hx=jx$である.Monoids do not "have enough points."
    \vspace{1cm}

    6. BAにて,$\mathrm{Hom}_{BA}(1,B)\; (B\ne 1)$は空集合である.実際,$f:1\to B$をその元とし,$f(0)=b\in B$と置くと,$f$は一項演算$\lnot$の構造を保つために$f(\lnot 0)=\lnot b\ne b$が必要だが,これは$0=\lnot 0\in 1$による$f(0)=f(\lnot 0)$と両立しない.
\end{example}
\vspace{3cm}

\begin{definition*}
    対象$A$に対して,勝手な対象$X$からの射$x:X\to A$を,$A$のgeneralized elementまたはvariable elementという.
    特に$X=1$の時,global element, points, constantsなどという.
\end{definition*}
\vspace{1cm}
\begin{remark*} 

    1. Computer scientistsとlogiciansは,射$1\to A$を定数や閉項とし,一般の射$X\to A$を任意の項とする.
\vspace{1cm}

    2(Good for testing). $f:A\to B$がmonicであるとは,任意の$x,x'\in\hom(X,A)$について,$x\ne x'\Rightarrow fx\ne fx'$であることだが,これは「$f$がmonicとは,一般化された元について"単射"であることである」と言い換えられる.
    \vspace{1cm}

    3. $C$の図式が可換$\alpha f=\beta g$であるとは,全ての一般化された元$x$について$\alpha fx=\beta gx$であるということである.(元の場合は$x=1_A$の場合に当たる.)
    \vspace{1cm}
\end{remark*}
\begin{proposition*}[射の相等の特徴付け] 
    
    全ての圏$C$における任意の射$f,g:C\to D$について,次の2条件は同値である.
    
    1. $f=g$である.

    2. $\forall X\in C\; \forall x\in\mathrm{Hom}_C(X,C)\; (fx=gx)$
\end{proposition*}
\begin{proof}
    1.$\Rightarrow$2.は明らかだから2.$\Rightarrow$1.を示す.特に$X=C$として,$x=id_C$とすると,$f\circ id_C=g\circ id_C$より$f=g$が従う.
\end{proof}
\vspace{3cm}

\begin{example}[一般化された元は,定数よりも,より多くの構造に言及できる.]\label{example-generalized-elements} 

    次のような2つのPoset category $X,A$とその間の射$f:X\to A$(点線で表した)を考える.
    \begin{center}
        \begin{tikzcd}
            &    & & c \\
            y \ar[rrr, dashed, bend left=20, "f"] & & z \ar[ur, dashed] & b \ar[u] \\
                & x\ar[ul] \ar[ur] \ar[rr, dashed] & & a \ar[u] \\
                & X & & A
        \end{tikzcd}
    \end{center}
    これはPosにてmonicかつepicであるが,同型ではない.さらに進んで,$X\simeq A$でないことを示すには,「$X,A$を区別する保存量(invariant)を見つける」と良い.

    実際,今回$\mathrm{Hom}_{Pos}(1,X)\simeq\mathrm{Hom}_{Pos}(1,A)$であるが,$|\mathrm{Hom}_{Pos}(2,X)|=5$と$|\mathrm{Hom}_{Pos}(2,A)|=6$は要素の数が違う.
    \vspace{1cm}

    \begin{proposition*}
        全ての圏$C$において,$P\simeq Q$ならば,$\mathrm{Hom}_C(2,P)\simeq\mathrm{Hom}_C(2,Q)$である.
    \end{proposition*}
    \begin{proof}
        $i:P\to Q$を同型とする.写像$i_*:\mathrm{Hom}(2,P)\to\mathrm{Hom}(2,Q)$を次のように定める.
        \begin{center}
            \begin{tikzcd}
                i_*:\mathrm{Hom}(2,P)\ar[r] \ar[d, phantom, "\rotatebox{90}{$\in$}"] & \mathrm{Hom}(2,Q) \ar[d, phantom, "\rotatebox{90}{$\in$}"] \\
                f \ar[r, mapsto] & i\circ f
            \end{tikzcd}
        \end{center}
        と定めると,$i:P\to Q$の逆射$j$から同様に定めた写像$j_*:\mathrm{Hom}(2,Q)\to\mathrm{Hom}(2,P)$が$i_*$の逆射となる.
    \end{proof}
    \vspace{3cm}
    \begin{remark*}
        $\mathrm{Hom}(X,-)$は常に関手になり,関手は常に同型を保存する.
    \end{remark*}
    \vspace{1cm}
\end{example}

\begin{example} 

    1. 一般化された元$t:T\to A$のうち,特定の$T$が特異的に意味を持つことが多い(revealing).
    そこで,このような$t$をfigures of shape $T$ in $A$と呼ぶ.
    \vspace{1cm}

    2. 前の例\ref{example-generalized-elements}でPosの射$2\to P$が$P$内の$p\le p'$を満たす組$(p,p')$と対応した.これはa figure of shape $2$ in $P$の例であり,非常にgeometricな直観に合う.
\vspace{1cm}

    3. Monの圏では,終対象からの射は常に1つしかなかった.しかし,$M(1)$からの射(figures of shape $\mathbb{N}$ in $M$)については次が成り立つ.
    \begin{proposition*}[$M(1)$-値点がモノイドの射を決定する]
        圏Monの射$f,g:M\to M'$について,次の2つは同値.

        1. $f=g$である.

        2. 任意の$x\in \mathrm{Hom}_{Mon}(M(1),M)$について$fx=gx$である.
    \end{proposition*}
    \begin{proof}
        2.$\Rightarrow$1.を示す.
        
        $U(M)$上の任意のglobal element $y:1\to U(M)$を取る.すると,これによって定まるモノイドの射$\overline{y}:M(1)\to M$について,仮定より,$f\circ \overline{y}=g\circ\overline{y}$が成り立つ.これから,$U(f\circ\overline{y})=U(g\circ\overline{y})$が従う.
        
        この時,下の図式は可換であるから,特に左側のSets上の図式の一番外側の大回りの図式も可換になる.これより,$U(f)\circ y=U(g)\circ y$が従う.
        \begin{center}
            \begin{tikzcd}
                U(M(1)) \ar[r, dashed, "U(\overline{y})"] \ar[rr, bend left=30, "{U(f\circ\overline{y}),U(g\circ\overline{y})}"] & U(M) \ar[r, "{U(f),U(g)}"] & U(M') & M(1) \ar[r, "\overline{y}"] \ar[dr, dashed, "f\circ\overline{y}=g\circ\overline{y}=\overline{U(f)\circ y}"'] & M \ar[d, bend right=20, "f"'] \ar[d, bend left=20, "g"] \\
                1 \ar[u, "i"] \ar[ur, "y"] \ar[urr, "{U(f)\circ y, U(g)\circ y}"'] & & & & M'
            \end{tikzcd}
        \end{center}

        以上のことが任意にとった$y$について成り立つのだから,次が導けたことになる.
        \[\forall y\in U(M)\; U(f)(y)=U(g)(y) \]
        即ち,モノイドの射$f,g$が写像として等しいことを得た.
    \end{proof}
\vspace{3cm}

    4. 次が成り立つ.モノイド$M$の台集合$U(M)$は,一般化された要素$M(1)\to M$,即ち,$M$内の全ての figures of shape $\mathbb{N}$によって定まる.
    \[ U(M)\simeq \mathrm{Hom}_{Sets}(1,U(M))\simeq \mathrm{Hom}_{Mon}(M(1),M) \]
    これより,モノイド$M$からの写像$U(M)\to -$を考えるときは,$M$の元の代わりに,$M$内の全ての$\mathbb{N}$の型$\mathbb{N}\to M$を考えればいい.
\end{example}

\clearpage

\subsection{Products}

\begin{proposition*}
    Setsにおいて,集合$A,B\in\mathrm{Sets}$の直積(cartesian product) $A\times B:=\{ (a,b)\mid a\in A,b\in B \}$
    からの写像$\pi_1:A\times B\to A$, $\pi_2:A\times B\to B$をそれぞれ第一射影と第二射影とすると,次の図式は可換になる.
    \begin{center}
        \begin{tikzcd}
            & 1 \ar[dl, "a"'] \ar[d, "{(a,b)}"] \ar[dr, "b"] & \\
            A & A\times B \ar[r, "\pi_2"'] \ar[l, "\pi_1"] & B
        \end{tikzcd}
    \end{center}
\end{proposition*}

\vspace{3cm}

\begin{definition}[UMP of a product diagram]
    任意の圏$C$において,対象$A,B$の直積図式(product diagram)とは,対象$P$とそれからの射$p_1:P\to A, p_2:P\to B$があって,次を満たすもののことである.
\begin{quote}
    任意の対象$X$と任意の射$x_1:X\to A, x_2:X\to B$について,ただ一つの射$u:X\to P$が存在して,次の図式を可換にする.
\end{quote}
    \begin{center}
        \begin{tikzcd}
            & X \ar[dl, "x_1"'] \ar[dr, "x_2"] \ar[d, dashed, "u"] & \\
            A & P \ar[l, "p_1"] \ar[r, "p_2"'] & B
        \end{tikzcd}
    \end{center}
\end{definition}
\vspace{3cm}
\begin{remark}
    次の2つに主張が分解できる.

    1(存在). $x_1=p_1u\;\land\; x_2=p_2u$を満たす射$u:X\to U$が存在する.

    2(一意性). $v:X\to U$も$x_1=p_1v\;\land\; x_2=p_2v$を満たすならば,$v=u$である.
\end{remark}
\vspace{1cm}

\begin{proposition}
    圏$C$の対象$A,B$の積$P$は,同型を除いて一意的である.
\end{proposition}
\begin{proof}
    $P$と射$p_1:P\to A, p_2:P\to B$と,$Q$と射$q_1:Q\to A,q_2:Q\to B$のいずれも,$A$と$B$の積のUMPを満たすとする.

    $Q$についてのUMPより,$i:P\to Q$が,$P$についてのUMPより,$j:Q\to P$がそれぞれ存在し,次の図式を可換にする.
    \begin{center}
        \begin{tikzcd}
            & P \ar[d, dashed, "i"] \ar[dl, "p_1"'] \ar[dr, "p_2"] & \\
            A & Q \ar[l, "q_1"'] \ar[r, "q_2"] \ar[d, dashed, "j"] & B \\
            & P \ar[ul, "p_1"] \ar[ur, "p_2"'] &
        \end{tikzcd}
    \end{center}
    すると,特に$p_1\circ j\circ i=p_1$かつ$p_2\circ j\circ i=p_2$が成り立つ.
    これを$p_1\circ 1_P=p_1$と$p_2\circ 1_P=p_2$と見比べると,UMPの一意性条件により,$j\circ i=1_P$かつ$i\circ j=1_P$である.

    同様にして,$i\circ j=1_Q$も得る.

    以上より,$P\simeq Q$である.
\end{proof}
\vspace{3cm}

\begin{notation*} 

    1. この$A$と$B$についての一意的な積を,$A\times B$と書く.
\vspace{1cm}

    2. UMPの定義内にある記号について,射$u:X\to A\times B$を$\langle x_1,x_2\rangle$と書く.
    \vspace{1cm}

    3. 積への射$f:X\to A\times B$は,射の組$(f_1:X\to A,f_2:X\to B)$と一対一対応する.
\vspace{1cm}

    4. 積からの射$g:A\times B\to Y$は,一般化された元$f=(f_1,f_2)$について,一般化された元$g\langle f_1,f_2\rangle$が対応するので,いわば「一般化された2変数関数」と言える.
    \begin{center}
        \begin{tikzcd}
            & X \ar[dl, "f_1"'] \ar[d, "f"] \ar[dr, "f_2"] & \\
            A & A\times B\ar[l, "p_1"'] \ar[r, "p_2"] \ar[d, "g"] & B \\
            & Y &
        \end{tikzcd}
    \end{center}
\end{notation*}

\end{document}