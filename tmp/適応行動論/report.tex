\documentclass[uplatex, 10.5pt, dvipdfmx]{jsarticle}
    
\title{適応行動論 第一回レポート}
\author{担当:大槻久先生}
\date{\today}
\pagestyle{empty} \setcounter{secnumdepth}{4}
\usepackage{amsmath, amsfonts, amsthm, amssymb, ascmac, color, comment, wrap fig}

\usepackage{mathtools}
\mathtoolsset{showonlyrefs=true} %labelを附した数式にのみ附番される.

\usepackage{tikz, tikz-cd}
\usepackage[all]{xy}
\def\objectstyle{\displaystyle} %デフォルトではxymatrix中の数式が文中数式モードになるので,それを直した.

%化学式をTikZで簡単に書くためのパッケージ.
\usepackage[version=4]{mhchem} %texdoc mhchem
%化学構造式をTikZで描くためのパッケージ.
\usepackage{chemfig}
%IS単位を書くためのパッケージ
\usepackage{siunitx}
%取り消し線を引くためのパッケージ
\usepackage{ulem}

%\rotateboxコマンドを,文字列の中心で回転させるオプション.
%他rotatebox, scalebox, reflectbox, resizeboxなどのコマンド.
\usepackage{graphicx}

%加藤晃史さんがフル活用していたtcolorboxを,途中改ページ可能で.
\usepackage[breakable]{tcolorbox}

%enumerate環境を凝らせる.
\usepackage{enumerate}

%日本語にルビをふる
\usepackage{pxrubrica}

%足助さんからもらったオプション
%\usepackage[shortlabels,inline]{enumitem}
%\usepackage[top=15truemm,bottom=15truemm,left=10truemm,right=10truemm]{geometry}

%以下,ソースコードを表示する環境の設定.
\usepackage{listings,jvlisting} %日本語のコメントアウトをする場合jlistingが必要
%ここからソースコードの表示に関する設定
\lstset{
  basicstyle={\ttfamily},
  identifierstyle={\small},
  commentstyle={\smallitshape},
  keywordstyle={\small\bfseries},
  ndkeywordstyle={\small},
  stringstyle={\small\ttfamily},
  frame={tb},
  breaklines=true,
  columns=[l]{fullflexible},
  numbers=left,
  xrightmargin=0zw,
  xleftmargin=3zw,
  numberstyle={\scriptsize},
  stepnumber=1,
  numbersep=1zw,
  lineskip=-0.5ex
}
%lstlisting環境で,[caption=hoge,label=fuga]などのoptionを付けられる.
\makeatletter
    \AtBeginDocument{
    \renewcommand*{\thelstlisting}{\arabic{chapter}.\arabic{section}.\arabic{lstlisting}}
    \@addtoreset{lstlisting}{section}
    }
\makeatother
%caption番号を「[chapter番号].[section番号].[subsection番号]-[そのsubsection内においてn番目]」に変更
\renewcommand{\lstlistingname}{program}
%caption名を"program"に変更

%%%
%%%フォント
%%%

%本文・数式の両方のフォントをTimesに変更するお手軽なパッケージだが,LaTeX標準数式記号の\jmath, \amalg, coprodはサポートされない.
% \usepackage{mathptmx}
%Palatinoの方が完成度が高いと美文書作成に書いてあった.
\usepackage[sc]{mathpazo} %オプションは,familyの指定.pplxにしている.
%2000年のYoung Ryuによる新しいTimes系.なおPalatinoもある.
% \usepackage{newtxtext, newtxmath}
%拡張数学記号.\textsectionでブルバキに!
\usepackage{textcomp, mathcomp}
\usepackage[T1]{fontenc} %8bitエンコーディングにする.comp系拡張数学文字の動作が安定する.
%AMS Euler.Computer Modernと相性が悪いとは…….
\usepackage{ccfonts, eulervm} %KnuthのConcrete Mathematicsの組み合わせ.
% \renewcommand{\rmdefault}{pplx} %makes LaTeX use Palatino in place of CM Roman.Do not use the Euler math fonts in conjunction with the default Computer Modern text fonts – this is ugly!

%%% newcommands
    %参考文献で⑦というのを出したかった.\circled{n}と打てば良い.LaTeX StackExchangeより.
\newcommand*\circled[1]{\tikz[baseline=(char.base)]{\node[shape=circle,draw,inner sep=0.8pt] (char) {#1};}}

%%%
%%% ショートカット 足助さんからのコピペ
%%%

\DeclareMathOperator{\grad}{\mathrm{grad}}
\DeclareMathOperator{\rot}{\mathrm{rot}}
\DeclareMathOperator{\divergence}{\mathrm{div}}
\newcommand\R{\mathbb{R}}
\newcommand\N{\mathbb{N}}
\newcommand\C{\mathbb{C}}
\newcommand\Z{\mathbb{Z}}
\newcommand\Q{\mathbb{Q}}
\newcommand\GL{\mathrm{GL}}
\newcommand\SL{\mathrm{SL}}
\newcommand\False{\mathrm{False}}
\newcommand\True{\mathrm{True}}
\newcommand\tr{\mathrm{tr}}
\newcommand\M{\mathcal{M}}
\newcommand\F{\mathbb{F}}
% \newcommand\H{\mathbb{H}} すでにある.
\newcommand\id{\mathrm{id}}
\newcommand\A{\mathcal{A}}
%\renewcommand\coprod{\rotatebox[origin=c]{180}{$\prod$}}
\newcommand\pr{\mathrm{pr}}
\newcommand\U{\mathfrak{U}}
\newcommand\Map{\mathrm{Map}}
\newcommand\dom{\mathrm{dom}}
\newcommand\cod{\mathrm{cod}}
\newcommand\supp{\mathrm{supp}}
%%% 複素解析学
\renewcommand\Re{\mathrm{Re}\;}
\renewcommand\Im{\mathrm{Im}\;}
\newcommand\Gal{\mathrm{Gal}}
\newcommand\PGL{\mathrm{PGL}}
\newcommand\PSL{\mathrm{PSL}}
%%% 解析力学
\newcommand\x{\mathbf{x}}
\newcommand\q{\mathbf{q}}
%%% 集合と位相
\newcommand\ORD{\mathrm{ORD}}

%%% 圏
\newcommand\Hom{\mathrm{Hom}}
\newcommand\Mor{\mathrm{Mor}}
\newcommand\Aut{\mathrm{Aut}}
\newcommand\End{\mathrm{End}}
\newcommand\op{\mathrm{op}}
\newcommand\ev{\mathrm{ev}}
\newcommand\Ob{\mathrm{Ob}}
\newcommand\Ar{\mathrm{Ar}}
\newcommand\Arr{\mathrm{Arr}}
\newcommand\Set{\mathrm{Set}}
\newcommand\Grp{\mathrm{Grp}}
\newcommand\Cat{\mathrm{Cat}}
\newcommand\Mon{\mathrm{Mon}}
\newcommand\CMon{\mathrm{CMon}}
\newcommand\Pos{\mathrm{Pos}}
\newcommand\Vect{\mathrm{Vect}}
\newcommand\FinVect{\mathrm{FinVect}}
\newcommand\Fun{\mathrm{Fun}}
\newcommand\Ord{\mathrm{Ord}}

%%%
%%% 定理環境 以下足助さんからのコピペ
%%%

\newtheoremstyle{StatementsWithStar}% ?name?
{3pt}% ?Space above? 1
{3pt}% ?Space below? 1
{}% ?Body font?
{}% ?Indent amount? 2
{\bfseries}% ?Theorem head font?
{\textbf{.}}% ?Punctuation after theorem head?
{.5em}% ?Space after theorem head? 3
{\textbf{\textup{#1~\thetheorem{}}}{}\,$^{\ast}$\thmnote{(#3)}}% ?Theorem head spec (can be left empty, meaning ‘normal’)?
%
\newtheoremstyle{StatementsWithStar2}% ?name?
{3pt}% ?Space above? 1
{3pt}% ?Space below? 1
{}% ?Body font?
{}% ?Indent amount? 2
{\bfseries}% ?Theorem head font?
{\textbf{.}}% ?Punctuation after theorem head?
{.5em}% ?Space after theorem head? 3
{\textbf{\textup{#1~\thetheorem{}}}{}\,$^{\ast\ast}$\thmnote{(#3)}}% ?Theorem head spec (can be left empty, meaning ‘normal’)?
%
\newtheoremstyle{StatementsWithStar3}% ?name?
{3pt}% ?Space above? 1
{3pt}% ?Space below? 1
{}% ?Body font?
{}% ?Indent amount? 2
{\bfseries}% ?Theorem head font?
{\textbf{.}}% ?Punctuation after theorem head?
{.5em}% ?Space after theorem head? 3
{\textbf{\textup{#1~\thetheorem{}}}{}\,$^{\ast\ast\ast}$\thmnote{(#3)}}% ?Theorem head spec (can be left empty, meaning ‘normal’)?
%
\newtheoremstyle{StatementsWithCCirc}% ?name?
{6pt}% ?Space above? 1
{6pt}% ?Space below? 1
{}% ?Body font?
{}% ?Indent amount? 2
{\bfseries}% ?Theorem head font?
{\textbf{.}}% ?Punctuation after theorem head?
{.5em}% ?Space after theorem head? 3
{\textbf{\textup{#1~\thetheorem{}}}{}\,$^{\circledcirc}$\thmnote{(#3)}}% ?Theorem head spec (can be left empty, meaning ‘normal’)?
%
\theoremstyle{definition}
 \newtheorem{theorem}{定理}[section]
 \newtheorem{axiom}[theorem]{公理}
 \newtheorem{corollary}[theorem]{系}
 \newtheorem{proposition}[theorem]{命題}
 \newtheorem*{proposition*}{命題}
 \newtheorem{lemma}[theorem]{補題}
 \newtheorem*{lemma*}{補題}
 \newtheorem*{theorem*}{定理}
 \newtheorem{definition}[theorem]{定義}
 \newtheorem{example}[theorem]{例}
 \newtheorem{notation}[theorem]{記法}
 \newtheorem*{notation*}{記法}
 \newtheorem{assumption}[theorem]{仮定}
 \newtheorem{question}[theorem]{問}
 \newtheorem{counterexample}[theorem]{反例}
 \newtheorem{reidai}[theorem]{例題}
 \newtheorem{problem}[theorem]{問題}
 \newtheorem*{solution*}{\bf{[解]}}
 \newtheorem{discussion}[theorem]{議論}
 \newtheorem{remark}[theorem]{注}
 \newtheorem{universality}[theorem]{普遍性} %非自明な例外がない.
 \newtheorem{universal tendency}[theorem]{普遍傾向} %例外が有意に少ない.
 \newtheorem{hypothesis}[theorem]{仮説} %実験で説明されていない理論.
 \newtheorem{theory}[theorem]{理論} %実験事実とその(さしあたり)整合的な説明.
 \newtheorem{fact}[theorem]{実験事実}
 \newtheorem{model}[theorem]{模型}
 \newtheorem{explanation}[theorem]{説明} %理論による実験事実の説明
 \newtheorem{anomaly}[theorem]{理論の限界}
 \newtheorem{application}[theorem]{応用例}
 \newtheorem{method}[theorem]{手法} %実験手法など,技術的問題.
 \newtheorem{history}[theorem]{歴史}
 \newtheorem{research}[theorem]{研究}
% \newtheorem*{remarknonum}{注}
 \newtheorem*{definition*}{定義}
 \newtheorem*{remark*}{注}
 \newtheorem*{question*}{問}
 \newtheorem*{axiom*}{公理}
 \newtheorem*{example*}{例}
%
\theoremstyle{StatementsWithStar}
 \newtheorem{definition_*}[theorem]{定義}
 \newtheorem{question_*}[theorem]{問}
 \newtheorem{example_*}[theorem]{例}
 \newtheorem{theorem_*}[theorem]{定理}
 \newtheorem{remark_*}[theorem]{注}
%
\theoremstyle{StatementsWithStar2}
 \newtheorem{definition_**}[theorem]{定義}
 \newtheorem{theorem_**}[theorem]{定理}
 \newtheorem{question_**}[theorem]{問}
 \newtheorem{remark_**}[theorem]{注}
%
\theoremstyle{StatementsWithStar3}
 \newtheorem{remark_***}[theorem]{注}
 \newtheorem{question_***}[theorem]{問}
%
\theoremstyle{StatementsWithCCirc}
 \newtheorem{definition_O}[theorem]{定義}
 \newtheorem{question_O}[theorem]{問}
 \newtheorem{example_O}[theorem]{例}
 \newtheorem{remark_O}[theorem]{注}
%
\theoremstyle{definition}
%
\raggedbottom
\allowdisplaybreaks

%証明環境のスタイル
\everymath{\displaystyle}
\renewcommand{\proofname}{\bf [証明]}
\renewcommand{\thefootnote}{\dag\arabic{footnote}}	%足助さんからもらった.どうなるんだ?

%mathptmxパッケージ下で,\jmath, \amalg, coprodの記号を出力するためのマクロ.TeX Wikiからのコピペ.
% \DeclareSymbolFont{cmletters}{OML}{cmm}{m}{it}
% \DeclareSymbolFont{cmsymbols}{OMS}{cmsy}{m}{n}
% \DeclareSymbolFont{cmlargesymbols}{OMX}{cmex}{m}{n}
% \DeclareMathSymbol{\myjmath}{\mathord}{cmletters}{"7C}
% \DeclareMathSymbol{\myamalg}{\mathbin}{cmsymbols}{"71}
% \DeclareMathSymbol{\mycoprod}{\mathop}{cmlargesymbols}{"60}
% \let\jmath\myjmath
% \let\amalg\myamalg
% \let\coprod\mycoprod
\begin{document}
\begin{flushright}
    J4-190549 司馬博文
\end{flushright}
\begin{center}
    適応行動論 第一回レポート

    担当:大槻久先生

    \today
\end{center}

\section{論点整理}
この節では、宗教と科学を、「Homo Sapiensの言語的な活動のうち特に、集団的な知恵の共有の働きをするもの」として同列に定義して、いずれも進化論的な産物として議論する枠組みを整理する。多少の語弊を恐れず、この活動を、まとめて「宗教」と呼んでしまうこととする。特にこの使い方をして居ることを強調するために括弧「」を附ける。

まず、宗教を大きく分けて2つの側面に分類する。社会的側面である「慣習」と、知識的側面である「知恵」である。

\subsection{「宗教的」の社会的側面「慣習」について}

宗教的な慣習について、人は行動の善悪を判断する道徳規範を、身の回りの仲間の行動から内面化していく。
勿論、これは必ずしも科学的根拠に基づいたものだと限らなければ、大義が明確に意識されて居るものでもない。
例えば「教義」という形で権威づけされて居るなどの形態を取り、まさに先祖代々受け継いだ慣習である。
また、これに類するものは科学に従事する人間の界隈でも多分に存在し、人間はそこから自由になることはない。
これはいつの時代でも、巨視的に見れば、一つの人間集団が問題に対処するにあたって、集団的思考の中で計算し出した「集団的知性(collective knowledge)」だとも言える。
例を挙げる。タブーの概念には、次のような効用が考えられる。

\begin{quotation}
    「タブーの心理は、宗教的な規範や性的な慣例に従わないことに対して激怒をよび起こし、おぞましい罰を求めさせることもあるが、逆に、人の心が危険な領域にはまっていくのを阻止してくれることもある。」\cite{暴力の人類学}
\end{quotation}

また、2つ目の例として、未知なものや異文化を反射的に嫌ってしまう傾向にも、効用がある。

\begin{quotation}
    「(前略)同じような傾向は、言語、宗教、自民族中心主義の地理的分布のなかにも見られる。北極・南極から赤道に近づくにつれて地域ごとの言語や宗教の数は増え、住民はより外国人嫌いになる。一見すると無関係にも思われるものの、これら三つの要素はすべて集団を区別するために役立っている。相手と同じ言語を話さず、宗教も共有せず、そもそも自分たちにほかのグループを嫌う傾向があるとき、外部の集団と交流する機会は減る。
    
    なぜ赤道に近づくと言語と宗教が増えるのか?その数の多さが自民族中心主義に関連するのはなぜか?それらの問いへの答えは、温帯・寒冷地域よりも熱帯地域のほうが病原体の密度がはるかに高いという事実に隠れている。あなたがスウェーデンに住んでいるとすれば、800キロ圏内にある土地のすべての集団の住人たちが、数少ない同じ病原体をもっている確率が高い。一方、アフリカ中央部にあるコンゴに住んでいるとしたら、谷の反対側の集団が未知の病原体をもっていてもなんらおかしくはない。」\cite{異文化嫌い}
\end{quotation}

そして勿論、この2つの生物的に組み込まれたと言えるヒトの傾向は、現代社会において多くの負の遺産も産んできた。

\begin{quotation}
    「人間の道徳感覚は、どんな残虐行為にも、その行為を犯す人の頭のなかに言い訳をもたせられる。そして、その人に何ら具体的な利益をもたらさない暴力行為への動機を与えられる。異端者や転向者への拷問、魔女の火あぶり、同性愛者の投獄、身持ちの悪い娘への名誉殺人等は、その一部の例にすぎない。」\cite{暴力の人類学}
\end{quotation}

\subsection{「宗教」の知識的側面「知恵」について}

人類は印刷革命を迎えた時から、「活字の形で客体化された思考を、操作可能な形で、多くの人に共有する」ことを可能とする人類内のメディア環境の変化を主原因として、科学と呼ばれる広い活字ネットワークの上での知的営みを発展させた。
活字として定式化された論考を、十分な識字率は前提として必要だが、たくさんの人が同時に読んで参加できる枠組みが揃った時、その活動の公共性ゆえ、「宗教」の知識的側面は不可逆的な変化を要請される。
その歴史的に重要な不連続点として、科学革命の騎手であるNewtonやKeplerらが土台とした観測事実を収集したTycho Braheに注目して、Bruno Latourは以下のように論じて居る。

\begin{quotation}
    「彼(ティコ・ブラーエのこと)の精神が突然に変化を遂げたのでは無い。彼の目が突然に古い偏見から自由になるのでは無い。以前の誰よりも夏の空を注意深く観測している訳でも無い。彼は、夏の空+自分の観測+同僚の観測+コペルニクスの書物+プトレマイオスの『アルマゲスト』の多くの版とを一望のもとに見て考えた最初の人物である。長いネットワークの始点と終点に座り、不変で結合可能な可動物を私が呼ぶものを生み出した最初の人物である。」\cite{ラトゥール}
\end{quotation}

このように、システム・慣習・作法の体系としての違いが、宗教の中から「科学」と呼ぶべき部分を区別して特徴付ける主な性質だと考えられる。そして科学者とは、その手続きの枠組みの中で作品を作り出す存在と言えるだろう。

\begin{quotation}
    「HelmholtzがMayerについて語っているように、素人を専門家から区別するものは,ただ素人がこれと決まった作業方法を欠き、従って与えられた思いつきについてその効果を判定し、評価し、かつこれを実現する能力を持たないということだけである。」\cite{ウェーバー}
\end{quotation}

逆にいえば、「宗教」と呼ぶべき営みの中で、「科学」の俎上に載せられない不妙な部分は、どの時代でも残り続けるであろう。

\section{共存の可能性}

即ち、メディアが違うだけであって、2つは本質的な対立構造を持つわけではない。2014年のローマ教皇の次の言葉にある通り、進化論と創造論を同時に信じることは可能であって、自分の中で、あるいは特定の局所的集団の中で理論化することは可能である。
しかし、科学のシステム内にそれを提出するかは別問題であり、それ以降には科学の作法がある。音楽には沢山のジャンルがあるが、それを同時に楽しむことは可能である。

\begin{quote}
    Evolution of nature is not inconsistent with the notion of creation because evolution presupposes the creation of beings which evolve.
\end{quote}

「宗教」活動において、意見の違いから生じる対立は避けがたい。
これは科学活動に話を限ってもそうである。
我々はむしろ、宗教と科学との共存の可能性の模索は、その1つの例に過ぎないと断じて、
宗教や各学問分野に限らず音楽、陶芸を初めとして、様々な文化活動を、積極的に発信やアウトリーチをすることで、認知容易性を下げたり、互いに対話可能なプラットホームを用意するなどの特にメディア的な努力によって、その間の論争をうまく乗りこなす枠組みを模索し、さらに花咲く文化を目指していくのみである。
進化論と創造論は、公平な取り扱いの中で提示されれば、それは立派な教育である。
進化論の枠組み自体はDarwinにまで遡るが、その枠組みの心理学を始めとした各分野の継承が、遅れたとはいえ、今現在本格的に始まりつつあるのは、私が今提示した共存の可能性の成功例の1つであると確信して居る。

\begin{thebibliography}{9}
    \bibitem{暴力の人類学}
        S. Pinker著、幾島幸子、塩原通緒訳『暴力の人類史』
        (青土社、2015)
    \bibitem{異文化嫌い}
        W. von Hippel著、濱野大道訳『われわれはなぜ嘘つきで自信過剰でお人好しなのか 進化心理学で読み解く、人類の驚くべき戦略』
        (ハーパーコリンズ・ジャパン、2019)
    \bibitem{ラトゥール}
    ブルーノ・ラトゥール著、川崎勝・高田紀代志訳『科学が作られているとき――人類学的考察』
    (産業図書、1999)
    \bibitem{ウェーバー}
    マックス・ヴェーバー著、尾高邦雄訳『職業としての学問』
    (岩波文庫、1936)
\end{thebibliography}
\end{document}