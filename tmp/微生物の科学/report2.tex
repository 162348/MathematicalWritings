\documentclass[uplatex, dvipdfmx]{jsarticle}
\title{微生物の科学 レポート\\福田良一先生担当授業「多様な酵母の能力とその応用」(6月5日開講)}
\author{司馬博文 J4-190549}
\pagestyle{empty} \setcounter{secnumdepth}{4}
\usepackage{amsmath, amsfonts, amsthm, amssymb, ascmac, color, comment, wrap fig}

\usepackage{mathtools}
\mathtoolsset{showonlyrefs=true} %labelを附した数式にのみ附番される.

\usepackage{tikz, tikz-cd}
\usepackage[all]{xy}
\def\objectstyle{\displaystyle} %デフォルトではxymatrix中の数式が文中数式モードになるので,それを直した.

%化学式をTikZで簡単に書くためのパッケージ.
\usepackage[version=4]{mhchem} %texdoc mhchem
%化学構造式をTikZで描くためのパッケージ.
\usepackage{chemfig}
%IS単位を書くためのパッケージ
\usepackage{siunitx}
%取り消し線を引くためのパッケージ
\usepackage{ulem}

%\rotateboxコマンドを,文字列の中心で回転させるオプション.
%他rotatebox, scalebox, reflectbox, resizeboxなどのコマンド.
\usepackage{graphicx}

%加藤晃史さんがフル活用していたtcolorboxを,途中改ページ可能で.
\usepackage[breakable]{tcolorbox}

%enumerate環境を凝らせる.
\usepackage{enumerate}

%日本語にルビをふる
\usepackage{pxrubrica}

%足助さんからもらったオプション
%\usepackage[shortlabels,inline]{enumitem}
%\usepackage[top=15truemm,bottom=15truemm,left=10truemm,right=10truemm]{geometry}

%以下,ソースコードを表示する環境の設定.
\usepackage{listings,jvlisting} %日本語のコメントアウトをする場合jlistingが必要
%ここからソースコードの表示に関する設定
\lstset{
  basicstyle={\ttfamily},
  identifierstyle={\small},
  commentstyle={\smallitshape},
  keywordstyle={\small\bfseries},
  ndkeywordstyle={\small},
  stringstyle={\small\ttfamily},
  frame={tb},
  breaklines=true,
  columns=[l]{fullflexible},
  numbers=left,
  xrightmargin=0zw,
  xleftmargin=3zw,
  numberstyle={\scriptsize},
  stepnumber=1,
  numbersep=1zw,
  lineskip=-0.5ex
}
%lstlisting環境で,[caption=hoge,label=fuga]などのoptionを付けられる.
\makeatletter
    \AtBeginDocument{
    \renewcommand*{\thelstlisting}{\arabic{chapter}.\arabic{section}.\arabic{lstlisting}}
    \@addtoreset{lstlisting}{section}
    }
\makeatother
%caption番号を「[chapter番号].[section番号].[subsection番号]-[そのsubsection内においてn番目]」に変更
\renewcommand{\lstlistingname}{program}
%caption名を"program"に変更

%%%
%%%フォント
%%%

%本文・数式の両方のフォントをTimesに変更するお手軽なパッケージだが,LaTeX標準数式記号の\jmath, \amalg, coprodはサポートされない.
% \usepackage{mathptmx}
%Palatinoの方が完成度が高いと美文書作成に書いてあった.
\usepackage[sc]{mathpazo} %オプションは,familyの指定.pplxにしている.
%2000年のYoung Ryuによる新しいTimes系.なおPalatinoもある.
% \usepackage{newtxtext, newtxmath}
%拡張数学記号.\textsectionでブルバキに!
\usepackage{textcomp, mathcomp}
\usepackage[T1]{fontenc} %8bitエンコーディングにする.comp系拡張数学文字の動作が安定する.
%AMS Euler.Computer Modernと相性が悪いとは…….
\usepackage{ccfonts, eulervm} %KnuthのConcrete Mathematicsの組み合わせ.
% \renewcommand{\rmdefault}{pplx} %makes LaTeX use Palatino in place of CM Roman.Do not use the Euler math fonts in conjunction with the default Computer Modern text fonts – this is ugly!

%%% newcommands
    %参考文献で⑦というのを出したかった.\circled{n}と打てば良い.LaTeX StackExchangeより.
\newcommand*\circled[1]{\tikz[baseline=(char.base)]{\node[shape=circle,draw,inner sep=0.8pt] (char) {#1};}}

%%%
%%% ショートカット 足助さんからのコピペ
%%%

\DeclareMathOperator{\grad}{\mathrm{grad}}
\DeclareMathOperator{\rot}{\mathrm{rot}}
\DeclareMathOperator{\divergence}{\mathrm{div}}
\newcommand\R{\mathbb{R}}
\newcommand\N{\mathbb{N}}
\newcommand\C{\mathbb{C}}
\newcommand\Z{\mathbb{Z}}
\newcommand\Q{\mathbb{Q}}
\newcommand\GL{\mathrm{GL}}
\newcommand\SL{\mathrm{SL}}
\newcommand\False{\mathrm{False}}
\newcommand\True{\mathrm{True}}
\newcommand\tr{\mathrm{tr}}
\newcommand\M{\mathcal{M}}
\newcommand\F{\mathbb{F}}
% \newcommand\H{\mathbb{H}} すでにある.
\newcommand\id{\mathrm{id}}
\newcommand\A{\mathcal{A}}
%\renewcommand\coprod{\rotatebox[origin=c]{180}{$\prod$}}
\newcommand\pr{\mathrm{pr}}
\newcommand\U{\mathfrak{U}}
\newcommand\Map{\mathrm{Map}}
\newcommand\dom{\mathrm{dom}}
\newcommand\cod{\mathrm{cod}}
\newcommand\supp{\mathrm{supp}}
%%% 複素解析学
\renewcommand\Re{\mathrm{Re}\;}
\renewcommand\Im{\mathrm{Im}\;}
\newcommand\Gal{\mathrm{Gal}}
\newcommand\PGL{\mathrm{PGL}}
\newcommand\PSL{\mathrm{PSL}}
%%% 解析力学
\newcommand\x{\mathbf{x}}
\newcommand\q{\mathbf{q}}
%%% 集合と位相
\newcommand\ORD{\mathrm{ORD}}

%%% 圏
\newcommand\Hom{\mathrm{Hom}}
\newcommand\Mor{\mathrm{Mor}}
\newcommand\Aut{\mathrm{Aut}}
\newcommand\End{\mathrm{End}}
\newcommand\op{\mathrm{op}}
\newcommand\ev{\mathrm{ev}}
\newcommand\Ob{\mathrm{Ob}}
\newcommand\Ar{\mathrm{Ar}}
\newcommand\Arr{\mathrm{Arr}}
\newcommand\Set{\mathrm{Set}}
\newcommand\Grp{\mathrm{Grp}}
\newcommand\Cat{\mathrm{Cat}}
\newcommand\Mon{\mathrm{Mon}}
\newcommand\CMon{\mathrm{CMon}}
\newcommand\Pos{\mathrm{Pos}}
\newcommand\Vect{\mathrm{Vect}}
\newcommand\FinVect{\mathrm{FinVect}}
\newcommand\Fun{\mathrm{Fun}}
\newcommand\Ord{\mathrm{Ord}}

%%%
%%% 定理環境 以下足助さんからのコピペ
%%%

\newtheoremstyle{StatementsWithStar}% ?name?
{3pt}% ?Space above? 1
{3pt}% ?Space below? 1
{}% ?Body font?
{}% ?Indent amount? 2
{\bfseries}% ?Theorem head font?
{\textbf{.}}% ?Punctuation after theorem head?
{.5em}% ?Space after theorem head? 3
{\textbf{\textup{#1~\thetheorem{}}}{}\,$^{\ast}$\thmnote{(#3)}}% ?Theorem head spec (can be left empty, meaning ‘normal’)?
%
\newtheoremstyle{StatementsWithStar2}% ?name?
{3pt}% ?Space above? 1
{3pt}% ?Space below? 1
{}% ?Body font?
{}% ?Indent amount? 2
{\bfseries}% ?Theorem head font?
{\textbf{.}}% ?Punctuation after theorem head?
{.5em}% ?Space after theorem head? 3
{\textbf{\textup{#1~\thetheorem{}}}{}\,$^{\ast\ast}$\thmnote{(#3)}}% ?Theorem head spec (can be left empty, meaning ‘normal’)?
%
\newtheoremstyle{StatementsWithStar3}% ?name?
{3pt}% ?Space above? 1
{3pt}% ?Space below? 1
{}% ?Body font?
{}% ?Indent amount? 2
{\bfseries}% ?Theorem head font?
{\textbf{.}}% ?Punctuation after theorem head?
{.5em}% ?Space after theorem head? 3
{\textbf{\textup{#1~\thetheorem{}}}{}\,$^{\ast\ast\ast}$\thmnote{(#3)}}% ?Theorem head spec (can be left empty, meaning ‘normal’)?
%
\newtheoremstyle{StatementsWithCCirc}% ?name?
{6pt}% ?Space above? 1
{6pt}% ?Space below? 1
{}% ?Body font?
{}% ?Indent amount? 2
{\bfseries}% ?Theorem head font?
{\textbf{.}}% ?Punctuation after theorem head?
{.5em}% ?Space after theorem head? 3
{\textbf{\textup{#1~\thetheorem{}}}{}\,$^{\circledcirc}$\thmnote{(#3)}}% ?Theorem head spec (can be left empty, meaning ‘normal’)?
%
\theoremstyle{definition}
 \newtheorem{theorem}{定理}[section]
 \newtheorem{axiom}[theorem]{公理}
 \newtheorem{corollary}[theorem]{系}
 \newtheorem{proposition}[theorem]{命題}
 \newtheorem*{proposition*}{命題}
 \newtheorem{lemma}[theorem]{補題}
 \newtheorem*{lemma*}{補題}
 \newtheorem*{theorem*}{定理}
 \newtheorem{definition}[theorem]{定義}
 \newtheorem{example}[theorem]{例}
 \newtheorem{notation}[theorem]{記法}
 \newtheorem*{notation*}{記法}
 \newtheorem{assumption}[theorem]{仮定}
 \newtheorem{question}[theorem]{問}
 \newtheorem{counterexample}[theorem]{反例}
 \newtheorem{reidai}[theorem]{例題}
 \newtheorem{problem}[theorem]{問題}
 \newtheorem*{solution*}{\bf{[解]}}
 \newtheorem{discussion}[theorem]{議論}
 \newtheorem{remark}[theorem]{注}
 \newtheorem{universality}[theorem]{普遍性} %非自明な例外がない.
 \newtheorem{universal tendency}[theorem]{普遍傾向} %例外が有意に少ない.
 \newtheorem{hypothesis}[theorem]{仮説} %実験で説明されていない理論.
 \newtheorem{theory}[theorem]{理論} %実験事実とその(さしあたり)整合的な説明.
 \newtheorem{fact}[theorem]{実験事実}
 \newtheorem{model}[theorem]{模型}
 \newtheorem{explanation}[theorem]{説明} %理論による実験事実の説明
 \newtheorem{anomaly}[theorem]{理論の限界}
 \newtheorem{application}[theorem]{応用例}
 \newtheorem{method}[theorem]{手法} %実験手法など,技術的問題.
 \newtheorem{history}[theorem]{歴史}
 \newtheorem{research}[theorem]{研究}
% \newtheorem*{remarknonum}{注}
 \newtheorem*{definition*}{定義}
 \newtheorem*{remark*}{注}
 \newtheorem*{question*}{問}
 \newtheorem*{axiom*}{公理}
 \newtheorem*{example*}{例}
%
\theoremstyle{StatementsWithStar}
 \newtheorem{definition_*}[theorem]{定義}
 \newtheorem{question_*}[theorem]{問}
 \newtheorem{example_*}[theorem]{例}
 \newtheorem{theorem_*}[theorem]{定理}
 \newtheorem{remark_*}[theorem]{注}
%
\theoremstyle{StatementsWithStar2}
 \newtheorem{definition_**}[theorem]{定義}
 \newtheorem{theorem_**}[theorem]{定理}
 \newtheorem{question_**}[theorem]{問}
 \newtheorem{remark_**}[theorem]{注}
%
\theoremstyle{StatementsWithStar3}
 \newtheorem{remark_***}[theorem]{注}
 \newtheorem{question_***}[theorem]{問}
%
\theoremstyle{StatementsWithCCirc}
 \newtheorem{definition_O}[theorem]{定義}
 \newtheorem{question_O}[theorem]{問}
 \newtheorem{example_O}[theorem]{例}
 \newtheorem{remark_O}[theorem]{注}
%
\theoremstyle{definition}
%
\raggedbottom
\allowdisplaybreaks

%証明環境のスタイル
\everymath{\displaystyle}
\renewcommand{\proofname}{\bf [証明]}
\renewcommand{\thefootnote}{\dag\arabic{footnote}}	%足助さんからもらった.どうなるんだ?

%mathptmxパッケージ下で,\jmath, \amalg, coprodの記号を出力するためのマクロ.TeX Wikiからのコピペ.
% \DeclareSymbolFont{cmletters}{OML}{cmm}{m}{it}
% \DeclareSymbolFont{cmsymbols}{OMS}{cmsy}{m}{n}
% \DeclareSymbolFont{cmlargesymbols}{OMX}{cmex}{m}{n}
% \DeclareMathSymbol{\myjmath}{\mathord}{cmletters}{"7C}
% \DeclareMathSymbol{\myamalg}{\mathbin}{cmsymbols}{"71}
% \DeclareMathSymbol{\mycoprod}{\mathop}{cmlargesymbols}{"60}
% \let\jmath\myjmath
% \let\amalg\myamalg
% \let\coprod\mycoprod

\begin{document}
\maketitle
\begin{abstract}
    講義の内容を,「酵母とは何か」「酵母の産業利用」「酵母の研究利用」「酵母の病原性」の4つの節に大別して,授業内容を構成し直した.自分で追加で調べたものについては,参考文献を付けた.
\end{abstract}

\section{酵母とはどんな微生物か}

\subsection{授業で扱われた微生物の一覧}

真菌(fungi)
\begin{center}
\begin{table}[h]\centering
    \begin{tabular}{|l|c|c|}\hline
        Ascomycota門(子嚢菌門)& Saccharomyces属 &cerevisiae:出芽酵母 \\
        && eubayanus\\
        && pastorianus:下面発酵ビール\\\hline
        &Schizosaccharomyces属&pombe:分裂酵母\\\hline
        &Zygosaccharomyces属&ouxii:醤油,味噌造り\\\hline
        &Aspergillus属&=コウジカビ,麹菌\\\hline
        &Yarrowia属&lipolytica:n-アルカン資化酵母\\\hline
        &Candida属&albicans:病原性\\\hline
    \end{tabular}
\end{table}
\end{center}

細菌(bacteria)
\begin{table}[h]\centering
    \begin{tabular}{|l|c|c|}\hline
        Proteobacteria門&Zymomonas属 &mobilis:pulque造り\\\hline
    \end{tabular}
\end{table}

\subsection{酵母とは?}

\begin{screen}\begin{definition*}
\noindent
酵母(yeast)とは,真菌類(fingi)であって,
\begin{quotation}
    1. その生活環内での栄養体が単細胞性を持ち

    2. 無性生殖(出芽や分裂など)により単細胞のまま増殖する
\end{quotation}
\noindent
もののことをいう.
\end{definition*}\end{screen}

これに当てはまる実際の生物には,子嚢菌門(Ascomycota)のうちのSaccharomyces属の真菌類などがあり\footnote{授業中には,“The Yeasts: A Taxonomic Study” 5th ed. には,144属1270種の酵母が記載されているとの紹介があった.},
狭義ではそのうちのS. cerevisiae種を指して「酵母菌」と呼ぶこともある.

\subsection{出芽酵母(Saccharomyces cerevisiae)の生活環}
出芽酵母とは,文字通りには出芽により繁殖する酵母のことであるが,狭義では特にSaccharomyces cerevisiaeのことを指す.

其の生活環は2倍体の場合と1倍体の場合で2通り存在し,2つの接合型$a,\alpha$の1倍体が出会うと接合によって2倍体を形成する.
2倍体細胞は減数分裂を行い,$a,\alpha$の各2つずつ1倍体胞子を形成する.接合型はMAT遺伝子にコードされて居るが,野生株では,出芽する度にMAT遺伝子座が変化し,接合型変換を起こす.これを自家和合性(Homothallism)という.
一方実験室株では,接合型変換に欠損を持つ為に,接合型は何度出芽しても不変である.これを自家不和合性(Heterothallism)という.\cite{酵母}

\subsection{酵母の分類}

酵母に分類される微生物は,分類体系上では2つの門・子嚢菌門(Ascomycota)と担子菌門(Basidiomycota)に渡る.
この2つの門は,其の有性生殖段階に於て,胞子を子嚢の中に形成するか,子実体など細胞(担子器\footnote{担子器とは,「担子菌類を特徴づける「担子胞子」を外生する特殊な細胞」のこと.(日本大百科全書・ニッポニカ,寺川博典).})の外に形成するかの形態の違いに拠って区別される.

酵母には,無性世代(不完全世代(anamorph)ともいう)に加えて有性世代(完全世代(telemorph)ともいう)も持つものも多く,
その場合はその胞子形成の形態に依って上述2門に分類されるが,無性世代しか持たない酵母も,DNA配列や生化学的性質に拠って上述いずれかに分類される.
それぞれを,子嚢菌系酵母(Ascomycetous yeasts)・担子菌系酵母(Basidiomycetous yeasts)という.

以降,酵母の3大利用用途「産業微生物」「モデル生物」「病原微生物」に大別して,それぞれを詳しく見ていく.

\section{産業微生物としての酵母}

\begin{itembox}[l]{酵母のモデル生物としての適性}
    1. 増殖が速く,培養が容易で,培地が安価に作れるため.

    2. 1タンパク質輸送系の細胞小器官(organelle)が発達しているため異種タンパク質の分泌生産に適する.

    3. 分子生物学的/遺伝学的な解析手法が取りやすく,その技術基盤が早期から確立したために,育種・改良がしやすい.

    4. 細菌に比べてウイルス汚染が少ない.

    5. 菌体を回収しやすい.
\end{itembox}

\subsection{出芽酵母によるアルコール発酵}

出芽酵母は,嫌気呼吸としてアルコール発酵を行う.

\begin{itembox}[l]{出芽酵母のアルコール発酵}
    1. グルコース1分子が,解糖系によりピルビン酸2分子に変換され,この際にATP2分子とNADH2分子を産出する.
    
    2. ピルビン酸はピルビン酸デカルボキシラーゼにより,アセトアルデヒドと二酸化炭素に分解される.

    3. アセトアルデヒドはアルコールデヒドロゲナーゼにより速やかにエタノールに還元される(この際NADHを消費する).

    以上をまとめると,\ce{C6H12O6 -> 2C2H5OH + 2CO2}
\end{itembox}
この反応は嫌気反応であるが,出芽酵母や分裂酵母(S. pombe)では,グルコースの濃度が十分(110 mg/L以上\cite{Nanba})にあれば,酸素の有無に関わらず(TCA回路による好気呼吸ではなく)このアルコール発酵を起こす.
これを発見者に因んで,Crabtree効果という.

\subsection{お酒造り}

以下,具体的な酒類を挙げて,その醸造過程と酵母の関係をみるが,殆どが出芽酵母により,また交雑による育種がなされた為,
遺伝学的な解析・考察は困難な現状となって居る.

\subsubsection{清酒造り}

清酒に使われる出芽酵母は,次の性質を持つ.
\begin{itembox}[l]{清酒酵母に見られる性質}
    1. Heterothallicな2倍体を形成する.

    2. 胞子形成効率と胞子出芽率が低い.

    3. ストレス応答(G1期停止とG0期以降の両方)の欠損がある(MSN4,PPT1,RIM15).
\end{itembox}

清酒造りは,この酵母とコウジカビを用いて,デンプンを糖化する過程とグルコースをエタノールに変換する過程を同一容器内で同時に発生させる.
これを並行複発酵という.この時糖の濃度はある一定値で安定する上に,G0期に移行してストレス耐性化しないが為に,出芽酵母の生存の限界までの濃度のアルコールを作ることができる.
これが約20度であり,他の一般の醸造酒よりも非常に高い濃度となって居る.

なお,吟醸酒に独特の果実的な香りがあるのも,清酒酵母の生産する酢酸イソアミル(バナナ・メロン風の香り)やカプロン酸エチル(りんご風の香り)による.

\subsubsection{ビール造り}

ビール作りは,まず大麦やその発芽した種子(malt)を,その他米などの副原料とともに混ぜ合わせて保持すると,麦芽に含まれる$\beta$-アミラーゼの働きで,でんぷんはマルトースに加水分解される.これを濾過してホップを加えて煮沸してから,発酵工程へ移る.
この時,$\beta$-マルトースを初めとした酵素は全て失活するため,糖化と発酵は別々に行われる.これを単行複発酵という.\footnote{ウイスキー製造では,ビールと違い糖化後に麦汁を煮沸せず,酵素の活性は失われない為,並行複発酵とも言える.\cite{酵母}}
こうして発酵過程に移るが,発酵のさせ方は大きく次の2つに大別される.
\begin{itembox}[l]{ビールの2大醸造法}
    1. 上面発酵(ale):麦芽を常温(15~20$^\circ\mathrm{C}$)で短い時間で発酵を行う.盛んに二酸化炭素が発生する為に,最終的に酵母が上面に浮かび上がって層を作る為に上面発酵と呼ばれる.

    2. 下面発酵(lager):麦芽を低温で長い時間で発酵を行う.ドイツ・バイエルン発祥で,冷却器などの整備と技術が整った19世紀以降に普及した醸造法で,発酵の間ゆっくりと酵母は沈殿し最終的に底に層を作るために下面発酵と呼ばれる.
    酵母としては,二糖類(maltoseやmelibiose)を発酵に用いることができるS. pastorianusを用いる.
\end{itembox}
甘みとのバランスを取るために苦味を付けること,また防腐や余分なタンパク質を凝固させて液を澄ませるなどの用途でホップが加えられるが,以前はgruit(ハーブや香辛料を混ぜたもの)が用いられていた.
2つの酵母S. cerevisiaeとS. pastorianusは,メリビオースの資化性を除けば形態的・微細構造上の区別を見出すのは難しいため,歴史的にはS. carlsbergensisと呼ばれていたものが,1984年にS. cerevisiaeに統合されたこともあるが\cite{有村},
現在ではS. cerevisiaeとS. eubayanusとの自然交雑の結果生じた種であることが判明して居る\cite{pastorianus}.S. eubayanusはパタゴニアから始まり,チベット,アメリカ,中国,ニュージーランドなど世界各地から発見された酵母であるが,これと交雑して得たS. pastorianusは自然界からは分離されていない.

\subsubsection{ワイン製造}

\begin{itembox}[l]{ワイン酵母に見られる性質}
    1. Homothallicな2倍体を形成する.

    2. 亜硫酸耐性を持つ.

    3. ブドウの果皮などにも存在する(従って,現代ではワイン醸造に適した酵母が加えられるが,原理的にはそのまま醸造可能).
\end{itembox}

\subsection{パン造り}

出芽酵母の代謝により発生する二酸化炭素でパンが膨らみ,またアルコールなどの副産物がそれぞれのパンに独特の風味をつける.

\subsection{醤油・味噌造り}

耐塩性酵母Zygosaccharomyces rouxiiは,食塩濃度約20\%の醤油もろみや味噌中で活発に繁殖してアルコール発酵を行う.
また,代謝の過程でHEMFと呼ばれる香味成分を生産する.

なお,欧州では,蜂蜜やシロップ,濃縮果汁,ジャム,ゼリーなどの含糖食品を変敗させる酵母であるとしてとして嫌忌されることもある.
実際,醤油や味噌の空気に触れる表面に,覆うような膜状の白いカビのような微生物がZ. rouxiiであり,風味を劣化させることがある.
また,産膜作用自体は,Saccharomyces属も持ち,ワインの貯蔵中にも起こる.アセトアルデヒドを主体とした悪臭を産膜臭という.
さらに,スペインで醸造されるシェリー酒の
一種であるフロールシェリーは,産膜を利用して作られるもので,エタノールから産膜酵母が代謝したアセトアルデヒドがシェリー酒特有の香りとなる.
なお,フロールとは,S. cerevisiaeの形成する膜を花に喩えて付けられた名前である.

\subsection{$n$-アルカン資化酵母}

Yarrowia lipolyticaは,炭素源として脂肪等の疎水性化合物も用いることが出来る所から$n$-アルカン資化酵母と呼ばれる.
\begin{itembox}[l]{n-アルカン資化酵母に見られる性質}
    1. リパーゼやプロテアーゼの高い分泌能(1~2 g/L).

    2. 有機酸(クエン酸,2-オキソグルタル酸,EPAなど)を産出する.

    3. 酵母型と菌糸型の二形性を持つ.2つの違う表現型で生育できる.
\end{itembox}

限りある石油資源から取れるn-アルカンや油脂のより効率良い利用手段として,
n-アルカン資化酵母を用いた油脂バイオマス資源を生産する手段が考えられて居る.

\section{モデル生物としての酵母}
出芽酵母(Saccharomyces cerevisiae)は,次のような特性を持つために,真核生物のモデル生物として,細胞周期,転写,翻訳,小胞輸送,シグナル伝達などの様々な基礎研究に貢献して居る.

\begin{itembox}[l]{酵母のモデル生物としての適性}
    1. 増殖が速く,培養が容易で,培地が安価に作れるため.

    2. 1倍体で生育する世代を持つために,変異株が簡単に単離できる.

    3. 分子生物学的/遺伝学的な解析手法が取りやすく,その技術基盤が早期から確立した.
\end{itembox}

\subsection{細胞周期の研究}
Leland H. HartwellとPaul M. Nurseは,cdc2 (cell division cycle)遺伝子が細胞周期の鍵となる制御因子であることの発見により2001年にノーベル賞を受賞した.
cdc2に欠損がある分裂酵母は,35$^\circ\mathrm{C}$では細胞周期を進められないという温度感受性変異を持つ.これと等価な遺伝子を出芽酵母から導入されたものは,細胞周期変異を回復する.\cite{Essential}

\subsection{小胞輸送の研究}
SEC遺伝子は,輸送と分泌に関わる温度感受性のタンパク質を指令して居るため,これに欠損がある酵母のタンパク質は25$^\circ\mathrm{C}$では正常に機能するが,35$^\circ\mathrm{C}$では不活化する.
従って,小胞体やゴルジ体や輸送小胞内に,分泌されるべきタンパク質が蓄積する\cite{Essential}.
この仕組みの解明により,Randy Schekmanは2013年にノーベル賞を受賞した.

\subsection{オートファジーの研究}
\begin{quotation}
    オートファジーは,タンパク質など細胞質成分のみならずオルガネラのような巨大な構造体を丸ごと分解する,真核生物に広く保存されたバルク分解系である.オートファジーの概念は最初に哺乳動物の系から提唱されたが,その分子実体が明らかになるのに40年もの年月を要した.このブレイクスルーは,もっともシンプルな真核生物のモデル系である出芽酵母によるものであった.\cite{大隈}
\end{quotation}
大隈良典先生は,オートファジーの機構の解明により2016年にノーベル賞を受賞した.オートファジーの機構の解明のブレイクスルーとなったのは,1992年に大隈先生が初めて出芽酵母でのオートファジーの観察からであるが,その理由として次の2点を挙げて居る.
\begin{quotation}
    1つ目に,この時期の出芽酵母の研究は,従来の変異体の分離,遺伝解析といった古典的な遺伝学にくわえ,遺伝子クローニング,逆遺伝学といった分子生物学的な解析の技術基盤が確立していただけでなく,真核生物においてはじめてのゲノムプロジェクトとしてゲノム全塩基配列の解読が精力的になされていて,ゲノム情報の利用が可能になっていた.出芽酵母ゲノムプロジェクトの完了は1997年である.
    2つ目として,動物細胞におけるオートファジーは電子顕微鏡でしかとらえられなかったのに対し,出芽酵母のオートファジーは液胞酵素の不活性株を用いることで位相差顕微鏡によりオートファジックボディーの蓄積として容易にとらえることができた.\cite{大隈}
\end{quotation}


\section{病原微生物としての酵母}

\subsection{カンジダ症}
カンジダ症(candidiasis)とは,Candida属に分類されるC. albicansやC. glabrataなどの酵母により引き起こされる,日和見感染症である.


\begin{thebibliography}{9}
    \bibitem{酵母}
        jp.wikipedia.orgによる「出芽酵母」「並行複発酵」などのエントリー(6/27/2020確認)
    \bibitem{Nanba}
        Nanba, A. et al.: J. Ferment. Technol., 59, 383 (1981).
    \bibitem{有村}
        有村治彦、ビール酵母 日本醸造協会誌 Vol.95 (2000) No.11 P.791-802.
    \bibitem{pastorianus}
        Libkind, Diego; Hittinger, Chris Todd; Valério, Elisabete; Gonçalves, Carla; Dover, Jim; Johnston, Mark; Gonçalves, Paula; Sampaio, José Paulo (2011-08-30). “Microbe domestication and the identification of the wild genetic stock of lager-brewing yeast”. Proceedings of the National Academy of Sciences of the United States of America 108 (35): 14539–14544.
    \bibitem{ビール}
        ビール酒造組合HP(brewers.or.jp)
    \bibitem{Essential}
        Bruce Alberts et al. "Essential Cell Biology" (3rd Edition)
        New York, Garland Science, 2014.
    \bibitem{大隈}
        荒木保弘、大隅良典「オートファジーを長き眠りからめざめさせた酵母」『領域融合レビュー』第1巻e005、2012年9月19日
\end{thebibliography}

\end{document}