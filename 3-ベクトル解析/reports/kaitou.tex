\documentclass[dvipdfmx,a4paper,uplatex]{jsarticle}
%
\newcommand\GAKUSEISHOBANGO{J4-190549}% 学生証番号
\newcommand\NAMAE{司馬博文}% 氏名
% 上の二行について,{}内に記入の上末尾の「解答欄」以降を適宜編集すれば良い.
\usepackage{tikz, tikz-cd}
\usepackage{amsmath, amsfonts, amsthm, amssymb, ascmac, color, comment, wrap fig}
\newtheorem*{lemma*}{補題}
\usepackage{amsmath,amssymb,amscd,amsthm,amsbsy,multicol}
\usepackage[shortlabels,inline]{enumitem}
\renewcommand\labelenumi{\theenumi)}
\renewcommand{\thefootnote}{\dag\arabic{footnote}}
\DeclareMathOperator{\Div}{\mathrm{div}}
\DeclareMathOperator{\grad}{\mathrm{grad}}
\pagestyle{plain}
%
\setlength{\paperwidth}{257mm}
\setlength{\paperheight}{364mm}
\setlength{\textwidth}{170mm}
\setlength{\textheight}{280mm}
\setlength{\topmargin}{-31mm}
%
\raggedbottom
% \allowdisplaybreaks
%
% \everymath{\displaystyle}
%
\begin{document}
\thispagestyle{empty}
\setlength{\parindent}{1zw}
\setlength{\baselineskip}{13pt}
\setcounter{section}{9}
\newcounter{version}
\setcounter{version}{1}
\noindent
2020年度ベクトル解析(足助担当)期末レポート解答用紙%\par\noindent
\hfil2020/7/6(月)\par\noindent
提出先:ITC-LMSのページの「課題」\par\noindent
提出期限:2020/7/9(木)\textbf{12:00}\par\noindent
解答は\textbf{1問につき1枚(両面使用可)}とすること(この用紙を4枚印刷するとよい.あるいは2枚目以降は白紙を用いても良いが,\textbf{全て縦向き}とすること.
また,いずれの解答用紙の表面の上の方に\textbf{学生証番号と氏名を記入すること}).
\vskip-18pt\noindent
\begin{table}[h]
\begin{tabular}{|c|c|} \hline
& \\[-13pt]
学生証番号& 氏名 \\[2pt] \hline
\rule{0pt}{16pt}%
\parbox[c]{9.2zw}{\GAKUSEISHOBANGO\hfill} & \parbox[c]{13.0zw}{\NAMAE\hfill} \\[6pt] \hline
\end{tabular}
\end{table}

\par\noindent
\bf{問1 1)} 次のように定める正則な$C^\infty$級曲線$\gamma_1,\gamma_2,\gamma_3$について,その和を$\gamma:=\gamma_1+\gamma_2+\gamma_3$と置くと(向き付き曲線の和は問4の回答中での補題の証明中$(\star^1)$にて定義した),$\gamma$は$C$の向きと整合的な区分的$C^\infty$級の正則なパラメータ付けである.
\begin{center}
    \begin{tikzcd}
        \gamma_1:[0,1] \ar[r] \ar[d, phantom, "\rotatebox{90}{$\in$}"] & \mathbb{R}^2 \ar[d, phantom, "\rotatebox{90}{$\in$}"] & \gamma_2:\left[0,\frac{\pi}{2}\right] \ar[d, phantom, "\rotatebox{90}{$\in$}"] \ar[r] & \mathbb{R}^2 \ar[d, phantom, "\rotatebox{90}{$\in$}"] & \gamma_3:[0,1] \ar[d, phantom, "\rotatebox{90}{$\in$}"] \ar[r] & \mathbb{R}^2\ar[d, phantom, "\rotatebox{90}{$\in$}"]\\
        t \ar[r, mapsto] & \begin{pmatrix}t\\0\end{pmatrix} & \theta \ar[r, mapsto] & \begin{pmatrix}\cos\theta\\\sin\theta\end{pmatrix} & t \ar[r, mapsto] & \begin{pmatrix}0\\1-t\end{pmatrix}
    \end{tikzcd}
\end{center}
従って,これを用いて,求める線積分の値は次のように計算できる.
\begin{align*}
    \int_CX(p)\cdot dp &= \int_\gamma X(p)\cdot dp = \int_{\gamma_1}X(p)\cdot dp + \int_{\gamma_2}X(p)\cdot dp + \int_{\gamma_3}X(p)\cdot dp \\
    &= \int^1_0\left\langle X(\gamma_1(t))\;\middle|\;\frac{d\gamma_1}{dt}(t)\right\rangle dt + \int^{\frac{\pi}{2}}_0\left\langle X(\gamma_2(\theta))\;\middle|\;\frac{d\gamma_2}{d\theta}(\theta)\right\rangle d\theta + \int^1_0\left\langle X(\gamma_3(t))\;\middle|\;\frac{d\gamma_3}{dt}(t)\right\rangle dt \\
    &= \int^1_0\left\langle\begin{pmatrix}t^3\\0\end{pmatrix}\;\middle|\; \begin{pmatrix}1\\0\end{pmatrix}\right\rangle dt + \int^{\frac{\pi}{2}}_0\left\langle\begin{pmatrix}\cos^3\theta-3\cos\theta\sin^2\theta\\3\cos^2\theta\sin\theta-\sin^3\theta\end{pmatrix}\;\middle|\; \begin{pmatrix}-\sin\theta\\\cos\theta\end{pmatrix}\right\rangle d\theta + \int^1_0\left\langle\begin{pmatrix}0\\-(1-t)^3\end{pmatrix}\;\middle|\; \begin{pmatrix}0\\-1\end{pmatrix}\right\rangle dt \\
    &= \int^1_0\left( t^3+(1-t)^3 \right)dt + \int^{\frac{\pi}{2}}_0\sin 2\theta d\theta = \left[ \frac{t^4}{4} -\frac{(1-t)^4}{4} \right]^1_0 + \left[ -\frac{\cos 2\theta}{2} \right]^{\frac{\pi}{2}}_0 = \frac{3}{2}
\end{align*}
\begin{flushright}
    $\blacksquare$
\end{flushright}






\bf{2)} 曲線$C_d$とは,原点中心の単位球面と,${}^t(1,1,1)$を法線ベクトルとする原点からの距離$d$の平面との共通部分である.従って,$C_d=\emptyset\;(|d|>1の時)$であり,この時線積分は$\int_{C_d}f(p)|dp|=0\;(|d|>1)$であるから,以降$|d|\le 1$の場合を考える.
いま,$C_d$の$y$軸を中心とする$\frac{\pi}{4}$回転変換による像を$C'_d$とすると,これは単位球面と平面$x=d$との共通部分であるから,次の正則な$C^\infty$級関数$\gamma_d$によりパラメータ付けされる.
\begin{center}
    \begin{tikzcd}
        \gamma_d:[0,2\pi] \ar[r] \ar[d, phantom, "\rotatebox{90}{$\in$}"] & \mathbb{R}^3 \ar[d, phantom, "\rotatebox{90}{$\in$}"]\\
        \theta \ar[r, mapsto] & \begin{pmatrix}d\\\sqrt{1-d^2}\cos\theta\\\sqrt{1-d^2}\sin\theta\end{pmatrix}
    \end{tikzcd}
\end{center}
従って,標準基底${}^t(x,y,z)$に同様に$\frac{\pi}{4}$回転変換を施して得る基底${}^t\left(-\frac{1}{\sqrt{2}}x+\frac{1}{\sqrt{2}}z,y,\frac{1}{\sqrt{2}}x+\frac{1}{\sqrt{2}}z\right)=:{}^t(x',y',z')$で表された関数$f(x',y',z')=x'^2$を
座標${}^t(x,y,z)$について表示した関数を$g(x,y,z):=\left(-\frac{1}{\sqrt{2}}x+\frac{1}{\sqrt{2}}z\right)^2=\frac{1}{2}(z^2-2xz+x^2)$とすると,$|d|\le 1$の下で,求める線積分は次のように計算できる.
\begin{align*}
    \int_{C_d}f(p)|dp| &= \int_{C'_d}g(p)|dp| = \int_{\gamma_d}g(p)|dp| \\
    &= \int^{2\pi}_0 g(\gamma_d(\theta))\left\| \frac{d\gamma_d}{d\theta} \right\|d\theta \\
    &= \frac{1-d^2}{2}\int^{2\pi}_0\left( \frac{1+d^2}{2} -2d\sqrt{1-d^2}\sin\theta - \frac{1-d^2}{2}\cos 2\theta \right)d\theta \\
    &= \frac{1-d^2}{2}\left[ \frac{1+d^2}{2}\theta + 2d\sqrt{1-d^2}\cos\theta - \frac{1-d^2}{4}\sin 2\theta \right]^{2\pi}_0 \\
    &= \frac{\pi}{2}(1-d^4)\;\;\;(|d|\le 1)\hspace{11cm}\blacksquare
\end{align*}
\clearpage





\bf{問2} 境界を含む有界な領域$D$とその境界である曲面$\partial D$について,Gau\ss の発散定理より,
\begin{align*}
    \int_{\partial D}X\cdot dA &= \int_D (\Div X)\;d\mathrm{vol} \\
    &= \int_D 3x dxdydz
\end{align*}
であるから,次のように$D_1,D_2,D_3$を定めると,$D=D_1\cup D_2\cup D_3$であり,$D_1\cap D_3=\emptyset$,また$D_1\cap D_2,D_2\cap D_3$はいずれも単位円板で体積$0$であるから,$\int_D 3xdxdydz=\int_{D_1}3xdxdydz + \int_{D_2}3xdxdydz + \int_{D_3}3xdxdydz$が成立する.
\begin{align*}
    D_1 &= \left\{ \varphi_1(r,\theta,\varphi)=\begin{pmatrix}r\sin\theta\cos\varphi\\r\sin\theta\sin\varphi\\r\cos\theta\end{pmatrix}\in D \;\middle|\; 0\le r\le 1,0\le\theta\le\pi,\frac{\pi}{2}\le\varphi\le\frac{3}{2}\pi \right\} \\
    D_2 &= \left\{ \varphi_2(r,\theta,\varphi)=\begin{pmatrix}x\\r\cos\theta\\r\sin\theta\end{pmatrix}\in D \;\middle|\; 0\le r\le 1,0\le x\le 1,0\le\theta\le 2\pi \right\} \\
    D_3 &= \left\{ \varphi_3(r,\theta,\varphi)=\begin{pmatrix}1+r\sin\theta\cos\varphi\\r\sin\theta\sin\varphi\\r\cos\theta\end{pmatrix}\in D \;\middle|\; 0\le r\le 1,0\le\theta\le\pi,-\frac{\pi}{2}\le\varphi\le +\frac{\pi}{2} \right\}
\end{align*}
それぞれの境界を含んだ領域$D_1,D_2,D_3$のパラメータ付け$\varphi_1,\varphi_2,\varphi_3$のYacobianが,
\begin{align*}
    \det D\varphi_1=\det D\varphi_3&=\det\begin{pmatrix}\sin\theta&r\cos\theta\cos\varphi&-r\sin\theta\sin\varphi\\\sin\theta\sin\varphi&r\cos\theta\sin\varphi&r\sin\theta\cos\varphi\\\cos\theta&-r\sin\theta&0\end{pmatrix} = r^2\sin\theta \\
    \det D\varphi_2&=\det\begin{pmatrix}1&0&0\\0&\cos\theta&\sin\theta\\0&-r\sin\theta&r\cos\theta\end{pmatrix} = r &
\end{align*}
であることに注意すれば,変数変換公式より,それぞれの重積分は次のように計算できる.
\begin{align*}
    \int_{D_1}xdxdydz &= \int^1_0\int^\pi_0\int^{\frac{3}{2}\pi}_{\frac{\pi}{2}}r\sin\theta\cos\varphi\cdot r^2\sin\theta d\varphi d\theta dr =-\frac{\pi}{4} \\
    \int_{D_2}xdxdydz &= \int^{2\pi}_0\int^1_0\int^1_0 x\cdot rdrdxd\theta = \frac{\pi}{2} \\
    \int_{D_3}xdxdydz &= \int^1_0\int^\pi_0\int^{\frac{\pi}{2}}_{-\frac{\pi}{2}}(1+r\sin\theta\cos\varphi)r^2\sin\theta d\varphi d\theta dr = \frac{11}{12}\pi
\end{align*}
従って,
\begin{align*}
    \int_{\partial D}X\cdot dA &= \int_D (\Div X)\;d\mathrm{vol} \\
    &= \int_D 3x dxdydz\\
    &=3\left(\int_{D_1}xdxdydz + \int_{D_2}xdxdydz + \int_{D_3}xdxdydz\right)\\
    &=3\left(-\frac{\pi}{4}+\frac{\pi}{2}+\frac{11}{12}\pi\right)=\frac{7}{2}\pi
\end{align*}
を得る.
\begin{flushright}
    $\blacksquare$
\end{flushright}
\clearpage





\bf{問3 1)} $f^1(x,y)=\frac{-2y}{x^2+y^2}+\frac{-(y-2)}{x^2+(y-2)^2},\; f^2(x,y)=\frac{2x}{x^2+y^2}+\frac{x}{x^2+(y-2)^2}$と置くとこれはいずれも$C^\infty$級の関数で,$X=f^1\frac{\partial}{\partial x}+f^2\frac{\partial}{\partial y}$と表せる.
これが定める$D$上の$C^\infty$級の1-形式$\omega:=f^1dx+f^2dy$を,次の$C^\infty$級関数$\gamma={}^t(\gamma^1,\gamma^2)$で引き戻して考える.
\begin{center}
    \begin{tikzcd}
        \gamma:\mathbb{R}^3 \ar[r] \ar[d, phantom, "\rotatebox{90}{$\in$}"] & D \ar[d, phantom, "\rotatebox{90}{$\in$}"]\\
        \begin{pmatrix}p\\q\\r\end{pmatrix} \ar[r, mapsto] & \begin{pmatrix}p+t\\q\end{pmatrix}
    \end{tikzcd}
\end{center}
この$\gamma$による$D$上の1-形式$dx,dy$の引き戻しは
\begin{align*}
    \gamma^*dx &= d\gamma^1 \\
    &= \frac{\partial\gamma^1}{\partial p}dp + \frac{\partial\gamma^1}{\partial q}dq + \frac{\partial\gamma^1}{\partial t}dt\\
    &= dp+dt\\
    \gamma^*dy &= \frac{\partial\gamma^2}{\partial p}dp + \frac{\partial\gamma^2}{\partial q}dq + \frac{\partial\gamma^2}{\partial t}dt\\
    &= dq
\end{align*}
であることより,$\gamma$による$\omega$の引き戻しは
\begin{align*}
    \gamma^*\omega &= (\gamma^*f^1)\;(\gamma^*dx) + (\gamma^*f^2)\;(\gamma^*dy) \\
    &= (f^1\circ\gamma) dp + (f^2\circ\gamma )dq + (f^1\circ\gamma) dt
\end{align*}
となる.定数分の差は重要ではないから,$dt$の付く項のみを採用して,また$p$を$x$,$q$を$y$を書き直して$D$上の$C^\infty$級関数$g(x,y)$を次のように定義することとする.
\[ g(x,y):=-\int^\infty_0f^1\circ\gamma(x,y,t) dt = -\int^\infty_0 \left( \frac{-2y}{(x+t)^2+y^2} + \frac{-(y-2)}{(x+t)^2+(y-2)^2} \right)dt \]
なお,$t\ge 0$に於いて,${}^t(x,y)\in D$より$x>0$に注意して,三角不等式より,
\begin{align*}
    (0\le)\left| \frac{-2y}{(x+t)^2+y^2} + \frac{-(y-2)}{(x+t)^2+(y-2)^2} \right| &= \left| \frac{-2y}{(x^2+y^2+t^2+2xt)} + \frac{-(y-2)}{(x^2+y^2+t^2+2xt-4y+4)} \right|\\
    &\le \left| \frac{-2y}{x^2+y^2+t^2} \right| + \left| \frac{-(y-2)}{x^2+y^2+t^2} \right|
\end{align*}
が成り立つから,
\begin{align*}
    \int^\infty_0 \left| \frac{-2y}{(x+t)^2+y^2} + \frac{-(y-2)}{(x+t)^2+(y-2)^2} \right|dt &\le \int^\infty_0 \left(\left| \frac{-2y}{x^2+y^2+t^2} \right| + \left| \frac{-(y-2)}{x^2+y^2+t^2} \right|\right)dt \\
    &= |-2y|\int^{\frac{\pi}{2}}_0 \left| \frac{\cos^2\theta}{x^2+y^2} \right|\frac{\sqrt{x^2+y^2}}{\cos^2\theta}d\theta + |-(y-2)| \int^{\frac{\pi}{2}}_0 \left| \frac{\cos^2\theta}{x^2+y^2} \right|\frac{\sqrt{x^2+y^2}}{\cos^2\theta}d\theta\\
    &= (2|y|+|y-2|)\int^{\frac{\pi}{2}}_0\left| \frac{1}{\sqrt{x^2+y^2}} \right|d\theta = \frac{\pi}{2}\frac{2|y|+|y-2|}{\sqrt{x^2+y^2}}
\end{align*}
と評価できるが,最右辺は有限な値だから,最左辺は有界である.従って,積分$g(x,y)=-\int^\infty_0f^1\circ\gamma(x,y,t) dt$は絶対収束し,関数$g$は確かに$D$上で定義される.
この関数$g$がベクトル場$X$の領域$D$への制限$X|_U$のスカラーポテンシャルとなっていること,即ち$\grad g = X|_U$を証明する.$\grad g=\frac{\partial g}{\partial x}\frac{\partial}{\partial x} + \frac{\partial g}{\partial y}\frac{\partial}{\partial y}$であるから,
係数が等しいこと,即ち$f^1=\frac{\partial g}{\partial x},f^2=\frac{\partial g}{\partial y}$を示せば良い.

関数$g$の非積分関数$h(x,y,t):=\frac{-2y}{(x+t)^2+y^2} + \frac{-(y-2)}{(x+t)^2+(y-2)^2}$は$C^\infty$級で,特に連続であるから,その積分$G(x,y,s):=-\int^s_0\left( \frac{-2y}{(x+t)^2+y^2} + \frac{-(y-2)}{(x+t)^2+(y-2)^2} \right)dt$も連続で,従って$s\to\infty$の時に$g(x,y)$には一様収束する.
このことより,$\frac{\partial g}{\partial x}$は次のように計算できる.
\begin{align*}
    \frac{\partial g}{\partial x} &= \frac{\partial}{\partial x}\int^0_\infty \left( \frac{-2y}{(x+t)^2+y^2} + \frac{-(y-2)}{(x+t)^2+(y-2)^2} \right)dt\\
    &= \int^0_\infty \frac{\partial}{\partial x}\left( \frac{-2y}{(x+t)^2+y^2} + \frac{-(y-2)}{(x+t)^2+(y-2)^2} \right)dt\\
    &= \left[ \frac{-2y}{(x+t)^2+y^2} + \frac{-(y-2)}{(x+t)^2+(y-2)^2} \right]^0_\infty = \frac{-2y}{x^2+y^2}+\frac{-(y-2)}{x^2+(y-2)^2} = f^1(x,y)
\end{align*}
続いて,$\frac{\partial g}{\partial y}$は次のように計算できる.
\begin{align*}
    \frac{\partial g}{\partial y} &= \frac{\partial}{\partial y}\int^0_\infty \left( \frac{-2y}{(x+t)^2+y^2} + \frac{-(y-2)}{(x+t)^2+(y-2)^2} \right)dt\\
    &= \int^0_\infty \frac{\partial}{\partial y}\left( \frac{-2y}{(x+t)^2+y^2} + \frac{-(y-2)}{(x+t)^2+(y-2)^2} \right)dt\\
    &= \int^0_\infty \left( \frac{-2((x+t)^2+y^2)+2y(2y)}{((x+t)^2+y^2)^2}+\frac{-((x+t)^2+(y-2)^2)+2(y-2)^2}{((x+t)^2+(y-2)^2} \right)dt \\
    &= \int^0_\infty \left( -2\left( \frac{2(x+t)^2}{((x+t)^2+y^2)^2}-\frac{1}{(x+t)^2+y^2} \right) - \left( \frac{2(x+t)^2}{((x+t)^2+(y-2)^2)^2}-\frac{1}{(x+t)^2+(y-2)^2}\right) \right)dt\\
    &= -4\int^0_\infty\frac{(x+t)^2}{((x+t)^2+y^2)^2}dt + 2\int^0_\infty\frac{dt}{(x+t)^2+y^2}-\int^0_\infty\frac{2(x+t)^2}{((x+t)^2+(y-2)^2)^2}dt+\int^0_\infty\frac{dt}{(x+t)^2+(y-2)^2}\;\;\;\cdots(\ast)\\
\end{align*}
部分積分の方法により,
\begin{align*}
    2\int^0_\infty\frac{(x+t)^2}{((x+t)^2+y^2)^2}dt &= \int^0_\infty -(x+t)\left(\frac{1}{(x+t)^2+y^2}\right)'dt\\
    &= -\left[\frac{x+t}{(x+t)^2+y^2} \right]^0_\infty + \int^0_\infty\frac{dt}{(x+t)^2+y^2}\\
    &= -\frac{x}{x^2+y^2} + \int^0_\infty\frac{dt}{(x+t)^2+y^2}
\end{align*}
であるから,$(\ast)$は,
\begin{align*}
    &\left( 2\frac{x}{x^2+y^2}-2\int^0_\infty\frac{dt}{(x+t)^2+y^2} \right) + 2\int^0_\infty\frac{dt}{(x+t)^2+y^2}+\left( \frac{x}{x^2+(y-2)^2} - \int^0_\infty\frac{dt}{(x+t)^2+(y-2)^2} \right)\\*
    &\hphantom{{}={}}+ \int^0_\infty\frac{dt}{(x+t)^2+(y-2)^2} \\
    &= \frac{2x}{x^2+y^2}+\frac{x}{x^2+(y-2)^2}=f^2(x,y)
\end{align*}
と計算できる.
以上より,$D$上の関数$g(x,y)=-\int^\infty_0 \left( \frac{-2y}{(x+t)^2+y^2} + \frac{-(y-2)}{(x+t)^2+(y-2)^2} \right)dt$は,ベクトル場$X$の領域$D$への制限のスカラーポテンシャルである.
%これが定める$C^\infty$級の1-形式を$\omega:=f^1dx+f^2dy$と置くと,これは閉形式であることが,次のように確かめられる.
%\begin{align*}
%    d\omega &= d(f^1dx*f^2dy) \\
%    &= df^1\wedge dx + df^2\wedge dy\\
%    &= \left( \frac{\partial f^2}{\partial x}-\frac{\partial f^1}{\partial y} \right) dx\wedge dy \\
%    &= \left( 2\frac{(x^2+y^2)-2x^2}{(x^2+y^2)^2} + \frac{x^2+(y-2)^2-2x^2}{(x^2+(y-2)^2)^2} +\right.\\*
%    &\hphantom{{}={}}- \left. 2\frac{-(x^2+y^2)+2y^2}{(x^2+y^2)^2} - \frac{-(x^2+(y-2)^2)+2(y-2)^2}{(x^2+(y-2)^2)^2} \right) dx\wedge dy \\
%    &= 0
%\end{align*}
%領域$D$は星形だから(任意の$p_0\in D$について,$\forall p\in D\; \overline{pp_0}\subset D$が成り立つ.ただし,$\overline{pp_0}$とは2点$p,p_0$を結ぶ線分とした),
%Poincaréの補題より,0-形式,即ち$C^\infty$級関数$g$が存在して$dg=\omega$を満たす.
\begin{flushright}
    $\blacksquare$
\end{flushright}





\bf{2)} $\mathbb{R}^2\setminus\{(0,0),(0,2)\}$上で定義された関数$g$が存在して$X=\grad g$を満たすと仮定し,矛盾を導く.
$\gamma(\theta)=\begin{pmatrix}\cos\theta\\\sin\theta\end{pmatrix}$と自己交叉のない閉曲線$\gamma:[0,2\pi]\to \mathbb{R}^2$を定めると,$\gamma([0,2\pi])\subset\mathbb{R}^2\setminus\{(0,0),(0,2)\}$であるから,勾配ベクトル場に関する積分定理より,次のようにして曲線$\gamma$に沿ったベクトル場$X$の線積分を求めることができる.
\begin{align*}
    \int_\gamma X\cdot dx &= \int_\gamma\grad g\cdot dx \\
%    &= \int_\gamma \frac{\partial g}{\partial x}dx + \frac{\partial g}{\partial y}dy \\
%    &= \int_\gamma dg
    &= \int_{\partial\gamma}g
\end{align*}
ここで$\gamma$は単純閉曲線であるから,$\partial\gamma=\emptyset$.従って,$\int_\gamma X\cdot dx=\int_{\partial\gamma}g=0$が導かれる.

一方,この$\gamma$に沿ったベクトル場$X$の線積分の値は,次の$g$を用いない場合の計算結果と矛盾する.
\begin{align*}
    \int_\gamma X\cdot dx &= \int^{2\pi}_0\left\langle X(\gamma(\theta))\;\middle|\;\frac{d\gamma}{d\theta}(\theta)\right\rangle d\theta \\
    &= \int^{2\pi}_0\left\langle\begin{pmatrix}-2\sin\theta+\frac{-(\sin\theta-2)}{5-4\sin\theta}\\2\cos\theta+\frac{\cos\theta}{5-4\sin\theta}\end{pmatrix}\;\middle|\;\begin{pmatrix}-\sin\theta\\\cos\theta\end{pmatrix}\right\rangle d\theta \\
    &= \int^{2\pi}_0 \left(\frac{5}{2}-\frac{3}{2}\frac{1}{5-4\sin\theta}\right) \\
    &\ge\int^{2\pi}_0\left(\frac{5}{2}-\frac{3}{2}\frac{1}{5-4}\right) = 2\pi
\end{align*}
従って,$\mathbb{R}^2\setminus\{(0,0),(0,2)\}$上で定義された$X$のスカラーポテンシャルは存在しない.$\blacksquare$
% 実際にこの積分を計算すると$4\pi$.特異点が原点に2つ縮退しているからである.
\clearpage





\bf{問4} $f^1(x,y)=\frac{-y}{x^2+y^2}+\frac{-(y-2)}{x^2+(y-2)^2}+\frac{-(y+2)}{x^2+(y+2)^2},\; f^2(x,y)=\frac{x}{x^2+y^2}+\frac{x}{x^2+(y-2)^2}+\frac{x}{x^2+(y+2)^2}$と置くとこれはいずれも$C^\infty$級で,$X=f^1\frac{\partial}{\partial x}+f^2\frac{\partial}{\partial y}$と表せる.
曲線$\gamma([0,1])$上の相異なる2点$a=\gamma(s),b=\gamma(t)\;(s,t\in[0,1],s<t)$を結ぶ区分的に$C^1$級かつ正則で,自己交叉を持たず,$\gamma([0,1])\setminus \{a,b\}$と共通部分を持たず,$s=0$または$t=1$である場合を除いて$C$とも共通部分を持たない曲線$\gamma':[s,t]\to\mathbb{R}^2\setminus\{p,q,r\}$を1つ取る.但し,向きは$\gamma'(s)=a,\gamma'(t)=b$とする.これについて次のように$\gamma$を部分的に変更した曲線
\[\gamma''(u)=\begin{cases}
    \gamma(u) & u\in [0,1]\setminus [s,t]\\
    \gamma'(u) & u\in [s,t]
\end{cases}\]
は再び区分的に$C^1$級かつ正則な曲線で,i),ii),iii)を満たし,$\gamma''([0,1])$には$\gamma''$による自然な向きが定まる.この場合も同様に,$C$と$\gamma''$を繋げて得られる閉曲線を$C_{\gamma''}$とする.
これについて,次の補題が成り立つ.
\begin{lemma*}
    線積分$\int_{C_\gamma}X(p)\cdot dp$の値は,積分路$C_\gamma$が囲む特異点$p,q,r$の組合わせで定まり,積分路$C_\gamma$を特異点$p,q,r$を跨がない範囲で変更しても値は変わらない.
    但し,証明を簡明化するため,$C_\gamma$から$C_{\gamma''}$への変化分の領域(証明中の言葉では領域$\Delta D$)は星形とする.即ち,上記の通り定めた閉曲線$C_{\gamma''}$であって,$p,q,r$のいずれも通らずに$C_\gamma$から$C_{\gamma''}$へと連続的に変形できる場合(証明中に定義する記号では,$p,q,r\notin \Delta D$の場合)について,次が成り立つ.
    \[ \int_{C_\gamma}X(p)\cdot dp = \int_{C_\gamma''}X(p)\cdot dp \]
\end{lemma*}
\begin{proof}[\bf{証明}]
条件i),ii),iii)より,$C_\gamma,C_{\gamma''}$はいずれも単純閉曲線であるから,Jordanの閉曲線定理より,それぞれについて$\mathbb{R}^2\setminus C_\gamma,\mathbb{R}^2\setminus C_{\gamma''}$の2つの連結部分のうち有界な方が取れる.これをそれぞれ領域$D_\gamma,D_{\gamma''}$とする.
この時,$\gamma'$は端点を除いて$\gamma$と共通部分を持たないので,$D_\gamma\subset D_{\gamma''}$か$D_{\gamma''}\subset D_\gamma$かのいずれかが成り立つ.ぞれぞれの場合について,$\Delta D:=D_{\gamma''}\setminus D_\gamma$,$\Delta D:=D_\gamma\setminus D_{\gamma''}$と定め,$\Delta D$が$p,q,r$のいずれをも含まない場合を考える.
ただし,曲線$\zeta:[c,d]\to\mathbb{R}^2$に対して曲線$-\zeta:[c,d]\to\mathbb{R}^2$を$-\zeta(u):=\zeta(c+d-u)\;(u\in[c,d])$とすることにして,向き付きの閉曲線$\partial(\Delta D)\subset\mathbb{R}^2$を,それぞれの場合についてパラメータ$\partial(\Delta D)=\gamma'+(-(\gamma|_{[s,t]}))$または$\partial(\Delta D)=\gamma|_{[s,t]}+(-\gamma')$が定める閉曲線と定める.
(但し,端点のみを共有する曲線$\zeta:[c,d]\to\mathbb{R}^2,\zeta':[c',d']\to\mathbb{R}^2$かつ$\zeta(d)=\zeta'(c')$に対して曲線$\zeta+\zeta'$とは,
$(\zeta+\zeta')(u)=\begin{cases}
    \zeta(u) & u\in[c,d] \\
    \zeta'(u) & u\in[d,d+(d'-c')]
\end{cases}$により定まる曲線$\zeta+\zeta':[c,d+d'-c']\to\mathbb{R}^2$とした$\cdots(\star^1)$).
いま,$X$の定める1-形式$f^1dx+f^2dy$は,
\begin{align*}
    d(f^1dx+f^2dy) &= df^1\wedge dx+df^2\wedge dy\\
    &= \left( \frac{\partial f^2}{\partial x}-\frac{\partial f^1}{\partial y} \right) dx\wedge dy \\
    &= \left( \frac{(x^2+y^2)-2x^2}{(x^2+y^2)^2} + \frac{x^2+(y-2)^2-2x^2}{(x^2+(y-2)^2)^2} + \frac{(x^2+(y+2)^2-2x^2)}{(x^2+(y+2)^2)^2}\right.\\*
    &\hphantom{{}={}}- \left. \frac{-(x^2+y^2)+2y^2}{(x^2+y^2)^2} - \frac{-(x^2+(y-2)^2)+2(y-2)^2}{(x^2+(y-2)^2)^2} - \frac{-(x^2+(y+2)^2)+2(y+2)^2}{(x^2+(y+2)^2)^2} \right) dx\wedge dy \\
    &= 0
\end{align*}
より閉形式である.従って,仮定より
$p,q,r\notin \Delta D\cup\partial(\Delta D)$であり($p,q,r\notin\gamma''([s,t]),p,q,r\notin\gamma([0,1])$であるので,$p,q,r\notin \Delta D$ならばこれを満たす),また領域$\Delta D$は星形としたから,$\Delta D\cup\partial(\Delta D)\subset U$を満たす或る$p,q,r$を含まない領域$U$が存在して星形になり,Poincaréの補題より,ベクトル場$X$の$U$への制限$X|_U$は,$U$上で定義されたスカラーポテンシャル$g$が存在して$\grad X|_U=g$を満たす.
従って,$D_\gamma\subset D_{\gamma''}$の時($D_{\gamma''}\subset D_\gamma$の場合も同様),勾配ベクトル場に関する積分定理と,$\partial(\Delta D)$が区分的に正則かつ$C^1$級な単純閉曲線$\gamma'+(-\gamma|_{[s,t]})$であることより,
\begin{align*}
    \int_{\partial(\Delta D)}X(p)\cdot dp &= \int_{\gamma'}X(p)\cdot dp + \int_{-\gamma|_{[s,t]}}X(p)\cdot dp \\
    &= \int_{\gamma'}X(p)\cdot dp - \int_{\gamma|_{[s,t]}}X(p)\cdot dp 
    = \int_{\partial(\partial(\Delta D))} g = 0
\end{align*}
であるから,
\[ \int_{\gamma'}X(p)\cdot dp = \int_{\gamma|_{[s,t]}}X(p)\cdot dp \]
従って,
\[ \int_{C_\gamma}X(p)\cdot dp = \int_{C_{\gamma''}}X(p)\cdot dp \]
が成り立つ.以上より,積分路を$C_\gamma$から$C_{\gamma''}$へと,特異点$p,q,r$を跨がない範囲かつ$\Delta D$が星形になる範囲で変更しても,線積分の値は一定である.
また積分路の$C_\gamma$から$C_{\gamma''}$への変更の仮定を十分細かく分解することにより,一般の場合についても主張は成り立つ.(補題の証明終わり)
\end{proof}

補題より,積分路$C_\gamma$が内部$D_\gamma$に含む特異点$p,q,r$の数と種類に依って積分の値が変化することが分かる.
これを調べるために,まず「$x$軸に沿った積分路$\gamma^{x=v_1\to v_2}_{y=l}$」と「$y$軸に沿った積分路$\gamma^{x=k}_{y=h_1\to h_2}$」の2つの場合について一般的な場合で線積分の値を求め,
この結果を用いて具体的に4種類の曲線$C_{\gamma_0},R_p,R_q,R_r$に沿った線積分を計算し,この結果から線積分$\int_{C_\gamma}X(p)\cdot dp$の値は領域$D_\gamma$内の特異点の個数のみに依ることを導き,
実際にこれらの積分路を組み合わせることで,特異点を0,1,2,3個含む積分路を全ての場合について構成して,線積分の値を求める.

まず,次の2つの曲線(線分)${\gamma^{x=v_1\to v_2}_{y=l}}:{[v_1,v_2]}\to\mathbb{R}^2\setminus\{p,q,r\},{\gamma^{x=k}_{y=h_1\to h_2}}:{[h_1,h_2]}\to\mathbb{R}^2\setminus\{p,q,r\}$を,それぞれ$t\mapsto\begin{pmatrix}t\\l\end{pmatrix},t\mapsto\begin{pmatrix}k\\t\end{pmatrix}$によって定義する$\cdots(\star^2)$.
%\begin{center}
%    \begin{tikzcd}
%        {\gamma^{x=v_1\to v_2}_{y=l}}:{[v_1,v_2]}\ar[r]\ar[d, phantom, "\rotatebox{90}{$\in$}"] & \mathbb{R}^2\ar[d, phantom, "\rotatebox{90}{$\in$}"]&&{\gamma^{x=k}_{y=h_1\to h_2}}:{[h_1,h_2]}\ar[r]\ar[d, phantom, "\rotatebox{90}{$\in$}"]&\mathbb{R}^2\ar[d, phantom, "\rotatebox{90}{$\in$}"]\\
%        t \ar[r, mapsto] & \begin{pmatrix}t\\l\end{pmatrix}&&t\ar[r,mapsto]&\begin{pmatrix}k\\t\end{pmatrix}
%    \end{tikzcd}
%\end{center}
$\gamma^{x=v_1\to v_2}_{y=l}$に沿った線積分の値は,各項について変数変換$t=l\tan\theta,t=(l-2)\tan\theta,t=(l+2)\tan\theta$による置換積分により,次のように計算できる($l\ne 0,\pm 2$の時).
\begin{align}
    \int_{\gamma^{x=v_1\to v_2}_{y=l}} X(p)\cdot dp &= \int^{v_2}_{v_1}\left( \frac{-l}{t^2+l^2}+\frac{-(l-2)}{t^2+(l-2)^2}+\frac{-(l+2)}{t^2+(l+2)^2} \right)dt \\
    &= -l\int^{\arctan\left(\frac{v_2}{l}\right)}_{\arctan\left(\frac{v_1}{l}\right)}\frac{d\theta}{l} - (l-2)\int^{\arctan\left(\frac{v_2}{l-2}\right)}_{\arctan\left(\frac{v_1}{l-2}\right)}\frac{d\theta}{l-2} - (l+2)\int^{\arctan\left(\frac{v_2}{l+2}\right)}_{\arctan\left(\frac{v_1}{l+2}\right)}\frac{d\theta}{l+2}\\
    &= -\arctan\left(\frac{v_2}{l}\right) + \arctan\left(\frac{v_1}{l}\right) - \arctan\left(\frac{v_2}{l-2}\right) + \arctan\left(\frac{v_1}{l-2}\right) \\*
    &\hphantom{{}={}}- \arctan\left(\frac{v_2}{l+2}\right) + \arctan\left(\frac{v_1}{l+2}\right) \;(l\ne 0,\pm 2)
\end{align}
なお,$l=0$の時は,$\int_{\gamma^{x=v_1\to v_2}_{y=l}} X(p)\cdot dp =\int^{v_2}_{v_1} \left(0+\frac{2}{t^2+4}+\frac{-2}{t^2+4}\right)dt = 0\;(l=0)\cdots\;(*)$である.また,$\gamma^{x=k}_{y=h_1\to h_2}$に沿った線積分の値も,同様の計算により次の結果を得る.
\begin{align}
    \int_{\gamma^{x=k}_{y=h_1\to h_2}} X(p)\cdot dp &= \arctan\left(\frac{h_2}{k}\right) - \arctan\left(\frac{h_1}{k}\right) + \arctan\left(\frac{h_2-2}{k}\right) - \arctan\left(\frac{h_1-2}{k}\right) \\*
    &\hphantom{{}={}}+ \arctan\left(\frac{h_2+2}{k}\right) - \arctan\left(\frac{h_1+2}{k}\right) \;(k\ne 0,\pm 2)
\end{align}
$\arctan$は奇関数であるから,どちらの積分の値も,それぞれ積分路に対する次の2種の変換「$y$軸対称変換$v_1\mapsto -v_1,v_2\mapsto -v_2$または$h_1\mapsto -h_1,h_2\mapsto -h_2$」と「$x$軸対称変換$l\mapsto -l$または$h\mapsto -h$」に対して符号が逆転することに注目して,次の4種の積分路について積分を計算する.
($\star^1$)で定義した向き付き曲線の和(但し$\zeta+(-\zeta)=\emptyset$する)の記法を用いる.
まず特異点を1つも囲まない長方形閉曲線$C_{\gamma_0}$を$C$と$\gamma_0:=\gamma^{x=2}_{y=0\to 3}+\gamma^{x=2\to 1}_{y=3}+\gamma^{x=1}_{y=3\to 0}$を繋げたものとし,各点$q,p,r$を中心とした一辺$2$の正方形閉曲線$R_q,R_p,R_r$を
それぞれ曲線$\gamma_q:=\gamma^{x=-1\to 1}_{y=1}+\gamma^{x=1}_{y=1\to 3}+\gamma^{x=1\to -1}_{y=3}+\gamma^{x=-1}_{y=3\to 1},\;\gamma_p:=\gamma^{x=-1\to 1}_{y=-1}+\gamma^{x=1}_{y=-1\to 1}+\gamma^{x=1\to -1}_{y=1}+\gamma^{x=-1}_{y=1\to -1},\;\gamma_r:=\gamma^{x=-1\to 1}_{y=-3}+\gamma^{x=1}_{y=-3\to -1}+\gamma^{x=1\to -1}_{y=-1}+\gamma^{x=-1}_{y=-1\to -3}$
とそれが定める自然な向きを持った閉路とする.
それぞれに沿った線積分の値を,式$(3),(4)$と$(5),(6)$と$(*)$に当てはめて,前述した奇関数性を利用して計算すると,まず$R_0,R_q$について次のようになる.
\begin{align*}
    \int_{R_0}X(p)\cdot dp &= \int_CX(p)\cdot dp + \int_{\gamma^{x=2}_{y=0\to 3}}X(p)\cdot dp + \int_{\gamma^{x=2\to 1}_{y=3}}X(p)\cdot dp + \int_{\gamma^{x=1}_{y=3\to 0}}X(p)\cdot dp\\
    &= 0 + \left( \arctan\left(\frac{1}{2}\right) + \arctan\left(\frac{3}{2}\right) + \arctan\left(\frac{5}{2}\right) \right) + \left( -\arctan\left(\frac{1}{3}\right) + \arctan\left(\frac{2}{3}\right) - \arctan\left(1\right) \right. \\*
    &\hphantom{{}={}}+ \left. \arctan(2) - \arctan\left(\frac{1}{5}\right) + \arctan\left(\frac{2}{5}\right) \right)
    -\left(\arctan(1)+\arctan(3)+\arctan(5)\right) = 0 \\
    \int_{R_q}X(p)\cdot dp &= \int_{\gamma^{x=-1\to 1}_{y=1}}X(p)\cdot dp + \int_{\gamma^{x=1}_{y=1\to 3}}X(p)\cdot dp + \int_{\gamma^{x=1\to -1}_{y=3}}X(p)\cdot dp + \int_{\gamma^{x=-1}_{y=3\to 1}}X(p)\cdot dp \\
    &= 2\left(\arctan(2)-\arctan(3)+\arctan(4)\right) + 2\left(\arctan(1)-\arctan(2)+\arctan(3)-\arctan(4) \right. \\*
    &\hphantom{{}={}}+ \left. \arctan(5)\right) +2\left(\arctan(1)+\arctan(1/3)+\arctan(1/5)\right) - 2\arctan(1/3)
    =2\pi
\end{align*}
前述した奇関数性に注意すれば,同様の計算により$\int_{R_p}X(p)\cdot dp=\int_{R_r}X(p)\cdot dp=2\pi$も得る.
積分路$R_p,R_q,R_r$での線積分での値がどれも等しいから,補題と併せて,線積分の値は積分路が囲む特異点の数に依ることが分かる.
すると,特異点を1つのみ囲む閉じた経路$C_{\gamma_0+\gamma_q}$について,$\int_{C_{\gamma_0+\gamma_q}}X(p)\cdot dp=2\pi$である.
以降同様に,$\int_{C_{\gamma_0+\gamma_q+\gamma_p}}X(p)\cdot dp=4\pi, \int_{C_{\gamma_0+\gamma_q+\gamma_p+\gamma_r}}X(p)\cdot dp=6\pi$を得る.

以上の結果を,($\star^2$)で定義した曲線の記法を用いてまとめると,線積分$\int_{C_\gamma}X(p)\cdot dp$の取り得る値とその時の積分路の例は次の通りとなる.
\[
    \int_{C_\gamma}X(p)\cdot dp = 
    \begin{cases}
        0 & \gamma=\gamma_0=\gamma^{x=2}_{y=0\to 3}+\gamma^{x=2\to 1}_{y=3}+\gamma^{x=1}_{y=3\to 0}の時\\
        2\pi & \gamma=\gamma_0+\gamma_q=\gamma^{x=2}_{y=0\to 3}+\gamma^{x=2\to -1}_{y=3}+\gamma^{x=-1}_{y=3\to 1}+\gamma^{x=-1\to 1}_{y=1}+\gamma^{x=1}_{y=1\to 0}の時\\
        4\pi & \gamma=\gamma_0+\gamma_q+\gamma_p=\gamma^{x=2}_{y=0\to 3}+\gamma^{x=2\to -1}_{y=3}+\gamma^{x=-1}_{y=3\to -1}+\gamma^{x=-1\to 1}_{y=-1}+\gamma^{x=1}_{y=-1\to 0}の時\\
        6\pi & \gamma=\gamma_0+\gamma_q+\gamma_p+\gamma_r=\gamma^{x=2}_{y=0\to 3}+\gamma^{x=2\to -1}_{y=3}+\gamma^{x=-1}_{y=3\to -3}+\gamma^{x=-1\to 1}_{y=-3}+\gamma^{x=1}_{y=-3\to 0}の時\;\;\;\blacksquare
    \end{cases}
\]
\end{document}
