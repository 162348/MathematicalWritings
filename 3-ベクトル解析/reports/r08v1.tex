\documentclass[dvipdfmx,uplatex,nosetpagesize]{jsarticle}
%
\newcommand\GAKUSEISHOBANGO{J4-190549}% 学生証番号
\newcommand\NAMAE{司馬博文}% 氏名
\newcommand\KYODOSAKUSEISHA{なし}% 共同作成者(ある場合)
% 上の三行について,{}内に記入の上末尾の「解答欄」以降を適宜編集すれば良い.
%
\usepackage{amsmath,amssymb,amscd,amsthm,amsbsy,multicol}
\usepackage[shortlabels,inline]{enumitem}
\renewcommand\labelenumi{\theenumi)}
\renewcommand{\thefootnote}{\dag\arabic{footnote}}
\DeclareMathOperator{\grad}{\mathrm{grad}}
\newcommand\R{\mathbb{R}}
\pagestyle{plain}
%
\setlength{\paperwidth}{257mm}
\setlength{\paperheight}{364mm}
\setlength{\textwidth}{170mm}
\setlength{\textheight}{280mm}
% \setlength{\oddsidemargin}{-2.0cm}
% \setlength{\evensidemargin}{-.3cm}
\setlength{\topmargin}{-31mm}
%\setlength{\footskip}{2cm}
%
\newtheoremstyle{StatementsWithStar}% ?name?
{3pt}% ?Space above? 1
{3pt}% ?Space below? 1
{}% ?Body font?
{}% ?Indent amount? 2
{\bfseries}% ?Theorem head font?
{\textbf{.}}% ?Punctuation after theorem head?
{.5em}% ?Space after theorem head? 3
{\textbf{\textup{#1~\thetheorem{}}}{}\,$^{\ast}$\thmnote{(#3)}}% ?Theorem head spec (can be left empty, meaning ‘normal’)?
%
\newtheoremstyle{StatementsWithStar2}% ?name?
{3pt}% ?Space above? 1
{3pt}% ?Space below? 1
{}% ?Body font?
{}% ?Indent amount? 2
{\bfseries}% ?Theorem head font?
{\textbf{.}}% ?Punctuation after theorem head?
{.5em}% ?Space after theorem head? 3
{\textbf{\textup{#1~\thetheorem{}}}{}\,$^{\ast\ast}$\thmnote{(#3)}}% ?Theorem head spec (can be left empty, meaning ‘normal’)?
%
\newtheoremstyle{StatementsWithStar3}% ?name?
{3pt}% ?Space above? 1
{3pt}% ?Space below? 1
{}% ?Body font?
{}% ?Indent amount? 2
{\bfseries}% ?Theorem head font?
{\textbf{.}}% ?Punctuation after theorem head?
{.5em}% ?Space after theorem head? 3
{\textbf{\textup{#1~\thetheorem{}}}{}\,$^{\ast\ast\ast}$\thmnote{(#3)}}% ?Theorem head spec (can be left empty, meaning ‘normal’)?
%
\newtheoremstyle{StatementsWithCCirc}% ?name?
{6pt}% ?Space above? 1
{6pt}% ?Space below? 1
{}% ?Body font?
{}% ?Indent amount? 2
{\bfseries}% ?Theorem head font?
{\textbf{.}}% ?Punctuation after theorem head?
{.5em}% ?Space after theorem head? 3
{\textbf{\textup{#1~\thetheorem{}}}{}\,$^{\circledcirc}$\thmnote{(#3)}}% ?Theorem head spec (can be left empty, meaning ‘normal’)?
%
\theoremstyle{definition}
 \newtheorem{theorem}{定理}[section]
 \newtheorem{corollary}[theorem]{系}
 \newtheorem{proposition}[theorem]{命題}
 \newtheorem*{proposition*}{命題}
 \newtheorem{lemma}[theorem]{補題}
 \newtheorem*{lemma*}{補題}
 \newtheorem*{theorem*}{定理}
 \newtheorem{definition}[theorem]{定義}
 \newtheorem{example}[theorem]{例}
 \newtheorem{notation}[theorem]{記号}
 \newtheorem*{notation*}{記号}
 \newtheorem{assumption}[theorem]{仮定}
 \newtheorem{question}[theorem]{問}
 \newtheorem{reidai}[theorem]{例題}
 \newtheorem{remark}[theorem]{注}
% \newtheorem*{remarknonum}{注}
 \newtheorem*{definition*}{定義}
 \newtheorem*{remark*}{注}
 \newtheorem*{question*}{問}
%
\theoremstyle{StatementsWithStar}
 \newtheorem{definition_*}[theorem]{定義}
 \newtheorem{question_*}[theorem]{問}
 \newtheorem{example_*}[theorem]{例}
 \newtheorem{theorem_*}[theorem]{定理}
 \newtheorem{remark_*}[theorem]{注}
%
\theoremstyle{StatementsWithStar2}
 \newtheorem{definition_**}[theorem]{定義}
 \newtheorem{theorem_**}[theorem]{定理}
 \newtheorem{question_**}[theorem]{問}
 \newtheorem{remark_**}[theorem]{注}
%
\theoremstyle{StatementsWithStar3}
 \newtheorem{remark_***}[theorem]{注}
 \newtheorem{question_***}[theorem]{問}
%
\theoremstyle{StatementsWithCCirc}
 \newtheorem{definition_O}[theorem]{定義}
 \newtheorem{question_O}[theorem]{問}
 \newtheorem{example_O}[theorem]{例}
 \newtheorem{remark_O}[theorem]{注}
%
\theoremstyle{definition}
%
\renewcommand{\proofname}{\underline{証明}}
%
\raggedbottom
\allowdisplaybreaks
%
\everymath{\displaystyle}
%
\begin{document}
\thispagestyle{empty}
\setlength{\parindent}{1zw}
\setlength{\baselineskip}{13pt}
\setcounter{section}{8}
\newcounter{version}
\setcounter{version}{1}
\noindent
2020年度ベクトル解析(足助担当)レポート問題~\thesection~v\theversion%\par\noindent
\hfil2020/6/22(月)\par\noindent
提出先:ITC-LMSのページの「課題」\par\noindent
提出期間:2020/6/22(月)$\sim$ 2020/6/29(月)\textbf{9:00}\par\noindent
返却はITC-LMSを用いて7/6日(月)以降に行う.\par\noindent
※ レポートの作成方法は特に指定しないが,提出ファイルはPDF形式とすること.
なお,ファイル名は,「``回数"+``学生証番号の下7桁.pdf\/"」(例:84123456.pdf)とすること.
ファイルの作成にあたって印刷やスキャンなどに困難があれば速やかに足助まで申し出ること.
\vskip-18pt\noindent
\begin{table}[h]
\begin{tabular}{|c|c|c|} \hline
& & \\[-13pt]
学生証番号& 氏名 & 共同作成者(ある場合)\\[2pt] \hline
\rule{0pt}{16pt}%
\parbox[c]{9.2zw}{\GAKUSEISHOBANGO\hfill} & \parbox[c]{13.0zw}{\NAMAE\hfill} & \parbox[c]{25.6zw}{\KYODOSAKUSEISHA\hfill}\\[6pt] \hline
%「\hfill」の前に必要事項を記入すること.
\end{tabular}
\end{table}

\noindent
% 5/31 v2:積分区間が$[0,s]$となっていたのを$[0,t]$に修正.\par
% \noindent
% 5/23 v3:1)の$x^i$が誤って$x^j$となっていたので修正.\par
% \ \par
% ここでは函数などは全て$C^\infty$級とする.
\begin{question*}
$(x,y,z)$を$\R^3$の標準的な座標とする.
$p={}^t(0,0,0),\ q={}^t(1,0,0)$とし,$\R^3\setminus\{p,q\}$上のベクトル場$X$を
\begin{align*}
X&=\frac{x}{(x^2+y^2+z^2)^{3/2}}\frac{\partial}{\partial x}+\frac{y}{(x^2+y^2+z^2)^{3/2}}\frac{\partial}{\partial y}+\frac{z}{(x^2+y^2+z^2)^{3/2}}\frac{\partial}{\partial z}\\*
&\hphantom{{}={}}+\frac{x-1}{((x-1)^2+y^2+z^2)^{3/2}}\frac{\partial}{\partial x}+\frac{y}{((x-1)^2+y^2+z^2)^{3/2}}\frac{\partial}{\partial y}+\frac{z}{((x-1)^2+y^2+z^2)^{3/2}}\frac{\partial}{\partial z}
\end{align*}
により定める.
さて,
\begin{align*}
D&=\{{}^t(x,y,z)\in\R^3\mid(x-3)^2+(y-3)^2+(z-3)^2\leq36\},\\*
\Sigma&=\partial D=\{{}^t(x,y,z)\in\R^3\mid(x-3)^2+(y-3)^2+(z-3)^2=36\}
\end{align*}とし,$D$には$\R^3$の標準的な向きから自然に向きを入れ,$\Sigma$には$D$の境界としての向きを入れる.
このとき,
\[
\int_{\Sigma}X\cdot dA
\]
を求めよ.
\end{question*}
\par
\ \par
\noindent
{\small
※ 参考文献がある場合には最後にまとめて箇条書きで示すこと.\par\noindent
※ \textbf{全体として2ページに収めること.}\par\noindent
※ 共同作成者に記載がないにもかかわらず,ほかのレポートとほぼ同一であるレポートが散見される.
誰かと共同してレポートを作成することは構わないが,そのことは明記すること.
それをしなければ剽窃であって,これは学術上の致命的な不正行為である.
万一,写される側がそのことを承知していなかったことが露見した場合には重大な結果をもたらす可能性がある.
}

\rightline{(以上)}\par
%
% 以下が解答欄である.2ページ以内に収まるように注意すること.なお,紙面レイアウトやフォントサイズを変更しないこと.
%
\noindent
\begin{proof}[解答]
    \begin{align*}
        f^1(x,y,z) &= \frac{x}{(x^2+y^2+z^2)^{3/2}}+\frac{x-1}{((x-1)^2+y^2+z^2)^{3/2}} \\
        f^2(x,y,z) &= \frac{y}{(x^2+y^2+z^2)^{3/2}}+\frac{y}{((x-1)^2+y^2+z^2)^{3/2}} \\
        f^3(x,y,z) &= \frac{z}{(x^2+y^2+z^2)^{3/2}}+\frac{z}{((x-1)^2+y^2+z^2)^{3/2}} 
    \end{align*}
    と置くと,$X=f^1\frac{\partial}{\partial x}+f^2\frac{\partial}{\partial y}+f^3\frac{\partial}{\partial z}$と表せる.
    従って,2-形式$\omega$を$\omega=f^1dy\wedge dz+f^2dz\wedge dx+f^3dx\wedge dy$と定めれば,Gaussの発散定理より,
    \begin{align*}
        \int_{\partial D}X\cdot dA &= \int_{\partial D}\omega \\
        &= \int_Dd\omega\\
        &= \int_D\left(\frac{\partial f^1}{\partial x}+\frac{\partial f^2}{\partial y}+\frac{\partial f^3}{\partial z}\right)dxdydz 
    \end{align*}
    となる.いま,
    \begin{align*}
        \frac{\partial f^1}{\partial x} &= \frac{-2x^2+y^2+z^2}{(x^2+y^2+z^2)^{5/2}} + \frac{-2(x-1)^2+y^2+z^2}{((x-1)^2+y^2+z^2)^{5/2}} \\
        \frac{\partial f^2}{\partial y} &= \frac{x^2-2y^2+z^2}{(x^2+y^2+z^2)^{5/2}} + \frac{(x-1)^2-2y^2+z^2}{((x-1)^2+y^2+z^2)^{5/2}} \\
        \frac{\partial f^3}{\partial z} &= \frac{x^2+y^2-2z^2}{(x^2+y^2+z^2)^{5/2}} + \frac{(x-1)^2+y^2-2z^2}{((x-1)^2+y^2+z^2)^{5/2}} 
    \end{align*}
    より,
    \begin{align*}
        \int_{\partial D}X\cdot dA &= \int_{\partial D}\omega \\
        &= \int_D\left(\frac{\partial f^1}{\partial x}+\frac{\partial f^2}{\partial y}+\frac{\partial f^3}{\partial z}\right)dxdydz \\
        &= \int_D 0 \;dxdydz = 0
    \end{align*}
    を得る.
\end{proof}
\end{document}
