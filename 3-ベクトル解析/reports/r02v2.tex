\documentclass[dvipdfmx,nosetpagesize, uplatex]{jsarticle}
%
\newcommand\GAKUSEISHOBANGO{J4-190549}% 学生証番号
\newcommand\NAMAE{司馬 博文}% 氏名
\newcommand\KYODOSAKUSEISHA{なし}% 共同作成者(ある場合)
% 上の三行について,{}内に記入の上181行以降を適宜編集すれば良い.
%
\usepackage{amsmath,amssymb,amscd,amsthm,amsbsy,multicol}
\usepackage[shortlabels,inline]{enumitem}
\renewcommand\labelenumi{\theenumi)}
\renewcommand{\thefootnote}{\dag\arabic{footnote}}
\pagestyle{plain}
%
\setlength{\paperwidth}{257mm}
\setlength{\paperheight}{364mm}
\setlength{\textwidth}{170mm}
\setlength{\textheight}{280mm}
% \setlength{\oddsidemargin}{-2.0cm}
% \setlength{\evensidemargin}{-.3cm}
\setlength{\topmargin}{-31mm}
%\setlength{\footskip}{2cm}
%
\newtheoremstyle{StatementsWithStar}% ?name?
{3pt}% ?Space above? 1
{3pt}% ?Space below? 1
{}% ?Body font?
{}% ?Indent amount? 2
{\bfseries}% ?Theorem head font?
{\textbf{.}}% ?Punctuation after theorem head?
{.5em}% ?Space after theorem head? 3
{\textbf{\textup{#1~\thetheorem{}}}{}\,$^{\ast}$\thmnote{(#3)}}% ?Theorem head spec (can be left empty, meaning ‘normal’)?
%
\newtheoremstyle{StatementsWithStar2}% ?name?
{3pt}% ?Space above? 1
{3pt}% ?Space below? 1
{}% ?Body font?
{}% ?Indent amount? 2
{\bfseries}% ?Theorem head font?
{\textbf{.}}% ?Punctuation after theorem head?
{.5em}% ?Space after theorem head? 3
{\textbf{\textup{#1~\thetheorem{}}}{}\,$^{\ast\ast}$\thmnote{(#3)}}% ?Theorem head spec (can be left empty, meaning ‘normal’)?
%
\newtheoremstyle{StatementsWithStar3}% ?name?
{3pt}% ?Space above? 1
{3pt}% ?Space below? 1
{}% ?Body font?
{}% ?Indent amount? 2
{\bfseries}% ?Theorem head font?
{\textbf{.}}% ?Punctuation after theorem head?
{.5em}% ?Space after theorem head? 3
{\textbf{\textup{#1~\thetheorem{}}}{}\,$^{\ast\ast\ast}$\thmnote{(#3)}}% ?Theorem head spec (can be left empty, meaning ‘normal’)?
%
\newtheoremstyle{StatementsWithCCirc}% ?name?
{6pt}% ?Space above? 1
{6pt}% ?Space below? 1
{}% ?Body font?
{}% ?Indent amount? 2
{\bfseries}% ?Theorem head font?
{\textbf{.}}% ?Punctuation after theorem head?
{.5em}% ?Space after theorem head? 3
{\textbf{\textup{#1~\thetheorem{}}}{}\,$^{\circledcirc}$\thmnote{(#3)}}% ?Theorem head spec (can be left empty, meaning ‘normal’)?
%
\theoremstyle{definition}
 \newtheorem{theorem}{定理}[section]
 \newtheorem{corollary}[theorem]{系}
 \newtheorem{proposition}[theorem]{命題}
 \newtheorem*{proposition*}{命題}
 \newtheorem{lemma}[theorem]{補題}
 \newtheorem*{lemma*}{補題}
 \newtheorem*{theorem*}{定理}
 \newtheorem{definition}[theorem]{定義}
 \newtheorem{example}[theorem]{例}
 \newtheorem{notation}[theorem]{記号}
 \newtheorem*{notation*}{記号}
 \newtheorem{assumption}[theorem]{仮定}
 \newtheorem{question}[theorem]{問}
 \newtheorem{reidai}[theorem]{例題}
 \newtheorem{remark}[theorem]{注}
% \newtheorem*{remarknonum}{注}
 \newtheorem*{definition*}{定義}
 \newtheorem*{remark*}{注}
 \newtheorem*{question*}{問}
%
\theoremstyle{StatementsWithStar}
 \newtheorem{definition_*}[theorem]{定義}
 \newtheorem{question_*}[theorem]{問}
 \newtheorem{example_*}[theorem]{例}
 \newtheorem{theorem_*}[theorem]{定理}
 \newtheorem{remark_*}[theorem]{注}
%
\theoremstyle{StatementsWithStar2}
 \newtheorem{definition_**}[theorem]{定義}
 \newtheorem{theorem_**}[theorem]{定理}
 \newtheorem{question_**}[theorem]{問}
 \newtheorem{remark_**}[theorem]{注}
%
\theoremstyle{StatementsWithStar3}
 \newtheorem{remark_***}[theorem]{注}
 \newtheorem{question_***}[theorem]{問}
%
\theoremstyle{StatementsWithCCirc}
 \newtheorem{definition_O}[theorem]{定義}
 \newtheorem{question_O}[theorem]{問}
 \newtheorem{example_O}[theorem]{例}
 \newtheorem{remark_O}[theorem]{注}
%
\theoremstyle{definition}
%
\renewcommand{\proofname}{\underline{証明}}
%
\raggedbottom
\allowdisplaybreaks
%
\everymath{\displaystyle}
%
\begin{document}
\thispagestyle{empty}
\setlength{\parindent}{1zw}
\setlength{\baselineskip}{13pt}
\setcounter{section}{2}
\newcounter{version}
\setcounter{version}{1}
\noindent
2020年度ベクトル解析(足助担当)レポート問題~\thesection~v\theversion%\par\noindent
\hfil2020/4/27(月),同修正\par\noindent
提出先:ITC-LMSのページの「課題」\par\noindent
提出期間:2020/4/27(月)$\sim$ 2020/5/11(月)\textbf{9:00}\par\noindent
返却はITC-LMSを用いて5/18日(月)以降に行う.\par\noindent
※ レポートの作成方法は特に指定しないが,提出ファイルはPDF形式とすること.
なお,ファイル名は(学生証番号の下7桁.pdf)とすること.
ファイルの作成にあたって印刷やスキャンなどに困難があれば速やかに足助まで申し出ること.
\vskip-18pt\noindent
\begin{table}[h]
\begin{tabular}{|c|c|c|} \hline
& & \\[-13pt]
学生証番号& 氏名 & 共同作成者(ある場合)\\[2pt] \hline
\rule{0pt}{16pt}%
\parbox[c]{9.2zw}{\GAKUSEISHOBANGO\hfill} & \parbox[c]{13.0zw}{\NAMAE\hfill} & \parbox[c]{25.6zw}{\KYODOSAKUSEISHA\hfill}\\[6pt] \hline
%「\hfill」の前に必要事項を記入すること.
\end{tabular}
\end{table}
\vskip-12pt\noindent
v2:$\gamma_n$の定義の$\theta$を$t$に修正.
% 
\begin{question*}
$n\in\mathbb{Z}$とし,$\gamma_n\colon[0,1]\to\mathbb{R}^2$を$\gamma_n(t)={}^t(\cos2n t,\sin2n t)$により定める.
また,$(x,y)$を$\mathbb{R}^2$の標準的な座標とする.
\begin{enumerate}
\item
$\mathbb{R}^2$上のベクトル場$X$を
\[
X(x,y)=-y\frac{\partial}{\partial x}+x\frac{\partial}{\partial y}
\]
により定める.
$X$の$\gamma_n$に沿った積分(線積分)$\int_{\gamma_n}X(p)\cdot dp$を求めよ.
\item
$\mathbb{R}^2$上の函数(スカラー場)$f$を
\[
f(x,y)=1
\]
により定める.
$f$の$\gamma_n$に沿った積分(線積分)$\int_{\gamma_n}f(p)|dp|$を求めよ.
\end{enumerate}
\end{question*}
\par
\ \par
\noindent
{\small
※ 参考文献がある場合には最後にまとめて箇条書きで示すこと.\par\noindent
※ \textbf{全体として2ページに収めること.}\par\noindent
※ 共同作成者に記載がないにもかかわらず,ほかのレポートとほぼ同一であるレポートが散見される.
誰かと共同してレポートを作成することは構わないが,そのことは明記すること.
それをしなければ剽窃であって,これは学術上の致命的な不正行為である.
万一,写される側がそのことを承知していなかったことが露見した場合には重大な結果をもたらす可能性がある.
}

\rightline{(以上)}\par
%
% 以下が解答欄である.2ページ以内に収まるように注意すること.なお,紙面レイアウトやフォントサイズを変更しないこと.
%
\noindent

1) 
\begin{eqnarray*}
    \int_{\gamma_n}X(p)\cdot dp &=& \int^1_0 (-\sin 2nt\;\; \cos 2nt)\left(\begin{array}{c}-2n\sin 2nt \\ 2n\cos 2nt\end{array}\right)dt \\
        &=& 2n\int^1_0 (\sin^2 2nt + \cos^2 2nt)dt \\
        &=& 2n\int^1_0 dt = \underline{2n}
\end{eqnarray*}

2)
\begin{eqnarray*}
    \int_{\gamma_n}f(p)|dp| &=& \int^1_0 \left| \left(\begin{array}{c}-2n\sin 2nt \\ 2n\cos 2nt\end{array}\right) \right|dt \\
        &=& 2n\int^1_0 dt = \underline{2n}
\end{eqnarray*}

$\gamma_n$が単位円の弧長によるパラメタ付けであること,ベクトル場$X$は曲線$|\gamma_n|$上では全て単位長さのベクトルであることが効いた結果となった.


\end{document}
