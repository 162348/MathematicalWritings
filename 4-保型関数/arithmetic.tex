\documentclass[uplatex, 12pt, dvipdfmx]{jsreport}
\title{数論}
\author{司馬博文}
\date{\today}
\pagestyle{headings} \setcounter{secnumdepth}{4}
\usepackage{amsmath, amsfonts, amsthm, amssymb, ascmac, color, comment, wrap fig}

\setcounter{tocdepth}{2}
%2はsubsectionまで
\usepackage{mathtools}
%\mathtoolsset{showonlyrefs=true} %labelを附した数式にのみ附番される.

%%% 生成されるPDFファイルにおいて、\tableofcontents によって書き出された目次をクリックすると該当する見出しへジャンプしたり、 さらには、\label{ラベル名} を番号で参照する \ref{ラベル名} や thebibliography環境において \bibitem{ラベル名} を文献番号で参照する \cite{ラベル名} においても番号をクリックすると該当箇所にジャンプする
\usepackage[dvipdfmx]{hyperref}
\usepackage{pxjahyper}

\usepackage{tikz, tikz-cd}
\usepackage[all]{xy}
\def\objectstyle{\displaystyle} %デフォルトではxymatrix中の数式が文中数式モードになるので,それを直した.

%化学式をTikZで簡単に書くためのパッケージ.
\usepackage[version=4]{mhchem} %texdoc mhchem
%化学構造式をTikZで描くためのパッケージ.
\usepackage{chemfig}
%IS単位を書くためのパッケージ
\usepackage{siunitx}

%取り消し線を引くためのパッケージ
\usepackage{ulem}

%\rotateboxコマンドを,文字列の中心で回転させるオプション.
%他rotatebox, scalebox, reflectbox, resizeboxなどのコマンド.
\usepackage{graphicx}

%加藤晃史さんがフル活用していたtcolorboxを,途中改ページ可能で.
\usepackage[breakable]{tcolorbox}

%足助さんからもらったオプション
% \usepackage[shortlabels,inline]{enumitem}
% \usepackage[top=15truemm,bottom=15truemm,left=10truemm,right=10truemm]{geometry}

%enumerate環境を凝らせる.
\usepackage{enumerate}

%日本語にルビをふる
\usepackage{pxrubrica}

%以下,ソースコードを表示する環境の設定.
\usepackage{listings,jvlisting} %日本語のコメントアウトをする場合jlistingが必要
%ここからソースコードの表示に関する設定
\lstset{
  basicstyle={\ttfamily},
  identifierstyle={\small},
  commentstyle={\smallitshape},
  keywordstyle={\small\bfseries},
  ndkeywordstyle={\small},
  stringstyle={\small\ttfamily},
  frame={tb},
  breaklines=true,
  columns=[l]{fullflexible},
  numbers=left,
  xrightmargin=0zw,
  xleftmargin=3zw,
  numberstyle={\scriptsize},
  stepnumber=1,
  numbersep=1zw,
  lineskip=-0.5ex
}
%lstlisting環境で,[caption=hoge,label=fuga]などのoptionを付けられる.

%%%
%%%フォント
%%%

%本文・数式の両方のフォントをTimesに変更するお手軽なパッケージだが,LaTeX標準数式記号の\jmath, \amalg, coprodはサポートされない.
\usepackage{mathptmx}
%Palatinoの方が完成度が高いと美文書作成に書いてあった.
% \usepackage[sc]{mathpazo} %オプションは,familyの指定.pplxにしている.
%2000年のYoung Ryuによる新しいTimes系.なおPalatinoもある.
% \usepackage{newtxtext, newtxmath}
%拡張数学記号.\textsectionでブルバキに!
% \usepackage{textcomp, mathcomp}
% \usepackage[T1]{fontenc} %8bitエンコーディングにする.comp系拡張数学文字の動作が安定する.
%AMS Euler.Computer Modernと相性が悪いとは…….
% \usepackage{ccfonts, eulervm} %KnuthのConcrete Mathematicsの組み合わせ.
% \renewcommand{\rmdefault}{pplx} %makes LaTeX use Palatino in place of CM Roman.Do not use the Euler math fonts in conjunction with the default Computer Modern text fonts – this is ugly!

%%% newcommands
    %参考文献で⑦というのを出したかった.\circled{n}と打てば良い.LaTeX StackExchangeより.
\newcommand*\circled[1]{\tikz[baseline=(char.base)]{\node[shape=circle,draw,inner sep=0.8pt] (char) {#1};}}

%%%
%%% ショートカット 足助さんからのコピペ
%%%

\DeclareMathOperator{\grad}{\mathrm{grad}}
\DeclareMathOperator{\rot}{\mathrm{rot}}
\DeclareMathOperator{\divergence}{\mathrm{div}}
\newcommand\R{\mathbb{R}}
\newcommand\N{\mathbb{N}}
\newcommand\C{\mathbb{C}}
\newcommand\Z{\mathbb{Z}}
\newcommand\Q{\mathbb{Q}}
\newcommand\GL{\mathrm{GL}}
\newcommand\SL{\mathrm{SL}}
\newcommand\False{\mathrm{False}}
\newcommand\True{\mathrm{True}}
\newcommand\tr{\mathrm{tr}}
\newcommand\M{\mathcal{M}}
\newcommand\F{\mathbb{F}}
\renewcommand\H{\mathbb{H}}
\newcommand\id{\mathrm{id}}
\newcommand\A{\mathcal{A}}
\renewcommand\coprod{\rotatebox[origin=c]{180}{$\prod$}}
\newcommand\pr{\mathrm{pr}}
\newcommand\U{\mathfrak{U}}
\newcommand\Map{\mathrm{Map}}
\newcommand\dom{\mathrm{dom}}
\newcommand\cod{\mathrm{cod}}
\newcommand\supp{\mathrm{supp}\;}
\newcommand\Ker{\mathrm{Ker}\;}
%%% 複素解析学
\renewcommand\Re{\mathrm{Re}\;}
\renewcommand\Im{\mathrm{Im}\;}
\newcommand\Gal{\mathrm{Gal}}
\newcommand\PGL{\mathrm{PGL}}
\newcommand\PSL{\mathrm{PSL}}
%%% 解析力学
\newcommand\x{\mathbf{x}}
\newcommand\q{\mathbf{q}}
%%% 集合と位相
\newcommand\ORD{\mathrm{ORD}}
%%% 形式言語理論
\newcommand\REGEX{\mathrm{REGEX}}

%%% 圏
\newcommand\Hom{\mathrm{Hom}}
\newcommand\Mor{\mathrm{Mor}}
\newcommand\Aut{\mathrm{Aut}}
\newcommand\End{\mathrm{End}}
\newcommand\op{\mathrm{op}}
\newcommand\ev{\mathrm{ev}}
\newcommand\Ob{\mathrm{Ob}}
\newcommand\Ar{\mathrm{Ar}}
\newcommand\Arr{\mathrm{Arr}}
\newcommand\Set{\mathrm{Set}}
\newcommand\Grp{\mathrm{Grp}}
\newcommand\Cat{\mathrm{Cat}}
\newcommand\Mon{\mathrm{Mon}}
\newcommand\CMon{\mathrm{CMon}}
\newcommand\Pos{\mathrm{Pos}}
\newcommand\Vect{\mathrm{Vect}}
\newcommand\FinVect{\mathrm{FinVect}}
\newcommand\Fun{\mathrm{Fun}}
\newcommand\Ord{\mathrm{Ord}}
\newcommand\eq{\mathrm{eq}}
\newcommand\coeq{\mathrm{coeq}}

%%%
%%% 定理環境 以下足助さんからのコピペ
%%%

\newtheoremstyle{StatementsWithStar}% ?name?
{3pt}% ?Space above? 1
{3pt}% ?Space below? 1
{}% ?Body font?
{}% ?Indent amount? 2
{\bfseries}% ?Theorem head font?
{\textbf{.}}% ?Punctuation after theorem head?
{.5em}% ?Space after theorem head? 3
{\textbf{\textup{#1~\thetheorem{}}}{}\,$^{\ast}$\thmnote{(#3)}}% ?Theorem head spec (can be left empty, meaning ‘normal’)?
%
\newtheoremstyle{StatementsWithStar2}% ?name?
{3pt}% ?Space above? 1
{3pt}% ?Space below? 1
{}% ?Body font?
{}% ?Indent amount? 2
{\bfseries}% ?Theorem head font?
{\textbf{.}}% ?Punctuation after theorem head?
{.5em}% ?Space after theorem head? 3
{\textbf{\textup{#1~\thetheorem{}}}{}\,$^{\ast\ast}$\thmnote{(#3)}}% ?Theorem head spec (can be left empty, meaning ‘normal’)?
%
\newtheoremstyle{StatementsWithStar3}% ?name?
{3pt}% ?Space above? 1
{3pt}% ?Space below? 1
{}% ?Body font?
{}% ?Indent amount? 2
{\bfseries}% ?Theorem head font?
{\textbf{.}}% ?Punctuation after theorem head?
{.5em}% ?Space after theorem head? 3
{\textbf{\textup{#1~\thetheorem{}}}{}\,$^{\ast\ast\ast}$\thmnote{(#3)}}% ?Theorem head spec (can be left empty, meaning ‘normal’)?
%
\newtheoremstyle{StatementsWithCCirc}% ?name?
{6pt}% ?Space above? 1
{6pt}% ?Space below? 1
{}% ?Body font?
{}% ?Indent amount? 2
{\bfseries}% ?Theorem head font?
{\textbf{.}}% ?Punctuation after theorem head?
{.5em}% ?Space after theorem head? 3
{\textbf{\textup{#1~\thetheorem{}}}{}\,$^{\circledcirc}$\thmnote{(#3)}}% ?Theorem head spec (can be left empty, meaning ‘normal’)?
%
\theoremstyle{definition}
 \newtheorem{theorem}{定理}[section]
 \newtheorem{axiom}[theorem]{公理}
 \newtheorem{corollary}[theorem]{系}
 \newtheorem{proposition}[theorem]{命題}
 \newtheorem*{proposition*}{命題}
 \newtheorem{lemma}[theorem]{補題}
 \newtheorem*{lemma*}{補題}
 \newtheorem*{theorem*}{定理}
 \newtheorem{definition}[theorem]{定義}
 \newtheorem{example}[theorem]{例}
 \newtheorem{notation}[theorem]{記法}
 \newtheorem*{notation*}{記法}
 \newtheorem{assumption}[theorem]{仮定}
 \newtheorem{question}[theorem]{問}
 \newtheorem{counterexample}[theorem]{反例}
 \newtheorem{reidai}[theorem]{例題}
 \newtheorem{problem}[theorem]{問題}
 \newtheorem*{solution*}{\bf{[解]}}
 \newtheorem{discussion}[theorem]{議論}
 \newtheorem{remark}[theorem]{注}
 \newtheorem{universality}[theorem]{普遍性} %非自明な例外がない.
 \newtheorem{universal tendency}[theorem]{普遍傾向} %例外が有意に少ない.
 \newtheorem{hypothesis}[theorem]{仮説} %実験で説明されていない理論.
 \newtheorem{theory}[theorem]{理論} %実験事実とその(さしあたり)整合的な説明.
 \newtheorem{fact}[theorem]{実験事実}
 \newtheorem{model}[theorem]{模型}
 \newtheorem{explanation}[theorem]{説明} %理論による実験事実の説明
 \newtheorem{anomaly}[theorem]{理論の限界}
 \newtheorem{application}[theorem]{応用例}
 \newtheorem{method}[theorem]{手法} %実験手法など,技術的問題.
 \newtheorem{history}[theorem]{歴史}
 \newtheorem{research}[theorem]{研究}
% \newtheorem*{remarknonum}{注}
 \newtheorem*{definition*}{定義}
 \newtheorem*{remark*}{注}
 \newtheorem*{question*}{問}
 \newtheorem*{axiom*}{公理}
 \newtheorem*{example*}{例}
%
\theoremstyle{StatementsWithStar}
 \newtheorem{definition_*}[theorem]{定義}
 \newtheorem{question_*}[theorem]{問}
 \newtheorem{example_*}[theorem]{例}
 \newtheorem{theorem_*}[theorem]{定理}
 \newtheorem{remark_*}[theorem]{注}
%
\theoremstyle{StatementsWithStar2}
 \newtheorem{definition_**}[theorem]{定義}
 \newtheorem{theorem_**}[theorem]{定理}
 \newtheorem{question_**}[theorem]{問}
 \newtheorem{remark_**}[theorem]{注}
%
\theoremstyle{StatementsWithStar3}
 \newtheorem{remark_***}[theorem]{注}
 \newtheorem{question_***}[theorem]{問}
%
\theoremstyle{StatementsWithCCirc}
 \newtheorem{definition_O}[theorem]{定義}
 \newtheorem{question_O}[theorem]{問}
 \newtheorem{example_O}[theorem]{例}
 \newtheorem{remark_O}[theorem]{注}
%
\theoremstyle{definition}
%
\raggedbottom
\allowdisplaybreaks

%証明環境のスタイル
\everymath{\displaystyle}
\renewcommand{\proofname}{\bf [証明]}
\renewcommand{\thefootnote}{\dag\arabic{footnote}}	%足助さんからもらった.どうなるんだ?

%mathptmxパッケージ下で,\jmath, \amalg, coprodの記号を出力するためのマクロ.TeX Wikiからのコピペ.
% \DeclareSymbolFont{cmletters}{OML}{cmm}{m}{it}
% \DeclareSymbolFont{cmsymbols}{OMS}{cmsy}{m}{n}
% \DeclareSymbolFont{cmlargesymbols}{OMX}{cmex}{m}{n}
% \DeclareMathSymbol{\myjmath}{\mathord}{cmletters}{"7C}
% \DeclareMathSymbol{\myamalg}{\mathbin}{cmsymbols}{"71}
% \DeclareMathSymbol{\mycoprod}{\mathop}{cmlargesymbols}{"60}
% \let\jmath\myjmath
% \let\amalg\myamalg
% \let\coprod\mycoprod
\begin{document}
\tableofcontents

\chapter{保型関数入門(担当:松本久義先生)}

\begin{quotation}
    離散部分群$\SL(2,\Z)$の作用で上半平面$\H$を割った空間はRiemann面の構造を入れることができ,この上の微分形式は保型形式と呼ばれる数論的対象を定める.

    同様にLie群の表現論の舞台ともなる.Lie群$\GL(2,\R)$が計量を保って作用するが,これをMöbius変換という.

    また,双曲幾何も,Gauss平面での実現を持つ,これをPoincaréの上半平面モデルという.このモデルは単位円板モデルと計量を保って写り合う,即ち,2つのモデルがRiemann面として解析的同型(多変数複素解析の文脈で,2つの$C^n$上の領域間に,正則写像が両方向に存在すること)である.
\end{quotation}

\section{非Euclid幾何}

\section{一次分数変換}

\begin{screen}
    射影一般線型群$\GL_2(\C)$は$\hat{\C}$に一次分数変換によって作用する.
    この作用は計量を保ち,特に射影幾何学の言葉で言えば円を保つ(射影空間の射).
\end{screen}

$\GL_2(\C)$を標準分解する.

\begin{theorem}[$\GL_2(\C)$の標準分解]\label{standard-decomposition-of-GL2C}
    任意の一次分数変換は,次の変換の合成によって表せる.
    \begin{enumerate}
        \item $z\mapsto az\;(a\in\C^\times)$,
        \item $z\mapsto z+c\;(c\in\C)$,
        \item $z\mapsto\frac{1}{z}$.
    \end{enumerate}
\end{theorem}

\begin{definition}[circle]
    $\hat{\C}$の\textbf{円}とは,次のことをいう.
    \begin{enumerate}
        \item $\C$上の円,
        \item $\C$上の直線と$\infty$との合併.
    \end{enumerate}
\end{definition}

\begin{theorem}[円円対応]
    一次分数変換は$\hat{\C}$上の円を円に移す.
\end{theorem}
\begin{proof}
    定理\ref{standard-decomposition-of-GL2C}より,変換$z\mapsto\frac{1}{z}$が
    円を保つことを示せば良い.
\end{proof}

\begin{thebibliography}[projective general linear group is triply transitive]
    $(x_1,x_2,x_3),(x'_1,x'_2,x'_3)\in\hat{\C}$をそれぞれの組のどの2つも等しくないとする.
    この時ある一次分数変換が存在して,$x_1\mapsto x'_1,x_2\mapsto x'_2,x_3\mapsto x'_3$を満たす.
\end{thebibliography}

\begin{definition}[orbit]
    群$G$が集合$X$に作用しているとする.
    \[ Gx:=\{gx\mid g\in G\} \]
    を$x$を通る\textbf{軌道}という.
    逆に,$X$の部分集合のうち$G$の軌道としても得られるものを\textbf{$G$-軌道}という.
\end{definition}
\begin{remark}
    線型空間も,体の作用と見れるのだろうか.すると,生成する空間とは軌道概念の拡張になる.
\end{remark}

\begin{definition}[transitive]
    群作用が\textbf{推移的}であるとは,空でない$X$に対して,
    \[ \forall x\in X,\; Gx=X \]
    が成り立つことをいう.
\end{definition}

\begin{theorem}
    Riemann面$\hat{\C}$への$\SL_2(\R)$の群作用の軌道は次の3つである.
    \begin{enumerate}
        \item 射影直線$\R\cup\{\infty\}$.
        \item 上半平面$\H$.
        \item 下半平面$\H_-$
    \end{enumerate}
\end{theorem}
\begin{proof}
    それぞれ,$0,i,-i$の軌道として構成し,これらが$\hat{\C}$の類別となっていることを確認する.
\end{proof}

\section{上半平面と束}

\begin{screen}
    平面上に基底を2つ定めると,これらが作る座標系を得る.これを斜交座標の場合も含めて,束という代数系のことばでGauss平面上で捉える.
\end{screen}

\begin{definition}[lattice and its morphism]\mbox{}
    \begin{enumerate}
        \item 加法群としての$\C$の部分群$L$が\textbf{束}であるとは,ある$w_1,w_2\in\C$が存在して次を満たすことをいう:
        \begin{enumerate}[(1)]
            \item $L=\{mw_1+nw_2\mid m,n\in\Z\}=:\langle w_1,w_2\rangle_\Z$.
            \item $w_1,w_2$は$\R$上一次独立.(i.e. $\frac{w_1}{w_2}\notin\R\cup\{\pm\infty\}$).
        \end{enumerate}
        この時の$w_1,w_2\in\C$を$L$の\textbf{基底}と呼ぶ.
        \item $L_1,L_2\subset\C$を束とする.これらが\textbf{同型}であるとは,
        \[ L_1\simeq L_2:\;\Leftrightarrow\; \exists a\in\C^\times,\; aL_1=L_2 \]
        とする.ただし,$aL_1=\{aL\mid l\in L_1\}=\langle aw_1,aw_2\rangle_\Z$とした.
        \item $w_1,w_2\in\C$を一次独立とする時,これらが張る平行四辺形の内部を
        \[ P(w_1,w_2):=\{tw_1+sw_2\mid s,t\in (0,1)\} \]
        とする.
    \end{enumerate}
\end{definition}
\begin{remark}\mbox{}
    \begin{enumerate}
        \item $\C$上の点を2つ取ると,これを基底とした座標系を得る.それを束と呼ぶ.
        \item このように加法群$\C$の言葉で定義した束が,等角写像で写り合う時,同型であるという.
    \end{enumerate}
\end{remark}

\begin{definition}[upper half-plane]\mbox{}
    \begin{enumerate}
        \item $\H:=\{t\in\C\mid \Im t>0\}$を\textbf{上半平面}と呼ぶ.$H,\mathfrak{H},H^+$などとも表す.
        \item これがRiemann球面に埋め込まれているとみなした時,その閉包を\textbf{閉上半平面}と呼ぶ:$\overline{\H}=\H\cup\partial\H=\H\cup\R\cup\{\infty\}$.
        \item $t\in\H$に対して,これが上半平面上に定める束を
        \[ \Omega_t:=\{m+nt\mid m,n\in\Z\}=\langle 1,t\rangle_\Z \]
        と置く.
    \end{enumerate}
\end{definition}
\begin{remark}[上半平面に注目すれば良い理由]
    $L=\langle w_1,w_2\rangle_\Z$とすると,基底は一次独立であることより$\Im w_1/w_2\ne 0$である.
    この時必要なら順番を入れ替えることで$w_1/w_2\in\H$と出来る(なす角のうち「狭い方」を取れば$\pi$より小さく$0$より大きい).
    従って,$L=w_2\Omega_{w_1/w_2}$である.
\end{remark}

\subsection{$\H$上の束の同型類を定めたい}

\begin{screen}
    前節で,束を考えるには上半平面のみに注目したクラス$\Omega_t\;(t\in\C)$に注目すれば良いとして代表系を取った.
    次に,これらの同型類を定めたい.
\end{screen}

ここで,上半平面に対する実行列の作用を観察する.
まず,実行列の固有ベクトルにより強く分類できる.
なぜなら,複素共軛による双対命題が常に成り立つので,
1つのベクトルの行き先に言及するだけで同時に2つ目も定めていることになる.

\begin{lemma}
    $A\in M_2(\R),t\in\H$が
    \[ A\begin{pmatrix}t\\1\end{pmatrix}=\begin{pmatrix}t\\1\end{pmatrix} \]
    を満たす時,$A=I$である.
\end{lemma}
\begin{proof}
    $A\in M_2(\R)$より,
    \[ A\begin{pmatrix}\overline{t}\\1\end{pmatrix}=\begin{pmatrix}\overline{t}\\1\end{pmatrix} \]
    も成り立つ.$t\in\H$としたから,$\begin{pmatrix}t\\1\end{pmatrix},\begin{pmatrix}\overline{t}\\1\end{pmatrix}$は$\R$上一次独立より,$A$の定める写像は$\H$上の恒等写像である.
    従って,$A=I$.
\end{proof}

次の定理は,楕円関数と保型形式の間の関係の土台となる対応を示す.
それは,上半平面上の束$\Omega_t$が同型であるとは,$\SL_2(\Z)$の作用に対して,
同じ軌道に乗る
\[\SL_2(\Z)\cdot t_1=\SL_2(\Z)t_2\]
ことに同値であることを示す.

\begin{theorem}
    $t_1,t_2\in\H$に対して,以下は同値である.
    \begin{enumerate}
        \item $\Omega_{t_1}\simeq\Omega_{t_2}$.
        \item $\exists g\in\SL_2(\Z)$.
    \end{enumerate}
\end{theorem}

\section{基本領域}

\section{楕円関数}

\section{Eisenstein級数}

\section{Fourier展開}

\section{保型形式}

\section{保型形式の極と零点}

\section{保型関数体}

\end{document}