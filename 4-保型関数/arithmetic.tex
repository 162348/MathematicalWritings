\documentclass[uplatex, 12pt, dvipdfmx]{jsreport}
\title{数論}
\author{司馬博文}
\date{\today}
\pagestyle{headings} \setcounter{secnumdepth}{4}
\input{/Users/hirofumi.shiba48/Desktop/数理科学/preamble_CM.tex}
\begin{document}
\tableofcontents

\chapter{保型関数入門(担当:松本久義先生)}

\begin{quotation}
    離散部分群$\SL(2,\Z)$の作用で上半平面$\H$を割った空間はRiemann面の構造を入れることができ,この上の微分形式は保型形式と呼ばれる数論的対象を定める.

    同様にLie群の表現論の舞台ともなる.Lie群$\GL(2,\R)$が計量を保って作用するが,これをMöbius変換という.

    また,双曲幾何も,Gauss平面での実現を持つ,これをPoincaréの上半平面モデルという.このモデルは単位円板モデルと計量を保って写り合う,即ち,2つのモデルがRiemann面として解析的同型(多変数複素解析の文脈で,2つの$C^n$上の領域間に,正則写像が両方向に存在すること)である.
\end{quotation}

\section{非Euclid幾何}

\section{一次分数変換}

\begin{screen}
    射影一般線型群$\GL_2(\C)$は$\hat{\C}$に一次分数変換によって作用する.
    この作用は計量を保ち,特に射影幾何学の言葉で言えば円を保つ(射影空間の射).
\end{screen}

$\GL_2(\C)$を標準分解する.

\begin{theorem}[$\GL_2(\C)$の標準分解]\label{standard-decomposition-of-GL2C}
    任意の一次分数変換は,次の変換の合成によって表せる.
    \begin{enumerate}
        \item $z\mapsto az\;(a\in\C^\times)$,
        \item $z\mapsto z+c\;(c\in\C)$,
        \item $z\mapsto\frac{1}{z}$.
    \end{enumerate}
\end{theorem}

\begin{definition}[circle]
    $\hat{\C}$の\textbf{円}とは,次のことをいう.
    \begin{enumerate}
        \item $\C$上の円,
        \item $\C$上の直線と$\infty$との合併.
    \end{enumerate}
\end{definition}

\begin{theorem}[円円対応]
    一次分数変換は$\hat{\C}$上の円を円に移す.
\end{theorem}
\begin{proof}
    定理\ref{standard-decomposition-of-GL2C}より,変換$z\mapsto\frac{1}{z}$が
    円を保つことを示せば良い.
\end{proof}

\begin{thebibliography}[projective general linear group is triply transitive]
    $(x_1,x_2,x_3),(x'_1,x'_2,x'_3)\in\hat{\C}$をそれぞれの組のどの2つも等しくないとする.
    この時ある一次分数変換が存在して,$x_1\mapsto x'_1,x_2\mapsto x'_2,x_3\mapsto x'_3$を満たす.
\end{thebibliography}

\begin{definition}[orbit]
    群$G$が集合$X$に作用しているとする.
    \[ Gx:=\{gx\mid g\in G\} \]
    を$x$を通る\textbf{軌道}という.
    逆に,$X$の部分集合のうち$G$の軌道としても得られるものを\textbf{$G$-軌道}という.
\end{definition}
\begin{remark}
    線型空間も,体の作用と見れるのだろうか.すると,生成する空間とは軌道概念の拡張になる.
\end{remark}

\begin{definition}[transitive]
    群作用が\textbf{推移的}であるとは,空でない$X$に対して,
    \[ \forall x\in X,\; Gx=X \]
    が成り立つことをいう.
\end{definition}

\begin{theorem}
    Riemann面$\hat{\C}$への$\SL_2(\R)$の群作用の軌道は次の3つである.
    \begin{enumerate}
        \item 射影直線$\R\cup\{\infty\}$.
        \item 上半平面$\H$.
        \item 下半平面$\H_-$
    \end{enumerate}
\end{theorem}
\begin{proof}
    それぞれ,$0,i,-i$の軌道として構成し,これらが$\hat{\C}$の類別となっていることを確認する.
\end{proof}

\section{上半平面と束}

\begin{screen}
    平面上に基底を2つ定めると,これらが作る座標系を得る.これを斜交座標の場合も含めて,束という代数系のことばでGauss平面上で捉える.
\end{screen}

\begin{definition}[lattice and its morphism]\mbox{}
    \begin{enumerate}
        \item 加法群としての$\C$の部分群$L$が\textbf{束}であるとは,ある$w_1,w_2\in\C$が存在して次を満たすことをいう:
        \begin{enumerate}[(1)]
            \item $L=\{mw_1+nw_2\mid m,n\in\Z\}=:\langle w_1,w_2\rangle_\Z$.
            \item $w_1,w_2$は$\R$上一次独立.(i.e. $\frac{w_1}{w_2}\notin\R\cup\{\pm\infty\}$).
        \end{enumerate}
        この時の$w_1,w_2\in\C$を$L$の\textbf{基底}と呼ぶ.
        \item $L_1,L_2\subset\C$を束とする.これらが\textbf{同型}であるとは,
        \[ L_1\simeq L_2:\;\Leftrightarrow\; \exists a\in\C^\times,\; aL_1=L_2 \]
        とする.ただし,$aL_1=\{aL\mid l\in L_1\}=\langle aw_1,aw_2\rangle_\Z$とした.
        \item $w_1,w_2\in\C$を一次独立とする時,これらが張る平行四辺形の内部を
        \[ P(w_1,w_2):=\{tw_1+sw_2\mid s,t\in (0,1)\} \]
        とする.
    \end{enumerate}
\end{definition}
\begin{remark}\mbox{}
    \begin{enumerate}
        \item $\C$上の点を2つ取ると,これを基底とした座標系を得る.それを束と呼ぶ.
        \item このように加法群$\C$の言葉で定義した束が,等角写像で写り合う時,同型であるという.
    \end{enumerate}
\end{remark}

\begin{definition}[upper half-plane]\mbox{}
    \begin{enumerate}
        \item $\H:=\{t\in\C\mid \Im t>0\}$を\textbf{上半平面}と呼ぶ.$H,\mathfrak{H},H^+$などとも表す.
        \item これがRiemann球面に埋め込まれているとみなした時,その閉包を\textbf{閉上半平面}と呼ぶ:$\overline{\H}=\H\cup\partial\H=\H\cup\R\cup\{\infty\}$.
        \item $t\in\H$に対して,これが上半平面上に定める束を
        \[ \Omega_t:=\{m+nt\mid m,n\in\Z\}=\langle 1,t\rangle_\Z \]
        と置く.
    \end{enumerate}
\end{definition}
\begin{remark}[上半平面に注目すれば良い理由]
    $L=\langle w_1,w_2\rangle_\Z$とすると,基底は一次独立であることより$\Im w_1/w_2\ne 0$である.
    この時必要なら順番を入れ替えることで$w_1/w_2\in\H$と出来る(なす角のうち「狭い方」を取れば$\pi$より小さく$0$より大きい).
    従って,$L=w_2\Omega_{w_1/w_2}$である.
\end{remark}

\subsection{$\H$上の束の同型類を定めたい}

\begin{screen}
    前節で,束を考えるには上半平面のみに注目したクラス$\Omega_t\;(t\in\C)$に注目すれば良いとして代表系を取った.
    次に,これらの同型類を定めたい.
\end{screen}

ここで,上半平面に対する実行列の作用を観察する.
まず,実行列の固有ベクトルにより強く分類できる.
なぜなら,複素共軛による双対命題が常に成り立つので,
1つのベクトルの行き先に言及するだけで同時に2つ目も定めていることになる.

\begin{lemma}
    $A\in M_2(\R),t\in\H$が
    \[ A\begin{pmatrix}t\\1\end{pmatrix}=\begin{pmatrix}t\\1\end{pmatrix} \]
    を満たす時,$A=I$である.
\end{lemma}
\begin{proof}
    $A\in M_2(\R)$より,
    \[ A\begin{pmatrix}\overline{t}\\1\end{pmatrix}=\begin{pmatrix}\overline{t}\\1\end{pmatrix} \]
    も成り立つ.$t\in\H$としたから,$\begin{pmatrix}t\\1\end{pmatrix},\begin{pmatrix}\overline{t}\\1\end{pmatrix}$は$\R$上一次独立より,$A$の定める写像は$\H$上の恒等写像である.
    従って,$A=I$.
\end{proof}

次の定理は,楕円関数と保型形式の間の関係の土台となる対応を示す.
それは,上半平面上の束$\Omega_t$が同型であるとは,$\SL_2(\Z)$の作用に対して,
同じ軌道に乗る
\[\SL_2(\Z)\cdot t_1=\SL_2(\Z)t_2\]
ことに同値であることを示す.

\begin{theorem}
    $t_1,t_2\in\H$に対して,以下は同値である.
    \begin{enumerate}
        \item $\Omega_{t_1}\simeq\Omega_{t_2}$.
        \item $\exists g\in\SL_2(\Z)$.
    \end{enumerate}
\end{theorem}

\section{基本領域}

\section{楕円関数}

\section{Eisenstein級数}

\section{Fourier展開}

\section{保型形式}

\section{保型形式の極と零点}

\section{保型関数体}

\end{document}