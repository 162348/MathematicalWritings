\documentclass[dvipdfmx,nosetpagesize, uplatex]{jsarticle}
%
% 133行目および163行以降を適宜編集すれば良い.
%
\usepackage{amsmath,amssymb,amscd,amsthm,amsbsy,multicol}
\usepackage[shortlabels,inline]{enumitem}
\renewcommand{\thefootnote}{\dag\arabic{footnote}}
\pagestyle{plain}
%
\setlength{\paperwidth}{257mm}
\setlength{\paperheight}{364mm}
\setlength{\textwidth}{170mm}
\setlength{\textheight}{280mm}
% \setlength{\oddsidemargin}{-2.0cm}
% \setlength{\evensidemargin}{-.3cm}
\setlength{\topmargin}{-31mm}
%\setlength{\footskip}{2cm}
%
\newtheoremstyle{StatementsWithStar}% ?name?
{3pt}% ?Space above? 1
{3pt}% ?Space below? 1
{}% ?Body font?
{}% ?Indent amount? 2
{\bfseries}% ?Theorem head font?
{\textbf{.}}% ?Punctuation after theorem head?
{.5em}% ?Space after theorem head? 3
{\textbf{\textup{#1~\thetheorem{}}}{}\,$^{\ast}$\thmnote{(#3)}}% ?Theorem head spec (can be left empty, meaning ‘normal’)?
%
\newtheoremstyle{StatementsWithStar2}% ?name?
{3pt}% ?Space above? 1
{3pt}% ?Space below? 1
{}% ?Body font?
{}% ?Indent amount? 2
{\bfseries}% ?Theorem head font?
{\textbf{.}}% ?Punctuation after theorem head?
{.5em}% ?Space after theorem head? 3
{\textbf{\textup{#1~\thetheorem{}}}{}\,$^{\ast\ast}$\thmnote{(#3)}}% ?Theorem head spec (can be left empty, meaning ‘normal’)?
%
\newtheoremstyle{StatementsWithStar3}% ?name?
{3pt}% ?Space above? 1
{3pt}% ?Space below? 1
{}% ?Body font?
{}% ?Indent amount? 2
{\bfseries}% ?Theorem head font?
{\textbf{.}}% ?Punctuation after theorem head?
{.5em}% ?Space after theorem head? 3
{\textbf{\textup{#1~\thetheorem{}}}{}\,$^{\ast\ast\ast}$\thmnote{(#3)}}% ?Theorem head spec (can be left empty, meaning ‘normal’)?
%
\newtheoremstyle{StatementsWithCCirc}% ?name?
{6pt}% ?Space above? 1
{6pt}% ?Space below? 1
{}% ?Body font?
{}% ?Indent amount? 2
{\bfseries}% ?Theorem head font?
{\textbf{.}}% ?Punctuation after theorem head?
{.5em}% ?Space after theorem head? 3
{\textbf{\textup{#1~\thetheorem{}}}{}\,$^{\circledcirc}$\thmnote{(#3)}}% ?Theorem head spec (can be left empty, meaning ‘normal’)?
%
\theoremstyle{definition}
 \newtheorem{theorem}{定理}[section]
 \newtheorem{corollary}[theorem]{系}
 \newtheorem{proposition}[theorem]{命題}
 \newtheorem*{proposition*}{命題}
 \newtheorem{lemma}[theorem]{補題}
 \newtheorem*{lemma*}{補題}
 \newtheorem*{theorem*}{定理}
 \newtheorem{definition}[theorem]{定義}
 \newtheorem{example}[theorem]{例}
 \newtheorem{notation}[theorem]{記号}
 \newtheorem*{notation*}{記号}
 \newtheorem{assumption}[theorem]{仮定}
 \newtheorem{question}[theorem]{問}
 \newtheorem{reidai}[theorem]{例題}
 \newtheorem{remark}[theorem]{注}
% \newtheorem*{remarknonum}{注}
 \newtheorem*{definition*}{定義}
 \newtheorem*{remark*}{注}
 \newtheorem*{question*}{問}
%
\theoremstyle{StatementsWithStar}
 \newtheorem{definition_*}[theorem]{定義}
 \newtheorem{question_*}[theorem]{問}
 \newtheorem{example_*}[theorem]{例}
 \newtheorem{theorem_*}[theorem]{定理}
 \newtheorem{remark_*}[theorem]{注}
%
\theoremstyle{StatementsWithStar2}
 \newtheorem{definition_**}[theorem]{定義}
 \newtheorem{theorem_**}[theorem]{定理}
 \newtheorem{question_**}[theorem]{問}
 \newtheorem{remark_**}[theorem]{注}
%
\theoremstyle{StatementsWithStar3}
 \newtheorem{remark_***}[theorem]{注}
 \newtheorem{question_***}[theorem]{問}
%
\theoremstyle{StatementsWithCCirc}
 \newtheorem{definition_O}[theorem]{定義}
 \newtheorem{question_O}[theorem]{問}
 \newtheorem{example_O}[theorem]{例}
 \newtheorem{remark_O}[theorem]{注}
%
\theoremstyle{definition}
%
\renewcommand{\proofname}{\underline{証明}}
%
\raggedbottom
\allowdisplaybreaks
%
\everymath{\displaystyle}
%
\begin{document}
\thispagestyle{empty}
\setlength{\parindent}{1zw}
\setlength{\baselineskip}{13pt}
\setcounter{section}{0}
\newcounter{version}
\setcounter{version}{1}
\noindent
2020年度ベクトル解析(足助担当)レポート問題~\thesection~v\theversion%\par\noindent
\hfil2020/4/13(月)\par\noindent
提出先:ITC-LMSのページの「課題」\par\noindent
提出期間:2020/4/13(月)$\sim$ 2020/4/27(月)\textbf{9:00}\par\noindent
返却はITC-LMSを用いて4/27日(月)以降に行う.
このレポートは添削しない予定である.\par\noindent
※ 印刷してレポートを作成し,スキャンして提出しても構わない.
その際,印刷が困難ならば速やかに足助まで申し出ること.
\vskip-18pt\noindent
\begin{table}[h]
\begin{tabular}{|c|c|c|} \hline
& & \\[-13pt]
学生証番号& 氏名 & 共同作成者(ある場合)\\[2pt] \hline
\parbox[c]{9.2zw}{J4-190549\centering} & \parbox[c]{13.0zw}{司馬 博文\centering} & \parbox[c]{25.6zw}{なし\centering}\\[12pt] \hline
%「\hfill」の前に必要事項を記入すること.
\end{tabular}
\end{table}
\vskip-12pt\noindent
% 
\begin{question*}
$\gamma\colon[0,2\pi]\to\mathbb{R}^2$を$\gamma(t)={}^t(\cos t,\sin t)$により定める.
また,$\gamma={}^t(\gamma^1,\gamma^2)$と成分を用いて表す.
\[
\int_0^{2\pi}\sqrt{\left(\dfrac{d\gamma^1}{dt}(t)\right)^2+\left(\dfrac{d\gamma^2}{dt}(t)\right)^2}dt
\]
を求めよ.
\end{question*}
\par
\ \par
\noindent
{\small
※ 参考文献がある場合には最後にまとめて箇条書きで示すこと.\par\noindent
※ \textbf{全体として2ページに収めること.}\par\noindent
※ 共同作成者に記載がないにもかかわらず,ほかのレポートとほぼ同一であるレポートが散見される.
誰かと共同してレポートを作成することは構わないが,そのことは明記すること.
それをしなければ剽窃であって,これは学術上の致命的な不正行為である.
万一,写される側がそのことを承知していなかったことが露見した場合には重大な結果をもたらす可能性がある.
}

\rightline{(以上)}\par
%
% 以下が解答欄である.2ページ以内に収まるように注意すること.なお,紙面レイアウトやフォントサイズを変更しないこと.
%
\noindent
解答欄

$\gamma^1(t)=\cos(t),\gamma^2(t)=\sin(t)$であるから,
\begin{eqnarray*}
    \int_0^{2\pi}\sqrt{\left(\dfrac{d\gamma^1}{dt}(t)\right)^2+\left(\dfrac{d\gamma^2}{dt}(t)\right)^2}dt
    &=& \int_0^{2\pi}\sqrt{\left( -\sin t\right)^2+\cos^2t}\hspace{0.5em}dt \\
    &=& \int_0^{2\pi}dt = 2\pi
\end{eqnarray*}
\rightline{$\blacksquare$}

\end{document}
