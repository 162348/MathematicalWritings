\documentclass[uplatex, 12pt, dvipdfmx, twocolumn]{jsarticle}
\title{微分積分学 石毛和弘\thanks{メールアドレスはishige@ms.u-tokyo.ac.jp} \\ 講義ノート}
\author{司馬博文 J4-190549 \\ hirofumi-shiba48@g.ecc.u-tokyo.ac.jp}
\date{2019年9月30日}
\pagestyle{headings} \setcounter{secnumdepth}{4}
\usepackage{amsmath, amsfonts, amsthm, mathptmx, amssymb, ascmac, color, comment, graphicx, wrap fig}
\usepackage{tikz, tikz-cd}
\usepackage[top=15truemm,bottom=15truemm,left=10truemm,right=10truemm]{geometry}
\newtheorem{theorem}{定理} \newtheorem{definition}{定義} \newtheorem{proposition}[theorem]{命題} \newtheorem{corollary}[theorem]{系} \newtheorem{lemma}[theorem]{補題} \newtheorem{problem}{問} \newtheorem{solution}{解}
\begin{document}




\section*{夜行}

ねぇ,このまま夜が来たら,

僕ら,どうなるんだろうね.

列車にでも乗って行くかい.

僕は,どこでもいいかな.

\vspace{\baselineskip}

君はまだ分からないだろうけど,

空も言葉で出来ているんだ.

「そっか,隣街ならついて行くよ.」

\vspace{\baselineskip}

ハラハラハラハラ

ハラリハルルハラ

君が詠む詩やイチリンソウ

他には何にも要らないから.

\vspace{\baselineskip}

波立つ夏原

涙尽きぬまま

鳴くやひぐらしは夕悠悠

夏が,終わって行くんだね.

そうなんだね.

\vspace{15cm}
 
\vspace{5.3mm}

ねぇ,いつか大人になったら,

僕ら,どうなるんだろうね.

何かしたいことはあるのかい.

僕は,それが見たいかな.

\vspace{\baselineskip}
\vspace{0.2mm}

君は忘れてしまうだろうけど,

思い出だけが本当なんだ.

「そっか,道の先ならついて行くよ.」

\vspace{\baselineskip}

サラサラサラサラ

サラサラサラサラ

花風揺られやイチリンソウ

言葉は何にも要らないから.

\vspace{\baselineskip}

君発つ夏原

髪はなびくまま

泣くや雨模様憂悠悠

夏が,終わって行くんだね.

そうなんだね.

\vspace{\baselineskip}
\vspace{\baselineskip}

そっか.大人になったんだね.

\vspace{\baselineskip}

ハラハラハラハラ

ハラリハルルハラ

君が詠む詩やイチリンソウ

他には何にも要らないから.

\vspace{\baselineskip}

波立つ夏原

涙尽きぬまま

鳴くやひぐらしは夕悠悠

夏が,終わって行くんだね.

\vspace{\baselineskip}

僕は,此処に残るんだね.

ずっと向こうへ行くんだね.

そうなんだね.

\end{document}