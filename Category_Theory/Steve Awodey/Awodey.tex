\documentclass[uplatex, 12pt, dvipdfmx]{jsarticle}
\title{}
\author{司馬博文 J4-190549\\hirofumi-shiba48@g.ecc.u-tokyo.ac.jp}
\date{\today}
\pagestyle{headings} \setcounter{secnumdepth}{4}
\input{/Users/hirofumi.shiba48/Desktop/数理科学/preamble_CM.tex}
\begin{document}
\tableofcontents

\section{Categories}

\subsection{Introduction}
\subsection{Functions of sets}
\subsection{Definition of a category}
\begin{definition}[Category] 

    1. 対象$A,B,C,\cdots$というものがある.

    2. 射$f,g,h,\cdots$というものがある.

    3. 各射には$\mathrm{dom}(f)=A, \mathrm{cod}(f)=B$という対象が紐づけられていて,その関係を$f:A\to B$と書く.

    4. $\mathrm{cod}(f)=\mathrm{dom}(g)$を満たす射$f,g$に対し,$g\circ f:\mathrm{dom}(f)\to \mathrm{cod}(g)$という射が定義される.

    5. 各対象$A$には$1_A:A\to A$という特別な射が定義される(単位射).

    6. 射は結合律を満たす.$h\circ (g\circ f)=(h\circ g)\circ f$

    7. 単位射は合成について単位的である.$f:A\to B$として,$f\circ 1_A=f=1_B\circ f$
\end{definition}

\subsection{Examples of categories}

1. 集合の圏$\mathbf{Sets}$と,有限集合の圏$\mathbf{Sets}_\mathrm{fin}$
\begin{example_*}[集合の圏から,対象の集合と射の集合に特定の制限を付け加えることで,自由に部分圏が作れる他の例.] 

    1. 対象:有限集合,射:単射

    2. 対象:集合,射:ファイバーが高々2元集合である写像

    3. 対象:集合,射:ファイバーが高々有限集合である写像

    4. 対象:集合,射:ファイバーは無限でも良い多価写像
\end{example_*}

2. Category of structured sets
\begin{definition*}[具体圏]
    圏$C$が,忘却関手$U:C\to\mathbf{Set}$を持つとき,これを具体圏と呼ぶ.
\end{definition*}

3. 順序集合と単調写像の圏$\mathbf{Pos}$

4. 二項関係の圏$\mathbf{Rel}$:写像は特別な二項関係と見れるから,$\mathbf{Sets}$はこの部分圏である.

射$f:A\to B$は$A\times B$の部分集合で,単位射$1_A$は恒等写像$id_A$のグラフと共通の「恒等関係」となる.
合成は,2つの関係$R\subset A\times B, S\subset B\times C$から作れる「相対関係$(a,c)\in S\circ R:\Leftrightarrow \exists (a,b)\in R, (b,c)\in S$」として作れば確かに閉じている.

5. 有限圏としての自然数:射は順序関係である.

6. 圏の圏$\mathbf{Cat}$
\begin{definition}[Functor]
    関手$F:\mathbf{C}\to\mathbf{D}$とは,次を満たす対象写像と射写像の組である.

    1. $F(f:A\to B)=F(f):F(A)\to F(B)$

    2. $F(1_A)=1_F(A)$

    3. $F(g\circ f)=F(g)\circ F(g)$
\end{definition}

7. 圏としてのpreorder:任意の2つの間に射が1つしか存在しない圏(細い圏).

\begin{definition*}[thin category]
    圏$C$が次の条件を満たす時,細い圏であるという.
    
    任意の2つの対象$x,y\in C$について,
    \begin{center}\begin{tikzcd}
        x \ar[r, "f"] \ar[r, "g"'] & g
    \end{tikzcd}\end{center}
    となっている時,必ず$f=g$である.
\end{definition*}
\begin{remark*}
    細い圏に於いて,2つの対象間で双方向に射が存在する場合,これは互いに逆射になる.
\end{remark*}
\begin{proposition*}
    細い圏は,prosetと同型で,posetと同値である.
\end{proposition*}
\begin{proof}
    圏$C$の対象の集合を集合$P$とし,その間の関係$x\le y$を
    \[ x\le y:\Leftrightarrow \exists f:x\to y\in C \]
    と定めると,この関係は反射性と推移性を満たし,前順序集合(preordered set)となる.
    今,関手$F:C\to P$を対象集合は$1_P$,射集合は$f:x\to y\mapsto x\le y$とすると,これはいずれも可逆で,確かに圏の同型である.

    この時,集合$P$について,次のように約束する.
    \[ x\le y\land y\le x\Rightarrow x=y \]
    すると集合$P/=$は順序集合(partially ordered set)である.
    関手$F':C\to P/=$は厳密な意味では可逆ではない.
\end{proof}

8. 圏としてのposet:poset categories

9. 位相空間からの例
\begin{proposition*}
    $T_0$ spaces $X$ are posets under the specialization ordering:
    \[ x\le y \Leftrightarrow \forall U\in O(X)\; (x\in U\Rightarrow y\in U) \]
\end{proposition*}

10. 数理論理学からの例:演繹体系に付随する圏 category of proofs
対象を式とし,その間に証明がある$\varphi\vdash\psi$時,射$\varphi\to\psi$を定義する.

11. 計算機科学からの例:関数型プログラミング言語Lに付随する圏$C(L)$
対象は$L$のデータ型,射は関数とする.単位射はdo nothing programで,合成は関数の連続適用$g\circ f=f:g$である.

12. 集合$X$に付随する離散圏$\mathbf{Dis}(X)$

13. 単一対象圏としてのmonoid

射が対象の間に持つ構造「2つの対象と順番付きで紐づけられている」と「単位射の存在」と「合成についての閉性(=推移性)」とを,そっくりそのまま,順序関係に翻訳すれば前順序である.
射自体の持つ構造「結合性」と「単位射の存在」を,代数構造に翻訳すればモノイドである.いずれも最低限の圏である.
それぞれに付加構造として対称性を加えれば,半順序と群を得る.半順序とモノイドが,この本の主要な例になる.

8., 13.の観点から,posetの射とは関手だし,monoidの射も関手と見做せる.

\subsection{Isomorphisms}

\begin{definition}[同型]
    圏$C$に於いて,次を満たす射$f:A\to B$を同型という.
    \[ \exists g:B\to A\in C\; g\circ f=1_A \land f\circ g=1_B \]
\end{definition}
\begin{remark*}[note that, for example in Pos, the category theoretic definition gives the right notion, while there are "bijective homomorphisms" between non-isomorphic posets.]
    射を何らかの写像だとすると,この同型であるための条件は全単射であることと同値.従ってこの定義は,具体圏に於ける台写像の「全単射」性を一般の圏に写し取ったものに思える.
    だから,全単射でないのに同型になることはないはずだ.
    だが,全単射な射は可逆だとは限らない.
\end{remark*}

\begin{definition}[群]
    群とは,可逆なモノイドのことである.従って,全ての射が同型であるような単一対象圏のことである.
\end{definition}

\begin{theorem*}[Cayley]
    群$G=(G,\cdot,e,{}^{-1})$は,$\mathrm{Aut}(G)$の或る部分群と同型になる.
\end{theorem*}
\begin{proof}
    Cayley representation $\overline{G}\subset\mathrm{Aut}(G)$を構成する.各$g\in G$に対して,$\overline{g}\in\overline{G}\subset\mathrm{Aut}(G)$を次のような射として定める.
    \begin{center}\begin{tikzcd}
        \overline{g}=g^*:G\ar[r] \ar[d, phantom, "\rotatebox{90}{$\in$}"] & G \ar[d, phantom, "\rotatebox{90}{$\in$}"] \\
        h \ar[r, mapsto] & g\cdot h
    \end{tikzcd}\end{center}
    この時,$\overline{G}$は群になっていることを,写像$F:G\to \overline{G}$が群の射であることを示すことによって確認する.
    $F(f\cdot g)=F(f)\circ F(g)$は$G$の演算$\cdot$の結合性より,また$F(e)=1_G$も成り立つ.
    なお,各射の可逆性については,$F(f\cdot f^{-1})=F(f)\circ F(f^{-1})=1_G=F(e)$より成り立つ.

    群の射$F:G\to \overline{G}$の逆射$H$を構成する.
    \begin{center}\begin{tikzcd}
        H:\overline{G}\ar[r] \ar[d, phantom, "\rotatebox{90}{$\in$}"] & G \ar[d, phantom, "\rotatebox{90}{$\in$}"] \\
        \overline{g} \ar[r, mapsto] & g=\overline{g}(e)
    \end{tikzcd}\end{center}
    これについて,確かに$F\circ H=1_{\overline{G}}, H\circ F=1_G$が成り立つ.従って,$G\simeq \overline{G}$
\end{proof}

\begin{remark}[Two different levels of isomorphisms]
    構成した群$\overline{G}\subset\mathrm{Aut}(G)$の元である,$g$を集合$G$に左から作用させる写像$\overline{g}$は,群$G$の置換であり,集合の同型である.
    一方,構成した関手$F,H$は群の同型である.
\end{remark}

\begin{theorem}
    任意の圏$C$は,或る具体圏と同型である.
\end{theorem}
\begin{proof}
    圏$C$から,同型な圏$\overline{C}$を構成する.関手$\overline{ }:C\to\overline{C}$の対象写像を次のように定める.
    \begin{center}\begin{tikzcd}
        C \ar[r] \ar[d, phantom, "\rotatebox{90}{$\in$}"] & \overline{C} \ar[d, phantom, "\rotatebox{90}{$\in$}"] \\
        c \ar[r, mapsto] & \overline{c}=\{ f\in\mathrm{arr}(C)\mid \mathrm{cod}(f)=c \}
    \end{tikzcd}\end{center}
    射関手を次のように定める.
    \begin{center}\begin{tikzcd}
        C \ar[r] \ar[d, phantom, "\rotatebox{90}{$\in$}"] & \overline{C} \ar[d, phantom, "\rotatebox{90}{$\in$}"]\\
        g:c\to d \ar[r, mapsto] & \overline{g}=g^*:\hom_C(-,c)\to\hom_C(-,d)
    \end{tikzcd}\end{center}
    ただし,この写像$g^*$は,任意の対象$x\in C$に対して,
    \begin{center}\begin{tikzcd}
        \hom_C(x,c) \ar[r] \ar[d, phantom, "\rotatebox{90}{$\in$}"] & \hom_C(x,d)\ar[d, phantom, "\rotatebox{90}{$\in$}"]\\
        f:x\to c \ar[r, mapsto] & g\circ f:x\to d
    \end{tikzcd}\end{center}
    と対応づける写像(関手の射/自然変換)である.
    この関手は可逆であり,逆関手の$\overline{x}\in\overline{C}$成分は射写像は次の通りである.
    \begin{center}\begin{tikzcd}
        \overline{C} \ar[r] \ar[d, phantom, "\rotatebox{90}{$\in$}"] & C \ar[d, phantom, "\rotatebox{90}{$\in$}"]\\
        \overline{g}:\hom_C(-,c)\to\hom_C(-,d) \ar[r, mapsto] & \overline{g}(1_c)
    \end{tikzcd}\end{center}
\end{proof}
\begin{remark*}
    これが「表現」という述語の出処であろう.この時点ではまだ素朴の意味で「$C$の表現$\overline{C}$」という感覚である.
    また,これが「ホム関手」「ホム集合」という概念の出処でもある.集合での表現を持つから,我々の「具体」性という得意分野に引きずりこめるのだ.
    また,集合に頼り過ぎないで,純粋に圏論的なまま理論を豊かにしていくのも大事である.(群論だってそうなのだろう).
    例えば,一般の圏を白紙から考えるとき,対象の間の射全体の集まりは「集合」であるとは限らないのだ.
\end{remark*}

\subsection{Constructions on categories}

\subsubsection{Product}

圏$C\times D$は$(c,d)$という形の対象をもち,射も,合成も,単位射も,直接の「要素毎」の考え方で,新しい圏を想定出来る.

\subsubsection{Opposite}

$f:C\to D\in C$に対して,$f^*:D^*\to C^*\in C^{op}$で,合成の順序も逆にしたもの.

dualityとは,ある圏が,別の圏の反対(の部分圏)になるという対応が成り立つこと(を主張する命題のこと)である.

\subsubsection{arrow category}

圏$C$に対して,その射を対象とし,その間の射を$g:(f:A\to B)\to (f':A'\to B')$を,次の$f'\circ g_1=g_2\circ f$を主張する可換図式,つまり,圏$C$の射の組$g:=(g_1,g_2)$とする圏である.
\begin{center}\begin{tikzcd}
    A \ar[r, "g_1"] \ar[d, "f"'] & A' \ar[d, "f'"]\\
    B \ar[r, "g_2"] & B'
\end{tikzcd}\end{center}
合成は,可換図式を繋げて外回りを取ること,つまり成分毎$(h_1,h_2)\circ (g_1,g_2)=(h_1\circ g_1, h_2\circ g_2)$で,従って単位射は$1_f=(1_A,1_B)$

対象は射$f:A\to B$だが,要は$(A,B)$,これはどう考えても$C\times C$あるいは$[2,C]$と同型になる.即ち,次の関手が存在する.
\begin{center}\begin{tikzcd}
    C & \overrightarrow{C} \ar[l, "\mathrm{dom}"'] \ar[r, "\mathrm{cod}"] & C
\end{tikzcd}\end{center}

\subsubsection{slice category}

圏$C$と対象$c\in C$について,$\{ f\in\mathrm{arr}(C)\mid \mathrm{cod}(f)=c \}$を対象全体の集合とし,2つの対象$f:x\to c, f':x'\to c$の間の射は次の$C$の図式を可換にする射$a:x'\to x\in C$である($f=f'\circ a$).
\begin{center}\begin{tikzcd}
    x\ar[rr, "a"] \ar[dr, "f"'] & & x' \ar[dl, "f'"] \\
    &c&
\end{tikzcd}\end{center}
これはarrow categoryの部分圏であろう.

対象について,そのcodomain$c$を忘れ,射$a:(x,c)\to (x',c)$についても$c$を忘れれば,忘却関手$C/c\to C$を定める.
これは一種の具体圏だったのか.

$C$の射$g:c\to d$に対して,関手$g^*:C/c\to C/d$が定まる.
\begin{center}\begin{tikzcd}
    C/c \ar[r] \ar[d, phantom, "\rotatebox{90}{$\in$}"] & C/d\ar[d, phantom, "\rotatebox{90}{$\in$}"]\\
    f:x\to c \ar[r, mapsto] & g\circ f:x\to d \\
    a:(f:x\to c)\to (f':x'\to c) \ar[r] & a:(g\circ f:x\to d)\to (g\circ f':x'\to d)
\end{tikzcd}\end{center}

slice categoryの構成は,関手$C/(-):C\to \mathbf{Cat}$を定める.これは圏$C$に対して,関手圏としての表現を与える表現関手,と思うことが出来る.

\subsection{Free categories}
\subsection{Foundations: large, small, and locally small}
\subsection{Exercises}

\section{Abstract structures}

\subsection{Epis and monos}
\subsection{Initial and terminal objects}
\subsection{Generalized elements}
\subsection{Products}
\subsection{Examples of products}
\subsection{Categories with products}
\subsection{Hom-sets}
\subsection{Exercises}

\section{Duality}

\subsection{The duality principle}
\subsection{Coproducts}
\subsection{Equalizers}
\subsection{Coequalizers}
\subsection{Exercises}

\section{Groups and categories}

\subsection{Groups in category}
\subsection{The category of groups}
\subsection{Groups as categories}
\subsection{Finitely presented categories}
\subsection{Exercises}

\section{Limits and colimits}

\subsection{Subobjects}
\subsection{Pullbacks}
\subsection{Properties of pullbacks}
\subsection{Limits}
\subsection{Perservation of limits}
\subsection{Colimits}
\subsection{Exercises}

\section{Exponentials}

\subsection{Exponential in a category}

\subsection{Cartesian closed categories}

\subsection{Heyting algebras}

\subsection{Propositional definition of CCC}

\subsection{$\lambda$-calculus}

\subsection{Variable sets}

\subsection{exercises}

\section{Naturality}

\subsection{Category of categories}

\subsection{Representable structure}

\subsection{Stone duality}

\subsection{Naturality}

\subsection{Examples of natural transformations}

\subsection{Exponentials of categories}

\subsection{Functor categories}

\subsection{Monoidal categories}

\subsection{Equivalence of categories}

\subsection{Examples of equivalence}

\subsection{Exercises}

\section{Categories of diagrams}

\subsection{Set-valued functor categories}

\subsection{The Yoneda embedding}

\subsection{The Yoneda lemma}

\subsection{Applications of the Yoneda lemma}

\subsection{Limits in categories of diagrams}

\subsection{Colimits in categories of diagrams}

\subsection{Exponentials in categories of diagrams}

\subsection{Topoi}

\subsection{Exercises}

\section{Adjoints}

\subsection{Preliminary definition}

\subsection{Hom-set definition}

\subsection{Examples of adjoints}

\subsection{Order adjoints}

\subsection{Quantifiers as adjoints}

\subsection{RAPL}

\subsection{Locally cartesian closed categories}

\subsection{Adjoint functor theorem}

\subsection{Exercises}

\section{Monads and algebras}

\subsection{The triange identities}

\subsection{Mondas and adjoints}

\subsection{Algebras for a monad}

\subsection{Comonads and coalgebras}

\subsection{Algebras for endofunctors}

\subsection{Exercises}

\end{document}