\documentclass[uplatex, 12pt, dvipdfmx]{jsarticle}
\title{}
\author{司馬博文 J4-190549\\hirofumi-shiba48@g.ecc.u-tokyo.ac.jp}
\date{\today}
\pagestyle{headings} \setcounter{secnumdepth}{4}
\usepackage{amsmath, amsfonts, amsthm, amssymb, ascmac, color, comment, wrap fig}

\setcounter{tocdepth}{2}
%2はsubsectionまで
\usepackage{mathtools}
%\mathtoolsset{showonlyrefs=true} %labelを附した数式にのみ附番される.

%%% 生成されるPDFファイルにおいて、\tableofcontents によって書き出された目次をクリックすると該当する見出しへジャンプしたり、 さらには、\label{ラベル名} を番号で参照する \ref{ラベル名} や thebibliography環境において \bibitem{ラベル名} を文献番号で参照する \cite{ラベル名} においても番号をクリックすると該当箇所にジャンプする
\usepackage[dvipdfmx]{hyperref}
\usepackage{pxjahyper}

\usepackage{tikz, tikz-cd}
\usepackage[all]{xy}
\def\objectstyle{\displaystyle} %デフォルトではxymatrix中の数式が文中数式モードになるので,それを直した.

%化学式をTikZで簡単に書くためのパッケージ.
\usepackage[version=4]{mhchem} %texdoc mhchem
%化学構造式をTikZで描くためのパッケージ.
\usepackage{chemfig}
%IS単位を書くためのパッケージ
\usepackage{siunitx}

%取り消し線を引くためのパッケージ
\usepackage{ulem}

%\rotateboxコマンドを,文字列の中心で回転させるオプション.
%他rotatebox, scalebox, reflectbox, resizeboxなどのコマンド.
\usepackage{graphicx}

%加藤晃史さんがフル活用していたtcolorboxを,途中改ページ可能で.
\usepackage[breakable]{tcolorbox}

%足助さんからもらったオプション
% \usepackage[shortlabels,inline]{enumitem}
% \usepackage[top=15truemm,bottom=15truemm,left=10truemm,right=10truemm]{geometry}

%enumerate環境を凝らせる.
\usepackage{enumerate}

%日本語にルビをふる
\usepackage{pxrubrica}

%以下,ソースコードを表示する環境の設定.
\usepackage{listings,jvlisting} %日本語のコメントアウトをする場合jlistingが必要
%ここからソースコードの表示に関する設定
\lstset{
  basicstyle={\ttfamily},
  identifierstyle={\small},
  commentstyle={\smallitshape},
  keywordstyle={\small\bfseries},
  ndkeywordstyle={\small},
  stringstyle={\small\ttfamily},
  frame={tb},
  breaklines=true,
  columns=[l]{fullflexible},
  numbers=left,
  xrightmargin=0zw,
  xleftmargin=3zw,
  numberstyle={\scriptsize},
  stepnumber=1,
  numbersep=1zw,
  lineskip=-0.5ex
}
%lstlisting環境で,[caption=hoge,label=fuga]などのoptionを付けられる.

%%%
%%%フォント
%%%

%本文・数式の両方のフォントをTimesに変更するお手軽なパッケージだが,LaTeX標準数式記号の\jmath, \amalg, coprodはサポートされない.
\usepackage{mathptmx}
%Palatinoの方が完成度が高いと美文書作成に書いてあった.
% \usepackage[sc]{mathpazo} %オプションは,familyの指定.pplxにしている.
%2000年のYoung Ryuによる新しいTimes系.なおPalatinoもある.
% \usepackage{newtxtext, newtxmath}
%拡張数学記号.\textsectionでブルバキに!
% \usepackage{textcomp, mathcomp}
% \usepackage[T1]{fontenc} %8bitエンコーディングにする.comp系拡張数学文字の動作が安定する.
%AMS Euler.Computer Modernと相性が悪いとは…….
% \usepackage{ccfonts, eulervm} %KnuthのConcrete Mathematicsの組み合わせ.
% \renewcommand{\rmdefault}{pplx} %makes LaTeX use Palatino in place of CM Roman.Do not use the Euler math fonts in conjunction with the default Computer Modern text fonts – this is ugly!

%%% newcommands
    %参考文献で⑦というのを出したかった.\circled{n}と打てば良い.LaTeX StackExchangeより.
\newcommand*\circled[1]{\tikz[baseline=(char.base)]{\node[shape=circle,draw,inner sep=0.8pt] (char) {#1};}}

%%%
%%% ショートカット 足助さんからのコピペ
%%%

\DeclareMathOperator{\grad}{\mathrm{grad}}
\DeclareMathOperator{\rot}{\mathrm{rot}}
\DeclareMathOperator{\divergence}{\mathrm{div}}
\newcommand\R{\mathbb{R}}
\newcommand\N{\mathbb{N}}
\newcommand\C{\mathbb{C}}
\newcommand\Z{\mathbb{Z}}
\newcommand\Q{\mathbb{Q}}
\newcommand\GL{\mathrm{GL}}
\newcommand\SL{\mathrm{SL}}
\newcommand\False{\mathrm{False}}
\newcommand\True{\mathrm{True}}
\newcommand\tr{\mathrm{tr}}
\newcommand\M{\mathcal{M}}
\newcommand\F{\mathbb{F}}
\renewcommand\H{\mathbb{H}}
\newcommand\id{\mathrm{id}}
\newcommand\A{\mathcal{A}}
\renewcommand\coprod{\rotatebox[origin=c]{180}{$\prod$}}
\newcommand\pr{\mathrm{pr}}
\newcommand\U{\mathfrak{U}}
\newcommand\Map{\mathrm{Map}}
\newcommand\dom{\mathrm{dom}}
\newcommand\cod{\mathrm{cod}}
\newcommand\supp{\mathrm{supp}\;}
\newcommand\Ker{\mathrm{Ker}\;}
%%% 複素解析学
\renewcommand\Re{\mathrm{Re}\;}
\renewcommand\Im{\mathrm{Im}\;}
\newcommand\Gal{\mathrm{Gal}}
\newcommand\PGL{\mathrm{PGL}}
\newcommand\PSL{\mathrm{PSL}}
%%% 解析力学
\newcommand\x{\mathbf{x}}
\newcommand\q{\mathbf{q}}
%%% 集合と位相
\newcommand\ORD{\mathrm{ORD}}
%%% 形式言語理論
\newcommand\REGEX{\mathrm{REGEX}}

%%% 圏
\newcommand\Hom{\mathrm{Hom}}
\newcommand\Mor{\mathrm{Mor}}
\newcommand\Aut{\mathrm{Aut}}
\newcommand\End{\mathrm{End}}
\newcommand\op{\mathrm{op}}
\newcommand\ev{\mathrm{ev}}
\newcommand\Ob{\mathrm{Ob}}
\newcommand\Ar{\mathrm{Ar}}
\newcommand\Arr{\mathrm{Arr}}
\newcommand\Set{\mathrm{Set}}
\newcommand\Grp{\mathrm{Grp}}
\newcommand\Cat{\mathrm{Cat}}
\newcommand\Mon{\mathrm{Mon}}
\newcommand\CMon{\mathrm{CMon}}
\newcommand\Pos{\mathrm{Pos}}
\newcommand\Vect{\mathrm{Vect}}
\newcommand\FinVect{\mathrm{FinVect}}
\newcommand\Fun{\mathrm{Fun}}
\newcommand\Ord{\mathrm{Ord}}
\newcommand\eq{\mathrm{eq}}
\newcommand\coeq{\mathrm{coeq}}

%%%
%%% 定理環境 以下足助さんからのコピペ
%%%

\newtheoremstyle{StatementsWithStar}% ?name?
{3pt}% ?Space above? 1
{3pt}% ?Space below? 1
{}% ?Body font?
{}% ?Indent amount? 2
{\bfseries}% ?Theorem head font?
{\textbf{.}}% ?Punctuation after theorem head?
{.5em}% ?Space after theorem head? 3
{\textbf{\textup{#1~\thetheorem{}}}{}\,$^{\ast}$\thmnote{(#3)}}% ?Theorem head spec (can be left empty, meaning ‘normal’)?
%
\newtheoremstyle{StatementsWithStar2}% ?name?
{3pt}% ?Space above? 1
{3pt}% ?Space below? 1
{}% ?Body font?
{}% ?Indent amount? 2
{\bfseries}% ?Theorem head font?
{\textbf{.}}% ?Punctuation after theorem head?
{.5em}% ?Space after theorem head? 3
{\textbf{\textup{#1~\thetheorem{}}}{}\,$^{\ast\ast}$\thmnote{(#3)}}% ?Theorem head spec (can be left empty, meaning ‘normal’)?
%
\newtheoremstyle{StatementsWithStar3}% ?name?
{3pt}% ?Space above? 1
{3pt}% ?Space below? 1
{}% ?Body font?
{}% ?Indent amount? 2
{\bfseries}% ?Theorem head font?
{\textbf{.}}% ?Punctuation after theorem head?
{.5em}% ?Space after theorem head? 3
{\textbf{\textup{#1~\thetheorem{}}}{}\,$^{\ast\ast\ast}$\thmnote{(#3)}}% ?Theorem head spec (can be left empty, meaning ‘normal’)?
%
\newtheoremstyle{StatementsWithCCirc}% ?name?
{6pt}% ?Space above? 1
{6pt}% ?Space below? 1
{}% ?Body font?
{}% ?Indent amount? 2
{\bfseries}% ?Theorem head font?
{\textbf{.}}% ?Punctuation after theorem head?
{.5em}% ?Space after theorem head? 3
{\textbf{\textup{#1~\thetheorem{}}}{}\,$^{\circledcirc}$\thmnote{(#3)}}% ?Theorem head spec (can be left empty, meaning ‘normal’)?
%
\theoremstyle{definition}
 \newtheorem{theorem}{定理}[section]
 \newtheorem{axiom}[theorem]{公理}
 \newtheorem{corollary}[theorem]{系}
 \newtheorem{proposition}[theorem]{命題}
 \newtheorem*{proposition*}{命題}
 \newtheorem{lemma}[theorem]{補題}
 \newtheorem*{lemma*}{補題}
 \newtheorem*{theorem*}{定理}
 \newtheorem{definition}[theorem]{定義}
 \newtheorem{example}[theorem]{例}
 \newtheorem{notation}[theorem]{記法}
 \newtheorem*{notation*}{記法}
 \newtheorem{assumption}[theorem]{仮定}
 \newtheorem{question}[theorem]{問}
 \newtheorem{counterexample}[theorem]{反例}
 \newtheorem{reidai}[theorem]{例題}
 \newtheorem{problem}[theorem]{問題}
 \newtheorem*{solution*}{\bf{[解]}}
 \newtheorem{discussion}[theorem]{議論}
 \newtheorem{remark}[theorem]{注}
 \newtheorem{universality}[theorem]{普遍性} %非自明な例外がない.
 \newtheorem{universal tendency}[theorem]{普遍傾向} %例外が有意に少ない.
 \newtheorem{hypothesis}[theorem]{仮説} %実験で説明されていない理論.
 \newtheorem{theory}[theorem]{理論} %実験事実とその(さしあたり)整合的な説明.
 \newtheorem{fact}[theorem]{実験事実}
 \newtheorem{model}[theorem]{模型}
 \newtheorem{explanation}[theorem]{説明} %理論による実験事実の説明
 \newtheorem{anomaly}[theorem]{理論の限界}
 \newtheorem{application}[theorem]{応用例}
 \newtheorem{method}[theorem]{手法} %実験手法など,技術的問題.
 \newtheorem{history}[theorem]{歴史}
 \newtheorem{research}[theorem]{研究}
% \newtheorem*{remarknonum}{注}
 \newtheorem*{definition*}{定義}
 \newtheorem*{remark*}{注}
 \newtheorem*{question*}{問}
 \newtheorem*{axiom*}{公理}
 \newtheorem*{example*}{例}
%
\theoremstyle{StatementsWithStar}
 \newtheorem{definition_*}[theorem]{定義}
 \newtheorem{question_*}[theorem]{問}
 \newtheorem{example_*}[theorem]{例}
 \newtheorem{theorem_*}[theorem]{定理}
 \newtheorem{remark_*}[theorem]{注}
%
\theoremstyle{StatementsWithStar2}
 \newtheorem{definition_**}[theorem]{定義}
 \newtheorem{theorem_**}[theorem]{定理}
 \newtheorem{question_**}[theorem]{問}
 \newtheorem{remark_**}[theorem]{注}
%
\theoremstyle{StatementsWithStar3}
 \newtheorem{remark_***}[theorem]{注}
 \newtheorem{question_***}[theorem]{問}
%
\theoremstyle{StatementsWithCCirc}
 \newtheorem{definition_O}[theorem]{定義}
 \newtheorem{question_O}[theorem]{問}
 \newtheorem{example_O}[theorem]{例}
 \newtheorem{remark_O}[theorem]{注}
%
\theoremstyle{definition}
%
\raggedbottom
\allowdisplaybreaks

%証明環境のスタイル
\everymath{\displaystyle}
\renewcommand{\proofname}{\bf [証明]}
\renewcommand{\thefootnote}{\dag\arabic{footnote}}	%足助さんからもらった.どうなるんだ?

%mathptmxパッケージ下で,\jmath, \amalg, coprodの記号を出力するためのマクロ.TeX Wikiからのコピペ.
% \DeclareSymbolFont{cmletters}{OML}{cmm}{m}{it}
% \DeclareSymbolFont{cmsymbols}{OMS}{cmsy}{m}{n}
% \DeclareSymbolFont{cmlargesymbols}{OMX}{cmex}{m}{n}
% \DeclareMathSymbol{\myjmath}{\mathord}{cmletters}{"7C}
% \DeclareMathSymbol{\myamalg}{\mathbin}{cmsymbols}{"71}
% \DeclareMathSymbol{\mycoprod}{\mathop}{cmlargesymbols}{"60}
% \let\jmath\myjmath
% \let\amalg\myamalg
% \let\coprod\mycoprod
\begin{document}
\tableofcontents

\section{Categories}

\subsection{Introduction}
\subsection{Functions of sets}
\subsection{Definition of a category}
\begin{definition}[Category] 

    1. 対象$A,B,C,\cdots$というものがある.

    2. 射$f,g,h,\cdots$というものがある.

    3. 各射には$\mathrm{dom}(f)=A, \mathrm{cod}(f)=B$という対象が紐づけられていて,その関係を$f:A\to B$と書く.

    4. $\mathrm{cod}(f)=\mathrm{dom}(g)$を満たす射$f,g$に対し,$g\circ f:\mathrm{dom}(f)\to \mathrm{cod}(g)$という射が定義される.

    5. 各対象$A$には$1_A:A\to A$という特別な射が定義される(単位射).

    6. 射は結合律を満たす.$h\circ (g\circ f)=(h\circ g)\circ f$

    7. 単位射は合成について単位的である.$f:A\to B$として,$f\circ 1_A=f=1_B\circ f$
\end{definition}

\subsection{Examples of categories}

1. 集合の圏$\mathbf{Sets}$と,有限集合の圏$\mathbf{Sets}_\mathrm{fin}$
\begin{example_*}[集合の圏から,対象の集合と射の集合に特定の制限を付け加えることで,自由に部分圏が作れる他の例.] 

    1. 対象:有限集合,射:単射

    2. 対象:集合,射:ファイバーが高々2元集合である写像

    3. 対象:集合,射:ファイバーが高々有限集合である写像

    4. 対象:集合,射:ファイバーは無限でも良い多価写像
\end{example_*}

2. Category of structured sets
\begin{definition*}[具体圏]
    圏$C$が,忘却関手$U:C\to\mathbf{Set}$を持つとき,これを具体圏と呼ぶ.
\end{definition*}

3. 順序集合と単調写像の圏$\mathbf{Pos}$

4. 二項関係の圏$\mathbf{Rel}$:写像は特別な二項関係と見れるから,$\mathbf{Sets}$はこの部分圏である.

射$f:A\to B$は$A\times B$の部分集合で,単位射$1_A$は恒等写像$id_A$のグラフと共通の「恒等関係」となる.
合成は,2つの関係$R\subset A\times B, S\subset B\times C$から作れる「相対関係$(a,c)\in S\circ R:\Leftrightarrow \exists (a,b)\in R, (b,c)\in S$」として作れば確かに閉じている.

5. 有限圏としての自然数:射は順序関係である.

6. 圏の圏$\mathbf{Cat}$
\begin{definition}[Functor]
    関手$F:\mathbf{C}\to\mathbf{D}$とは,次を満たす対象写像と射写像の組である.

    1. $F(f:A\to B)=F(f):F(A)\to F(B)$

    2. $F(1_A)=1_F(A)$

    3. $F(g\circ f)=F(g)\circ F(g)$
\end{definition}

7. 圏としてのpreorder:任意の2つの間に射が1つしか存在しない圏(細い圏).

\begin{definition*}[thin category]
    圏$C$が次の条件を満たす時,細い圏であるという.
    
    任意の2つの対象$x,y\in C$について,
    \begin{center}\begin{tikzcd}
        x \ar[r, "f"] \ar[r, "g"'] & g
    \end{tikzcd}\end{center}
    となっている時,必ず$f=g$である.
\end{definition*}
\begin{remark*}
    細い圏に於いて,2つの対象間で双方向に射が存在する場合,これは互いに逆射になる.
\end{remark*}
\begin{proposition*}
    細い圏は,prosetと同型で,posetと同値である.
\end{proposition*}
\begin{proof}
    圏$C$の対象の集合を集合$P$とし,その間の関係$x\le y$を
    \[ x\le y:\Leftrightarrow \exists f:x\to y\in C \]
    と定めると,この関係は反射性と推移性を満たし,前順序集合(preordered set)となる.
    今,関手$F:C\to P$を対象集合は$1_P$,射集合は$f:x\to y\mapsto x\le y$とすると,これはいずれも可逆で,確かに圏の同型である.

    この時,集合$P$について,次のように約束する.
    \[ x\le y\land y\le x\Rightarrow x=y \]
    すると集合$P/=$は順序集合(partially ordered set)である.
    関手$F':C\to P/=$は厳密な意味では可逆ではない.
\end{proof}

8. 圏としてのposet:poset categories

9. 位相空間からの例
\begin{proposition*}
    $T_0$ spaces $X$ are posets under the specialization ordering:
    \[ x\le y \Leftrightarrow \forall U\in O(X)\; (x\in U\Rightarrow y\in U) \]
\end{proposition*}

10. 数理論理学からの例:演繹体系に付随する圏 category of proofs
対象を式とし,その間に証明がある$\varphi\vdash\psi$時,射$\varphi\to\psi$を定義する.

11. 計算機科学からの例:関数型プログラミング言語Lに付随する圏$C(L)$
対象は$L$のデータ型,射は関数とする.単位射はdo nothing programで,合成は関数の連続適用$g\circ f=f:g$である.

12. 集合$X$に付随する離散圏$\mathbf{Dis}(X)$

13. 単一対象圏としてのmonoid

射が対象の間に持つ構造「2つの対象と順番付きで紐づけられている」と「単位射の存在」と「合成についての閉性(=推移性)」とを,そっくりそのまま,順序関係に翻訳すれば前順序である.
射自体の持つ構造「結合性」と「単位射の存在」を,代数構造に翻訳すればモノイドである.いずれも最低限の圏である.
それぞれに付加構造として対称性を加えれば,半順序と群を得る.半順序とモノイドが,この本の主要な例になる.

8., 13.の観点から,posetの射とは関手だし,monoidの射も関手と見做せる.

\subsection{Isomorphisms}

\begin{definition}[同型]
    圏$C$に於いて,次を満たす射$f:A\to B$を同型という.
    \[ \exists g:B\to A\in C\; g\circ f=1_A \land f\circ g=1_B \]
\end{definition}
\begin{remark*}[note that, for example in Pos, the category theoretic definition gives the right notion, while there are "bijective homomorphisms" between non-isomorphic posets.]
    射を何らかの写像だとすると,この同型であるための条件は全単射であることと同値.従ってこの定義は,具体圏に於ける台写像の「全単射」性を一般の圏に写し取ったものに思える.
    だから,全単射でないのに同型になることはないはずだ.
    だが,全単射な射は可逆だとは限らない.
\end{remark*}

\begin{definition}[群]
    群とは,可逆なモノイドのことである.従って,全ての射が同型であるような単一対象圏のことである.
\end{definition}

\begin{theorem*}[Cayley]
    群$G=(G,\cdot,e,{}^{-1})$は,$\mathrm{Aut}(G)$の或る部分群と同型になる.
\end{theorem*}
\begin{proof}
    Cayley representation $\overline{G}\subset\mathrm{Aut}(G)$を構成する.各$g\in G$に対して,$\overline{g}\in\overline{G}\subset\mathrm{Aut}(G)$を次のような射として定める.
    \begin{center}\begin{tikzcd}
        \overline{g}=g^*:G\ar[r] \ar[d, phantom, "\rotatebox{90}{$\in$}"] & G \ar[d, phantom, "\rotatebox{90}{$\in$}"] \\
        h \ar[r, mapsto] & g\cdot h
    \end{tikzcd}\end{center}
    この時,$\overline{G}$は群になっていることを,写像$F:G\to \overline{G}$が群の射であることを示すことによって確認する.
    $F(f\cdot g)=F(f)\circ F(g)$は$G$の演算$\cdot$の結合性より,また$F(e)=1_G$も成り立つ.
    なお,各射の可逆性については,$F(f\cdot f^{-1})=F(f)\circ F(f^{-1})=1_G=F(e)$より成り立つ.

    群の射$F:G\to \overline{G}$の逆射$H$を構成する.
    \begin{center}\begin{tikzcd}
        H:\overline{G}\ar[r] \ar[d, phantom, "\rotatebox{90}{$\in$}"] & G \ar[d, phantom, "\rotatebox{90}{$\in$}"] \\
        \overline{g} \ar[r, mapsto] & g=\overline{g}(e)
    \end{tikzcd}\end{center}
    これについて,確かに$F\circ H=1_{\overline{G}}, H\circ F=1_G$が成り立つ.従って,$G\simeq \overline{G}$
\end{proof}

\begin{remark}[Two different levels of isomorphisms]
    構成した群$\overline{G}\subset\mathrm{Aut}(G)$の元である,$g$を集合$G$に左から作用させる写像$\overline{g}$は,群$G$の置換であり,集合の同型である.
    一方,構成した関手$F,H$は群の同型である.
\end{remark}

\begin{theorem}
    任意の圏$C$は,或る具体圏と同型である.
\end{theorem}
\begin{proof}
    圏$C$から,同型な圏$\overline{C}$を構成する.関手$\overline{ }:C\to\overline{C}$の対象写像を次のように定める.
    \begin{center}\begin{tikzcd}
        C \ar[r] \ar[d, phantom, "\rotatebox{90}{$\in$}"] & \overline{C} \ar[d, phantom, "\rotatebox{90}{$\in$}"] \\
        c \ar[r, mapsto] & \overline{c}=\{ f\in\mathrm{arr}(C)\mid \mathrm{cod}(f)=c \}
    \end{tikzcd}\end{center}
    射関手を次のように定める.
    \begin{center}\begin{tikzcd}
        C \ar[r] \ar[d, phantom, "\rotatebox{90}{$\in$}"] & \overline{C} \ar[d, phantom, "\rotatebox{90}{$\in$}"]\\
        g:c\to d \ar[r, mapsto] & \overline{g}=g^*:\hom_C(-,c)\to\hom_C(-,d)
    \end{tikzcd}\end{center}
    ただし,この写像$g^*$は,任意の対象$x\in C$に対して,
    \begin{center}\begin{tikzcd}
        \hom_C(x,c) \ar[r] \ar[d, phantom, "\rotatebox{90}{$\in$}"] & \hom_C(x,d)\ar[d, phantom, "\rotatebox{90}{$\in$}"]\\
        f:x\to c \ar[r, mapsto] & g\circ f:x\to d
    \end{tikzcd}\end{center}
    と対応づける写像(関手の射/自然変換)である.
    この関手は可逆であり,逆関手の$\overline{x}\in\overline{C}$成分は射写像は次の通りである.
    \begin{center}\begin{tikzcd}
        \overline{C} \ar[r] \ar[d, phantom, "\rotatebox{90}{$\in$}"] & C \ar[d, phantom, "\rotatebox{90}{$\in$}"]\\
        \overline{g}:\hom_C(-,c)\to\hom_C(-,d) \ar[r, mapsto] & \overline{g}(1_c)
    \end{tikzcd}\end{center}
\end{proof}
\begin{remark*}
    これが「表現」という述語の出処であろう.この時点ではまだ素朴の意味で「$C$の表現$\overline{C}$」という感覚である.
    また,これが「ホム関手」「ホム集合」という概念の出処でもある.集合での表現を持つから,我々の「具体」性という得意分野に引きずりこめるのだ.
    また,集合に頼り過ぎないで,純粋に圏論的なまま理論を豊かにしていくのも大事である.(群論だってそうなのだろう).
    例えば,一般の圏を白紙から考えるとき,対象の間の射全体の集まりは「集合」であるとは限らないのだ.
\end{remark*}

\subsection{Constructions on categories}

\subsubsection{Product}

圏$C\times D$は$(c,d)$という形の対象をもち,射も,合成も,単位射も,直接の「要素毎」の考え方で,新しい圏を想定出来る.

\subsubsection{Opposite}

$f:C\to D\in C$に対して,$f^*:D^*\to C^*\in C^{op}$で,合成の順序も逆にしたもの.

dualityとは,ある圏が,別の圏の反対(の部分圏)になるという対応が成り立つこと(を主張する命題のこと)である.

\subsubsection{arrow category}

圏$C$に対して,その射を対象とし,その間の射を$g:(f:A\to B)\to (f':A'\to B')$を,次の$f'\circ g_1=g_2\circ f$を主張する可換図式,つまり,圏$C$の射の組$g:=(g_1,g_2)$とする圏である.
\begin{center}\begin{tikzcd}
    A \ar[r, "g_1"] \ar[d, "f"'] & A' \ar[d, "f'"]\\
    B \ar[r, "g_2"] & B'
\end{tikzcd}\end{center}
合成は,可換図式を繋げて外回りを取ること,つまり成分毎$(h_1,h_2)\circ (g_1,g_2)=(h_1\circ g_1, h_2\circ g_2)$で,従って単位射は$1_f=(1_A,1_B)$

対象は射$f:A\to B$だが,要は$(A,B)$,これはどう考えても$C\times C$あるいは$[2,C]$と同型になる.即ち,次の関手が存在する.
\begin{center}\begin{tikzcd}
    C & \overrightarrow{C} \ar[l, "\mathrm{dom}"'] \ar[r, "\mathrm{cod}"] & C
\end{tikzcd}\end{center}

\subsubsection{slice category}

圏$C$と対象$c\in C$について,$\{ f\in\mathrm{arr}(C)\mid \mathrm{cod}(f)=c \}$を対象全体の集合とし,2つの対象$f:x\to c, f':x'\to c$の間の射は次の$C$の図式を可換にする射$a:x'\to x\in C$である($f=f'\circ a$).
\begin{center}\begin{tikzcd}
    x\ar[rr, "a"] \ar[dr, "f"'] & & x' \ar[dl, "f'"] \\
    &c&
\end{tikzcd}\end{center}
これはarrow categoryの部分圏であろう.

対象について,そのcodomain$c$を忘れ,射$a:(x,c)\to (x',c)$についても$c$を忘れれば,忘却関手$C/c\to C$を定める.
これは一種の具体圏だったのか.

$C$の射$g:c\to d$に対して,関手$g^*:C/c\to C/d$が定まる.
\begin{center}\begin{tikzcd}
    C/c \ar[r] \ar[d, phantom, "\rotatebox{90}{$\in$}"] & C/d\ar[d, phantom, "\rotatebox{90}{$\in$}"]\\
    f:x\to c \ar[r, mapsto] & g\circ f:x\to d \\
    a:(f:x\to c)\to (f':x'\to c) \ar[r] & a:(g\circ f:x\to d)\to (g\circ f':x'\to d)
\end{tikzcd}\end{center}

slice categoryの構成は,関手$C/(-):C\to \mathbf{Cat}$を定める.これは圏$C$に対して,関手圏としての表現を与える表現関手,と思うことが出来る.

\subsection{Free categories}
\subsection{Foundations: large, small, and locally small}
\subsection{Exercises}

\section{Abstract structures}

\subsection{Epis and monos}
\subsection{Initial and terminal objects}
\subsection{Generalized elements}
\subsection{Products}
\subsection{Examples of products}
\subsection{Categories with products}
\subsection{Hom-sets}
\subsection{Exercises}

\section{Duality}

\subsection{The duality principle}
\subsection{Coproducts}
\subsection{Equalizers}
\subsection{Coequalizers}
\subsection{Exercises}

\section{Groups and categories}

\subsection{Groups in category}
\subsection{The category of groups}
\subsection{Groups as categories}
\subsection{Finitely presented categories}
\subsection{Exercises}

\section{Limits and colimits}

\subsection{Subobjects}
\subsection{Pullbacks}
\subsection{Properties of pullbacks}
\subsection{Limits}
\subsection{Perservation of limits}
\subsection{Colimits}
\subsection{Exercises}

\section{Exponentials}

\subsection{Exponential in a category}

\subsection{Cartesian closed categories}

\subsection{Heyting algebras}

\subsection{Propositional definition of CCC}

\subsection{$\lambda$-calculus}

\subsection{Variable sets}

\subsection{exercises}

\section{Naturality}

\subsection{Category of categories}

\subsection{Representable structure}

\subsection{Stone duality}

\subsection{Naturality}

\subsection{Examples of natural transformations}

\subsection{Exponentials of categories}

\subsection{Functor categories}

\subsection{Monoidal categories}

\subsection{Equivalence of categories}

\subsection{Examples of equivalence}

\subsection{Exercises}

\section{Categories of diagrams}

\subsection{Set-valued functor categories}

\subsection{The Yoneda embedding}

\subsection{The Yoneda lemma}

\subsection{Applications of the Yoneda lemma}

\subsection{Limits in categories of diagrams}

\subsection{Colimits in categories of diagrams}

\subsection{Exponentials in categories of diagrams}

\subsection{Topoi}

\subsection{Exercises}

\section{Adjoints}

\subsection{Preliminary definition}

\subsection{Hom-set definition}

\subsection{Examples of adjoints}

\subsection{Order adjoints}

\subsection{Quantifiers as adjoints}

\subsection{RAPL}

\subsection{Locally cartesian closed categories}

\subsection{Adjoint functor theorem}

\subsection{Exercises}

\section{Monads and algebras}

\subsection{The triange identities}

\subsection{Mondas and adjoints}

\subsection{Algebras for a monad}

\subsection{Comonads and coalgebras}

\subsection{Algebras for endofunctors}

\subsection{Exercises}

\end{document}